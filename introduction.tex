\section*{Вступ}
\addcontentsline{toc}{section}{Вступ}
Підвищення ефективності експлуатації 
\nomenclature{АТ}{авіаційна техніка}, рівня безпеки 
\nomenclature{ПС}{повітряне судно}, зниження затрат на 
\nomenclature{ТОіР}{технічне обслуговування та ремонт} і 
комплектуючих виробів є основними задачами в сфері цивільної 
авіації. їх розв'язок можливий за уови корінної перебудови 
всієї системи ТОіР ПС, основу якої складає напрацювання і 
ресурс виробів.

Планово-попереджувальна система ТОіР не відповідає підвищеним 
вимогам до АТ. Перспективною є система ТОіР ПС по стану, яка 
передбачає збільшення часу екстплуатації АТ, зниження 
експлуатаційних витрат і підвищення рівня безпеки польотів.

При розробці методів і засобів настройки окремих елементів і 
функціональних сиситем ПС, необхідних для впровадження системи 
ТОіР за станом окремих типів ПС та їх комплектуючих виробів, 
значну увагу приділяють системам автоматичного керування руху 
як системам, що суттєво впливають на безпеку та економічність 
польоту ПС.

Під час експлуатації конструкція повітряного судна, його 
агрегати й окремі частини знаходяться під дією різноманітних 
навантажень, що спричиняють поступову зміну параметрів 
математичної моделі ПС.  Характер дії цих сил може бути 
різноманітними. Постійним від завантаження літака, підіймальної 
сили, змінним від дії аеродинамічних сил, епізодичним при 
ударних навантаженнях при посадці, зіткненні з нерівностями на 
злітно-посадковій смузі та ін. В результаті дії цих сил 
накопичуються втомні деформації, змінюються параметри механічних 
з'єднань важелів управління з керуючими поверхнями, що спричинює 
до невідповідності реакцій еталонної математичної моделі та 
реальної системи.

На сьогоднішній день експлуатаційне обслуговування ПС, зокрема 
діагностика та настройка \nomenclature{САК}{система автоматичного керування} 
ПС ведеться згідно регламенту, в 
якому передбачається перевірка основних конструктивних параметрів 
складових елементів САК через визначений час експлуатації, ремонт 
чи заміна окремих вузлів та елементів конструкції згідно вимог 
експлуатаційної документації. Терміни, склад і порядок виконання 
регламентних робіт складають на основі результатів 
передексплуатаційних випробувань. При цьому основні способи 
настройки спрямовані на утримання основних конструктивних 
параметрах в заданому значенні з відповідними похибками, а 
дослідження всієї системи в цілому можливе лише під час її 
активної роботи, тобто під час польоту ПС.
