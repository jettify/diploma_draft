\section*{ВСТУП}
\addcontentsline{toc}{section}{Вступ}

З розвитком та вдосконаленням літальних апаратів, ускладненням та розширенням 
виконуваних ними польотних завдань, розвивались та вдосконалювались пілотажно-навігаційні 
прилади та системи, які з впровадженням в склад бортового обладнання високопродуктивної 
обчислювальної техніки стало об’єднувати в пілотажно-навігаційні комплекси (ПНК).

ПНК є логічним наслідком еволюції систем навігації та управління і являє собою якісно 
новий ступінь в автоматизації літаководіння. В склад ПНК сучасного літального апарату 
будь-якого класу входять декілька навігаційних систем, зокрема інерціальні 
(ІНС) та супутникові (СНС) системи навігації. Завдяки різній фізичній природі 
та різним принципам формування навігаційного алгоритмічного забезпечення,  ІНС 
та СНС добре доповнюють одна одну, що природно визначило їхню інтеграцію в 
складі сучасних ПНК, у якості  інтегрованих інерціально-супутникових систем 
навігації (ІССН). Доцільність спільного використання ІНС та СНС дозволяє, з 
одного боку, обмежити зростання похибок ІНС (головний недолік цієї системи) а, 
з іншого боку, знизити шумову складову похибок СНС, підвищити темп видачі 
інформації бортовим споживачам, істотно підняти рівень завадозахищеності (недоліки СНС), 
крім того забезпечується висока інформативність інтегрованої системи. 

В результаті комплексування ІНС та СНС досягаються:
\begin{itemize}
 \item підвищення точності визначення координат, висоти, швидкості і часу споживача;
 \item уточнення кутів орієнтації (курсу, крену і тангажа); 
 \item оцінка й уточнення параметрів калібрування навігаційних датчиків, таких, як 
дрейфи гіроскопів, масштабні коефіцієнти, зсуви акселерометрів тощо;
 \item забезпечення на цій основі безперервності навігаційних визначень на 
всіх етапах руху, у тому числі і при тимчасовій непрацездатності приймача СНС у 
випадках впливу  завад або енергійних маневрів ЛА.
\end{itemize}
Вищезазначені причини призводять до необхідності застосування інтегрованих 
інерціально-супутникових систем для навігації і керування ЛА практично всіх типів. 
Тому комітет міжнародної організації цивільної авіації (ІКАО) з майбутніх навігаційних 
систем (FANS- Future Air Navigation System) прийняв рішення про обов'язкове використання 
систем супутникової навігації в поєднанні з ІНС.

Інтеграція інерціальної та супутникової систем реалізується шляхом комплексування двох
систем. При вирішенні задачі комплексної обробки інформації в інерціально супутникових
системах навігації найбільш привабливим і розповсюдженим є, безумовно, калманівська 
фільтрація.

Фильтр калмана (ФК \nomenclature{ФК}{фільтр Калмана}) --- ефективний рекурсивний фільтр оцінюючий вектор стану системи
ряд неповних і зашумлених вимірів.

ФК призначений для рекурсивного дооцінювання вектора стану апріорно відомої динамічної
системи, для розрахунку поточного стану системи необхідно знати поточні виміри, а
також попередній стан фільтра. Таким чином фільтр Калмана, як і безліч інших
рекурсивних фільтрів, реалізовані в часовому представленні, а не в частотному.
Дана особливість відрізняє його від пакетних фільтрів,які вимагають в поточний
такт роботи знати історію змін і/або оцінок.

В ряді випадків, кількість параметрів, що задають стан об'єкта, більше, ніж кількість
спостерігаємих параметрів, доступно для вимірів. За допомогою моделі об'єкта
по ряду доступних вимірів фільтр Калмана дозволяє отримати оцінку всього вектора
внутрішнього стану об'єкта. 

\textbf{[TODO  добавить с 10 и 13-1 главы монографии выдержки]}

Отже розробка та дослідження працездатності алгоритмів роботи інтегрованих 
інерціально-супутникових систем для навігації і керування ЛА, оцінка 
ступіню впливу похибок датчиків первинної інформації безплатформної 
інерціальної системи (БІНС) та похибок супутникової навігаційної системи 
(СНС) на точнісні характеристики числення навігаційних параметрів і динаміку 
зміни похибок комплексної системи, оцінка впливу перерв у роботі СНС на 
траєкторний рух ЛА при польоті за складним маршрутом складають зміст 
магістерської роботи. Саме тому тема роботи є досить актуальною на 
сьогоднішній час. Робота виконувалася у рамках держбюджетної 
НДР „”(\textbf{[TODO  добавить номер НДР]})