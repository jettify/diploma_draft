\section*{ВСТУП}
\addcontentsline{toc}{section}{Вступ}

Сформована в даний момент практика створення і застосування навігаційних 
систем заснована на використанні інтегрованих інерціально-супутникових 
систем навігації. Інтеграція  інерціальної та супутникової систем реалізується шляхом комплексування двох систем. 

При вирішенні задачі комплексної обробки інформації в інерціально-супутникових сист
емах навігації найбільш привабливою є, безумовно, Калмановська фільтрація. Проте, 
використання фільтра Калмана зустрічає певних труднощів при його практичній реалізації 
на борті ЛА. При комплексуванні оцінюється  положення і швидкість ЛА, причому ці дані 
надходять не тільки споживачам, але і контурам спостереження за затримкою  і за фазою  
приймачів СНС. Причому зв'язок блоку фільтра Калмана з контурами приймача СНС дуже жорсткий, 
тому фільтр Калмана повинний бути дуже швидкодіючій, що обмежується характеристиками 
процесорів бортових ЦОМ. 

Основною перевагою Калманівської фільтрації є те, що при комплексуванні СНС і БІНС на 
виході фільтра Калмана відновлюються  оцінки інструментальних похибок БІНС (похибки 
зсуву нулів гіроскопів і акселерометрів, похибки масштабних коефіцієнтів і т. ін.), 
які використовуються для корекції інерціальних датчиків. Тому при перервах надходження 
даних із приймача отримані раніше оцінки похибок ІНС і її вимірювальних елементів дозволяють 
поліпшити точнісні характеристики ІНС в автономному режимі. 

