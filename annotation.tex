% \documentclass[ukrainian,utf8,simple]{eskdtext}
% \usepackage[numberright]{eskdplain}
% % variables.tex
% This file contains information about author and other specific
% people for use in eskdx collection.

\title{\fontsize{12}{12} \selectfont Інтегрована інерціально-супутникова система навігації, що базується на принципах комплексної обробки інформації
з використанням калманівської фільтрації}
% smaller size of font set for the title in frame
\author{НовікМ.В.}

\ESKDchecker{ФіляшкінМ.К.}
\ESKDnormContr{КозловА.П.}
\ESKDapprovedBy{СинєглазовВ.М.}

\ESKDdepartment{Міністерство освіти і науки України}
\ESKDcompany{Національний авіаційний університет}

\ESKDsignature{НАУ 11 09 02 000 ПЗ}
\ESKDgroup{ІАСУ 608}

\ESKDsectAlign{section}{Center}
\ESKDsectAlign{subsection}{Center}
\ESKDsectAlign{subsubsection}{Center}


% \title{Реферат}
% \begin{document}

\section*{РЕФЕРАТ}
Пояснювальна записка до дипломного проекту <<Інтегрована інерціально-супутникова система навігації, що базується на принципах комплексної обробки інформації з використанням калманівської фільтрації>>: стор. --- \ESKDtotal{page}  , рис. --- \ESKDtotal{figure}, використаних джерел --- \ESKDtotal{bibitem}.



ІНЕРЦІАЛЬНА НАВІГАЦІЙНА СИСТЕМА, МЕТОДИ КОМПЛЕКСНОЇ ОБРОБКИ ІНФОРМАЦІЇ,ФІЛЬТР КАЛМАНА, КОМП’ЮТЕРНО ІНТЕГРОВАНИЙ КОМПЛЕКС.

Об’єкт дослідження --- методи та алгоритми комплексної обробки інформації, принципи побудови інтегрованих навігаційних комплексів, на базі процедури
оптимальної калманівської фільтрації.

Мета диплому --- аналіз та вибір схеми комплексної інерціально-супутникової навігаційної системи та схем оцінювання та корекції в цій системі і, як наслідок, розробка слабко зв’язаної схеми інтеграції, дослідження ступеню впливу похибок датчиків первинної інформації  безплатформної інерціальної системи та точнісні характеристики числення навігаційних параметрів і динаміку зміни похибок, впливу перерв у роботі СНС на траєкторний рух ЛА, моделювання зміни похибок комплексної інерціально-супутникової навігаційної системи.

Метод дослідження --- математичне моделювання.

Розробленей алгоритм авіаційного бортового навігаційного комплексу, що включає безплатформенну інерціальну навігаційну систему, супутникову навігаційну систему та баровисотомір, дозволяє ефективно оцінити навігаційні параметри, залишивши переваги кожної із підсистем і значно знизити вплив їх недоліків.

Матеріали дипломного проекту рекомендується використовувати при проведені наукових досліджень та у навчальному процесі.

%\end{document}
