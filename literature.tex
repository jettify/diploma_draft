\begin{thebibliography}{99}
\bibitem{bib:1} М.К. Філяшкін В.О. Рогожин, А.В. Скрипець, Т.І. Лукінова Інерціально-супутникові навігаційні  системи. - К.: Вид-во НАУ, 2009. - 306 с.
\bibitem{bib:7} Ільін О.Ю., Філяшкін М.К., Черних Ю.О. Пілотажно-навігаційні системи та комплекси. - К.: Вид-во КІ ВПС, 1999. - 335 с.
\bibitem{bib:pnk} Рогожин В.О., Синєглазов В.М., Філяшкін М.К. Пілотажно-навігаційні комплекси повітряних суден -К.: Вид-во НАУ, 2005. - 316с.
\bibitem{bib:zaharin1} Захарин М.И. Некоторые вопросы кинематики инерциальных систем навигации. – Киев: КВИАВУ, 1952.-100 с.
\bibitem{bib:zaharin2} Захарин М.И., Захарин Ф.М. Кинематика инерциальных навигационных систем. – М.: Машиностроение, 1968. – 236 с.
\bibitem{bib:zaharin3} Захарин М.И., Панов А.П., Златкин Ю.М. 40 лет развития теории бесплатформенных инерциальных систем в Украине // Гироскопия и навигация, 1997. No 15.С. 82 – 85.
\bibitem{bib:zaharin4} Захарін М.І. Головні результати розробки теорії безплатформових інерціальних навігаційних систем (БІНС), добуті київською секцією «Ради з проблем навігації та керування рухом» при АН України // Механіка гіроскопічних систем, 1997. No14. С. 108-112.
\bibitem{bib:8} Интегрированные инерциально-спутниковые навигационные системы // Под ред. В.А. Пешехонова. -  С.-Петербург: 2001. - 235 с.
\bibitem{bib:9} Интегрированные комплексы на базе ИНС и приемника «Навстар» // Новости зарубежной науки и техники, Серия «авиационные системы». ГосНИИАС, 1995, №10-12.
\bibitem{bib:10} Кузовков Н.Т., Салычев О.С. Инерциальная навигация и оптимальная фильтрация. - М.: Машиностроение, 1988.  - 216 с.


\bibitem{bib:gps_ins} Grewal M. S. Global Positioning Systems, Inertial Navigation and Integration / M. S. Grewal, L.R.Weill, A. P. Andrews. - A John Wiley and Sons, Inc., Publication, 2007. - 525 p.
\bibitem{bib:12} Соловьев Ю.А. Системы спутниковой навигации - М.: ЭКО-ТРЕНДЗ, 2000.  - 270 с. 
\bibitem{bib:gps1} J. Zander, B. Slimane, and L. Ahlin, Principles of Wireless Communications.// Stockholm: Royal Institute of Technology, 2005.
\bibitem{bib:gps2} A. El-Rabbany, Introduction to GPS - The Global Positioning System // London: Artech House, 2002
\bibitem{bib:gps3} USCG Navigation Center, GPS Standard Positioning Service - Performance Standard, // USCG Navigation Center, Oct. 2001.
\bibitem{bib:gps4} B. W. Parkinson and J. J. Spilker, Global Positioning System: Theory and
 Applications. Progress in Astronautics and Aeronautics, 1996, vol. 163.
\bibitem{bib:joseph} R. S. Bucy and P. D. Joseph, Filtering for Stochastic Processes, with Applications to Guidance, Wiley, New York, 1968.
\bibitem{bib:kalman_1} R. E. Kalman, A new approach to linear filtering and prediction problems, ASME Journal of Basic Engineering, Vol. 82, pp. 34-45, 1960.
\bibitem{bib:kalman_2} R. E. Kalman, New methods in Wiener filtering,  in Proceeding of the First Symposium on Engineering Applications of 
Random Function Theory and Probability, Wiley, New York, 1963.
\bibitem{bib:2} Авиационные приборы и навигационные системы // Под  ред. Бабича О.А. - М.: Изд-во ВВИА им. проф. Н.Е. Жуковского, 1981. - 648 с.
\bibitem{bib:3} Бабич О.А. Обработка информации в навигационных комплексах. - М.: Машиностроение, 1988.  - 212 с.
\bibitem{bib:4} Власенко А. В. Интегральные гироскопы iMEMS - датчики угловой скорости фирмы Analog Devices (Интернет-издание), 2006.
\bibitem{bib:5} Воробьев В.Г., Глухов В.В., Кадышев И.К. Авиационные приборы, информационно-измерительные системы и комплексы. - М.: Транспорт, 1992.  - 399 с.
\bibitem{bib:6} Глобальная спутниковая навигационная система ГЛОНАСС // под ред. В.Н. Харисова, А.И.Петрова, В.А.Болдина. - М.: ИПРЖР, 1998.  - 400 с. 
\bibitem{bib:observ} I. Rhee, M.F. Abdel-Hafez, and J.L. Speyer, Observability of anintegrated GPS/INS during maneuvers. IEEE Transactions onAerospace  Electronic Systems, Apr. 2004, pp. 526-535


\bibitem{bib:g1} G. L. Bierman, Factorization Methods for Discrete Sequential Estimation, Mathematics
in Science and Engineering, Vol. 128, Academic, New York, 1977.
\bibitem{bib:g1} A. Bjorck, "Solving Least Squares Problems by Orthogonalization," BIT, Vol. 7, pp.
1-21, 1967.
\bibitem{bib:g1} F. R. Bletzacker, D. H. Eller, T. M. Gorgette, G. L. Seibert, J. L. Vavrus, and M. D. Wade,
"Kalman Filter Design for Integration of Phase III GPS with an Inertial Navigation
System," in Proceedings of the Institute of Navigation, (Santa Barbara , CA) Jan. 26-29,
1988, pp. 113-129, ION, Alexandria, VA, 1988.
\bibitem{bib:g1} G. Blewitt, "An Automatic Editing Algorithm for GPS Data," Geophysical Research
Letters, Vol. 17, No. 3, pp. 199-202, 1990.
\bibitem{bib:g1} M. S. Braasch, "Improved Modeling of GPS Selective Availability," in Proceedings of
The Institute of Navigation (ION) Annual Technical Meeting, ION, Alexandria, VA,
1993.
\bibitem{bib:g1} M. S. Braasch, "A Signal Model for GPS," NAVIGATION: Journal of the Institute of
Navigation, Vol. 37, No. 4, pp. 363-379, 1990.
\bibitem{bib:g1} E. A. Bretz, "X Marks the Spot, Maybe," IEEE Spectrum, Vol. 37, No. 4, pp. 26-36,
2000.
\bibitem{bib:g1} R. G. Brown and P. Y. C. Hwang, Introduction to Random Signals and Applied Kalman
Filtering, 2nd ed., Wiley, New York, 1992.
\bibitem{bib:g1} R. G. Brown and P. Y C. Hwhang, Introduction to Random Signals and Applied Kalman
Filtering: With Matlab Exercises and Solutions, 3rd ed. Wiley, New York, 1997.
\bibitem{bib:g1} R. S. Bucy and P. D. Joseph, Filtering for Stochastic Processes with Applications to
Guidance, Chelsea, New York,   1968  (republished by the American Mathematical
Society).
\bibitem{bib:g1} N. A. Carlson, "Fast Triangular Formulation of the Square Root Filter," AIAA Journal,
Vol. 11, No. 9, pp. 1259-1265, 1973.
\bibitem{bib:g1} Y Chao and B. W Parkinson, "The Statistics of Selective Availability and Its Effects on
Differential GPS," in Proceedings of the 6th International Technical Meeting of the
Satellite Division of (ION) GPS-93, (Salt Lake City, UT) Sept. 22-24, 1993, ION;
Alexandria, VA, 1993. pp. 1509-1516.
\bibitem{bib:g1} A. B. Chatfield, Fundamentals of High Accuracy Inertial Navigation, American Institute
of Aeronautics and Astronautics, New York, 1997.
\bibitem{bib:g1} S. Cooper and H. Durrant-Whyte, "A Kalman Filter Model for GPS Navigation of Land
Vehicles," Intelligent Robots and Systems, IEEE, Piscataway, NJ, 1994.
\bibitem{bib:g1} J. P. Costas, "Synchronous Communications," Proceedings oftheIRE, Vol. 45, pp. 1713-
1718, 1956.
\bibitem{bib:g1} E. Copros, J. Spiller, T Underwood, and C. Vialet, "An Improved Space Segment for the
End-State WAAS and EGNOS Final Operational Capability," in Proceedings of the
Institute of Navigation, ION GPS-96 (Kansas City, MO), pp. 1119-1125 ION, Alexan
dria, VA, Sept. 1996.
\bibitem{bib:g1} G. G. Coriolis, "Sur les equations du mouvement relatif des systemes de corps," Ecole
Polytechnique, Paris, 1835.
\bibitem{bib:g1} C. C. Counselman, III "Multipath-Rejecting GPS Antennas," Proceedings of the IEEE,
Vol. 87, No. 1, pp. 86-91, 1999.
\bibitem{bib:g1} P. Daum, J. Beyer, and T. F. W Kohler, "Aided Inertial LAnd NAvigation system
 
\bibitem{bib:g1} REFERENCES
(ILANA) with a Minimum Set of Inertial Sensors," in Proceedings of IEEE Position, Locations and Navigation Conference, Las Vegas, 1994.
\bibitem{bib:g1} A. J. Dierendonck,  "Understanding GPS Receiver Terminology: A Tutorial,"  GPS
WORLD January 1995, pp. 34-44.
\bibitem{bib:g1} Mani  Djodat,   "Comparison of Various  Differential  Global  Positioning  Systems,"
California State University, Fullerton Masters' Thesis, Fullerton, CA, 1996.
\bibitem{bib:g1} R. M. du Plessis, "Poor Man's Explanation of Kalman Filtering, or How I Stopped
Worrying and Learned to Love Matrix Inversion," Autonetics Technical Note, Anaheim,
CA, 1967, republished by Taygeta Scientific Incorporated, Monterey, CA, 1996.
\bibitem{bib:g1} P. Dyer and S. McReynolds, "Extension of Square-Root Filtering to Include Process
Noise,"  Journal of Optimization  Theory and Applications, Vol.  3,  pp.  444:58,
1969.
\bibitem{bib:g1} M. B. El-Arini, El-Arini, Robert S. Conker, Thomas W. Albertson, James K. Reagan,
John A. Klobuchar, and Patricia H. Doherty, "Comparison of Real-Time Ionospheric
Algorithms for a GPS Wide-Area Augmentation System," Navigation, Vol. 41, no. 4, pp.
393-413, Winter 1994/1995.
\bibitem{bib:g1} J. A. Farrell and M. Barth, The Global Positioning System \& Inertial Navigation,
McGraw-Hill, New York, 1998.
\bibitem{bib:g1} Federal Aviation Administration (U.S.A.), FAA Specification WAAS FAA-E-2892 B, Oct.
1997.
\bibitem{bib:g1} C. M. Feit, "GPS Range Updates in an Automatic Flight Inspection System: Simulation,
Static and Flight Test Results," in Proceedings of The Institute of Navigation (ION) GPS-
92, pp. 75-86 ION, Alexandria, VA, Sept. 1992.
\bibitem{bib:g1} W A. Feess and S. G. Stephens, "Evaluation of GPS Ionospheric Time Delay Algorithm
for Single Frequency Users," in Proceedings of the IEEE Position, Location, and
Navigation Symposium (PLANS '86) (Las Vegas, NV), Nov. 4-7, 1986, pp. 206-213.
New York, NY, 1986.
\bibitem{bib:g1} M. E. Frerking, "Fifty Years of Progress in Quartz Crystal Frequency Standards,"
Proceedings of the 1996 IEEE International Frequency Control Symposium, IEEE,
New York, 1996, pp. 33-46.
\bibitem{bib:g1} L. Garin, F. van Diggelen, and J. Rousseau, "Strobe and Edge Correlator Multipath
Mitigation for Code," in Proceedings of ION GPS-96, the Ninth International Technical
Meeting of the Satellite Division of the Institute of Navigation, (Kansas City, MO), ION,
Alexandria, VA, 1996. pp. 657-664.
\bibitem{bib:g1} A. Gelb (Editor). Applied Optimal Estimation, MIT Press, Cambridge, MA, 1974.
\bibitem{bib:g1} G. H. Golub and C. F. Van Loan, Matrix Computations, 2nd ed., Johns Hopkins
University Press, Baltimore, MD, 1989.
\bibitem{bib:g1} GPS Interface Control Document ICD-GPS-200, Rockwell International Corporation,
Satellite Systems Division, Revision B, July 3, 1991.
\bibitem{bib:g1} R. L. Greenspan, "Inertial Navigation Technology from 1970-1995," Navigation, The
Journal of the Institute of Navigation, Vol. 42, No. 1, pp. 165-186, 1995.
\bibitem{bib:g1} M. S. Grewal, "GEO Uplink Subsystem (GUS) Clock Steering Algorithms Performance
and Validation Results," in Proceedings of 1999 ION Conference, Vision 2010: Present \&
Future, National Technical Meeting, (San Diego, CA) Jan. 25-27, pp. 853-859 ION,
Alexandria, VA, 1999.
 

\bibitem{bib:g1} M. S. Grewal and A. P. Andrews, Application of Kalman Filtering to GPS, INS, \&
Navigation, Short Course Notes, Kalman Filtering Consulting Associates, Anaheim, CA,
June 2000.
\bibitem{bib:g1} M. S. Grewal and A. P. Andrews, Kalman Filtering: Theory and Practice, 2nd ed., Wiley,
New York, 2000.
\bibitem{bib:g1} M. S. Grewal, W. Brown, S. Evans, P. Hsu, and R. Lucy, "Ionospheric Delay Validation
Using Dual Frequency Signal from GPS at GEO Uplink Subsystem (GUS) Locations,"
Proceedings of ION GPS '99, Satellite Division of the Institute of Navigation Twelfth
International Technical Meeting, Session C4, Atmospheric Effects, (Nashville, TN)
September 14-17, ION, Alexandria, VA, 1999.
\bibitem{bib:g1} M. S. Grewal, W. Brown, and R. Lucy, "Test Results of Geostationary Satellite (GEO)
Uplink Sub-System (GUS) Using GEO Navigation Payloads," Monographs of the Global
Positioning System: Papers Published in Navigation ("Redbook"), Vol. VI, Institute of
Navigation, pp. 339-348, ION, Alexandria, VA, 1999.
\bibitem{bib:g1} M. S. Grewal, W. Brown, P. Hsu, and R. Lucy, "GEO Uplink Subsystem (GUS) Clock
Steering Algorithms Performance, Validation and Test Results," in Proceedings of Thirty-
First Annual Precise Time and Time Interval (PTTI) Systems and Applications Meeting,
(Dana Point, CA), December 7-9, 1999. Time Services Dept, US Naval Observatory,
Washington, DC.
\bibitem{bib:g1} M. S. Grewal, N. Pandya, J. Wu, and E. Carolipio, "Dependence of User Differential
Ranging Error (UDRE) on Augmentation Systems—Ground Station Geometries," in
Proceedings of the Institute of Navigation's (ION) 2000 National Technical Meeting,
(Anaheim, CA) January 26-28, 2000, pp. 80-91, ION Alexandria, VA, 2000.
\bibitem{bib:g1} L. Hagerman,  "Effects of Multipath on Coherent and Noncoherent PRN Ranging
Receiver,"   Aerospace  Report  No.   TOR-0073(3020-03)-3,   Aerospace   Corporation,
Development Planning Division, El Segundo, CA, May 15, 1973.
\bibitem{bib:g1} B. Hassibi, A. H. Sayed, and T Kailath, Indefinite Quadratic Estimation and Control: A
Unified Approach to H2 Theories, SIAM, Philadelphia, PA, 1998.
\bibitem{bib:g1} H. V Henderson and S. R. Searle, "On Deriving the Inverse of a Sum of Matrices," SIAM
Review, Vol. 23, pp. 53-60, 1981.
\bibitem{bib:g1} T A. Herring, "The Global Positioning System," Scientific American, Feb. 1996, pp. 44-
50.
\bibitem{bib:g1} D. Herskovitz,  "A Sampling of Global Positioning System Receivers," Journal of
Electronic Defense, pp. 61-66, May 1994.
\bibitem{bib:g1} B. Hofmann-Wellenhof, H. Lichtenegger, and J. Collins, GPS: Theory and Practice,
Springer-Verlag, Vienna, 1997.

\end{thebibliography}

