\begin{thebibliography}{99}
\bibitem{1} М.К. Філяшкін В.О. Рогожин, А.В. Скрипець, Т.І. Лукінова Інерціально-супутникові навігаційні  системи. - К.: Вид-во НАУ, 2009. - 306 с.
\bibitem{2} Авиационные приборы и навигационные системы // Под  ред. Бабича О.А. - М.: Изд-во ВВИА им. проф. Н.Е. Жуковского, 1981. - 648 с.
\bibitem{3} БАБИЧ О.А. Обработка информации в навигационных комплексах. - М.: Машиностроение, 1988.  - 212 с.
\bibitem{4} ВЛАСЕНКО А. В. Интегральные гироскопы iMEMS - датчики угловой скорости фирмы Analog Devices (Интернет-издание), 2006.
\bibitem{5} ВОРОБЬЕВ В.Г., ГЛУХОВ В.В., КАДЫШЕВ И.К. Авиационные приборы, информационно-измерительные системы и комплексы. - М.: Транспорт, 1992.  - 399 с.
\bibitem{6} Глобальная спутниковая навигационная система ГЛОНАСС // под ред. В.Н. Харисова, А.И.Петрова, В.А.Болдина. - М.: ИПРЖР, 1998.  - 400 с. 
\bibitem{7} Ільін О.Ю., Філяшкін М.К., Черних Ю.О. Пілотажно-навігаційні системи та комплекси. - К.: Вид-во КІ ВПС, 1999. - 335 с.
\bibitem{8} Интегрированные инерциально-спутниковые навигационные системы // Под ред. В.А. Пешехонова. -  С.-Петербург: 2001. - 235 с.
\bibitem{9} Интегрированные комплексы на базе ИНС и приемника «Навстар» // Новости зарубежной науки и техники, Серия «авиационные системы». ГосНИИАС, 1995, №10-12.
\bibitem{10} КУЗОВКОВ Н.Т., САЛЫЧЕВ О.С. Инерциальная навигация и оптимальная фильтрация. - М.: Машиностроение, 1988.  - 216 с.
\bibitem{11} РОГОЖИН В.О., СИНЄГЛАЗОВ В.М., Філяшкін М.К. Пілотажно-навігаційні комплекси повітряних суден -К.: Вид-во НАУ, 2005. - 316с.
\bibitem{12} СОЛОВЬЕВ Ю.А. Системы спутниковой навигации - М.: ЭКО-ТРЕНДЗ, 2000.  - 270 с. 
\bibitem{gps1} J. Zander, B. Slimane, and L. Ahlin, Principles of Wireless Communications.// Stockholm: Royal Institute of Technology, 2005.
\bibitem{gps2} A. El-Rabbany, Introduction to GPS - The Global Positioning System // London: Artech House, 2002
\bibitem{gps3} USCG Navigation Center, GPS Standard Positioning Service - Performance Standard, // USCG Navigation Center, Oct. 2001.
\bibitem{gps4} B. W. Parkinson and J. J. Spilker, Global Positioning System: Theory and
 Applications. Progress in Astronautics and Aeronautics, 1996, vol. 163.
\end{thebibliography}

% \bibitem{13} E. Anderson, Z. Bai, C. Bischoff. LAPACK: A portable linear algebra package for high-performance computers. //In Proceedings of Supercomputing '90, pages 1-10. IEEE Press, 1990. 
% \bibitem{14} Netlib BLAS --- http://www.netlib.org/blas/index.html.
% \bibitem{15} Blitz++ C++ Class Library for Scientific Computing --- http://oonumerics.org/blitz, 1996.

% \bibitem{14) A. El-Rabbany, Introduction to GPS - The Global Positioning System.// London: Artech House, 2002.
% \bibitem{15) USCG Navigation Center, GPS Standard Positioning Service - Performance Standard,// USCG Navigation Center, Oct. 2001.


% \bibitem{13} Li P. GNSS/Pseudolite Signal Re-Acquisition with the aid of INS in Short Signal Blockage Scenarios http://www.gmat.unsw.edu.au/snap/publications/lip_etal2008a.pdf
% \bibitem{16} Wang J. Integration of GPS/INS/Vision sensors to navigate Unmanned Aerial Vehicles /Wang J., Garratt M., Lambert A. and other // Proceedings of XXI Congress of the Int. Society of Photogrammetry and Remote Sensing. - Beijing, China, 2008. - P. 963 - 970.
% \bibitem{18} Bhatti U. I. Improved integrity algorithms for integrated GPS/INS systems in the presence of slowly growing errors: PhD thesis / U. I. Bhatti. - Department of Civil and Environmental Engineering. - Imperial College London, United Kingdom, 2007. - P. 363.
% \bibitem{19} Titterton, D. H. Strapdown Inertial Navigation Technology, Peter Peregrinus Press. - London, 2004. - 549 p.16. 
% \bibitem{20} Grewal M. S. Global Positioning Systems, Inertial Navigation and Integration / M. S. Grewal,L. R. Weill, A. P. Andrews. - A John Wiley & Sons, Inc., Publication, 2007. - 525 p.
