\section{Розробка алгоримів оптимального комплексування в інерціально-супутникових
систем навігації}

Загальною вимогою для організації процесу комплексування є наявність математичних 
моделей підсистем, що підлягають комплексуванню. Сучасний стан обчислювальної техніки, 
знань в області інерціальної та супутникової навігації дозволяють скласти досить 
повні й адекватні моделі цих систем. У комплексі системи описуються на рівні їхніх 
похибок. Таким чином, для нормальної роботи комплексу потрібний адекватний опис похибок 
підсистем, включаючи неконтрольовані джерела похибок. 

\subsection{Моделі похибок  інерціальних навігаційних систем }

Рівняння похибок БІНС описують збурений режим роботи системи і є основою при аналізі 
її точності, при організації корекції, при побудові оптимальних навігаційних алгоритмів.

Матриця переходу від зв'язаної СК до географічної  СК  $B(\psi ,\vartheta ,\gamma )$ має 
вид:
\begin{equation}
\label{eq:noname_1} 
\scriptstyle
B(\psi ,\vartheta ,\gamma )=\left(
\begin{array}{ccc} 
{\scriptstyle \sin \psi \cos \vartheta } & 
{\scriptstyle\cos \psi \sin \gamma -\sin \psi \cos \gamma \sin \vartheta } & 
{\scriptstyle\cos \psi \cos \gamma +\sin \psi \sin \gamma \sin \vartheta } \\ 
{\scriptstyle\cos \psi \cos \vartheta } & 
{\scriptstyle-\sin \psi \sin \gamma -\cos \psi \cos \gamma \sin \vartheta } & 
{\scriptstyle-\sin \psi \cos \gamma +\cos \psi \sin \gamma \sin \vartheta } \\ 
{\scriptstyle\sin \vartheta } & 
{\scriptstyle\cos \gamma \cos \vartheta } & 
{\scriptstyle-\sin \gamma \cos \vartheta } 
\end{array}\right),
\end{equation}

\begin{ESKDexplanation}
\item де $\psi \left(t\right),\vartheta \left(t\right),\gamma \left(t\right)$- кути курсу, 
тангажа та крену ЛА відповідно. 
\end{ESKDexplanation}
Матриця переходу від географічної  СК до  рухомої 
екваторіальної СК $Q\left(\varphi \right)$ має вигляд:

\[Q\left(\varphi \right)=\left(\begin{array}{ccc} {1} & {0} & {0} \\ 
{0} & {\cos\varphi } & {\sin \varphi } \\ 
{0} & {-\sin \varphi } & {\cos \varphi } \end{array}\right),\] 

де $\varphi $- географічна широта.

Матриця переходу від зв'язаної СК до рухомої екваторіальної СК $C(\psi ,\vartheta,\gamma ,\varphi)$ 
задовольняє співвідношенням виду:
\[C\left(\psi ,\vartheta ,\gamma ,\varphi \right)=
Q\left(\varphi \right)\cdot B\left(\psi ,\vartheta ,\gamma \right).\] 
При розв`язанні задач повітряної навігації як основні навігаційні параметри ЛА можна 
розглядати поточні географічні координаті ( довготу $\lambda $, широту $\varphi $ и 
висоту над поверхнею земного еліпсоїда \textit{Н}), проекції шляхової швидкості $V_{E} 
,V_{N} ,V_{h} ,$а також елементи матриці переходу $B\left(\psi ,\vartheta ,\gamma 
\right)$, що характеризує орієнтацію ЛА у просторі.

Вказані навігаційні параметри задовольняє таким диференціальним рівнянням:
\begin{equation}
\left .
\begin{array}{l} 
{\dot{\lambda }=
\frac{V_{E} \left(t\right)}{\left(R_{1}+h\right)\cos \varphi \left(t\right)} } \\ 
{\dot{\varphi }=\frac{V_{N} \left(t\right)}{\left(R_{2} +h\right)} } \\
{\dot{h}=V_{h} \left(t\right)} \end{array}\right\};
\label{eq:coordinates}
\end{equation}
\begin{equation}
\dot{B}=B\Omega_{c} -\Omega_{\Gamma}B ;               
\label{eq:dBmatrix}
\end{equation}
\begin{equation}
\dot{\bar{V}}=B\bar{a}_{c} -\Delta \bar{n}\left(t\right)+\bar{g}_{T} ,     
\label{eq:dVector}
\end{equation}
\begin{ESKDexplanation}
\item де \eqref{eq:coordinates} -- рівняння для числення географічних координат; 
\item \eqref{eq:dBmatrix} -- матричне рівняння Пуассона для визначення матриці 
направляючих косинусів $B\left(\psi ,\vartheta ,\gamma \right)$; 
\item \eqref{eq:dVector} -- векторне рівняння відновно 
проекцій шляхової швидкості ЛА 
$\bar{V}=\left(\begin{array}{ccc} {V_{E} ,} & {V_{N} 
,} & {V_{h} } \end{array}\right)^{T} $; $\bar{a}_{c} \left(t\right)=\left(\begin{array}{ccc} 
{a_{x1} \left(t\right),} & {a_{y1} \left(t\right),} & {a_{z1} \left(t\right)} \end{array}
\right)^{T} $-- вектор проекцій уявного прискорення початку зв'язаної СК на її осі;
\end{ESKDexplanation}
\[\Omega_{c} =\left(\begin{array}{ccc} 
{0} & {-\omega {}_{z1} } & {\omega {}_{y1} } \\ 
{\omega {}_{z1} } & {0} & {-\omega {}_{x1} } \\ 
{-\omega {}_{y1} } & {\omega {}_{x1}} & {0} 
\end{array}\right);\] 
\[\Omega_{\Gamma } =\left(\begin{array}{ccc} 
{0} & {-(\dot{\lambda }+u)\sin \varphi } & {(\dot{\lambda }+u)\cos \varphi } \\ 
{(\dot{\lambda}+u)\sin \varphi } & {0} & {\dot{\varphi }} \\
{-(\dot{\lambda }+u)\cos \varphi } & {-\dot{\varphi }} & {0} 
\end{array}\right);\] 
\begin{ESKDexplanation}
\item $\omega_{x1}$ ,$\omega_{y1}$ ,$\omega_{z1}$-- проекції абсолютної кутової швидкості 
зв'язаної з ЛА СК на її осі; $u$-- кутова швидкість обертання Землі; 
\item $R_{1} $ и $R_{2} $-- головні радіуси кривизни обраного земного еліпсоїда;
\end{ESKDexplanation}

\[\begin{array}{l} 
{R_{1} =a\left[1-e^{2} \sin ^{2} \varphi (t)\right]^{-\frac{1}{2}};} \\ 
{R_{2} =a\left(1-e^{2} \right)\left[1-e^{2} \sin ^{2} \varphi(t)\right]^{-\frac{3}{2}};} 
\end{array}\] 
\begin{ESKDexplanation}
\item $a$,$e$-- велика піввісь и ексцентриситет земного еліпсоїда;
\item $\bar{g}_{T} =(\begin{array}{ccc}{g_{TE},}&{g_{TN},}&{g_{Th} }\end{array})^{T} $
-- вектор проекцій прискорення сили ваги на оси географічної СК;
\item $\Delta \bar{n}=\begin{array}{ccc} {\Delta n_{E} ,} & {\Delta n_{N} ,} & {
\Delta n_{h} ,} \end{array})^{T} $-- вектор проекцій суми переносного и кориолісова 
прискорень на осі географічної СК;
\end{ESKDexplanation}
\[\begin{array}{l} 
{\Delta n_{E} =\frac{V_{E} V_{h} }{R_{1} +h} -\frac{V_{E} V_{N}}{R_{1} +h} tg\varphi +2u\left(V_{h} \cos \varphi -V_{N} \sin \varphi \right);} \\ 
{\Delta n_{N} =\frac{V_{N} V_{h} }{R_{2} +h} +\frac{V_{E}^{2} }{R_{1} +h} tg\varphi+2uV_{E} \sin \varphi ;} \\ 
{\Delta n_{h} =-\frac{V_{E}^{2} }{R_{1} +h} -\frac{V_{N}^{2}}{R_{2} +h} -2uV_{E} \cos \varphi ;} 
\end{array}\] 
\begin{ESKDexplanation}
\item $\bar{g}_{T} =\left[0,0,g_{e} \right]^{T} $-- вектор проекцій нормального 
прискорення сили ваги на осі географічної СК 
$g_{e}=\mu//a^{2}$, $\mu=398600,44\cdot 10^{9} \left[\text{м}^{3}/c^{2} \right]$
\end{ESKDexplanation}


Маючи інформацію  про вихідні координати та проекції шляхової швидкості ЛА, про вихідну 
матрицю орієнтації $B_{0}$ (її визначення є предметом задачі початкового виставлення  
БІНС ), а також про моделі прискорення сили ваги $g^{T}$($\varphi $, $\lambda $, \textit{h}), 
на основі рівнянь \eqref{eq:coordinates}$\div $\eqref{eq:dVector} с використанням  
поточних показів ДУС и акселерометрів можна отримати поточні значення  шуканих навігаційних 
параметрів ЛА.

При точному завдані вихідних умов и при точній  моделі прискорення сили ваги, а також 
при відсутності похибок інерціальних ДПІ и похибок обчислення в наслідок інтегрування 
рівнянь \eqref{eq:coordinates}$\div $\eqref{eq:dVector}  будуть отримані істинні 
значення основних навігаційних параметрів ЛА.

Похибки завдання вихідних координат и проекцій шляхової швидкості ЛА, похибки  початкового 
виставлення , аномальні варіації прискорення сили ваги, похибки інерціальних ДПІ, 
методичні похибки алгоритмів обчислення и похибки через  кінцеву довжину розрядній 
сітці обчислювача (похибки округлення) будуть приводити до похибок визначення шуканих 
навігаційних параметрів ЛА.

У лінійному наближенні еволюція похибок БІНС у визначенні основних навігаційних параметрів 
у часі може бути описана лінійними диференціальними рівняннями похибок.

Рівняння похибок БІНС у визначенні координат випливає з динамічних рівнянь числення 
координат, що наведені в алгоритмах БІНС і мають вигляд:

\begin{equation} 
\label{eq:dRsdins} 
\begin{array}{l} 
{\Delta \dot{R}_{E} =\Delta V_{E}(t)\cdot \frac{R_{\text{З}} }{R\cos \varphi (t)} 
+\Delta R_{N} (t)\frac{V_{E}^{}(t)\sin \varphi (t)}{R_{\text{З}} R\cos ^{2} \varphi (t)} 
-\Delta h(t)\frac{R_{\text{З}} V_{E}^{}(t)}{R^{2} \cos \varphi (t)} ;} \\ 
{\Delta \dot{R}_{N} =\Delta V_{N}(t)\cdot \frac{R_{\text{З}}}{R} -\Delta h(t)\frac{R_{\text{З}} V_{N}(t)}{R^{2}};} \\ 
{\Delta \dot{h} =\Delta V_{h} (t);} \end{array} \end{equation} 
\begin{ESKDexplanation}
\item де $\Delta R_{E} (t)=\Delta \lambda (t)R_{\text{З}}$, $\Delta R_{N}(t)=\Delta\varphi(t)R_{\text{З}} $
-- похибка БІНС у визначенні приведених координат місцезнаходженняЛА; 
\item $\Delta \lambda(t)$,$\Delta \varphi (t)$,$\Delta H(t)$-- похибки БІНС у визначенні 
географічних координат; $\Delta V_{E}(t),\Delta V_{N}(t),\Delta V_{H}(t)$-- похибки 
БІНС у визначенні проекції шляхової швидкості ЛА; 
\item $R=R_{\text{З}}+H$; $R_{\text{З}}$  -- радіус земної сфери; 
\end{ESKDexplanation}
Еволюція похибок БІНС у визначенні проекції шляхової швидкості ЛА $\Delta V_{E}^(t)$,
$\Delta V_{N}(t)$,$\Delta V_{h}(t)$, також може бути отримана з динамічних 
рівнянь числення шляхової швидкості в алгоритмах БІНС, і описується наступною системою 
рівнянь: 
\begin{equation}
\begin{array}{l}{\Delta \dot{V}_{E} =a_{N} \alpha_{h} -a_{h} \alpha_{N} +\sum_{i=1}^{3}b_{1,i}  \Delta a_{i} -\Delta V_{h} U(t)\cos \varphi +\Delta V_{N}U(t)\sin \varphi +} \\ 
{+\frac{\Delta R_{N} }{R_{\text{З}} } \left(U(t)(V_{h} \sin \varphi +V_{N}\cos \varphi \right))-(\frac{\Delta V_{E} }{R\cos \varphi } +\frac{V_{E} \sin \varphi}{R\cos ^{2} \varphi } \frac{\Delta R_{N} }{R_{\text{З}} } )\times } \\ 
{\times (V_{h} \cos \varphi -V_{N} \sin \varphi )+\frac{\Delta hV_{E} }{R^{2} } (V_{h} -V_{N}tg\varphi);} \\
\\
{\Delta \dot{V}_{N} =-a_{E}\alpha_{h} +a_{h} \alpha_{E} +\sum_{i=1}^{3}b_{2,i}  \Delta a_{i} -\Delta V_{E}U(t)\sin \varphi -\Delta V_{h} \dot{\varphi }(t)-} \\ 
{-\frac{\Delta R_{N} }{R_{\text{З}}} V_{E} U(t)\cos \varphi -\frac{\Delta V_{N} }{R} V_{h} -(\frac{\Delta V_{E} }{R\cos \varphi } +\frac{V_{E} \sin \varphi }{R\cos ^{2} \varphi } \frac{\Delta R_{N} }{R_{\text{З}} } )V_{E} \sin \varphi +} \\ 
{+\frac{\Delta h}{R^{2} } (V_{E}^{2} tg\varphi +V_{N} V_{h} );} \\
\\
{\Delta \dot{V}_{h} =a_{E} \alpha_{N} -a_{N} \alpha_{E} +\sum_{i=1}^{3}b_{3,i}  \Delta a_{i} +\Delta V_{E} U(t)\cos \varphi +\Delta V_{N} \dot{\varphi }(t)-} \\ 
{-\frac{\Delta R_{N} }{R_{\text{З}} } V_{E} U(t)\sin \varphi +\frac{\Delta V_{N} }{R} V_{N} +(\frac{\Delta V_{E} }{R\cos \varphi } +\frac{V_{E} \sin \varphi }{R\cos ^{2} \varphi } \frac{\Delta R_{N} }{R_{\text{З}} } )V_{E} \cos \varphi +} \\ 
{+g_{e} \left(-\frac{2\Delta h}{a} +\frac{3}{2} e^{2} \sin \varphi \cos \varphi \frac{\Delta R_{N} }{R_{\text{З}} } \right)-\frac{\Delta h}{R^{2} } \left(V_{E}^{2} +V_{N}^{2} \right),} 
\end{array}
\label{eq:dVsdins}
\end{equation}
\begin{ESKDexplanation}
\item де $b_{ij}$ (i,j=1,2,3) -- елементи матриці направляючих косинусів \textbf{B}; 
\item $\Delta a_{i}$ (i=1,2,3) -- приведені похибки акселерометрів БІНС (з урахуванням 
похибок чисельного інтегрування рівняння  у бортовому обчислювачі); 
\item $a_{H}$ ,$a_{E}$,$a_{N}$ -- поточні значення проекцій уявного прискорення початку зв'язаної СК на 
осі географічної СК; 
\item $\alpha_{H}$, $\alpha_{E}$, $\alpha_{N}$ -- похибки моделювання 
в БІНС орієнтації географічного координатного тригранника ($\alpha_{E} $;
\item $\alpha_{N} $-- похибки побудови вертикалі, $\alpha_{H} $-- азимутальна похибка); 
\item $R=R_{\text{З}} +H$ -- поточна висота; 
\item $U(t)=2\Omega_{\text{З}} +\dot{\lambda }(t)$; $dot{\varphi}(t)=\frac{V_{N}}{R}$; $\dot{\lambda }(t)=\frac{V_{E}}{R\cos \varphi }.$ 
\end{ESKDexplanation}
  Аналіз показує, що еволюція параметрів $\alpha_{h}$, $\alpha_{E}$, $\alpha_{N}$ у часі описується наступною системою рівняннь:
\begin{equation} 
\label{eq:dasdins} \begin{array}{l} 
{\dot{\alpha }_{E} =-\omega_{N} \alpha_{h} +\omega_{h} \alpha_{N} -\frac{\Delta V_{N} }{R} -\sum_{i=1}^{3}b_{1,i}\varepsilon_{i} ,} \\
{\dot{\alpha }_{N} =-\omega_{h} \alpha_{E} +\omega_{E} \alpha_{h} +\frac{\Delta V_{E} }{R} -u\sin \varphi \frac{\Delta R_{N} }{R_{7} }
-\sum_{i=1}^{3}b_{2,i}  \varepsilon_{i} ,} \\ 
{\dot{\alpha }_{h} =-\omega_{E} \alpha_{N} +\omega_{N} \alpha_{E} +\frac{\Delta V_{E} }{R} tg\varphi +(u\cos \varphi +\frac{V_{E} }{R\cos ^{2} \varphi } )
\frac{\Delta R_{N} }{R_{7} } -\sum_{i=1}^{3}b_{3,i}\varepsilon_{i} ,} \end{array} \end{equation} 

\begin{ESKDexplanation}
\item де  $\omega_{E} =-\dot{\varphi }(t),\omega_{N} =\left[u+\dot{\lambda }(t)\right]
\cos \varphi ,\omega_{h} =\left[u+\dot{\lambda }(t)\right]\sin \varphi ,$
\item $\dot{\lambda }=\frac{V_{E} }{R\cos \varphi } ;$ $\dot{\varphi }=\frac{V_{N} }{R} $; $\varepsilon_{i}$
 (i=1,2,3)-- приведені похибкм ДУС БІНС;
\end{ESKDexplanation}
Аналіз показує, що похибки моделювання географічного тригранника $\alpha_{h} $,$\alpha_{E} $ ,
$\alpha_{N} $ зв'язані з похибками визначення координат $\Delta R_{N} $,$\Delta 
R_{E} $ і похибками моделювання орієнтації рухливої екваторіальної СК $\delta_{\xi} $, 
$\delta_{\eta } $, $\delta_{\zeta } $ такими   співвідношеннями:
\[\begin{array}{l} 
{\alpha_{E} =\delta_{\xi } -\frac{\Delta R_{N} }{R_{\text{З} } } ;} \\ 
{\alpha_{N} =\delta_{\eta } \cos \varphi -\delta \sin \varphi +\frac{\Delta R_{E} }{R_{\text{З} } } \cos \varphi ;} \\ 
{\alpha_{h} =\delta_{\eta } \sin \varphi-\delta_{\zeta } \cos \varphi +\frac{\Delta R_{E} }{R_{\text{З} } } \sin \varphi .} 
\end{array}\] 
Еволюція в часі похибок моделювання рухливої екваторіальної СК $\delta_{\xi } $, $\delta_{\eta } $,
$\delta_{\zeta } $ описується більш простими, ніж \eqref{eq:dasdins}, рівняннями:
\[\begin{array}{l} 
{\dot{\delta }_{\xi } =-(u+\dot{\lambda })\delta_{\zeta } -\varepsilon_{\zeta }(t)} \\ 
{\dot{\delta }_{\eta } =-\varepsilon_{\eta } (t)} \\ 
{\dot{\delta }_{\zeta } =-(u+\dot{\lambda })\delta_{\xi } -\varepsilon_{\zeta }(t)}\end{array};\] 
\begin{ESKDexplanation}
\item де $\dot{\lambda }=\frac{V_{E}(t)}{R\cos(t)} $.
\end{ESKDexplanation}
Якщо ввести в розгляд  інерціальну прямокутну геоцентричну СК $\xi_{u} \eta_{u} \zeta_{u} $, 
вісь $\eta_{u} $ якої збігається з віссю $\zeta $, 
а вісь $\xi_{u} $ у момент  \textit{t }= 0 лежить у площині Гринвіцького меридіана, 
то можна сказати, що похибки моделювання орієнтації такої СК $\delta_{\xi_{u}} $,
$\delta_{\eta_{u} } $,$\delta_{\zeta_{u} } $ зв'язані з параметрами $\delta_{\xi }$,
$\delta_{\eta } $,$\delta_{\zeta }$ співвідношеннями виду:
\[\begin{array}{l} 
{\delta_{\xi } =\delta_{\xi_{u} } Aos\lambda_{*} -\delta_{\xi_{u} } \sin \lambda_{*} } \\ 
{\delta_{\eta } =\delta_{\eta_{u} }} \\ 
{\delta_{\zeta } =\delta_{\xi_{u} } \sin \lambda_{*} -\delta_{\zeta_{u} } Aos\lambda_{*} } \end{array};\] 
\begin{ESKDexplanation}
\itemде  $\lambda_{*} =ut+\lambda (t)$ .
\end{ESKDexplanation}
Рівняння, що описують еволюцію в часі похибок моделювання інерціальної СК $\delta_{\xi_{u} } $,
$\delta_{\eta_{u} } $,$\delta_{\zeta_{u} } $ виявляється досить простими:

\[\begin{array}{l}{\dot{\delta }_{\xi_{u} } =-\varepsilon_{\xi_{u}} (t);} \\ 
{\dot{\delta}_{\eta_{u} } =-\varepsilon_{\eta_{u} } (t);} \\ 
{\dot{\delta}_{\zeta_{u} } =-\varepsilon_{\zeta_{u} } (t),} \end{array}\] 

де $\left(
\begin{array}{l} {\varepsilon_{\xi_{u} } } \\ 
{\varepsilon_{\eta_{{\rm u}} } } \\ 
{\varepsilon_{\zeta_{u} } } \end{array}\right)=\Delta {\bf C}(t){
\bf C}(t)\left(\begin{array}{l} {\varepsilon_{1} } \\ 
{\varepsilon_{2} } \\ 
{\varepsilon_{3} } \end{array}\right)$;

$\Delta {\bf C}(t)=\left(\begin{array}{ccccc} 
{\cos \lambda_{*} } & {} & {-\sin \lambda_{*} } & {} & {0} \\ 
{\sin \lambda_{*} } & {} & {\cos \lambda_{*} } & {} & {0} \\ 
{0} & {} & {0} & {} & {1} \end{array}\right)$ -- матриця переходу від рухливої  
екваторіальної СК до  інерціальної СК.

Таким чином, у моделі похибок БІНС можливе використання принаймні трьох груп параметрів, 
що характеризують похибки моделювання орієнтації СК:

\{$\alpha_{E} $,$\alpha_{N} $,$\alpha_{h} $\},\{$\delta_{\xi } $,$\delta_{\eta} $,
$\delta_{\zeta } $\}, \{$\delta_{\xi_{u} } $,$\delta_{\eta_{u}} $,$\delta_{\zeta_{u} } $\}.

Надалі в роботі використовуються параметри $\alpha_{E} $,$\alpha_{N} $,$\alpha_{h} $, 
що характеризують похибки  моделювання географічної СК і мають найбільш наочну 
фізичну інтерпретацію. Цим параметрам відповідають рівняння еволюції \eqref{eq:dasdins}.

Для замикання системи рівнянь похибок БІНС \eqref{eq:dRsdins}, \eqref{eq:dVsdins}, 
\eqref{eq:dasdins} необхідно вказати моделі еволюції приведених похибок ДПІ. 

З урахуванням вигляду моделі еволюції похибок ДПІ , 
рівняння похибок БІНС \eqref{eq:dRsdins}, \eqref{eq:dVsdins}, \eqref{eq:dasdins} 
можуть бути замкненні  наступними  рівняннями відносно 
$C_{\omega } $, $C_{a} $, $C_{\varepsilon} $, $D_{a} $, $\bar{\varepsilon }_{A} $, $\Delta \bar{a}_{c} $:

\begin{equation} \label{eq:dawsdins} \begin{array}{l} 
{\dot{C}_{\omega } =\xi_{A\omega } (t);} \\ 
{\dot{C}_{a} =\xi_{Aa} (t);} \\ 
{\dot{C}_{\varepsilon } =\xi_{A\varepsilon } (t);} \\ 
{\dot{D}_{a} =\xi_{Da}(t);} \\ 
{\dot{\bar{\varepsilon }}_{c} =\bar{\xi}_{A} (t);} \\ 
{\Delta \dot{\bar{a}}_{c} =\bar{\xi }_{\Delta a}(t),} \end{array} \end{equation} 
\begin{ESKDexplanation}
\item де $\xi_{A\omega } (t);$$\xi_{Aa} (t);$\item $\xi_{A\varepsilon } (t);$$\xi_{Da} (t);$
$\bar{\xi }_{A} (t);$$\bar{\xi }_{\Delta a} (t)$-- білошумні збурення відповідної розмірності, 
які характеризують дрейф квазістаціонарних  параметрів моделі ДПІ. % \eqref{eq:__6_13_}.
\end{ESKDexplanation}
Повертаючись до моделей похибок БІНС відзначимо, що коли  вектор-стовпець похибок БІНС $\bar{X}(t)$ прийняти 
у вигляді:
\[\bar{X}=(\Delta R_{E} ,\Delta R_{N} ,\Delta h,\Delta V_{E} ,\Delta V_{N} ,\Delta 
V_{h} ,\alpha_{E} ,\alpha_{N} ,\alpha_{h} ,\varepsilon_{c1} ,\varepsilon_{c2} 
,\varepsilon_{c3} ,\Delta a_{c1} ,\Delta a_{c2} ,\Delta a_{c3} ,)^{T} ,\] 
то модель еволюції похибок БІНС може бути подана у компактній формі
\begin{equation} 
\label{eq:matrix_sdins} \dot{\bar{X}}=F\bar{X}\left(t\right)+G\bar{\xi }(t), 
\end{equation} 
де F та G   --  матриці 15 $\times$ 15 і 15 $\times$ 21 відповідно; $\bar{\xi }(t)$ -- вектор-стовпець 
розмірності 21, компонентами якого є незалежні Гауссівські «білі» шуми з нульовими 
середніми значеннями и одиничними дисперсіями.

Відмінні від нуля елементи матриці $F$ мають вигляд:
\begin{equation}
\label{eq:Fsdins_} 
\begin{array}{l}
{f_{1,2} =\frac{\dot{\lambda }}{R_{\text{З}} } tg\varphi;f_{1,3} =\frac{-\dot{\lambda }R_{\text{З}} }{R};f_{1,4} =\frac{R_{\text{З}} }{R\cos \varphi } ;}\\
{f_{2,3} =\frac{-\dot{\varphi }R_{\text{З}} }{R};f_{2,5} =\frac{R_{\text{З}} }{R};f_{3,6} =1;}\\
{f_{4,2} =\frac{2u+\dot{\lambda }}{R_{\text{З}} } \left(V_{h} \sin \varphi +V_{N} \cos \varphi \right)-\frac{\dot{\lambda }}{R_{\text{З}} } tg\varphi \left(V_{h} \cos \varphi -V_{N} \sin \varphi \right);}\\
{f_{4,3}=\frac{V_{E} }{R^{2} } \left(V_{h} -V_{N} tg\varphi \right);}\\
{f_{4,4}=\frac{V_{N}\sin \varphi -V_{h} \cos \varphi }{R\cos \varphi } ;}\\
{f_{4,5}=\left(2u+\dot{\lambda }\right)\sin \varphi; f_{4,6}=-\left(2u+\dot{\lambda }\right)\cos \varphi ;}\\
{f_{4,8}=-a_{h};f_{4,9}=a_{N};f_{4,13}=b_{1,1};f_{4,14}=b_{1,2};f_{4,15}=b_{1,3};}\\
{f_{5,2}=-\frac{2u+\dot{\lambda }}{R_{\text{З}} }V_{E} \cos \varphi -\frac{V_{E}^{2} }{RR_{\text{З}} } tg^{2} \varphi ;}\\
{f_{5,3}=\frac{V_{E}^{2} tg\varphi +V_{h} V_{N} }{R^{2} } ;}\\ 
{f_{5,4}=-\left(2u+\dot{\lambda }\right)\sin \varphi;f_{5,5}=-\frac{V_{h} }{R};}\\
{f_{5,6}=-\dot{\varphi }(t);f_{5,7}=a_{h}; f_{5,9} =-a_{E};f_{5,13}=b_{2,1};f_{5,14}=b_{2,2};f_{5,15}=b_{2,3} ;}\\
{f_{6,2} =-2u\frac{V_{E}^{} \sin \varphi }{R} +\frac{3g_{e} }{2R_{\text{З}}} e^{2} \sin \varphi \cos \varphi ;}\\
{f_{6,3} =-\frac{2g_{e} }{a} -\frac{V_{E}^{2} +V_{N}^{2}}{R^{2} };f_{6,4} =\left(2u+\dot{\lambda }\right)\cos \varphi ;}\\
{f_{6,5}=\dot{\varphi }(t)+\frac{V_{N} }{R};f_{6,7} =-a_{N} ;f_{6,8} =a_{E} ;f_{6,13}=b_{3,1};f_{6,14}=b_{3,2}; f_{6,15}=b_{3,3} ;}\\ 
{f_{7,5}=-\frac{1}{R} ;f_{7,8}=\omega_{h}; f_{7,9}=-\omega_{N} ;}\\
{f_{7,10}=-b_{1,1};f_{7,11}=-b_{1,2};f_{7,12}=-b_{1,3} ;}\\ 
{f_{8,2} =-\frac{u}{R} \sin \varphi; f_{8,4} =\frac{1}{R};f_{8,7} =-\omega_{h};f_{8,9} =\omega_{E};}\\
{f_{8,10} =-b_{2,1}; f_{8,11} =-b_{2,2}; f_{8,12} =-b_{2,3} ;}\\ 
{f_{9,2} =\frac{1}{R}_{7} (u\cos \varphi +\frac{\dot{\lambda }}{\cos \varphi });}\\
{f_{9,4} =\frac{tg\varphi }{R}; f_{9,7} =\omega_{N}; f_{9,8} =-\omega_{E};}\\ 
{f_{9,10} =-b_{3,1};f_{9,11} =-b_{3,2};f_{9,12} =-b_{3,3}.}
\end{array}
\end{equation} 
Відрізні від нуля елементи матриці \textit{G}(15$\times $21) задовольняють таким 
співвідношенням:
\begin{equation} 
\label{eq:Gmatrix_} 
\begin{array}{cc} 
g_{i,i} =\sigma_{i},& i=1,..,15; \\ 
g_{i+3,j+18}=b_{i,j} \sigma_{a},& i=1,2,3,j=1,2,3;\\ 
g_{i+6,j+15}=-\sigma_{\omega } b_{i,j} & i=1,2,3,j=1,2,3; 
\end{array} 
\end{equation} 
\begin{ESKDexplanation}
\item де $\sigma_{1} \div \sigma_{15} $- середньоквадратичні значення (СКЗ) білошумних збурень, 
що характеризують вплив різних факторів ($\sigma_{1} \div \sigma_{3} $-- похибок 
численного інтегрування рівняння \eqref{eq:coordinates}; 
\item  $\sigma_{4} \div \sigma_{6} $ --  підсумковий ефект аномалій гравітаційного поля и похибок численного інтегрування 
рівняння \eqref{eq:dVector}, 
\item $\sigma_{7} \div \sigma_{9} $-- похибок численного інтегрування рівняння для параметрів орієнтації \eqref{eq:dBmatrix};  
\item $\sigma_{10} \div \sigma_{15} $ -- випадкового дрейфу квазістаціонарних зведених погрішностей 
ДПІ  $\bar{\varepsilon }_{A} $ и $\Delta \bar{0}_{A} $);
\item $\sigma_{a} $, $\sigma_{\omega } $ -- СКЗ білошумних складових погрішностей акселерометрів і ДКШ БІНС.
\end{ESKDexplanation}

Елементи матриць \textbf{F} и \textbf{G}, що випливає з аналізу співвідношень \eqref{eq:Fsdins_} 
и \eqref{eq:Gmatrix_}, залежать від поточних значень навігаційних параметрів польоту 
ЛА.

Безперервної моделі еволюції похибок БІНС  \eqref{eq:matrix_sdins} відповідає такий 
дискретний аналог:

\[\bar{X}_{k+1} =\Phi_{k} \bar{X}_{k} +G_{k} \bar{\xi}_{k} ,\] 
\begin{ESKDexplanation}
\item де $\Phi_{k}=E+F(t_{k})\Delta t$, $G_{k} =G(t_{k})\cdot \Delta t;$ 
\item $\Delta t$--  крок дискретизації часу;
\item $E$ -- одинична матриця  $15\times 15$.
\end{ESKDexplanation}

\subsection{Математичні моделі похибок супутникової системи навігації}

Для опису  похибок СНС у визначенні координат і проекцій шляхової швидкості ЛА пропонується 
використовувати математичні моделі, що містять  Марківські і гаусовські складові 
похибок:
\begin{equation} \label{eq:sns_errors} 
\begin{array}{l} 
{\Delta R_{Es,k} =\Delta R_{Ec,k} +\frac{\sigma_{Rs} }{\cos \varphi_{k} } \eta_{REs,k} +\frac{\sigma_{\delta Rs} }{\cos \varphi_{k} } \eta_{\delta RE,k} ;} \\ 
{\Delta R_{Ns,k} =\Delta R_{Nc,k} +\sigma_{Rs} \eta_{RNs,k} +\sigma_{\delta Rs} \eta_{\delta RN,k} ;} \\ 
{\Delta H_{s,k} =\Delta H_{c,k} +\sigma_{Hs} \eta_{Hs,k} +\sigma_{\delta Rs} \eta_{\delta H,k} }\\ 
{\Delta V_{ls,k} =\Delta V_{lc,k} +\sigma_{Vs} \eta_{V\, ls,k} +\sigma_{\delta Vs} \eta_{\delta V\, ls,k}, \text{при } l=E,N,H;} 
\end{array} \end{equation} 

\begin{ESKDexplanation}
\item де $\Delta R_{ls,k}$, (l=E,N); $\Delta H_{s,k}$; $\Delta V_{ls,k}$ 
 (l=E,N,H) -- похибки СНС у визначенні приведених  координат, висоти і складових 
шляхової швидкості ЛА;
\item $\Delta R_{lc,k}$ (l=E,N);  $\Delta H_{c,k}$; $\Delta V_{lc,k}$ 
 (l=E,N,H) -- корельовані (Марківські) складові  похибок СНС;
\item $\sigma_{Rs} $, $\sigma_{Hs}$ , $\sigma_{Vs}$  --  СКЗ білошумових складових 
похибок СНС;
\item $\sigma_{\delta Rs} $, $\sigma_{\delta Hs} $, $\sigma_{\delta Vs} $ -- СКЗ додаткових 
білошумових складових похибок СНС, що виникають тільки за умови, що $t_{k}$ -- момент 
зміни сузір'я навігаційних супутників; 
\item $\eta_{Rls,k}$, $\eta_{\delta Rls,k}$, (l=E,N); $\eta_{Hs,k}$ 
, $\eta_{\delta Hs,k}$; $\eta_{V\, ls,k}$ ,$\eta_{\delta V\, ls,k}$ 
 (l=E,N,H)-- стандартні білі дискретні шуми зі СКЗ.
\end{ESKDexplanation}

Корельовані складові похибок СНС описуються наступними співвідношеннями:
\begin{equation}\label{eq:sns_errors_cor} 
\begin{array}{l} 
{\Delta R_{Ec,k}=W_{R} \Delta R_{Ec,k-1} +q_{R} \frac{\sigma_{Rc} }{\cos \varphi_{k} } \eta_{REc,k} +\frac{\sigma_{\delta RC} }{\cos \varphi_{k} } \eta_{\delta REc,k} ;} \\ 
{\Delta R_{Nc,k}=W_{R} \Delta R_{Nc,k-1} +q_{R} \sigma_{Rc} \eta_{RNc,k} +\sigma_{\delta RC} \eta_{\delta RNc,k} ;} \\ 
{\Delta H_{c,k}=W_{R}  \Delta H_{c,k-1}  +q_{R} \sigma_{Hc} \eta_{Hc,k} +\sigma_{\delta Hc} \eta_{\delta Hc,k} ;} \\ 
{\Delta V_{lc,k} =W_{V} \Delta V_{lc,k-1} +q_{V}\sigma_{Vc} \eta_{V lc,k} +\sigma_{\delta Vc} \eta_{\delta V lc,k}, \text{при } l=E,N,H,} 
\end{array} \end{equation} 
\begin{ESKDexplanation}
\item де  
\[
\begin{array}{l}
{W_{R} =e^{-(\lambda_{s} V_{\text{Ш}} +\lambda_{st} )\Delta t} ; }
{q_{R} =\left[1-\exp \left(-2\left(\lambda_{s} V_{\text{Ш}} +\lambda_{st} \right)\Delta t\right)\right]^{0,5};}\\
{W_{V} =e^{-\lambda_{V} \Delta t};}
{q_{V} =\left[1-\exp \left(-2 \lambda_{V} \Delta t\right)\right]^{0,5};}
\end{array}\] 
\item $\lambda_{s} $-- показник просторової кореляції похибки СНС за координатами; 
\item $\lambda_{V} ,\lambda_{st} $-- показник часової кореляції похибок СНС за швидкістю та за 
координатами; 
\item $V_{\text{Ш}}$-- шляхова швидкість ЛА;
\item $\Delta t$-- дискрета оновлення вихідної інформації СНС у часі;
\item $\sigma_{Rc} ,\sigma_{Hc} ,\sigma_{Vc} $ -- СКЗ корельованих складових похибок СНС;
\item $\sigma_{\delta Rc} $,  $\sigma_{\delta Hc} $, $\sigma_{\delta_{Vc}} $  -- СКЗ 
додаткових гаусовських збурень у моменти зміни сузір'я навігаційних супутників;
$\eta_{Rlc,k}$ ,  $\eta_{\delta Rlc,k}$  (l=E,N),  $\eta_{Hc,k}$ 
, $\eta_{\delta Hc,k}$, $\eta_{V\, lc,k}$ ,$\eta_{\delta V\, lc,k}$ (l=E,N,H) -- 
стандартні центровані дискретні білі шуми з одиничною інтенсивністю.
\end{ESKDexplanation}

Для стандартного режиму СНС типу GPS NAVSTAR можуть бути рекомендовані наступні значення 
параметрів моделей \eqref{eq:sns_errors}, \eqref{eq:sns_errors_cor}:

\[\begin{array}{l}
{\lambda_{s} =4\cdot 10^{-6} \text{м}^{-1} ;} 
{\lambda_{st} =5\cdot 10^{-4} \text{с}^{-1}; }
{\lambda_{V} =\left(0,0017\div 0,05\right)\, \text{с}^{-1};}\\
{\sigma_{Rs} =\left(1\div 3\right) \text{м};}
{\sigma_{Hs} =\left(1,5\div 4\right)\, \text{м};}
{\sigma_{Vs} =\left(0,01\div 0,05\right) \text{м}/c;}\\
{\sigma_{\delta Rs} =\left(1\div 4\right) \text{м}; }
{\sigma_{\delta_{Vs} } =\left(0,02\div 0,2\right) \text{м}/c;} 
{\sigma_{Rc} =\left(5\div 7\right) \text{м};}\\
{\sigma_{Hc} =\left(7\div 10\right) \text{м};  }
{\sigma_{Vc} =\left(0,02\div 0,3\right) \text{м}/c;}
{\sigma_{\delta Rc} =\left(2\div 5\right) \text{м};}\\
{\sigma_{\delta_{Vc} } =\left(0,01\div 0,02\right) \text{м}/c;}
{\sigma_{\delta Hc} =\left(3\div 7\right) \text{м}.}
\end{array}\] 

% Головною задачею СНС є визначення псевдодальностей $D_{sl}^{} $ і псевдошвидкостей $V_{sl}^{} $ (l 
% = 1, \dots , N  -- число видимих навігаційних супутників),  які задовольняють співвідношенням 
% виду: 
% 
% \[\begin{array}{l} {D_{sl,k} =\{ \left[x_{l} \left(t_{k} -\tau_{l} \right)-x\left(t_{k} 
% \right)\right]^{2} +\left[y_{l} \left(t_{k} -\tau_{l} \right)-y\left(t_{k} \right)
% \right]^{2} +} \\ {\, \, \, \, \, \, \, \, \, \, \, \, \, \, \, \, +\left[z_{l} \left(t_{k} 
% -\tau_{l} \right)-z\left(t_{k} \right)\right]^{2} \} ^{\frac{1}{2} } +c\Delta \tau 
%_{k} ;} \end{array}\] 
% 
% \[\begin{array}{l} {V_{sl,k} =\{ \left[V_{xl} \left(t_{k} -\tau_{l} \right)-V_{x} 
% \left(t_{k} \right)\right]\left[x_{l} \left(t_{k} -\tau_{l} \right)-x\left(t_{k} 
% \right)\right]+} \\ {\, \, \, \, \, \, \, \, \, \, \, \, \, +\left[V_{yl} \left(t_{k} 
% -\tau_{l} \right)-V_{y} \left(t_{k} \right)\right]\left[y_{l} \left(t_{k} -\tau 
%_{l} \right)-y\left(t_{k} \right)\right]+} \\ {\, \, \, \, \, \, \, \, \, \, \, \, 
% \, +\left[V_{zl} \left(t_{k} -\tau_{l} \right)-V_{z} \left(t_{k} \right)\right]
% \left[z_{l} \left(t_{k} -\tau_{l} \right)-z\left(t_{k} \right)\right]+} \\ {\, \, 
% \, \, \, \, \, \, \, \, \, \, \, +\Omega_{{\rm }} x_{l} \left(t_{k} -\tau_{l} \right)y
% \left(t_{k} \right)\, -\Omega_{{\rm }} y_{l} \left(t_{k} -\tau_{l} \right)x\left(t_{k} 
% \right)\} \tilde{D}_{sl,k}^{-1} +cV,} \end{array}\] 
% 
% де $x_{l} \left(t_{k} -\tau_{l} \right)$,$y_{l} \left(t_{k} -\tau_{l} \right)$,$z_{l} 
% \left(t_{k} -\tau_{l} \right)$,$V_{xl} \left(t_{k} -\tau_{l} \right),V_{yl} \left(t_{k} 
% -\tau_{l} \right),V_{zl} \left(t_{k} -\tau_{l} \right)$-- координати і проекції 
% абсолютної швидкості $l$-го навігаційного супутника в прямокутної гринвіцькій  геоцентричній 
% СК XYZ;
% 
% $x\left(t_{k} \right),y\left(t_{k} \right),z\left(t_{k} \right),V_{x} \left(t_{k} 
% \right),V_{y} \left(t_{k} \right),V_{z} \left(t_{k} \right)$ -- координати і проекції 
% шляхової швидкості ЛА в СК XYZ; 
% 
% $\tau_{l} $ -- час проходження радіосигналу від l - го навігаційного супутника до ЛА;
% 
% \[\tilde{D}_{sl,k} 
% =D_{sl,k} -\Delta \tau ;\] 
% 
% $\Delta \tau_{k} =\Delta \tau_{k-1} +V_{\varepsilon_{k} } \Delta t+\sigma_{\zeta 
% \tau } \xi_{\tau ,k-1} \, {\rm B0}\, \, V_{\varepsilon_{k} } =V_{\varepsilon_{k-1} 
% } +\sigma_{\zeta V\tau } \xi_{V\tau ,k-1} $--зрушення та дрейф шкали часу в бортовій 
% апаратурі СНС;
% 
% $\Omega_{\text{З}} $ -- кутова швидкість обертання Землі; c -- швидкість  поширення 
% світла.
% 
% Похибки СНС при визначенні псевдодальностей $D_{sl}^{} $ и псевдошвидкостей $V_{sl}^{} $ можуть 
% бути описані наступними співвідношеннями:
% 
% \begin{equation} \label{eq:__6_16_} \begin{array}{c} {\Delta D_{sl,k} =\Delta 
% D_{scl,k} +\sigma_{DS} \eta_{Dl,k} ;} \\ {\Delta V_{sl,k} =\Delta V_{scl,k} +\sigma 
%_{VS} \eta_{Vl,k} ;} \end{array}\, \, \, \, \, \, \, \, \, \, (l=1,...,N), \end{equation} 
% тут 
% 
% $\begin{array}{l} 
% {\Delta D_{scl,k} =W_{R} \Delta D_{scl,k-1} +q_{R} \sigma_{Ds} \eta_{sl,k} } \\ 
% {\Delta V_{scl,k} =W_{V} \Delta V_{scl,k-1} +q_{V} \sigma_{Vc} \eta_{scl,k} } \end{array}$;                                                   
% \eqref{eq:__6_17_} 
% $\eta_{Dl,k} $,$\eta_{Vl,k} $,$\eta_{sl,k} $,$\eta_{scl,k} $,$\xi_{\tau ,k} $,$\xi 
%_{V\tau ,k} $ -- стандартні центровані дискретні білі шуми з одиничною інтенсивністю;
% 
% $\sigma 
%_{DS} $, $\sigma_{Vs} $ -- СКЗ гаусівських складових похибок СНС у визначенні псевдодальностей 
% і псевдошвидкостей;
% 
% $\sigma_{DC} $, $\sigma_{Vc} $ --  СКЗ корельованих складових похибок СНС у визначенні 
% псевдодальностей і псевдошвидкостей;
% 
%  Для стандартного режиму СНС GPS NAVSTAR можуть бути рекомендовані наступні значення 
% параметрів моделі \eqref{eq:__6_16_}, \eqref{eq:__6_17_}:
% 
% \[\begin{array}{l} {\sigma_{Ds} =(1\div 4){\rm <;}\, \, \, \, \, \sigma_{Vs} =(0,02
% \div 0,03){\rm </A};} \\ {\sigma_{Dc} =(5\div 9){\rm <;}\, \, \, \, \sigma_{Vc} 
% =(0,02\div 0,03){\rm </A};} \end{array}\] 
% 
% При застосуванні в навігаційних розрахунках комбінованих методів додаткову навігаційну 
% функцію дає вимірник висоти. Так, у далекомірному методі при наявності на борту ЛА 
% високоточної системи вимірювання висоти польоту Н, сфера з радіусом Rз + Н (де Rз 
% = 6371116 м -- радіус сфери, рівновеликої земному геоїду) може бути прийнята  за 
% додаткову поверхню положення. У цьому випадку можна замість вимірювань трьох дальностей 
% до НС обмежитися вимірюванням двох дальностей, тоді навігаційна функція буде включати 
% два рівняння сфери, а третє необхідне рівняння дає вимірник висоти 
% 
% (Rз + H)2 = x2 + y2 + z2.
% 
% Ось чому для реалізації процедур оптимального комплексування  інерціальної та супутникової 
% систем навігації необхідно мати додаткову модель похибок барометричного висотоміра.

\subsection{Математичні моделі похибок барометричного висотоміра}

Похибка барометричного висотоміра (БВ) у визначенні абсолютної висоти ЛА може бути 
описана співвідношенням вигляду:
\begin{equation}
 \label{eq:ba}
\Delta h(t_{k} )=\Delta h_{\text{вс}} +\sigma_{h} \eta_{n,k},               
\end{equation}

\begin{ESKDexplanation}
\item де $\Delta h_{\text{вс}}$-- квазістаціонарна похибка виміру барометричної висоти, що обумовлена 
неточністю початкової виставки, а також змінами температури та тиску атмосфери за 
час польоту;
\item $\sigma_{h} $-- СКЗ флюктуаційної складової похибки, що обумовлена пульсаціями 
тиску й іншими факторами;
\item $\eta_{n,k} $-- дискретний білий шум з одиничною інтенсивністю.
\end{ESKDexplanation}
У свою чергу дискретна модель еволюції квазістаціонарної похибки БВ може бути представлена 
в наступному вигляді:
\begin{equation}
\label{eq:ba_discrete}
\Delta h_{c,k} =\Delta h_{c,k-1} +\sigma_{\xi A} \xi_{k-1},                                       
\end{equation}
\begin{ESKDexplanation}
\item де  $\sigma_{\xi A} $-- заданий параметр; 
\item$\xi_{k-1} $ -- стандартний дискретний білий шум з одиничною інтенсивністю.
\end{ESKDexplanation}
Аналіз показує, що для моделі похибок БВ \eqref{eq:ba}, \eqref{eq:ba_discrete} 
можна рекомендувати наступні значення параметрів:
\[\begin{array}{l}
{\sigma_{h} =(0,5\div 1) \text{м};}\\
{\sigma_{\xi c} =(0,05\div 0,02) \text{м};}\\
{\sigma_{\Delta hc,0} =(3\div 5) \text{м};}
\end{array}\] 
\begin{ESKDexplanation}
\item де $\sigma_{\Delta hc,0} $ --  СКЗ похибки $\Delta $\textit{hс} у початковий момент часу.
\end{ESKDexplanation}


\subsection{Аналіз та розробка алгоритмів оптимальної комплексної обробки
навігаційної інформації}

Загальною вимогою для організації процесу комплексування є наявність математичних 
моделей підсистем, що підлягають комплексуванню. Сучасний стан обчислювальної техніки, 
знань в області інерціальної та супутникової навігації дозволяють скласти досить 
повні й адекватні моделі цих систем. У комплексі системи описуються на рівні їхніх 
похибок. Таким чином, для нормальної роботи комплексу потрібний адекватний опис похибок 
підсистем, включаючи неконтрольовані джерела похибок. Розробка 
алгоритмів комплексної обробки навігаційної інформації здійснюватиметься з використанням 
моделей похибок ІНС (\eqref{eq:dRsdins}, \eqref{eq:dVsdins}, \eqref{eq:dasdins}), СНС (\eqref{eq:sns_errors}) 
та баровисотоміра (\eqref{eq:ba}-\eqref{eq:ba_discrete}).

При розгляді слабкозв'язаної схеми інваріантного алгоритму комплексної обробки навігаційної  
інформації для розглянутого складу навігаційних підсистем рекомендується використовувати 
розширений вектор стану, що включає 22 компоненти, у тому числі: 15 компонент -- помилки 
нанотехнологічної БІНС, одна -- систематична помилка БВ, 6 компонент -- корельовані 
помилки СНС у визначенні координат і проекцій швидкості:

\begin{equation}\label{eq:state_vector}\begin{array}{l} 
{\bar{X}=(\Delta R_{E} ,\Delta R_{N} ,\Delta h,\Delta V_{E} ,\Delta V_{N} ,\Delta V_{h} ,\alpha_{E} ,\alpha_{N} ,\alpha_{h} ,
\varepsilon_{c1} ,\varepsilon_{c2} ,\varepsilon_{c3} , \Delta a_{c1} ,} \\ 
{ \Delta a_{c2} ,\Delta a_{c3} ,\Delta h_{\text{БВ}} ,\Delta R_{Ec} ,\Delta R_{Nc} ,\Delta h_{c} ,\Delta V_{Ec} ,\Delta V_{Nc} ,\Delta V_{hc} )^{T} } 
\end{array}\end{equation}

Дискретна 
модель еволюції вектора стану $\bar{X}_{p} $, що отримується на основі \eqref{eq:ba_discrete}, 
\eqref{eq:dasdins}, \eqref{eq:sns_errors}, має вигляд:

\begin{equation}
\label{eq:fullsystem}
\bar{X}_{p,k+1} =\Phi_{p,k} \bar{X}_{p,k} +G_{p,k} \bar{\xi }_{k}
\end{equation}
\begin{ESKDexplanation}
\item де $\Phi_{p,:} =E+F_{p,k} \Delta t;$
\item $\bar{\xi }_{k} $-- 28-мірний вектор центрованих гаусових дискретних білих шумів 
з одиничною інтенсивністю;
\end{ESKDexplanation}
\[ F_{p,k} =\left(\begin{array}{cccccccc} 
{F_{k} } & {.} & {.} & {.} & {.} & {.} & {.} & {.} \\ 
{.} & {0} & {.} & {.} & {.} & {.} & {.} & {.} \\ 
{.} & {.} & {W_{R}} & {.} & {.} & {.} & {.} & {.} \\ 
{.} & {.} & {.} & {W_{R} } & {.} & {.} & {.} & {.} \\ 
{.} & {.} & {.} & {.} & {W_{R} } & {.} & {.} & {.} \\ 
{.} & {.} & {.} & {.} & {.} & {W_{V} } & {.} & {.} \\ 
{.} & {.} & {.} & {.} & {.} & {.} & {W_{V} } & {.} \\ 
{.} & {.} & {.} & {.} & {.} & {.} & {.} & {W_{V} } 
\end{array}\right);\] 

\[G_{p,k} =\left(\begin{array}{ccc} 
{G_{k} } & {.} & {.} \\ 
{.} & {\sigma_{\text{БВ}} \sqrt{\Delta t}} & {.} \\ 
{.} & {.} & {G_{s,k} } \end{array}\right);\] 
\begin{equation*}
\scriptstyle
G_{S,k} = \left(\begin{array}{cccccccccccc} 
\scriptstyle{\frac{q_{R} \sigma_{Rc} }{\scriptstyle \cos \varphi_{k} } } & {.} & {.} & {.} & {.} & {.} & {\scriptstyle \frac{\mu \sigma_{\delta Rc} }{\cos \varphi_{k} } } & 
{.} & {.} & {.} & {.} & {.} \\ 
{.} & {\scriptstyle q_{R} \sigma_{Rc} } & {.} & {.} & {.} & {.} & {.} & {\scriptstyle \mu \sigma_{\delta Rc} } & {.} & {.} & {.} & {.} \\ 
{.} & {.} & {\scriptstyle q_{R} \sigma_{hc} } & {.} & {.} & {.} & {.} & {.} & {\scriptstyle \mu \sigma_{\delta hc} } & {.} & {.} & {.} \\ 
{.} & {.} & {.} & {\scriptstyle q_{V} \sigma_{Vc} } & {.} & {.} & {.} & {.} & {.} & {\scriptstyle \mu \sigma_{\delta Vc} } & {.} & {.} \\ 
{.} & {.} & {.} & {.} & {\scriptstyle q_{V} \sigma_{Vc} } & {.} & {.} & {.} & {.} & {.} & {\scriptstyle \mu \sigma_{\delta Vc} } & {.} \\ 
\scriptstyle{.} & {.} & {.} & {.} & {.} & {\scriptstyle q_{V} \sigma_{Vc} } & {.} & {.} & {.} & {.} & {.} & {\scriptstyle \mu \sigma_{\delta Vc} } \end{array}\right)
\end{equation*}
$\mu =1$ в момент зміни сузір'я $t_{k}^{*} $, $\mu =0$ в будь який інший момент $t_{k}^{} $.

До 
складу вектора спостережень  пропонується включити 8 компонент, у тому числі різницю 
оцінок висоти, видаваних  нанотехнологічної БІНС   і БВ, 3 різниці координат і 3 
різниці складові швидкості, вироблюваних нанотехнологічної БІНС і СНС відповідно, 
а також різниця оцінок висоти, видаваних БВ і СНС відповідно:

\begin{equation} 
\label{eq:measure_vector} 
\bar{Y}_{k} = 
\left(\begin{array}{l}
{\tilde{h}_{k} -\tilde{h}_{\text{БВ},k},}\\
{\tilde{R}_{E,K} -\tilde{R}_{ES,k},}\\
{\tilde{R}_{N,K} -\tilde{R}_{NS,k},}\\
{\tilde{h}_{k} -\tilde{h}_{s,k},}\\
{\tilde{V}_{E,k} -\tilde{V}_{ES,k},}\\
{\tilde{V}_{N,k} -\tilde{V}_{NS,k},}\\
{\tilde{V}_{h,k} -\tilde{V}_{hS,k},}\\
{\tilde{h}_{\text{БВ}} -\tilde{h}_{s,k}}
\end{array} \right)  
\end{equation} 

Рівняння спостережень у компактній  формі має вигляд:

\[\bar{Y}_{k} =H\bar{X}_{p,k} +Q_{p,k} \bar{\eta }_{k} ,\] 

де $\bar{\eta }_{k} $-- 13-мірний вектор стандартних центрованих гаусових дискретних 
білих шумів з одиничною інтенсивністю;

\[H=\left(\begin{array}{ccccccccccccc}    
{.} & {.} & {1} & {.} & {.} & {.} & {\ldots } & {-1} & {.} & {.} & {.} & {.} & {.} \\ 
{1} & {.} & {.} & {.} & {.} & {.} & {\ldots } & {.} & {-1} & {.} & {.} & {.} & {.} \\ 
{.} & {1} & {.} & {.} & {.} & {.} & {\ldots } & {.} & {.} & {-1} & {.} & {.} & {.} \\ 
{.} & {.} & {1} & {.} & {.} & {.} & {\ldots } & {.} & {.} & {.} & {-1} & {.} & {.} \\ 
{.} & {.} & {.} & {1} & {.} & {.} & {\ldots } & {.} & {.} & {.} & {.} & {-1} & {.} \\ 
{.} & {.} & {.} & {.} & {1} & {.} & {\ldots } & {.} & {.} & {.} & {.} & {.} & {-1} \\ 
{.} & {.} & {.} & {.} & {.} & {1} & {\ldots } & {.} & {.} & {.} & {.} & {.} & {.} \\ 
{.} & {.} & {.} & {.} & {.} & {.} & {\ldots } & {1} & {.} & {.} & {-1} & {.} & {.} 
\end{array}\right)\] 

\[Q_{p,k} =\left(
\begin{array}{ccccccccccccc} 
{\scriptstyle \sigma_{\text{БВ}} } & {.} & {.} & {.} & {.} & {.} & {.} & {.} & {.} & {.} & {.} & {.} & {.} \\ 
{.} & {\scriptstyle \frac{\sigma_{Rs}}{\cos \varphi_{k}}}& {.} & {.} & {.} & {.} & {.} & 
{\scriptstyle \frac{\mu \sigma_{\delta Rs}}{\cos \varphi_{k}}} & {.} & {.} & {.} & {.} & {.} \\ 
{.} & {.} & { \scriptstyle\sigma_{Rs} } & {.} & {.} & {.} & {.} & {.} & {\scriptstyle \mu \sigma_{\delta Rs} } & {.} & {.} & {.} & {.} \\ 
{.} & {.} & {.} & {\scriptstyle \sigma_{hs} } & {.} & {.} & {.} & {.} & {.} & {\scriptstyle \mu \sigma_{\delta Rs} } & {.} & {.} & {.} \\ 
{.} & {.} & {.} & {.} & {\scriptstyle \sigma_{Vs} } & {.} & {.} & {.} & {.} & {.} & {\scriptstyle \mu \sigma_{\delta Vs} } & {.} & {.} \\ 
{.} & {.} & {.} & {.} & {.} & {\scriptstyle \sigma_{Vs} } & {.} & {.} & {.} & {.} & {.} & {\scriptstyle \mu \sigma_{\delta Vs} } & {.} \\ 
{.} & {.} & {.} & {.} & {.} & {.} & {\scriptstyle \sigma_{Vs} } & {.} & {.} & {.} & {.} & {.} & {\scriptstyle \mu \sigma_{\delta Vs} } \\ 
{\scriptstyle \sigma_{\text{БВ}}} & {.} & {.} & { \scriptstyle\sigma_{hs} } & {.} & {.} & {.} & {.} & {.} & {\scriptstyle \mu \sigma_{\delta Rs}} & {.} & {.} & {.} 
\end{array}\right);\] 
\begin{ESKDexplanation}
\item $\mu =1$ в момент зміни сузір'я $t_{k}^{*} $, $\mu =0$ в будь який інший момент $t_{k}^{} $.
\end{ESKDexplanation}

Для оцінки вектора стану системи \eqref{eq:state_vector} за спостереженнями \eqref{eq:measure_vector} 
пропонується використовувати процедуру дискретного оптимального фільтра Калмана, 
у якій екстраполяція оцінки вектора стану $\stackrel{\frown}{\bar{X}}_{p,k-1} $ і 
коваріаційної матриці помилок оцінки $P_{k-1} $ здійснюється відповідно до формул:

\begin{equation} 
\label{eq:kalman_predict} \begin{array}{l} 
{\hat{\bar{X}}_{p,k} =\Phi_{p,k-1} \hat{\bar{X}}^{+}_{p,k-1} ,} \\ 
{P_{k} =\Phi_{p,k-1} P^{+}_{k-1} \Phi ^{T}_{p,k-1} +G_{p,k-1} G_{p,k-1}^{T} ;} \end{array} 
\end{equation} 

а корекція виконується згідно співвідношень виду:

\begin{equation} \label{eq:kalman_est} \begin{array}{l} 
{\hat{\bar{X}}^{+}_{p,k}=\hat{\bar{X}}_{p,k} +K_{k} (\bar{Y}_{k} -H\hat{\bar{X}}_{p,k} )} \\ 
{P^{+}_{k}=(E-K_{k} H)P_{k} \left(E-K_{k} H\right)^{T} +K_{k} Q_{p,k} Q_{p,k} ^{T} K_{k}^{T} } 
\end{array} \end{equation} 

де верхній індекс «+» є ознака корекції, виконаної  на відповідному кроці;

$K_{k} =P_{k} H^{T} (HP_{k} H^{T} +Q_{p,k} Q_{p,k} ^{T} )^{-1} $-- матричний коефіцієнт 
підсилення фільтра.

Процедура \eqref{eq:kalman_predict}, \eqref{eq:kalman_est} може бути доповнена операцією 
обмеження знизу значень діагональних елементів матриці коваріації $P^{+}_{k} $.

$P_{k,i} ^{+} $  якщо  $P_{k,i} ^{+} \ge \gamma_{i} $;

$\hat{P}_{k,i} ^{+} =\left\{\begin{array}{l} {P_{k,i} ^{+} ,\text{при }P_{k,i} ^{+} \ge \gamma_{i} } \\ 
{\gamma_{i} ,\text{при } P_{k,i} ^{+} <\gamma_{i} } \end{array} \right. $, 

% ( \textit{i} = 1,\dots , \textit{Np}),
\begin{ESKDexplanation}
 \item де $P_{k,i} ^{+} $-- \textit{i}-й діагональний елемент матриці $P_{k} ^{+} $;
 \item $\gamma_{i} $ якщо \textit{i} = 1,\dots , \textit{Np} -- задані нижні границі 
значень діагональних елементів.
\end{ESKDexplanation}

Як відзначалося вище, при комплексній обробці навігаційної  інформації необхідно 
здійснювати алгоритмічний контроль цілісності СНС. Можна вказати, принаймні, два  
підходи до розв'язання задачі контролю цілісності СНС.  Перший підхід зводиться до 
контролю за допуском вихідної позиційної і швидкісної інформації СНС. З цією метою  
здійснюється порівняння поточних показань СНС з географічних координат і проекцій 
шляхової швидкості з відповідними оцінками зазначених навігаційних параметрів, екстрапольованих 
з використанням навігаційних рівнянь \eqref{eq:coordinates}
з попереднього кроку (приймається гіпотеза про те, що оцінки навігаційних параметрів 
на попередньому кроці достовірні). Для такого підходу значення допусків можуть бути 
встановлені з урахуванням енергетичних можливостей ЛА.

Другий підхід випливає з теоретичних моделей процесу оптимальної калмановської фільтрації 
і передбачає аналіз характеристик так називаної оновленої послідовності спостережень

\begin{equation} 
\label{eq:y_calc} 
\Delta \bar{Y}_{j,k} =\bar{Y}_{j,k} -H\bar{X}_{p,k} ,{\rm \; \; \; \; }j={\rm 1,2,} \end{equation} 

де  $\bar{Y}_{j,k} $ (\textit{j} = 1,2) -- підвектори вектора спостережень $\bar{Y}_{p,k} $, 
які відповідають позиційної (компоненти 2 $\div $ 4) і швидкісний (компоненти  5 $\div $ 7) 
вихідної інформації СНС;

$h_{1,i,j} =h_{i+1,j} $  (\textit{i} = 1,2,3,  \textit{j} = 1, \dots , 22);
$h_{2,i,j} =h_{i+n,j} $ (\textit{i} = 1,2,3, \textit{j} = 1, \dots , 22).

Рішення про відмовлення позиційного або швидкісного каналів СНС приймається на основі 
аналізу умов нормальної роботи фільтра:
\begin{equation} 
\label{eq:y_sns_otkaz}
\frac{Sp(\Delta \bar{Y}_{j,k} \Delta \bar{Y}_{j,k} ^{T} )}{Sp(H_{j} P_{k} H_{j}^{T} +R_{j} )} <\delta ,\, \, \, \, j=1,2,    
\end{equation} 
\begin{ESKDexplanation}
\item де $Sp()$ -- символ сліду матриці;
\item $\delta $ -- задана константа $(\delta \ge 10);$
\item $R_{j} $ (\textit{j} = 1, 2) -- коваріаційні матриці відповідних підвекторів випадкових помилок вимірів.
\end{ESKDexplanation}
Якщо умови не виконуються на $k$-му  кроці для будь якого \textit{j}, то відповідний 
підвектор спостережень ігнорується при обробці інформації на цьому кроці.

Одержувані з виходу фільтра оцінки помилки нанотехнологічної БІНС   використовуються 
для виправлення вихідних навігаційних параметрів нанотехнологічної БІНС. Алгоритм 
виправлення оцінок координат і проекцій швидкості має вигляд:

\begin{equation} \label{eq:correct_lam_phi_V} 
\begin{array}{l} 
{h^{+}_{i} =h^{-}_{i} -\Delta \hat{h}_{i} ;} \\ 
{\varphi_{i}^{+} =\varphi ^{-}_{i} -\frac{\Delta \hat{R}_{Ni} }{R_{7} } ;} \\ 
{\lambda_{i}^{+} =\lambda ^{-}_{i} -\frac{\Delta \hat{R}_{Ei} }{R_{7} } ;} \\ 
{V^{+}_{l,i}=V^{-}_{l,i} -\Delta \hat{V}_{l} , l=E,N,h,} \end{array} 
\end{equation} 
\begin{ESKDexplanation}
\item де верхніми індексами  «-» і «+» позначені оцінки вихідних навігаційних параметрів 
до виправлення і після виправлення відповідно;
\item $\Delta \stackrel{\frown}{R}_{Ei} $, $\Delta \stackrel{\frown}{R}_{Ni} $, $\Delta 
\stackrel{\frown}{h}_{i} $, $\Delta \stackrel{\frown}{V}_{E} $, $\Delta \stackrel{
\frown}{V}_{N} $, $\Delta \stackrel{\frown}{V}_{h} $ -- поточні оцінки помилок нанотехнологічної 
БІНС, одержувані на виході фільтра.
\end{ESKDexplanation}
Виправлення одержуваної в нанотехнологічної БІНС оцінки матриці орієнтації $B_{i} $ виконується 
за допомогою наступної процедури:
\begin{equation}
\label{eq:correct_B}
\stackrel{\frown}{B}^{+}_{i} =\Delta B_{i} \stackrel{\frown}{B}^{-}_{i},
\end{equation}
\begin{ESKDexplanation}
\item де
$\Delta B_{i} =\left(\begin{array}{ccc} 
{1} & {-\hat{\alpha }_{h,i} } & {\hat{\alpha }_{N,i} }\\ 
{\hat{\alpha }_{h,i} } & {1} & {-\hat{\alpha }_{E,i} }\\ 
{-\hat{\alpha }_{N,i} } & {\hat{\alpha }_{E,i} } & {1} \end{array}\right)$,                                              
\item $\stackrel{\frown}{\alpha }_{E,i} $, $\stackrel{\frown}{\alpha }_{N,i} $, 
$\stackrel{\frown}{\alpha }_{h,i} $ -- поточні оцінки помилок  БІНС   у визначенні 
орієнтації географічної системи координат, одержувані на виході фільтра.
\end{ESKDexplanation}

Після виконання операції \eqref{eq:correct_B} варто перевіряти умови ортогональності 
матриці$\stackrel{\frown}{B}^{+}_{i} $ і при необхідності робити ортогоналізацію 
оцінки матриці направляючих косинусів $\stackrel{\frown}{B}^{+}_{i} $, наприклад, 
за допомогою процедури, запропонованої в роботі \cite{7}.

Як відзначалося вище, для випадку грубих або МЕМС ДПІ роботу нанотехнологічної БІНС 
необхідно періодично коректувати. Період корекції ${T}_{\text{кор}} $ може вибиратися 
з умови:

\[\Delta \alpha ({T}_{\text{кор}})=\Delta \alpha_{\text{доп}} ,\] 
\begin{ESKDexplanation}
\item де $\Delta \alpha ({T}_{\text{кор}})$ -- оцінка максимальної помилки моделювання 
орієнтації осей географічної системи координат у нанотехнологічної БІНС;
\item $\Delta \alpha_{\text{доп}} $ -- припустиме значення помилки, що забезпечує збереження 
лінійності моделі еволюції помилок нанотехнологічної БІНС.
\end{ESKDexplanation}
Аналіз показує, що для значень ${T}_{\text{кор}}$, які задовольняють умові ${T}_{\text{кор}} <<T_{\text{ш}} $
($T_{\text{ш}} $ = 84,4 хв -- період маятника Шулера), для оцінки  $\Delta \alpha(T)$ може бути використана формула виду:
\[\Delta \alpha_{\text{доп}}=\varepsilon {T}_{\text{кор}} +\Delta \alpha ^{*} ({T}_{\text{кор}}),\] 
\begin{ESKDexplanation}
\item де $\alpha ^{*} ({T}_{\text{кор}})=\left[\left(\Delta \alpha_{0} g+\Delta a\right)\frac{{T}_{\text{кор}}^{2} }{2} 
+g\frac{\varepsilon {T}_{\text{кор}}^{3} }{6} \right]R_{\text{З}}^{-1} $;     
\item $\Delta \alpha_{0} $ -- максимальне значення помилки початкової виставки нанотехнологічної БІНС ;
\item $\Delta a$,$\varepsilon $ -- максимальне значення помилки інерціальних ДПІ нанотехнологічної БІНС.
\end{ESKDexplanation}

У момент корекції роботи БІНС виконуються наступні операції:
\begin{itemize}
 \item вносяться виправлення  в обчислені значення оцінок координат, проекцій швидкості 
і матриці орієнтації \textit{В} у відповідності з формулами \eqref{eq:correct_lam_phi_V}, 
\eqref{eq:correct_B};
 \item обновляються оцінки приведених помилок датчиків нанотехнологічної БІНС  за формулами:
\[\begin{array}{l} 
{\varepsilon ^{*}_{i,l} =\varepsilon ^{*}_{i,l-1} +\stackrel{\frown}{\varepsilon 
}^{*}_{i,l} ;} \\ {\Delta a^{*}_{i,l} =\Delta a_{i,l-1} ^{*_{} } +\Delta \stackrel{
\frown}{a}_{i,l} ^{} ,} \end{array}\, \, \, (i=1,2,3),\] 
 де  $l$-- номер точки корекції (\textit{l} = 1,2\dots ...); 
$\varepsilon ^{*}_{i,0} =\Delta a^{*}_{i,0}=0$, \textit{i} = 1,2,3;
$\stackrel{\frown}{\varepsilon }_{i,e} $,$\Delta \stackrel{\frown}{a}_{i,e} $ -- оцінки 
помилок у точці корекції нанотехнологічної БІНС;
 \item онулюються компоненти вектора стану $X_{p}$ 1$\div$ 15, що відповідають помилкам 
нанотехнологічної БІНС.
\end{itemize}
Поточні оцінки приведених помилок ДКШ і МЕМС акселерометрів $\varepsilon_{i}^{*} $, $\Delta 
a_{i}^{*} $ (\textit{i }= 1,2,3)  використовуються в обчислювальних алгоритмах 
БІНС для внесення в показання ДПІ виправлень виду:

$\Delta \alpha_{i} =\Delta t_{\text{оптим}} \varepsilon ^{*}_{i} $ і $\Delta v_{i} 
=\Delta t_{\text{оптим}} \Delta a^{*}_{i} $ (\textit{i }= 1,2,3),
де $\Delta t_{\text{оптим}} $ -- крок опитування МЕМС ДПІ.

% \textbf{7.1.2 Жорсткозв'язана схема}
% 
% При реалізації жорстко зв'язаної схеми побудови інваріантного алгоритму комплексної 
% обробки навігаційної інформації для складу навігаційних підсистем, які  розглядаються, 
% рекомендується використовувати розширений вектор стану, що включає  $16+2*\left(M+1
% \right)$ компонентів ( \textit{М} -- число каналів у бортовій апаратурі СНС), у 
% тому числі 5 компонентів моделі помилок нанотехнологічної БІНС , систематичну помилку 
% БВ, зрушення і дрейф шкали часу бортової апаратури СНС, а також $2*M$ компонентів 
% корельованих помилок у визначенні псевдодальностей і псевдошвидкостей (цей підвектор 
% вектора стану заповнюється в порядку зростання умовних номерів видимих навігаційних 
% супутників, а якщо число  видимих супутників  $N<M$, те підвектор заповнюється частково):
% 
% \[\begin{array}{l} 
% {\bar{Z}_{P} =(\Delta R_{E} ,\Delta R_{N} ,\Delta h,\Delta V_{E} ,\Delta V_{N} ,
% \Delta V_{h} ,\alpha_{E} ,\alpha_{N} ,\alpha_{h} ,\varepsilon_{c1} ,\varepsilon 
%_{c2} ,\varepsilon_{c3} ,} \\ {\Delta a_{c1} ,\Delta a_{c2} ,\Delta a_{c3} ,\Delta 
% h_{cl} ,\Delta \tau ,V_{\tau } ,\Delta D_{scl} ,\Delta V_{scl} ,l=l_{1} ,\ldots ,l_{N} 
% )^{T} } \end{array}\] 
% 
% 
% 
% тут $l_{1} <l_{2} <\ldots <l_{N} ,\, \, \, \, N\le M$.
% 
% Дискретна модель еволюції вектора стану $\bar{Z}_{P} $ з урахуванням \eqref{eq:__6_13_}, 
% \eqref{eq:__6_17_}, \eqref{eq:__6_20_} може бути подана у вигляді:
% 
% $\bar{Z}_{P,K+1} =\Phi_{P,K} \bar{Z}_{P,K} +G_{P,K} \bar{\xi }_{K} $,            \eqref{eq:__7_7_}
% 
% де $\Phi 
%_{P,K} =E+F_{P,K} \Delta t$;
% 
% $F_{P,K} ,G_{P,K} $- матриці розміру $N_{S} \times N_{S} (N_{S} =16+2*(N+1))$ и $N_{S} 
% \times N_{S} (N_{S} +6)$, відповідно;
% 
% $\bar{\xi }_{k} -(N_{S} +6)$ вектор-стовбець центрированих дискретних білих шумів 
% з одиничною інтенсивністю;
% 
% \[F_{P,K} =\left(\begin{array}{ccccccccc} {F_{K} } & {.} & {.} & {.} & {.} & {.} 
% & {.} & {.} & {.} \\ {.} & {0} & {.} & {.} & {.} & {.} & {.} & {.} & {.} \\ {.} & 
% {.} & {0} & {1} & {.} & {.} & {.} & {.} & {.} \\ {.} & {.} & {0} & {0} & {0} & {.} 
% & {.} & {.} & {.} \\ {.} & {.} & {.} & {.} & {W_{R} } & {.} & {.} & {.} & {.} \\ 
% {.} & {.} & {.} & {.} & {.} & {W_{V} } & {.} & {.} & {.} \\ {.} & {.} & {.} & {.} 
% & {.} & {.} & {\ddots } & {.} & {.} \\ {.} & {.} & {.} & {.} & {.} & {.} & {.} & 
% {W_{R} } & {.} \\ {.} & {.} & {.} & {.} & {.} & {.} & {.} & {.} & {W_{V} } \end{array}
% \right);\] 
% 
% \[G_{P,K} =\left(\begin{array}{ccccccccc} {G_{K} \sqrt{\Delta t} } & {.} & {.} & 
% {.} & {.} & {.} & {.} & {.} & {.} \\ {.} & {\sigma_{\xi } \sqrt{\Delta t} } & {.} 
% & {.} & {.} & {.} & {.} & {.} & {.} \\ {.} & {.} & {\sigma_{\xi \tau } \sqrt{\Delta 
% t} } & {.} & {.} & {.} & {.} & {.} & {.} \\ {.} & {.} & {.} & {\sigma_{\xi V\tau 
% } \sqrt{\Delta t} } & {.} & {.} & {.} & {.} & {.} \\ {.} & {.} & {.} & {.} & {q_{R} 
% \sigma_{\xi s} } & {.} & {.} & {.} & {.} \\ {.} & {.} & {.} & {.} & {.} & {q_{V} 
% \sigma_{Vs} } & {.} & {.} & {.} \\ {.} & {.} & {.} & {.} & {.} & {.} & {\ddots } 
% & {.} & {.} \\ {.} & {.} & {.} & {.} & {.} & {.} & {.} & {q_{R} \sigma_{\xi s} } 
% & {.} \\ {.} & {.} & {.} & {.} & {.} & {.} & {.} & {.} & {q_{V} \sigma_{Vs} } \end{array}
% \right);\] 
% 
% $F_{K} ,G_{K} $-- матриці з елементами \eqref{eq:__6_11_} і \eqref{eq:__6_12_}.
% 
% Як 
% відзначалося вище, розмірність вектора спостереження для жорсткозв'язаної схеми $\bar{Y}_{P} $ залежить 
% від числа каналів приймача СНС і числа видимих навігаційних супутників
% 
% \begin{equation} \label{eq:__7_8_} \bar{Y}_{P,K} =\left(\tilde{h}_{K} -\tilde{h}_{,K} 
% ,\tilde{D}_{Sl_{1} ,K} -\tilde{D}_{l_{1} ,K} ,\tilde{V}_{Sl_{1} ,K} -\tilde{V}_{l_{1} 
% ,K} ,\ldots ,\tilde{D}_{SlN,K} -\tilde{D}_{lN,K} ,\tilde{V}_{SlN,K} -\tilde{V}_{lN,K} 
% \right)^{T}  \end{equation} 
% 
% де $\tilde{D}_{Sj,K} ,\tilde{V}_{Sj,K} (j=1,\ldots ,N\le M)$-- оцінки псевдодальностей, 
% які обчислюються  на основі значень вихідних навігаційних параметрів нанотехнологічної 
% БІНС   (узятих до внесення виправлень за результатами роботи фільтра).
% 
% Компактна форма подання рівняння  спостережень для жорсткозв'язаної схеми має вигляд:
% 
% \[\bar{Y}_{P,K} 
% =H_{K} \bar{Z}_{P,K} +Q_{P,K} \bar{\eta }_{K} \] 
% 
% де $\bar{Y}_{P,K} $ -- (1 + 2\textit{N})-мірний вектор спостережень;
% 
% $\bar{\eta }_{K} $ -- (1 + 2\textit{N})-мірний вектор центрованих дискретних білих 
% шумів з одиничною інтенсивністю;
% 
% $H_{K} $ -- матриця розміру $\left(1+2N\right)\times \left[16+2(N+1)\right]$;
% 
% $H^{\eqref{eq:__1_}} =\left(0,\, 0,\ldots ,-1,\, 0,\, 0,\, \ldots ,\, 0\right)$ -- перший 
% рядок матриці $H_{K} $ («--1» на позиції 16);
% 
% \[H^{(\ell )} =\left|\begin{array}{ccccccccccccc} {1} & {2} & {3} & {4} & {5} & {6} 
% & {\ldots } & {17} & {18} & {\ldots } & {\ell ^{*} +18} & {\ell ^{*} +19} & {\ldots 
% } \\ {\frac{\partial D_{sl} }{\partial R_{E} } } & {\frac{\partial D_{sl} }{\partial 
% R_{N} } } & {\frac{\partial D_{sl} }{\partial h} } & {0} & {0} & {0} & {\ldots } 
% & {c} & {0} & {\ldots } & {1} & {0} & {\ldots } \\ {\frac{\partial V_{sl} }{\partial 
% R_{E} } } & {\frac{\partial V_{sl} }{\partial R_{N} } } & {\frac{\partial V_{sl} 
% }{\partial h} } & {\frac{\partial V_{sl} }{\partial V_{E} } } & {\frac{\partial V_{sl} 
% }{\partial V_{N} } } & {\frac{\partial V_{sl} }{\partial V_{h} } } & {\ldots } & 
% {0} & {c} & {\ldots } & {0} & {1} & {\ldots } \\ {\ldots } & {\ldots } & {\ldots 
% } & {\ldots } & {\ldots } & {\ldots } & {\ldots } & {\ldots } & {\ldots } & {\ldots 
% } & {\ldots } & {\ldots } & {\ldots } \\ {} & {} & {} & {} & {} & {} & {} & {} & 
% {} & {} & {} & {} & {} \end{array}\right|\] 
% 
% блок з рядків  $\left(\ell ^{*} +1\right)$ і $\left(\ell ^{*} +2\right)$ матриці  $H_{K} $ $(
% \ell =1,\ldots ,N\le M,\ell ^{*} =2*(\ell -1))$;
% 
% \[\begin{array}{l} {\frac{\partial D_{sl} }{\partial R_{E} } =\frac{\partial D_{sl} 
% }{\partial X} \frac{\partial X}{\partial R_{E} } +\frac{\partial D_{sl} }{\partial 
% Y} \frac{\partial Y}{\partial R_{E} } +\frac{\partial D_{sl} }{\partial Z} \frac{
% \partial Z}{\partial R_{E} } ;} \\ {\frac{\partial D_{sl} }{\partial R_{N} } =\frac{
% \partial D_{sl} }{\partial X} \frac{\partial X}{\partial R_{N} } +\frac{\partial 
% D_{sl} }{\partial Y} \frac{\partial Y}{\partial R_{N} } +\frac{\partial D_{sl} }{
% \partial Z} \frac{\partial Z}{\partial R_{N} } ;} \\ {\frac{\partial D_{sl} }{\partial 
% h} =\frac{\partial D_{sl} }{\partial X} \frac{\partial X}{\partial h} +\frac{\partial 
% D_{sl} }{\partial Y} \frac{\partial Y}{\partial h} +\frac{\partial D_{sl} }{\partial 
% Z} \frac{\partial Z}{\partial h} ;} \\ {\frac{\partial V_{sl} }{\partial R_{E} } 
% =\frac{\partial V_{sl} }{\partial X} \frac{\partial X}{\partial R_{E} } +\frac{\partial 
% V_{sl} }{\partial Y} \frac{\partial Y}{\partial R_{E} } +\frac{\partial V_{sl} }{
% \partial Z} \frac{\partial Z}{\partial R_{E} } ;} \\ {\frac{\partial V_{sl} }{\partial 
% R_{N} } =\frac{\partial V_{sl} }{\partial X} \frac{\partial X}{\partial R_{N} } +
% \frac{\partial V_{sl} }{\partial Y} \frac{\partial Y}{\partial R_{N} } +\frac{\partial 
% V_{sl} }{\partial Z} \frac{\partial Z}{\partial R_{N} } ;} \\ {\frac{\partial V_{sl} 
% }{\partial h} =\frac{\partial V_{sl} }{\partial X} \frac{\partial X}{\partial h} 
% +\frac{\partial V_{sl} }{\partial Y} \frac{\partial Y}{\partial h} +\frac{\partial 
% V_{sl} }{\partial Z} \frac{\partial Z}{\partial h} ;} \\ {\frac{\partial V_{sl} }{
% \partial V_{E} } =\frac{\partial V_{sl} }{\partial V_{X} } \frac{\partial V_{X} }{
% \partial V_{E} } +\frac{\partial V_{sl} }{\partial V_{Y} } \frac{\partial V_{Y} }{
% \partial V_{E} } +\frac{\partial V_{sl} }{\partial V_{Z} } \frac{\partial V_{Z} }{
% \partial V_{E} } ;} \\ {\frac{\partial V_{sl} }{\partial V_{N} } =\frac{\partial 
% V_{sl} }{\partial V_{X} } \frac{\partial V_{X} }{\partial V_{N} } +\frac{\partial 
% V_{sl} }{\partial V_{Y} } \frac{\partial V_{Y} }{\partial V_{N} } +\frac{\partial 
% V_{sl} }{\partial V_{Z} } \frac{\partial V_{Z} }{\partial V_{N} } ;} \\ {\frac{\partial 
% V_{sl} }{\partial V_{h} } =\frac{\partial V_{sl} }{\partial V_{X} } \frac{\partial 
% V_{X} }{\partial V_{h} } +\frac{\partial V_{sl} }{\partial V_{Y} } \frac{\partial 
% V_{Y} }{\partial V_{h} } +\frac{\partial V_{sl} }{\partial V_{Z} } \frac{\partial 
% V_{Z} }{\partial V_{h} } ;} \end{array}\] 
% 
% тут
% 
% \[\begin{array}{l} {\frac{\partial D_{sl} }{\partial X} =\left(X-X_{l} \right)\tilde{D}_{sl}^{-1} 
% ;\frac{\partial D_{sl} }{\partial Y} =\left(Y-Y_{l} \right)\tilde{D}_{sl}^{-1} ;
% \frac{\partial D_{sl} }{\partial Z} =\left(Z-Z_{l} \right)\tilde{D}_{sl}^{-1} ;} 
% \\ {\frac{\partial X}{\partial R_{E} } =-\sin \lambda ;\frac{\partial X}{\partial 
% R_{N} } =-\sin \varphi \cos \lambda ;\frac{\partial X}{\partial R_{E} } =\cos \varphi 
% \cos \lambda ;} \\ {\frac{\partial Y}{\partial R_{E} } =\cos \lambda ;\frac{\partial 
% Y}{\partial R_{N} } =-\sin \varphi \sin \lambda ;\frac{\partial Y}{\partial R_{E} 
% } =\cos \varphi \sin \lambda ;} \\ {\frac{\partial Z}{\partial R_{E} } =0;\frac{
% \partial Z}{\partial R_{N} } =\cos \varphi ;\frac{\partial Z}{\partial R_{E} } =
% \sin \varphi ;} \\ {\frac{\partial V_{sl} }{\partial V_{X} } =\left(X-X_{l} \right)
% \tilde{D}_{sl}^{-1} ;\frac{\partial V_{sl} }{\partial V_{Y} } =\left(Y-Y_{l} \right)
% \tilde{D}_{sl}^{-1} ;\frac{\partial V_{sl} }{\partial V_{Z} } =\left(Z-Z_{l} \right)
% \tilde{D}_{sl}^{-1} ;} \\ {\frac{\partial V_{sl} }{\partial X} =\left(V_{X} -V_{Xl} 
% -uY_{E} \right)\tilde{D}_{sl}^{-1} ;\frac{\partial V_{sl} }{\partial Y} =\left(V_{Y} 
% -V_{Yl} -uX_{E} \right)\tilde{D}_{sl}^{-1} ;\frac{\partial V_{sl} }{\partial Z} =
% \left(V_{Z} -V_{Zl} \right)\tilde{D}_{sl}^{-1} ;} \\ {\frac{\partial V_{X} }{\partial 
% V_{E} } =-\sin \lambda ;\frac{\partial V_{X} }{\partial V_{N} } =-\sin \varphi \cos 
% \lambda ;\frac{\partial V_{X} }{\partial V_{h} } =\cos \varphi \cos \lambda ;} \\ 
% {\frac{\partial V_{Y} }{\partial V_{E} } =\cos \lambda ;\frac{\partial V_{Y} }{\partial 
% V_{N} } =-\sin \varphi \sin \lambda ;\frac{\partial V_{Y} }{\partial V_{h} } =\cos 
% \varphi \sin \lambda ;} \\ {\frac{\partial V_{Z} }{\partial V_{E} } =0;\frac{\partial 
% V_{Z} }{\partial V_{N} } =\cos \varphi ;\frac{\partial V_{Z} }{\partial V_{h} } =
% \sin \varphi ;} \end{array}\] 
% 
% \[Q_{R,K} =\left(\begin{array}{ccccccc} {\sigma_{h} } & {.} & {.} & {.} & {.} & 
% {.} & {.} \\ {.} & {\sigma_{DS} } & {.} & {.} & {.} & {.} & {.} \\ {.} & {.} & {
% \sigma_{VS} } & {.} & {.} & {.} & {.} \\ {.} & {.} & {.} & {\ddots } & {.} & {.} 
% & {.} \\ {.} & {.} & {.} & {.} & {\sigma_{DS} } & {.} & {.} \\ {.} & {.} & {.} & 
% {.} & {.} & {\sigma_{VS} } & {.} \\ {.} & {.} & {.} & {.} & {.} & {.} & {\ddots 
% } \end{array}\right).\] 
% 
% Аналіз показує, що для оцінки вектора стану $\bar{Z}_{P} $, еволюція якого описується 
% моделлю \eqref{eq:__7_7_}, за спостереженнями виду \eqref{eq:__7_8_} доцільно 
% використовувати так називану розщеплену форму оптимального дискретного фільтра Калмана 
% (алгоритм Карлсона), що припускає використання квадратного кореня з матриці коваріації 
% помилок оцінок і послідовну (скалярну) обробку компонентів вектора спостереження. 
% Процедура зазначеного фільтра відрізняється підвищеною обчислювальною стійкістю і 
% не вимагає обертання матриць.
% 
% Для такого фільтра операція екстраполяції оцінки вектора стану на один крок здійснюється 
% відповідно з  формулою виду:
% 
% \[\stackrel{\frown}{Z}_{P,K+1} =\Phi_{P,K} \stackrel{\frown}{Z}_{P,K} .\] 
% 
% Для екстраполяції кореня квадратного з матриці коваріації помилок оцінок \textit{Р}  застосовується 
% наступна процедура:
% 
% \[\begin{array}{l} {W_{K+1} =\Phi_{P,K} S_{K}^{+} ;} \\ {P_{K+1} =W_{K+1} W_{K+1}^{T} 
% +G_{P,K} G_{P,K}^{T} ;} \\ {S_{K+1}^{-} =\left[P_{K+1} \right]^{\frac{1}{2} } ,} 
% \end{array}\] 
% 
% де операція добування кореня квадратного з матриці  виконується за допомогою зворотної 
% процедури Холецького.
% 
% У свою чергу зворотна процедура Холецького з добування верхньотрикутного кореня квадратино 
% із симетричної матриці \textit{Р} (тобто $S=P^{0,5} ,\, \, \, SS^{{\rm T}} =P$) описується 
% співвідношеннями виду:
% 
% \[\begin{array}{l} {S_{n,n} =P_{n,n}^{\frac{1}{2} } ;} \\ {S_{i,n} ={\raise0.7ex
% \hbox{$ P_{i,n}^{}  $}\!\mathord{\left/{\vphantom{P_{i,n}^{}  S_{n,n} }}\right.\kern-
% \nulldelimiterspace}\!\lower0.7ex\hbox{$ S_{n,n}  $}} ;} \\ {S_{i,i} =\left(P_{i,i}^{} 
% -\sum_{k=i+1}^{n}S_{i,k}^{2}  \right)^{\frac{1}{2} } ;} \\ {S_{i,j} =\frac{\left(P_{i,j} 
% -\sum_{l=j+1}^{n}S_{i,l} S_{j,l}  \right)}{S_{i,i} } ,j>i;} \\ {S_{i,j} =0,j<i;} 
% \end{array}\] 
% 
% де  $i=n-1,\cdots ,\, 1$;  $n$-- розмірність вектора стану.
% 
% При обробці кожної $l$-ї компоненти вектора спостереження $\bar{Y}_{P} $ здійснюється 
% корекція оцінки вектора стану $\bar{Z}_{P} $ і матриці $S$ за формулами виду:
% 
% \[\begin{array}{l} {\hat{\bar{Z}}_{P,k}^{+} =\hat{\bar{Z}}_{P,k}^{-} +\left(\frac{
% \bar{b}_{n} }{\alpha_{n} } \right)\Delta Y_{l,K} ;} \\ {S_{K}^{+} =\left(S_{1}^{+} 
% ,...,S_{n}^{+} \right);} \end{array}\] 
% 
% де $\begin{array}{l} {\Delta Y_{l,K} =Y_{l,K} -\bar{h}_{l,K} \hat{\bar{Z}}_{p,k}^{-} 
% ;} \\ {S_{K}^{-} =\left(S_{1}^{-} ,...,S_{n}^{-} \right);} \end{array}$
% 
% 
% 
% \[\begin{array}{l} {\alpha_{i} =\alpha_{i-1}^{+} ;a_{i} =\left({\raise0.7ex\hbox{$ 
% \alpha_{i-1}^{}  $}\!\mathord{\left/{\vphantom{\alpha_{i-1}^{}  \alpha_{i}^{} 
% }}\right.\kern-\nulldelimiterspace}\!\lower0.7ex\hbox{$ \alpha_{i}^{}  $}} \right)^{
% \frac{1}{2} } } \\ {\bar{S}_{i}^{+} =\bar{S}_{i}^{-} a_{i} -\bar{b}_{i-1} d_{i} ;} 
% \\ {\bar{b}_{i} =\bar{b}_{i-1} +\bar{S}_{i}^{-} f_{i} ;} \\ {d_{i} ={f_{i}  \mathord{
% \left/{\vphantom{f_{i}  \left(\alpha_{i-1}^{} \alpha_{i}^{} \right)^{\frac{1}{2} 
% } ;i=1,2,...,n;}}\right.\kern-\nulldelimiterspace} \left(\alpha_{i-1}^{} \alpha 
%_{i}^{} \right)^{\frac{1}{2} } ;i=1,2,...,n;} } \\ {\bar{b}_{o} =0,\alpha_{0} =
% \sigma_{l}^{2} ;} \\ {\left(\begin{array}{cccc} {f_{1} ,} & {f_{2} ,} & {...,} & 
% {f_{n} } \end{array}\right)^{T} =\left(S_{k}^{-} \right)^{T} \bar{h}_{l}^{T} ;} \end{array}\] 
% 
% $\bar{h}_{l,K+1} $ -- $l$-ий 
% рядок матриці $H_{K+1} $;
% 
% $\sigma_{l}^{2} $ -- діагональний елемент матриці $Q_{P,K+1} $ з номером l;
% 
% «--» і «+» -- верхні індекси, що відповідають оцінкам вектора стану $\bar{Z}_{P} $ і 
% матриці $S$ до і після обробки компонента вектора спостережень.
% 
% Для жорсткозв'язаної схеми комплексної обробки навігаційної інформації операція контролю 
% цілісності СНС може бути проведена з використанням формул, аналогічних \eqref{eq:correct_lam_phi_V} 
% і \eqref{eq:correct_B}. Для цієї схеми контролюється вірогідність інформації про 
% псевдодальности і псевдошвидкості для кожного з видимих навігаційних супутників. 
% При порушенні умови нормальної роботи інформація від відповідного супутника виключається 
% з комплексної обробки.
% 
% Виправлення вихідних навігаційних параметрів нанотехнологічної БІНС   для жорсткозв'язаної 
% схеми проводиться по формулах \eqref{eq:correct_lam_phi_V}, \eqref{eq:correct_B}.
% 
% Операції періодичної корекції роботи нанотехнологічної БІНС також не  відрізняється 
% від відповідних операцій для слабкозв`язаної схеми.
% 
% 
% 
% 
% \end{document}
% 
% % == UNREGISTERED! == eq: Word-to-LaTeX 2007 ==
% 
