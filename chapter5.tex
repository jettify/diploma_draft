\section{Розробка алгоримів оптимального комплексування в інерціально-супутникових
систем навігації}

Загальною вимогою для організації процесу комплексування є наявність математичних 
моделей підсистем, що підлягають комплексуванню. Сучасний стан обчислювальної техніки, 
знань в області інерціальної та супутникової навігації дозволяють скласти досить 
повні й адекватні моделі цих систем. У комплексі системи описуються на рівні їхніх 
похибок. Таким чином, для нормальної роботи комплексу потрібний адекватний опис похибок 
підсистем, включаючи неконтрольовані джерела похибок. 

\subsection{Моделі похибок  інерціальних навігаційних систем }

Рівняння похибок БІНС описують збурений режим роботи системи і є основою при аналізі 
її точності, при організації корекції, при побудові оптимальних навігаційних алгоритмів.

Матриця переходу від зв'язаної СК до географічної  СК  $B(\psi ,\vartheta ,\gamma )$ має 
вид:
\begin{equation}
\label{eq:noname_1} 
\scriptstyle
B(\psi ,\vartheta ,\gamma )=\left(
\begin{array}{ccc} 
{\scriptstyle \sin \psi \cos \vartheta } & 
{\scriptstyle\cos \psi \sin \gamma -\sin \psi \cos \gamma \sin \vartheta } & 
{\scriptstyle\cos \psi \cos \gamma +\sin \psi \sin \gamma \sin \vartheta } \\ 
{\scriptstyle\cos \psi \cos \vartheta } & 
{\scriptstyle-\sin \psi \sin \gamma -\cos \psi \cos \gamma \sin \vartheta } & 
{\scriptstyle-\sin \psi \cos \gamma +\cos \psi \sin \gamma \sin \vartheta } \\ 
{\scriptstyle\sin \vartheta } & 
{\scriptstyle\cos \gamma \cos \vartheta } & 
{\scriptstyle-\sin \gamma \cos \vartheta } 
\end{array}\right),
\end{equation}

\begin{ESKDexplanation}
\item де $\psi \left(t\right),\vartheta \left(t\right),\gamma \left(t\right)$- кути курсу, 
тангажа та крену ЛА відповідно. 
\end{ESKDexplanation}
Матриця переходу від географічної  СК до  рухомої 
екваторіальної СК $Q\left(\varphi \right)$ має вигляд:

\[Q\left(\varphi \right)=\left(\begin{array}{ccc} {1} & {0} & {0} \\ 
{0} & {\cos\varphi } & {\sin \varphi } \\ 
{0} & {-\sin \varphi } & {\cos \varphi } \end{array}\right),\] 

де $\varphi $- географічна широта.

Матриця переходу від зв'язаної СК до рухомої екваторіальної СК $C(\psi ,\vartheta,\gamma ,\varphi)$ 
задовольняє співвідношенням виду:
\[C\left(\psi ,\vartheta ,\gamma ,\varphi \right)=
Q\left(\varphi \right)\cdot B\left(\psi ,\vartheta ,\gamma \right).\] 
При розв`язанні задач повітряної навігації як основні навігаційні параметри ЛА можна 
розглядати поточні географічні координаті ( довготу $\lambda $, широту $\varphi $ и 
висоту над поверхнею земного еліпсоїда \textit{Н}), проекції шляхової швидкості $V_{E} 
,V_{N} ,V_{h} ,$а також елементи матриці переходу $B\left(\psi ,\vartheta ,\gamma 
\right)$, що характеризує орієнтацію ЛА у просторі.

Вказані навігаційні параметри задовольняє таким диференціальним рівнянням:
\begin{equation}
\left .
\begin{array}{c} 
{\dot{\lambda }=
\frac{V_{E} \left(t\right)}{\left(R_{1}+h\right)\cos \varphi \left(t\right)} } \\ 
{\dot{\varphi }=\frac{V_{N} \left(t\right)}{\left(R_{2} +h\right)} } \\ $-$
{\dot{h}=V_{h} \left(t\right)} \end{array}\right\};
\label{eq:coordinates}
\end{equation}
\begin{equation}
\dot{B}=B\Omega_{c} -\Omega_{\Gamma}B ;               
\label{eq:dBmatrix}
\end{equation}
\begin{equation}
\dot{\bar{V}}=B\bar{a}_{c} -\Delta \bar{n}\left(t\right)+\bar{g}_{T} ,     
\label{eq:dVector}
\end{equation}
\begin{ESKDexplanation}
\item де \eqref{eq:coordinates} -- рівняння для числення географічних координат; 
\item \eqref{eq:dBmatrix} -- матричне рівняння Пуассона для визначення матриці 
направляючих косинусів $B\left(\psi ,\vartheta ,\gamma \right)$; 
\item \eqref{eq:dVector} -- векторне рівняння відновно 
проекцій шляхової швидкості ЛА 
$\bar{V}=\left(\begin{array}{ccc} {V_{E} ,} & {V_{N} 
,} & {V_{h} } \end{array}\right)^{T} $; $\bar{a}_{c} \left(t\right)=\left(\begin{array}{ccc} 
{a_{x1} \left(t\right),} & {a_{y1} \left(t\right),} & {a_{z1} \left(t\right)} \end{array}
\right)^{T} $-- вектор проекцій уявного прискорення початку зв'язаної СК на її осі;
\end{ESKDexplanation}
\[\Omega_{c} =\left(\begin{array}{ccc} 
{0} & {-\omega {}_{z1} } & {\omega {}_{y1} } \\ 
{\omega {}_{z1} } & {0} & {-\omega {}_{x1} } \\ 
{-\omega {}_{y1} } & {\omega {}_{x1}} & {0} 
\end{array}\right);\] 
\[\Omega _{\Gamma } =\left(\begin{array}{ccc} 
{0} & {-(\dot{\lambda }+u)\sin \varphi } & {(\dot{\lambda }+u)\cos \varphi } \\ 
{(\dot{\lambda}+u)\sin \varphi } & {0} & {\dot{\varphi }} \\
{-(\dot{\lambda }+u)\cos \varphi } & {-\dot{\varphi }} & {0} 
\end{array}\right);\] 
\begin{ESKDexplanation}
\item $\omega _{x1}$ ,$\omega _{y1}$ ,$\omega _{z1}$-- проекції абсолютної кутової швидкості 
зв'язаної з ЛА СК на її осі; $u$-- кутова швидкість обертання Землі; 
\item $R_{1} $ и $R_{2} $-- головні радіуси кривизни обраного земного еліпсоїда;
\end{ESKDexplanation}

\[\begin{array}{l} 
{R_{1} =a\left[1-e^{2} \sin ^{2} \varphi (t)\right]^{-\frac{1}{2}};} \\ 
{R_{2} =a\left(1-e^{2} \right)\left[1-e^{2} \sin ^{2} \varphi(t)\right]^{-\frac{3}{2}};} 
\end{array}\] 
\begin{ESKDexplanation}
\item $a$,$e$-- велика піввісь и ексцентриситет земного еліпсоїда;
\item $\bar{g}_{T} =(\begin{array}{ccc}{g_{TE},}&{g_{TN},}&{g_{Th} }\end{array})^{T} $
-- вектор проекцій прискорення сили ваги на оси географічної СК;
\item $\Delta \bar{n}=\begin{array}{ccc} {\Delta n_{E} ,} & {\Delta n_{N} ,} & {
\Delta n_{h} ,} \end{array})^{T} $-- вектор проекцій суми переносного и кориолісова 
прискорень на осі географічної СК;
\end{ESKDexplanation}
\[\begin{array}{l} 
{\Delta n_{E} =\frac{V_{E} V_{h} }{R_{1} +h} -\frac{V_{E} V_{N}}{R_{1} +h} tg\varphi +2u\left(V_{h} \cos \varphi -V_{N} \sin \varphi \right);} \\ 
{\Delta n_{N} =\frac{V_{N} V_{h} }{R_{2} +h} +\frac{V_{E}^{2} }{R_{1} +h} tg\varphi+2uV_{E} \sin \varphi ;} \\ 
{\Delta n_{h} =-\frac{V_{E}^{2} }{R_{1} +h} -\frac{V_{N}^{2}}{R_{2} +h} -2uV_{E} \cos \varphi ;} 
\end{array}\] 
\begin{ESKDexplanation}
\item $\bar{g}_{T} =\left[0,0,g_{e} \right]^{T} $-- вектор проекцій нормального 
прискорення сили ваги на осі географічної СК 
$g_{e}=\mu//a^{2}$, $\mu=398600,44\cdot 10^{9} \left[\text{м}^{3}/c^{2} \right]$
\end{ESKDexplanation}


Маючи інформацію  про вихідні координати та проекції шляхової швидкості ЛА, про вихідну 
матрицю орієнтації $B_{0}$ (її визначення є предметом задачі початкового виставлення  
БІНС ), а також про моделі прискорення сили ваги $g^{T}$($\varphi $, $\lambda $, \textit{h}), 
на основі рівнянь \eqref{eq:coordinates}$\div $\eqref{eq:dVector} с використанням  
поточних показів ДУС и акселерометрів можна отримати поточні значення  шуканих навігаційних 
параметрів ЛА.

При точному завдані вихідних умов и при точній  моделі прискорення сили ваги, а також 
при відсутності похибок інерціальних ДПІ и похибок обчислення в наслідок інтегрування 
рівнянь \eqref{eq:coordinates}$\div $\eqref{eq:dVector}  будуть отримані істинні 
значення основних навігаційних параметрів ЛА.

Похибки завдання вихідних координат и проекцій шляхової швидкості ЛА, похибки  початкового 
виставлення , аномальні варіації прискорення сили ваги, похибки інерціальних ДПІ, 
методичні похибки алгоритмів обчислення и похибки через  кінцеву довжину розрядній 
сітці обчислювача (похибки округлення) будуть приводити до похибок визначення шуканих 
навігаційних параметрів ЛА.

У лінійному наближенні еволюція похибок БІНС у визначенні основних навігаційних параметрів 
у часі може бути описана лінійними диференціальними рівняннями похибок.

Рівняння похибок БІНС у визначенні координат випливає з динамічних рівнянь числення 
координат, що наведені в алгоритмах БІНС і мають вигляд:

\begin{equation} 
\label{eq:dRsdins} 
\begin{array}{l} 
{\Delta \dot{R}_{E} =\Delta V_{E}(t)\cdot \frac{R_{\text{З}} }{R\cos \varphi (t)} 
+\Delta R_{N} (t)\frac{V_{E}^{}(t)\sin \varphi (t)}{R_{\text{З}} R\cos ^{2} \varphi (t)} 
-\Delta h(t)\frac{R_{} V_{E}^{}(t)}{R^{2} \cos \varphi (t)} ;} \\ 
{\Delta \dot{R}_{N} =\Delta V_{N}(t)\cdot \frac{R_{\text{З}}}{R} -\Delta h(t)\frac{R_{\text{З}} V_{N}(t)}{R^{2}};} \\ 
{\Delta \dot{h} =\Delta V_{h} (t);} \end{array} \end{equation} 
\begin{ESKDexplanation}
\item де $\Delta R_{E} (t)=\Delta \lambda (t)R_{{\rm }} ,\, \, \Delta R_{N} (t)=\Delta\varphi (t)R_{{\rm }} $
-- похибка БІНС у визначенні приведених координат місцезнаходженняЛА; 
\item $\Delta \lambda (t)$,$\Delta \varphi (t)$,$\Delta H(t)$-- похибки БІНС у визначенні 
географічних координат; $\Delta V_{E} (t),\Delta V_{N} (t),\Delta V_{H} (t)$-- похибки 
БІНС у визначенні проекції шляхової швидкості ЛА; 
\item $R=R_{\text{З}} +H$; $R_{\text{З}}$  -- радіус земної сфери; 
\end{ESKDexplanation}
Еволюція похибок БІНС у визначенні проекції шляхової швидкості ЛА $\Delta V_{E}^(t)$,
$\Delta V_{N}(t)$,$\Delta V_{h}(t)$, також може бути отримана з динамічних 
рівнянь числення шляхової швидкості в алгоритмах БІНС, і описується наступною системою 
рівнянь: 

\[\begin{array}{l} {\Delta \dot{V}_{E} =a_{N} \alpha _{h} -a_{h} \alpha _{N} +\sum 
_{i=1}^{3}b_{1,i}  \Delta a_{i} -\Delta V_{h} U(t)\cos \varphi +\Delta V_{N} U(t)
\sin \varphi +} \\ {+\frac{\Delta R_{N} }{R_{} } \left(U(t)(V_{h} \sin \varphi +V_{N} 
\cos \varphi \right))-(\frac{\Delta V_{E} }{R\cos \varphi } +\frac{V_{E} \sin \varphi 
}{R\cos ^{2} \varphi } \frac{\Delta R_{N} }{R_{} } )\times } \\ {\times (V_{h} \cos 
\varphi -V_{N} \sin \varphi )+\frac{\Delta hV_{E} }{R^{2} } (V_{h} -V_{N} tg\varphi 
);} \end{array}\] 

\begin{equation} \label{eq:__6_5_} \begin{array}{l} {\Delta \dot{V}_{N} =-a_{E} 
\alpha _{h} +a_{h} \alpha _{E} +\sum _{i=1}^{3}b_{2,i}  \Delta a_{i} -\Delta V_{E} 
U(t)\sin \varphi -\Delta V_{h} \dot{\varphi }(t)-} \\ {-\frac{\Delta R_{N} }{R_{} 
} V_{E} U(t)\cos \varphi -\frac{\Delta V_{N} }{R} V_{h} -(\frac{\Delta V_{E} }{R
\cos \varphi } +\frac{V_{E} \sin \varphi }{R\cos ^{2} \varphi } \frac{\Delta R_{N} 
}{R_{} } )V_{E} \sin \varphi +} \\ {+\frac{\Delta h}{R^{2} } (V_{E}^{2} tg\varphi 
+V_{N} V_{h} );} \end{array} \end{equation} 

\[\begin{array}{l} {\Delta \dot{V}_{h} =a_{E} \alpha _{N} -a_{N} \alpha _{E} +\sum 
_{i=1}^{3}b_{3,i}  \Delta a_{i} +\Delta V_{E} U(t)\cos \varphi +\Delta V_{N} \dot{
\varphi }(t)-} \\ {-\frac{\Delta R_{N} }{R_{} } V_{E} U(t)\sin \varphi +\frac{\Delta 
V_{N} }{R} V_{N} +(\frac{\Delta V_{E} }{R\cos \varphi } +\frac{V_{E} \sin \varphi 
}{R\cos ^{2} \varphi } \frac{\Delta R_{N} }{R_{} } )V_{E} \cos \varphi +} \\ {+g_{e} 
\left(-\frac{2\Delta h}{a} +\frac{3}{2} e^{2} \sin \varphi \cos \varphi \frac{\Delta 
R_{N} }{R_{} } \right)-\frac{\Delta h}{R^{2} } \left(V_{E}^{2} +V_{N}^{2} \right),} 
\end{array}\] 



де $b_{ij} \left(i,j=1,2,3\right)$ -- елементи матриці направляючих косинусів \textit{B}; $\Delta 
a_{i} \left(i=1,2,3\right)$ -- приведені похибки акселерометрів БІНС (з урахуванням 
похибок чисельного інтегрування рівняння  у бортовому обчислювачі); $a_{H} ,a_{E} 
,a_{N} $ -- поточні значення проекцій уявного прискорення початку зв'язаної СК на 
осі географічної СК; $\alpha _{H} ,\alpha _{E} ,\alpha _{N} $ -- похибки моделювання 
в БІНС орієнтації географічного координатного тригранника ($\alpha _{E} $ і $\alpha 
_{N} $-- похибки побудови вертикалі, $\alpha _{H} $-- азимутальна похибка); $R=R_{{
\rm }} +H$ -- поточна висота; 

\[U(t)=2\Omega _{{\rm }} +\dot{\lambda }(t);\, \, \dot{\varphi }(t)=\frac{V_{N} }{R} 
;\, \, \, \dot{\lambda }(t)=\frac{V_{E} }{R\cos \varphi } .\] 

  Аналіз показує, що еволюція параметрів $\alpha _{h} ,\, \, \alpha _{E} ,\, \, \alpha 
_{N} $ у часі описується наступною системою рівняннь:



\begin{equation} \label{eq:__6_6_} \begin{array}{l} {\dot{\alpha }_{E} =-\omega 
_{N} \alpha _{h} +\omega _{h} \alpha _{N} -\frac{\Delta V_{N} }{R} -\sum _{i=1}^{3}b_{1,i}  
\varepsilon _{i} ,} \\ {\dot{\alpha }_{N} =-\omega _{h} \alpha _{E} +\omega _{E} 
\alpha _{h} +\frac{\Delta V_{E} }{R} -u\sin \varphi \frac{\Delta R_{N} }{R_{7} } 
-\sum _{i=1}^{3}b_{2,i}  \varepsilon _{i} ,} \\ {\dot{\alpha }_{h} =-\omega _{E} 
\alpha _{N} +\omega _{N} \alpha _{E} +\frac{\Delta V_{E} }{R} tg\varphi +(u\cos \varphi 
+\frac{V_{E} }{R\cos ^{2} \varphi } )\frac{\Delta R_{N} }{R_{7} } -\sum _{i=1}^{3}b_{3,i}  
\varepsilon _{i} ,} \end{array} \end{equation} 

де  $\omega _{E} =-\dot{\varphi }(t),\omega _{N} =\left[u+\dot{\lambda }(t)\right]
\cos \varphi ,\omega _{h} =\left[u+\dot{\lambda }(t)\right]\sin \varphi ,$

$\dot{\lambda }=\frac{V_{E} }{R\cos \varphi } ;$ $\dot{\varphi }=\frac{V_{N} }{R} $; $\varepsilon 
_{i} (i=1,2,3)-$приведені похибкм ДУС БІНС;

Аналіз показує, що похибки моделювання географічного тригранника $\alpha _{h} $,$\alpha 
_{E} $ ,$\alpha _{N} $ зв'язані з похибками визначення координат $\Delta R_{N} $,$\Delta 
R_{} $ і похибками моделювання орієнтації рухливої екваторіальної СК $\delta _{\xi 
} $, $\delta _{\eta } $, $\delta _{\zeta } $ такими   співвідношеннями:

\[\begin{array}{l} {\alpha _{E} =\delta _{\xi } -\frac{\Delta R_{N} }{R_{{\rm }} 
} ;} \\ {\alpha _{N} =\delta _{\eta } \cos \varphi -\delta \sin \varphi +\frac{\Delta 
R_{E} }{R_{{\rm }} } \cos \varphi ;} \\ {\alpha _{h} =\delta _{\eta } \sin \varphi 
-\delta _{\zeta } \cos \varphi +\frac{\Delta R_{E} }{R_{{\rm }} } \sin \varphi .} 
\end{array}\] 

Еволюція в часі похибок моделювання рухливої екваторіальної СК $\delta _{\xi } $, $\delta 
_{\eta } $,$\delta _{\zeta } $ описується більш простими, ніж \eqref{eq:__6_6_}, 
рівняннями:

\[\begin{array}{l} {\dot{\delta }_{\xi } =-(u+\dot{\lambda })\delta _{\zeta } -\varepsilon 
_{\zeta } (t)} \\ {\dot{\delta }_{\eta } =-\varepsilon _{\eta } (t)} \\ {\dot{\delta 
}_{\zeta } =-(u+\dot{\lambda })\delta _{\xi } -\varepsilon _{\zeta } (t)} \end{array};\] 

де $\dot{
\lambda }=\frac{V_{E} (t)}{R\cos (t)} $.

Якщо ввести в розгляд  інерціальну прямокутну геоцентричну СК $\xi _{{\rm u}} \eta 
_{{\rm u}} \zeta _{{\rm u}} $, вісь $\eta _{{\rm u}} $ якої збігається з віссю $\zeta $, 
а вісь $\xi _{{\rm u}} $ у момент  \textit{t }= 0 лежить у площині Гринвіцького меридіана, 
то можна сказати, що похибки моделювання орієнтації такої СК $\delta _{\xi _{{\rm 
u}} } $,$\delta _{\eta _{{\rm u}} } $,$\delta _{\zeta _{{\rm u}} } $ зв'язані з параметрами $\delta 
_{\xi } $,$\delta _{\eta } $,$\delta _{\zeta } $ співвідношеннями виду:

\[\begin{array}{l} {\delta _{\xi } =\delta _{\xi _{{\rm u}} } Aos\lambda _{*} -\delta 
_{\xi _{u} } \sin \lambda _{*} } \\ {\delta _{\eta } =\delta _{\eta _{{\rm u}} } 
} \\ {\delta _{\zeta } =\delta _{\xi _{{\rm u}} } \sin \lambda _{*} -\delta _{\zeta 
_{{\rm u}} } Aos\lambda _{*} } \end{array};\] 

де  $\lambda _{*} =ut+\lambda (t)$ .

Рівняння, що описують еволюцію в часі похибок моделювання інерціальної СК $\delta 
_{\xi _{{\rm u}} } $,$\delta _{\eta _{{\rm u}} } $,$\delta _{\zeta _{{\rm u}} } $ виявляється 
досить простими:

\[\begin{array}{l} {\dot{\delta }_{\xi _{{\rm u}} } =-\varepsilon _{\xi _{{\rm u}} 
} (t);} \\ {\dot{\delta }_{\eta _{{\rm u}} } =-\varepsilon _{\eta _{{\rm u}} } (t);} 
\\ {\dot{\delta }_{\zeta _{{\rm u}} } =-\varepsilon _{\zeta _{{\rm u}} } (t),} \end{array}\] 

де $\left(
\begin{array}{l} {\varepsilon _{\xi _{{\rm u}} } } \\ {\varepsilon _{\eta _{{\rm 
u}} } } \\ {\varepsilon _{\zeta _{{\rm u}} } } \end{array}\right)=\Delta {\bf C}(t){
\bf C}(t)\left(\begin{array}{l} {\varepsilon _{1} } \\ {\varepsilon _{2} } \\ {\varepsilon 
_{3} } \end{array}\right)$;

$\Delta {\bf C}(t)=\left(\begin{array}{ccccc} {\cos \lambda _{*} } & {} & {-\sin 
\lambda _{*} } & {} & {0} \\ {\sin \lambda _{*} } & {} & {\cos \lambda _{*} } & {} 
& {0} \\ {0} & {} & {0} & {} & {1} \end{array}\right)$ -- матриця переходу від рухливої  
екваторіальної СК до  інерціальної СК.

Таким чином, у моделі похибок БІНС можливе використання принаймні трьох груп параметрів, 
що характеризують похибки моделювання орієнтації СК:

\{$\alpha _{E} $,$\alpha _{N} $,$\alpha _{h} $\},\{$\delta _{\xi } $,$\delta _{\eta 
} $,$\delta _{\zeta } $\}, \{$\delta _{\xi _{{\rm u}} } $,$\delta _{\eta _{{\rm u}} 
} $,$\delta _{\zeta _{{\rm u}} } $\}.

Надалі в роботі використовуються параметри $\alpha _{E} $,$\alpha _{N} $,$\alpha 
_{h} $, що характеризують похибки  моделювання географічної СК і мають найбільш наочну 
фізичну інтерпретацію. Цим параметрам відповідають рівняння еволюції \eqref{eq:__6_6_}.

Для 
замикання системи рівнянь похибок БІНС \eqref{eq:dRsdins}, \eqref{eq:__6_5_}, 
\eqref{eq:__6_6_} необхідно вказати моделі еволюції приведених похибок ДПІ. 

З 
урахуванням вигляду моделі еволюції похибок ДПІ \eqref{eq:__6_13_}, яка описується  
в п.п. 6.2, рівняння похибок БІНС \eqref{eq:dRsdins}, \eqref{eq:__6_5_}, \eqref{eq:__6_6_} 
можуть бути замкненні  наступними  рівняннями відносно $C_{\omega } $, $C_{a} $, $C_{
\varepsilon } $, $D_{a} $, $\bar{\varepsilon }_{A} $, $\Delta \bar{a}_{c} $:

\begin{equation} \label{eq:__6_9_} \begin{array}{l} {\dot{C}_{\omega }^{} =\xi 
_{A\omega } (t);} \\ {\dot{C}_{a}^{} =\xi _{Aa} (t);} \\ {\dot{C}_{\varepsilon }^{} 
=\xi _{A\varepsilon } (t);} \\ {\dot{D}_{a} =\xi _{Da} (t);} \\ {\dot{\bar{\varepsilon 
}}_{c} =\bar{\xi }_{A} (t);} \\ {\Delta \dot{\bar{a}}_{c} =\bar{\xi }_{\Delta a} 
(t),} \end{array} \end{equation} 

де $\xi _{A\omega } (t);$$\xi _{Aa} (t);$$\xi _{A\varepsilon } (t);$$\xi _{Da} (t);$$\bar{
\xi }_{A} (t);$$\bar{\xi }_{\Delta a} (t)$-- білошумні збурення відповідної розмірності, 
які характеризують дрейф квазістаціонарних  параметрів моделі ДПІ \eqref{eq:__6_13_}.

Повертаючись 
до моделей похибок БІНС відзначимо, що коли  вектор-стовпець похибок БІНС $\bar{X}(t)$ прийняти 
у вигляді:

\[\bar{X}=(\Delta R_{E} ,\Delta R_{N} ,\Delta h,\Delta V_{E} ,\Delta V_{N} ,\Delta 
V_{h} ,\alpha _{E} ,\alpha _{N} ,\alpha _{h} ,\varepsilon _{c1} ,\varepsilon _{c2} 
,\varepsilon _{c3} ,\Delta a_{c1} ,\Delta a_{c2} ,\Delta a_{c3} ,)^{T} ,\] 

то модель еволюції похибок БІНС може бути подана у компактній формі

\begin{equation} \label{eq:__6_10_} \dot{\bar{X}}=F\bar{X}\left(t\right)+G\bar{
\xi }(t), \end{equation} 

де F та G   --  матриці 15 $\times$ 15 і 15 $\times$ 21 відповідно; $\bar{\xi }(t)$ -- вектор-стовпець 
розмірності 21, компонентами якого є незалежні Гауссівські «білі» шуми з нульовими 
середніми значеннями и одиничними дисперсіями.

Відмінні від нуля елементи матриці $F$ мають вигляд:



\[f_{1,2} =\frac{\dot{\lambda }}{R_{} } tg\varphi ;f_{1,3} =\frac{-\dot{\lambda }R_{} 
}{R} ;f_{1,4} =\frac{R_{} }{R\cos \varphi } ;f_{2,3} =\frac{-\dot{\varphi }R_{} }{R} 
;f_{2,5} =\frac{R_{} }{R} ;f_{3,6} =1;\] 

\[f_{4,2} =\frac{2u+\dot{\lambda }}{R_{} } \left(V_{h} \sin \varphi +V_{N} \cos \varphi 
\right)-\frac{\dot{\lambda }}{R_{} } tg\varphi \left(V_{h} \cos \varphi -V_{N} \sin 
\varphi \right);\] 

\[f_{4,3} =\frac{V_{E} }{R^{2} } \left(V_{h} -V_{N} tg\varphi \right);f_{4,4} =\frac{V_{N} 
\sin \varphi -V_{h} \cos \varphi }{R\cos \varphi } ;\] 

\[f_{4,5} =\left(2u+\dot{\lambda }\right)\sin \varphi ;f_{4,6} =-\left(2u+\dot{\lambda 
}\right)\cos \varphi ;\] 

\[f_{4,8} =-a_{h} ;f_{4,9} =a_{N} ;f_{4,13} =b_{1,1} ;f_{4,14} =b_{1,2} ;f_{4,15} 
=b_{1,3} ;\] 

\begin{equation} \label{eq:__6_11_} f_{5,2} =-\frac{2u+\dot{\lambda }}{R_{} } 
V_{E} \cos \varphi -\frac{V_{E}^{2} }{RR_{} } tg^{2} \varphi ;f_{5,3} =\frac{V_{E}^{2} 
tg\varphi +V_{h} V_{N} }{R^{2} } ; \end{equation} 

\[f_{5,4} =-\left(2u+\dot{\lambda }\right)\sin \varphi ;f_{5,5} =-\frac{V_{h} }{R} 
;f_{5,6} =-\dot{\varphi }(t);\] 

\[f_{5,7} =a_{h} ;f_{5,9} =-a_{E} ;f_{5,13} =b_{2,1} ;f_{5,14} =b_{2,2} ;f_{5,15} 
=b_{2,3} ;\] 

\[\begin{array}{l} {f_{6,2} =-2u\frac{V_{E}^{} \sin \varphi }{R} +\frac{3g_{e} }{2R_{} 
} e^{2} \sin \varphi \cos \varphi ;f_{6,3} =-\frac{2g_{e} }{a} -\frac{V_{E}^{2} +V_{N}^{2} 
}{R^{2} } ;} \\ {f_{6,4} =\left(2u+\dot{\lambda }\right)\cos \varphi ;} \end{array}f_{6,5} 
=\dot{\varphi }(t)+\frac{V_{N} }{R} ;f_{6,7} =-a_{N} ;f_{6,8} =a_{E} ;f_{6,13} =b_{3,1} 
;f_{6,14} =b_{3,2} ;f_{6,15} =b_{3,3} ;\] 

\[f_{7,5} =-\frac{1}{R} ;f_{7,8} =\omega _{h} ;f_{7,9} =-\omega _{N} ;f_{7,10} =-b_{1,1} 
;f_{7,11} =-b_{1,2} ;f_{7,12} =-b_{1,3} ;\] 

\[\begin{array}{l} {f_{8,2} =-\frac{u}{R} \sin \varphi ;f_{8,4} =\frac{1}{R} ;f_{8,7} 
=-\omega _{h} ;f_{8,9} =\omega _{E} ;} \\ {f_{8,10} =-b_{2,1} ;f_{8,11} =-b_{2,2} 
;f_{8,12} =-b_{2,3} ;} \\ {f_{9,2} =\frac{1}{R} _{7} (u\cos \varphi +\frac{\dot{
\lambda }}{\cos \varphi } );f_{9,4} =\frac{tg\varphi }{R} ;f_{9,7} =\omega _{N} ;f_{9,8} 
=-\omega _{E} ;} \\ {f_{9,10} =-b_{3,1} ;f_{9,11} =-b_{3,2} ;f_{9,12} =-b_{3,3} .} 
\end{array}\] 

Відрізні від нуля елементи матриці \textit{G} (15$\times $21) задовольняють таким 
співвідношенням:

\begin{equation} \label{eq:__6_12_} \begin{array}{l} {g_{i,i} =\sigma _{i} ,
\, \, \, \, i=1,..,15;} \\ {g_{i+3,j+18} =b_{i,j} \sigma _{a} ,\, \, \, \, i=1,2,3,j=1,2,3;} 
\\ {g_{i+6,j+15} =-\sigma _{\omega } b_{i,j} ,\, \, \, \, \, i=1,2,3,j=1,2,3;} \end{array} \end{equation} 

де $\sigma 
_{1} \div \sigma _{15} $- середньоквадратичні значення (СКЗ) білошумних збурень, 
що характеризують вплив різних факторів ($\sigma _{1} \div \sigma _{3} $-- похибок 
численного інтегрування рівняння \eqref{eq:coordinates}; $\sigma _{4} \div \sigma 
_{6} $ --  підсумковий ефект аномалій гравітаційного поля и похибок численного інтегрування 
рівняння \eqref{eq:dVector}, $\sigma _{7} \div \sigma _{9} $-- похибок численного 
інтегрування рівняння для параметрів орієнтації \eqref{eq:dBmatrix};  $\sigma _{10} 
\div \sigma _{15} $ -- випадкового дрейфу квазістаціонарних зведених погрішностей 
ДПІ  $\bar{\varepsilon }_{A} $ и $\Delta \bar{0}_{A} $);

$\sigma _{a} $, $\sigma _{\omega } $ -- СКЗ білошумних складових погрішностей акселерометрів 
и ДКШ БІНС.

Елементи матриць \textit{F} и \textit{G, } що випливає з аналізу співвідношень \eqref{eq:__6_11_} 
и \eqref{eq:__6_12_}, залежать від поточних значень навігаційних параметрів польоту 
ЛА.

Безперервної моделі еволюції похибок БІНС  \eqref{eq:__6_10_} відповідає такий 
дискретний аналог:

\[\bar{\% }_{k+1} =\Phi _{k} \bar{\% }_{k} +G_{k} \bar{\xi }_{k} ,\] 

де $\Phi _{k} =E+F(t_{k} )\Delta t,$   $G_{k} =G(t_{k} )\cdot \Delta t;$ $\Delta 
t$--  крок дискретизації часу;

$E$ -- одинична матриця  $15\times 15$.





% \subsection{6.2  Математичні моделі похибок  МЕМC-датчиків }
% 
% Строго говорячи, кожен тип гіроскопа або акселерометра має свою модель з її характерними 
% компонентами і чисельними значеннями. Проте, можна задатися деякою узагальненою моделлю, 
% яка якісно враховує залежності похибок від того або іншого збурюючого фактора. Для 
% конкретного типу гіроскопів і акселерометрів коефіцієнти в цих моделях повинні одержати 
% відповідні чисельні значення, а частина членів, несуттєвих для приладів даного типу, 
% можуть прийняти нульові значення. Можна, однак, уявити собі й іншій ситуації, коли 
% така узагальнена модель для якогось типу приладу не буде мати істотної для нього 
% складової. У цьому випадку модель повинна бути доповнена відповідними компонентами.
% 
% Аналіз 
% характеристик ММГ показав, що в моделях похибок ММГ доцільно враховувати наступні 
% фактори:
% 
% - нестабільність масштабних коефіцієнтів;
% 
% - перекіс (неоктагональность) осей чутливості;
% 
% - систематичні складові дрейфів, які  характеризують
% 
%   зсув нулів від пуску до пуску;
% 
% -  випадкові складові дрейфів, які характеризують
% 
%     дрейф нулів у конкретному пуску;
% 
% - складові дрейфів через дію лінійних прискорень
% 
%   (перевантажень);
% 
% - флюктуационные складових дрейфів
% 
%   (шуми вихідних сигналів).
% 
% У свою чергу, у моделях похибок ММА доцільно враховувати:
% 
% - нестабільність масштабних коефіцієнтів;
% 
% - перекіс (неоктагональность) осей чутливості;
% 
% - систематичні зсуви нулів від пуску до пуску;
% 
% - випадкові дрейфи нулів у пуску;
% 
% - флюктуационные похибки.
% 
% Аналіз показує, що незалежно від типу конкретних ММГ і ММА похибки вихідної інформації 
% блоку мікромеханічних інерціальных датчиків можуть бути описані за допомогою наступної 
% узагальненої моделі:
% 
% \begin{equation} \label{eq:__6_13_} \begin{array}{l} {\Delta \bar{\omega }(t)=C_{
% \omega } \bar{\omega }(t)+C\bar{a}(t)+\Delta \bar{\omega }_{{\rm AB}} +\Delta \bar{
% \omega }_{{\rm A;}} (t{\rm )}+\bar{\eta }_{\omega } (t);} \\ {\Delta \bar{a}(t)=C_{a} 
% \bar{a}(t)+\Delta \bar{a}_{{\rm cB}} +\Delta \bar{a}_{{\rm A;}} (t{\rm )}+\bar{\eta 
% }_{a} (t),} \end{array} \end{equation} 
% 
% Де $C_{\omega } $ й$C_{a} $ -- матриці 3 $\times$ 3 коефіцієнтів систематичних похибок 
% датчиків (діагональні елементи цих матриць характеризують похибки масштабних коефіцієнтів, 
% а недіагональні елементи -- перекоси осей чутливості ММГ і ММА відносно осей ортогональної 
% системи координат, пов'язаної із блоком датчиків);
% 
% \textit{С} -- матриця 3 $\times$ 3 коефіцієнтів систематичних похибок ММГ, що залежать 
% від перевантажень;
% 
% $\bar{\omega }(t),\bar{a}(t)$-- вектори-стовпці поточної складової абсолютної кутової 
% швидкості й гаданого прискорення початку приладового координатного базису в його 
% осях;
% 
% $\Delta \bar{\omega }(t)$, $\Delta \bar{a}(t)$ - вектори-стовпці поточних похибок 
% виміру складової абсолютної кутової швидкості й уявного прискорення;
% 
% $\Delta \bar{\omega }_{{\rm AB}} $- вектор-стовпець систематичних дрейфів ММГ;
% 
% $\Delta \bar{\omega }_{{\rm A;}} (t{\rm )}$- вектор-стовпець випадкових дрейфів ММГ;
% 
% $\bar{
% \eta }_{\omega } (t)$ - вектор-стовпець вихідних шумів ММГ;
% 
% $\Delta \bar{a}_{{\rm cB}} $ - вектор-стовпець систематичних зсувів ММА;
% 
% $\Delta \bar{a}_{{\rm A;}} (t{\rm )}$ - вектор-стовпець випадкових дрейфів нулів 
% ММА;
% 
% $\bar{\eta }_{a} (t)$ - вектор-стовпець вихідних шумів ММА.
% 
% Параметри моделей: $C_{\omega } $ ,$C_{a} $ , \textit{С}, $\Delta \bar{\omega }_{{
% \rm AB}} $, $\Delta \bar{a}_{{\rm cB}} $ -- можна вважати квазістаціонарними, залежними 
% в основному від температурного режиму блоку датчиків. Флуктуаційні похибки $\bar{
% \eta }_{\omega } (t)$, $\bar{\eta }_{a} (t)$ можна розглядати як некорельовані «білі» 
% шуми. Випадкові дрейфи  $\Delta \bar{\omega }_{{\rm A;}} (t{\rm )}$й $\Delta \bar{a}_{{
% \rm A;}} (t{\rm )}$ можна інтерпретувати як марковські процеси першого порядку з 
% періодами кореляції \textit{Т}$\omega $ ≈ 600 с і \textit{Та}  ≈ 100 с;
% 
% \[\begin{array}{l} {\dot{\bar{\omega }}_{{\rm A;}} ({\rm t)}=-\frac{{\rm 1}}{T_{
% \omega } } \left[\bar{\omega }_{{\rm A;}} (t{\rm )}-\bar{\xi }_{\omega } {\rm (}t{
% \rm )}\right]\, \, {\rm ;}} \\ {\Delta \dot{\bar{a}}_{{\rm A;}} {\rm (t)}=-\frac{{
% \rm 1}}{T_{a} } \left[\Delta \bar{a}_{{\rm A;}} {\rm (}t{\rm )}-\bar{\xi }_{a} (t)
% \right]\, \, \, ,} \end{array}\] 
% 
% де $\bar{\xi }_{\omega } {\rm (}t{\rm )}$, $\bar{\xi }_{a} (t)$ - збурення типу «білого» 
% шуму із заданими інтенсивностями.
% 
% Основні підходи до підвищення точності блоків мікромеханічних інерціальних датчиків 
% у робочому діапазоні експлуатаційних температур наступні:
% 
% - калібрування елементів $C_{\omega } $ ,$C_{a} $ , \textit{С}, $\Delta \bar{\omega 
% }_{{\rm AB}} $, $\Delta \bar{a}_{{\rm cB}} $ у дискретних точках робочого діапазону 
% температур з наступною інтерполяцією за результатами вимірів температури й введення 
% відповідних виправлень (алгоритмічна компенсація погрішностей;
% 
% - термостабілізація блоків датчиків (виправлення вводяться тільки для розрахункової 
% температури блоку).
% 
% 
% \subsection{6.3 Математичні моделі похибок супутникової системи навігації}
% 
% Для опису  похибок СНС у визначенні координат і проекцій шляхової швидкості ЛА пропонується 
% використовувати математичні моделі, що містять  Марківські і гаусовські складові 
% похибок:
% 
% \begin{equation} \label{eq:__6_14_} \begin{array}{l} {\Delta R_{Es,k} =\Delta 
% R_{Ec,k} +\frac{\sigma _{Rs} }{\cos \varphi _{k} } \eta _{REs,k} +\frac{\sigma _{
% \delta Rs} }{\cos \varphi _{k} } \eta _{\delta RE,k} ;} \\ {\Delta R_{Ns,k} =\Delta 
% R_{Nc,k} +\sigma _{Rs} \eta _{RNs,k} +\sigma _{\delta Rs} \eta _{\delta RN,k} ;} 
% \\ {\Delta H_{s,k} =\Delta H_{c,k} +\sigma _{Hs} \eta _{Hs,k} +\sigma _{\delta Rs} 
% \eta _{\delta H,k} {\rm ,}\, \, } \\ {\Delta V_{ls,k} =\Delta V_{lc,k} +\sigma _{Vs} 
% \eta _{V\, ls,k} +\sigma _{\delta Vs} \eta _{\delta V\, ls,k} {\rm ,}\, \, \, \, 
% \, \, \, \, \, {\rm ?@8}\, \, \, \, l=E,N,H;} \end{array} \end{equation} 
% 
% 
% 
% де $\Delta R_{ls,k} \, (l=E,N);\, \, \, \Delta H_{s,k} \, ;\, \, \, \Delta V_{ls,k} 
% \, (l=E,N,H)\, \, $-- похибки СНС у визначенні приведених  координат, висоти і складових 
% шляхової швидкості ЛА;
% 
% $\Delta R_{lc,k} \, (l=E,N);\, \, \, \Delta H_{c,k} \, ;\, \, \, \Delta V_{lc,k} 
% \, (l=E,N,H)\, \, $-- корельовані (Марківські) складові  похибок СНС;
% 
% $\sigma _{Rs} ,\, \, \sigma _{Hs} ,\, \, \sigma _{Vs} $  --  СКЗ білошумових складових 
% похибок СНС;
% 
% $\sigma _{\delta Rs} $, $\sigma _{\delta Hs} $, $\sigma _{\delta Vs} $ -- СКЗ додаткових 
% білошумових складових похибок СНС, що виникають тільки за умови, що tk -- момент 
% зміни сузір'я навігаційних супутників; 
% 
% $\eta _{Rls,k} ,\, \, \, \, \eta _{\delta Rls,k} \, \, \, (l=E,N);\, \, \eta _{Hs,k} 
% ,\, \, \eta _{\delta Hs,k} {\rm ;}\, \, \, \eta _{V\, ls,k} ,\eta _{\delta V\, ls,k} 
% \, \, \, (l=E,N,H)$ -- стандартні білі дискретні шуми зі СКЗ.
% 
% Корельовані складові похибок СНС описуються наступними співвідношеннями:
% 
% \begin{equation} \label{eq:__6_15_} \begin{array}{l} {\Delta R_{Ec,k} =W_{R} 
% \Delta R_{Ec,k-1} +q_{R} \frac{\sigma _{Rc} }{\cos \varphi _{k} } \eta _{REc,k} +
% \frac{\sigma _{\delta RC} }{\cos \varphi _{k} } \eta _{\delta REc,k} ;} \\ {\Delta 
% R_{Nc,k} =W_{R} \Delta R_{Nc,k-1} +q_{R} \sigma _{Rc} \eta _{RNc,k} +\sigma _{\delta 
% RC} \eta _{\delta RNc,k} ;} \\ {\Delta H_{c,k} =W_{R} \Delta H_{c,k-1} +q_{R} \sigma 
% _{Hc} \eta _{Hc,k} +\sigma _{\delta Hc} \eta _{\delta Hc,k} ;} \\ {\Delta V_{lc,k} 
% =W_{V} \Delta V_{lc,k-1} +q_{V} \sigma _{Vc} \eta _{V\, lc,k} +\sigma _{\delta Vc} 
% \eta _{\delta V\, lc,k} \, \, \, \, \, \, \, {\rm ?@8}\, \, l=E,N,H,} \end{array} \end{equation} 
% 
% де  
% 
% \[W_{R} 
% =e^{-(\lambda _{s} V_{{\rm H}} +\lambda _{st} )\Delta t} ;\, \, \, \, \, \, q_{R} 
% =\left[1-\exp \left(-2\left(\lambda _{s} V_{{\rm H}} +\lambda _{st} \right)\Delta 
% t\right)\right]\, ^{0,5}   ;\] 
% 
% \[W_{V} =e^{-\lambda _{V} \Delta t} ;\, \, \, \, \, \, \, q_{V} =\left[1-\exp \left(-2
% \lambda _{V} \Delta t\right)\right]\, ^{0,5}  ;\] 
% 
% $\lambda _{s} $-- показник просторової кореляції похибки СНС за координатами; $\lambda 
% _{V} ,\lambda _{st} $-- показник часової кореляції похибок СНС за швидкістю та за 
% координатами; $V_{{\rm H}} $-- шляхова швидкість ЛА;
% 
% $\Delta t$-- дискрета оновлення вихідної інформації СНС у часі;
% 
% $\sigma _{Rc} ,\sigma _{Hc} ,\sigma _{Vc} $ -- СКЗ корельованих складових похибок 
% СНС;
% 
% $\sigma _{\delta Rc} $,  $\sigma _{\delta Hc} $, $\sigma _{\delta } _{Vc} $  -- СКЗ 
% додаткових гаусовських збурень у моменти зміни сузір'я навігаційних супутників;
% 
% $\eta _{Rlc,k} ,\, \, \, \, \eta _{\delta Rlc,k} \, \, \, (l=E,N),\, \, \eta _{Hc,k} 
% ,\, \, \eta _{\delta Hc,k} {\rm ,}\, \, \, \eta _{V\, lc,k} ,\eta _{\delta V\, lc,k} 
% \, \, (l=E,N,H)$ -- стандартні центровані дискретні білі шуми з одиничною інтенсивністю.
% 
% Для 
% стандартного режиму СНС типу GPS NAVSTAR можуть бути рекомендовані наступні значення 
% параметрів моделей \eqref{eq:__6_14_}, \eqref{eq:__6_15_}:
% 
% 
% 
% \[\lambda _{s} =4\cdot 10^{-6} {\rm <}^{-1} ; \lambda _{st} =5\cdot 10^{-4} {\rm 
% c}^{-1} ; \lambda _{V} =\left(0,0017\div 0,05\right)\, {\rm c}^{-1} ;\] 
% 
% \[\sigma _{Rs} =\left(1\div 3\right){\rm <};\sigma _{Hs} =\left(1,5\div 4\right)
% \, {\rm <};\sigma _{Vs} =\left(0,01\div 0,05\right){\rm <}/{\rm c};\] 
% 
% \[\sigma _{\delta Rs} =\left(1\div 4\right){\rm <}; \sigma _{\delta } _{Vs} =\left(0,02
% \div 0,2\right){\rm <}/{\rm c}; \sigma _{Rc} =\left(5\div 7\right){\rm <};\] 
% 
% \[\sigma _{Hc} =\left(7\div 10\right){\rm <};  \sigma _{Vc} =\left(0,02\div 0,3\right){
% \rm </c};\] 
% 
% \[\sigma _{\delta Rc} =\left(2\div 5\right){\rm <}; \sigma _{\delta } _{Vc} =\left(0,01
% \div 0,02\right){\rm <}/{\rm c}; \sigma _{\delta Hc} =\left(3\div 7\right){\rm <}.\] 
% 
% Головною 
% задачею СНС є визначення псевдодальностей $D_{sl}^{} $ і псевдошвидкостей $V_{sl}^{} $ (l 
% = 1, \dots , N  -- число видимих навігаційних супутників),  які задовольняють співвідношенням 
% виду: 
% 
% \[\begin{array}{l} {D_{sl,k} =\{ \left[x_{l} \left(t_{k} -\tau _{l} \right)-x\left(t_{k} 
% \right)\right]^{2} +\left[y_{l} \left(t_{k} -\tau _{l} \right)-y\left(t_{k} \right)
% \right]^{2} +} \\ {\, \, \, \, \, \, \, \, \, \, \, \, \, \, \, \, +\left[z_{l} \left(t_{k} 
% -\tau _{l} \right)-z\left(t_{k} \right)\right]^{2} \} ^{\frac{1}{2} } +c\Delta \tau 
% _{k} ;} \end{array}\] 
% 
% \[\begin{array}{l} {V_{sl,k} =\{ \left[V_{xl} \left(t_{k} -\tau _{l} \right)-V_{x} 
% \left(t_{k} \right)\right]\left[x_{l} \left(t_{k} -\tau _{l} \right)-x\left(t_{k} 
% \right)\right]+} \\ {\, \, \, \, \, \, \, \, \, \, \, \, \, +\left[V_{yl} \left(t_{k} 
% -\tau _{l} \right)-V_{y} \left(t_{k} \right)\right]\left[y_{l} \left(t_{k} -\tau 
% _{l} \right)-y\left(t_{k} \right)\right]+} \\ {\, \, \, \, \, \, \, \, \, \, \, \, 
% \, +\left[V_{zl} \left(t_{k} -\tau _{l} \right)-V_{z} \left(t_{k} \right)\right]
% \left[z_{l} \left(t_{k} -\tau _{l} \right)-z\left(t_{k} \right)\right]+} \\ {\, \, 
% \, \, \, \, \, \, \, \, \, \, \, +\Omega _{{\rm }} x_{l} \left(t_{k} -\tau _{l} \right)y
% \left(t_{k} \right)\, -\Omega _{{\rm }} y_{l} \left(t_{k} -\tau _{l} \right)x\left(t_{k} 
% \right)\} \tilde{D}_{sl,k}^{-1} +cV,} \end{array}\] 
% 
% де $x_{l} \left(t_{k} -\tau _{l} \right)$,$y_{l} \left(t_{k} -\tau _{l} \right)$,$z_{l} 
% \left(t_{k} -\tau _{l} \right)$,$V_{xl} \left(t_{k} -\tau _{l} \right),V_{yl} \left(t_{k} 
% -\tau _{l} \right),V_{zl} \left(t_{k} -\tau _{l} \right)$-- координати і проекції 
% абсолютної швидкості $l$-го навігаційного супутника в прямокутної гринвіцькій  геоцентричній 
% СК XYZ;
% 
% $x\left(t_{k} \right),y\left(t_{k} \right),z\left(t_{k} \right),V_{x} \left(t_{k} 
% \right),V_{y} \left(t_{k} \right),V_{z} \left(t_{k} \right)$ -- координати і проекції 
% шляхової швидкості ЛА в СК XYZ; 
% 
% $\tau _{l} $− час проходження радіосигналу від  l-го навігаційного супутника до ЛА;
% 
% \[\tilde{D}_{sl,k} 
% =D_{sl,k} -\Delta \tau ;\] 
% 
% $\Delta \tau _{k} =\Delta \tau _{k-1} +V_{\varepsilon _{k} } \Delta t+\sigma _{\zeta 
% \tau } \xi _{\tau ,k-1} \, {\rm B0}\, \, V_{\varepsilon _{k} } =V_{\varepsilon _{k-1} 
% } +\sigma _{\zeta V\tau } \xi _{V\tau ,k-1} $--зрушення та дрейф шкали часу в бортовій 
% апаратурі СНС;
% 
% $\Omega _{{\rm }} $ − кутова швидкість обертання Землі; с -- швидкість  поширення 
% світла.
% 
% Похибки СНС при визначенні псевдодальностей $D_{sl}^{} $ и псевдошвидкостей $V_{sl}^{} $ можуть 
% бути описані наступними співвідношеннями:
% 
% \begin{equation} \label{eq:__6_16_} \begin{array}{c} {\Delta D_{sl,k} =\Delta 
% D_{scl,k} +\sigma _{DS} \eta _{Dl,k} ;} \\ {\Delta V_{sl,k} =\Delta V_{scl,k} +\sigma 
% _{VS} \eta _{Vl,k} ;} \end{array}\, \, \, \, \, \, \, \, \, \, (l=1,...,N), \end{equation} 
% 
% 
% 
% тут 
% 
% $\begin{array}{l} 
% {\Delta D_{scl,k} =W_{R} \Delta D_{scl,k-1} +q_{R} \sigma _{Ds} \eta _{sl,k} } \\ 
% {\Delta V_{scl,k} =W_{V} \Delta V_{scl,k-1} +q_{V} \sigma _{Vc} \eta _{scl,k} } \end{array}$;                                                   
% \eqref{eq:__6_17_}
% 
% 
% 
% $\eta _{Dl,k} $,$\eta _{Vl,k} $,$\eta _{sl,k} $,$\eta _{scl,k} $,$\xi _{\tau ,k} $,$\xi 
% _{V\tau ,k} $ -- стандартні центровані дискретні білі шуми з одиничною інтенсивністю;
% 
% $\sigma 
% _{DS} $, $\sigma _{Vs} $ -- СКЗ гаусівських складових похибок СНС у визначенні псевдодальностей 
% і псевдошвидкостей;
% 
% $\sigma _{DC} $, $\sigma _{Vc} $ --  СКЗ корельованих складових похибок СНС у визначенні 
% псевдодальностей і псевдошвидкостей;
% 
%  Для стандартного режиму СНС GPS NAVSTAR можуть бути рекомендовані наступні значення 
% параметрів моделі \eqref{eq:__6_16_}, \eqref{eq:__6_17_}:
% 
% \[\begin{array}{l} {\sigma _{Ds} =(1\div 4){\rm <;}\, \, \, \, \, \sigma _{Vs} =(0,02
% \div 0,03){\rm </A};} \\ {\sigma _{Dc} =(5\div 9){\rm <;}\, \, \, \, \sigma _{Vc} 
% =(0,02\div 0,03){\rm </A};} \end{array}\] 
% 
% При застосуванні в навігаційних розрахунках комбінованих методів додаткову навігаційну 
% функцію дає вимірник висоти. Так, у далекомірному методі при наявності на борту ЛА 
% високоточної системи вимірювання висоти польоту Н, сфера з радіусом Rз + Н (де Rз 
% = 6371116 м -- радіус сфери, рівновеликої земному геоїду) може бути прийнята  за 
% додаткову поверхню положення. У цьому випадку можна замість вимірювань трьох дальностей 
% до НС обмежитися вимірюванням двох дальностей, тоді навігаційна функція буде включати 
% два рівняння сфери, а третє необхідне рівняння дає вимірник висоти 
% 
% (Rз + H)2 = x2 + y2 + z2.
% 
% Ось чому для реалізації процедур оптимального комплексування  інерціальної та супутникової 
% систем навігації необхідно мати додаткову модель похибок барометричного висотоміра.
% 
% 
% 
% 
% \subsection{6.4 Математичні 
% моделі похибок барометричного висотоміра}
% 
% Похибка барометричного висотоміра (БВ) у визначенні абсолютної висоти ЛА може бути 
% описана співвідношенням вигляду:
% 
% $\Delta h(t_{k} )=\Delta h_{{\rm 2A}} +\sigma _{h} \eta _{n,k} $ ,               \eqref{eq:__6_18_}
% 
% де $\Delta 
% h_{{\rm 2A}} $-- квазістаціонарна похибка виміру барометричної висоти, що обумовлена 
% неточністю початкової виставки, а також змінами температури та тиску атмосфери за 
% час польоту;
% 
% $\sigma _{h} _{} $-- СКЗ флюктуаційної складової похибки, що обумовлена пульсаціями 
% тиску й іншими факторами;
% 
% $\eta _{n,k} $-- дискретний білий шум з одиничною інтенсивністю.
% 
% У свою чергу дискретна модель еволюції квазістаціонарної похибки БВ може бути представлена 
% в наступному вигляді:
% 
% $\Delta h_{c,k} =\Delta h_{{\rm 2c},k-1} +\sigma _{\xi A} \xi _{k-1} $,                                       \eqref{eq:__6_19_}
% 
% де  $\sigma 
% _{\xi A} $-- заданий параметр; $\xi _{k-1} $ -- стандартний дискретний білий шум 
% і одинична інтенсивність.
% 
% Аналіз показує, що для моделі похибок БВ \eqref{eq:__6_18_}, \eqref{eq:__6_19_} 
% можна рекомендувати наступні значення параметрів:
% 
% \[\sigma _{h} =(0,5\div 1){\rm <;}\, \, \, \, \sigma _{\xi c} =(0,05\div 0,02){\rm 
% <;}\, \, \, \, \, \sigma _{\Delta hc,0} =(3\div 5){\rm <,}\] 
% 
% де $\sigma _{\Delta hc,0} $ --  СКЗ похибки $\Delta $\textit{hс} у початковий момент 
% часу.
% 
% 
% \subsection{6.5 Дослідження математичних моделей похибок MEMС- датчиків}
% 
% Математична модель похибок трикомпонентного МЕМС датчика кутової швидкості при проведені 
% моделюванні була представлена у більш розширеному вигляді ніж у п.п. 6.2.
% 
% Нижче приведена модель датчика кутової швидкості, яка використовувалася при проведенні 
% досліджень:
% 
% \[\begin{array}{l} {\omega _{x}^{m} =(1+K_{\omega x} )[\omega _{x} +K_{xz} \omega 
% _{y} -K_{xy} \omega _{z} +\varepsilon _{\omega x} ]+\varepsilon _{\omega x_{{\rm 
% c}} } ;} \\ {\omega _{y}^{m} =(1+K_{\omega y} )[\omega _{y} +K_{yx} \omega _{z} -K_{yz} 
% \omega _{x} +\varepsilon _{\omega y} ]+\varepsilon _{\omega y_{{\rm c}} } ;} \\ {
% \omega _{z}^{m} =(1+K_{\omega z} )[\omega _{z} +K_{zy} \omega _{z} -K_{yz} \omega 
% _{y} +\varepsilon _{\omega z} ]+\varepsilon _{\omega z_{{\rm c}} } ,} \end{array}\] 
% 
% де $\omega 
% _{x}^{m} ,{\rm \; }\omega _{y}^{m} ,{\rm \; }\omega _{z}^{m} $ --  обмірювані величини 
% кутових швидкостей; 
% 
% $\omega$\textit{x} , $\omega$\textit{y} , $\omega$\textit{z}  -- `` дійсні '' значення 
% кутових швидкостей; 
% 
% \textit{К}$\omega$\textit{x}, \textit{К} $\omega$\textit{y}, \textit{К} $\omega$\textit{z} -- 
% похибки масштабних лінійних коефіцієнтів; 
% 
% \textit{Кxy}, \textit{Кyz} \dots  \textit{Кzx}, \textit{Кzy} -- помилки невиставки 
% приладів у  відповідних площинах зв'язаної системи; 
% 
% $\epsilon$$\omega$\textit{x} , $\epsilon$$\omega$\textit{y} , $\epsilon$$\omega$\textit{z} -- 
% систематичні складові зсуви нулів датчиків,  $\epsilon$$\omega$\textit{xс} , $\epsilon$$\omega$\textit{yс} , $\epsilon$$\omega$\textit{zс} -- 
% випадкові складові зсуви нулів датчиків (шуми вимірів).
% 
% Похибки, зв'язані із систематичною і випадковою складовими зсувів нулів датчиків, 
% зі змінами масштабних лінійних коефіцієнтів, розподіляються таким чином, що при збільшенні 
% одного з них зростають всі інші. Цей факт ілюструється у табл. 6.1, 6.2, де показані 
% приклади зміни складових  похибок датчиків первинної інформації сучасних прецизійних 
% ІНС. 
% 
% 
% 
% 
% 
% 
% 
% 
% 
%   \textit{Таблиця }6.1\textit{}
% 
% \begin{tabular}{|p{2.5in}|p{0.7in}|p{0.7in}|p{0.7in}|} \hline 
% \multicolumn{4}{|p{1in}|}{Акселерометри} \\ \hline 
% Зсув показань, 10-3 & 0,01 & 0,05 & 0,1 \\ \hline 
% Масштабний коефіцієнт & 0,001 & 0,005 & 0,01 \\ \hline 
% Неортогональність, кут.с & 10 & 10 & 20 \\ \hline 
% Випадкова складова, м/с3/год & 0,009 & 0,01 & 0,02 \\ \hline 
% \end{tabular}
% 
% \textit{Таблиця }6.2\textit{}
% 
% \begin{tabular}{|p{2.8in}|p{0.6in}|p{0.6in}|p{0.6in}|} \hline 
% \multicolumn{4}{|p{1in}|}{Датчики кутової швидкості} \\ \hline 
% Дрейф, що не залежить від перевантаження, град/год & 0,005 & 0,01 & 0,1 \\ \hline 
% Дрейф, 
% що залежить від перевантаження, град/год & 0,0075 & 0,015 & 0,15 \\ \hline 
% Масштабний коефіцієнт & 0,0025 & 0,005 & 0,02 \\ \hline 
% Неортогональність, кут. сек & 20 & 60 & 120 \\ \hline 
% Випадкове блукання, град/год & 0,0005 & 0,001 & 0,01 \\ \hline 
% \end{tabular}
% 
% 
% 
% Для моделюванні похибок МЕМС датчика кутової швидкості були проаналізовані його джерела 
% похибок, кількісні оцінки яких приведені в табл. 6.3. 
% 
% \textit{Таблиця }6.3.\textit{}
% 
% \begin{tabular}{|p{3.3in}|p{1.3in}|} \hline 
% Систематична складова  зсуву нуля & 15 град/год \\ \hline 
% Випадкова складова  зсуву нуля (за час польоту) & 10 град/год \\ \hline 
% Похибка масштабного лінійного коефіцієнта & 1. 5 град/год \\ \hline 
% Невиставлення осі (похибки юстирування) за 1 віссю & 15 кутов. сек \\ \hline 
% Невиставлення осі (похибки юстирування) за 2 віссю & 15 кутов. сек \\ \hline 
% \end{tabular}
% 
% 
% 
% Зокрема величина систематичної складові зсуву нуля датчика кутової швидкості як величина 
% ($\pm$2$\sigma$)  = = $\pm$15\dots 20$\circ$/год була визначена в ході аналізу стабільності 
% швидкості дрейфу існуючих та перспективних (найближчі 10 років) МЕМС-гіроскопів.  
% 
% При 
% проведенні досліджень $\epsilon$$\omega$\textit{x} , $\epsilon$$\omega$\textit{y} , $\epsilon$$\omega$\textit{z} задавалися 
% в межах $\pm$0,01\dots 0,02$\circ$/с для невідкаліброваних датчиків кутової швидкості 
% і у два три рази менше для датчиків, що пройшли попереднє калібрування на етапі виставлення. 
% 
% Випадкова 
% складові зсуву нуля датчика кутової швидкості  за годину польоту  може досягати величини 
% того ж порядку, що і систематична складових невідкаліброваних датчиків кутової швидкості. 
% При проведенні досліджень випадкова складового зсуву нуля моделювалася за допомогою 
% формуючих фільтрів, на входи яких надходив випадковий процес типу білого шуму, а 
% на виході утворюється процес  $\epsilon$$\omega$\textit{x}с(\textit{t}), $\epsilon$$\omega$\textit{вус}(\textit{t}), $\epsilon$$\omega$\textit{z}с 
% (\textit{t}). Експериментально параметри формуючого фільтра минулого підібрані  таким 
% чином, щоб випадкові блукання нуля датчика кутової швидкості за годину польоту не 
% виходили за межі  $\pm$0,05\dots 0,01$\circ$/с.
% 
% Похибки масштабних лінійних коефіцієнтів датчиків кутової швидкості, також мають 
% величину того ж порядку, що і  систематична складових невідкаліброваних датчиків 
% кутової швидкості, але в два\dots три рази її перевищують. При проведенні досліджень 
% величини \textit{К}$\omega$\textit{x}, \textit{К}$\omega$\textit{y}, \textit{К}$\omega$\textit{z} задавалися 
% в межах $\pm$0,05\dots 0,008$\circ$/с. 
% 
% Помилки невиставки приладів у відповідних площинах зв'язаної системи  \textit{Кxy}, \textit{Кyz} \dots  \textit{Кzx}, \textit{Кzy}  залежать 
% від точності юстировки датчиків у блоці чутливих елементів (БЧЕ) і від точності юстировки 
% самого БЧЕ по осях ЛА. В даний час досягнута точність юстирування складає 5...10 
% кут. с., що в перерахуванні відповідає коефіцієнтам  \textit{Кxy}, \textit{Кyz} \dots  \textit{Кzx}, \textit{Кzy}  рівним  
% 2,5$.$10-5\dots 5$.$10-5$\circ$/с. При проведенні досліджень ці коефіцієнти були 
% збільшені на порядок, з метою урахування статичних деформацій конструкції ЛА в місці 
% установки БЧЕ. 
% 
% Як показали попередні дослідження моделей похибки датчиків кутової швидкості при 
% такому рівні систематичних і випадкових складових зсувів нулів датчиків мертва зона 
% і гістерезис практично не надають впливу на помилки обчислення навігаційних параметрів.  
% 
% Аналогічний 
% МЕМС датчику кутової швидкості мають вид моделі МЕМС акселерометрів: 
% 
% \[\begin{array}{l} {a_{x}^{m} =(1+K_{ax} )[a_{x} +K_{xy} a_{y} -K_{xz} a_{z} +Q_{x} 
% a_{x}^{2} +\varepsilon _{ax} +\omega _{z} L_{x} +\omega _{y} L_{x} ]+\varepsilon 
% _{ax_{{\rm c}} } ;} \\ {a_{y}^{m} =(1+K_{ay} )[a_{y} +K_{yx} a_{z} -K_{yz} a_{x} 
% +Q_{y} a_{y}^{2} +\varepsilon _{ay} +\omega _{x} L_{y} +\omega _{z} L_{y} ]+\varepsilon 
% _{ay_{{\rm c}} } ;} \\ {a_{z}^{m} =(1+K_{az} )[a_{z} +K_{zy} a_{x} -K_{zx} a_{y} 
% +Q_{z} a_{z}^{2} +\varepsilon _{az} +\omega _{x} L_{z} +\omega _{y} L_{z} ]+\varepsilon 
% _{az_{{\rm c}} } ,} \end{array}\] 
% 
% де $a_{x}^{m} ,{\rm \; }a_{y}^{m} ,{\rm \; }a_{z}^{m} $ -- обмірювані прискорення,  
% 
% \textit{ax , 
% ay , az} --  `` дійсні ''  прискорення,
% 
% $\omega$\textit{x , }$\omega$\textit{y , }$\omega$\textit{z} -- кутові швидкості 
% обертання ЛА,
% 
% \textit{Lx , Ly , Lz}  -- лінійні зсуви місця установки блоку чуттєвих елементів 
% від центра мас ЛА;
% 
% \textit{Кax} \dots  \textit{Кaz} -- похибки масштабних коефіцієнтів, 
% 
% \textit{Кxz} \dots  \textit{К}z\textit{x} -- похибки юстирування, 
% 
% \textit{Qx, Qy, Qz} -- коефіцієнти квадратичної похибки через нелінійність характеристики 
% приладу, 
% 
% $\epsilon$ах , $\epsilon$а\textit{y} , $\epsilon$а\textit{z} -- систематичні складові 
% зсувів нулів датчиків,  $\epsilon$а\textit{x}с , $\epsilon$а\textit{y}с , $\epsilon$а\textit{z}с 
% -- випадкові складові зсувів нулів датчиків (шуми вимірів).
% 
% Проаналізувавши розподіли погрішностей існуючих акселерометрів (табл. 6.4)  сформовані 
% значення складових погрішностей МЕМС акселерометрів. Зокрема  $\epsilon$ах , $\epsilon$а\textit{y} , $\epsilon$а\textit{z} задавалися 
% в діапазоні 5 $.$10-2 \textit{g}  для невідкаліброваних  датчиків і 2,5 $.$10-2 \textit{g}  для 
% датчиків, що пройшли передстартову виставку (калібрування).  Випадкові складові зсуви 
% нулів акселерометрів були підібрані експериментально так, щоб випадкові блукання 
% нуля датчика за годину польоту не виходили за межі  $\pm$(0.9...1,2) $.$10-2 \textit{g} . 
% Похибки масштабних коефіцієнтів \textit{Кax} \dots  \textit{Кaz}  задавалися в межах 
% 0,004\dots 0,005,а коефіцієнти квадратичної похибки \textit{Qx, Qy, Qz}  варіювалися 
% в межах 2,5 $.$10-4\dots 1,5 $.$10-4. Помилки невиставки приладів у відповідних площинах 
% зв'язаної системи формулювалися аналогічно датчикам кутових швидкостей.
% 
% \textit{Таблиця }6.4.\textit{}
% 
% \begin{tabular}{|p{3.3in}|p{1.3in}|} \hline 
% Мертва зона (гранична чутливість)\textbf{} & $10^{-3} g...10^{-2} g$ \\ \hline 
% Систематична складова  зсуву нуля & (5\dots 3) $.$10-2 \textit{g} \\ \hline 
% Випадкова складова  зсуву нуля (за час польоту) & 0.9 $.$10-2 \textit{g} \\ \hline 
% Похибка 
% масштабного лінійного коефіцієнта & 0.1\%  по всей шкале \\ \hline 
% Нелінійний коефіцієнт калібрування (коефіцієнт впливу  \textit{g}2) & 0,3 $.$10-2 \textit{g}/\textit{g}2 \\ \hline 
% Невиставлення 
% осі приладу (похибки юстирування) & 15 кутов. сек \\ \hline 
% Лінійний температурний коефіцієнт  & 10-2 \textit{g}/град \\ \hline 
% \end{tabular}
% 
% 
% 
% Випадкові складові і перекручування масштабного коефіцієнта моделювалися з використанням 
% генераторів "білого шуму" і формуючих фільтрів. При цьому вважалося, що кожен чуттєвий 
% елемент цілком визначається значеннями цих складових, а самі ці складові змінюються 
% таким чином, що при збільшенні одного з них зростають і всі інші.
% 
% При моделюванні білошумних погрішностей датчиків БІНС можна скористатися схемою  формування 
% випадкових  сигналів, приведеної на рис 6.1.
% 
% 
% 
% Складовими випадкової похибки датчиків БІНС є марковський процес і процес випадкового 
% блукання.
% 
% Марковський процес використовується для апроксимації високочастотного стаціонарного 
% випадкового процесу. Процес випадкового блукання є нестаціонарним.
% 
% Складові марковського процесу і випадкового блукання можуть збуджуватися окремими 
% білими шумами або спільним білим шумом, у цьому випадку виникає кореляція між процесами.  
% 
% Марковська 
% складова випадкової похибки датчиків БІНС $\varepsilon _{} $ має кореляційну функцію
% 
% \[K_{
% \varepsilon } (\tau )=A\cdot e^{-\mu \left|\tau \right|}  ,\] 
% 
% де    $A=\sigma ^{2} $ дисперсія випадкової похибки; $\sigma $ середньоквадратичне 
% відхилення; $\tilde{\mu }$ коефіцієнт згасання кореляційної функції; $T={1 \mathord{
% \left/{\vphantom{1 \mu }}\right.\kern-\nulldelimiterspace} \mu } $  стала часу кореляції, 
% що дорівнює 0.5...\dots 1год.
% 
% Зазвичай марковськую складову $\varepsilon _{} $ представляють у вигляді випадкового 
% процесу, зв'язаного з білим шумом диференціальним рівнянням першого порядку.
% 
% Так, кореляційної функції $_{\varepsilon } $ відповідає рівняння
% 
% \[\dot{\varepsilon }_{M} =-\mu \cdot \varepsilon _{M} +\sqrt{2\cdot A\cdot \mu } 
% \cdot w(t) ,\] 
% 
% де   \textit{w}(\textit{t})   вихідний білий шум одиничної інтенсивності з рівним 
% нулю математичним сподіванням  і кореляційною функцією $M\left[w(t)\cdot w(t)\right]=
% \delta (t-\tau )$.
% 
% \includegraphics[bb=0mm 0mm 208mm 296mm, width=170.6mm, height=183.8mm, viewport=3mm 
% 4mm 205mm 292mm]{image1.eps} Рисунок 6.2
% 
% Рівняння   такого виду   називають рівнянням формуючого фільтра, на вхід якого надходить 
% випадковий процес \textit{w}(\textit{t}) типу білого шуму, а на виході маємо процес $\varepsilon 
% (t)$з кореляційною функцією $K_{\varepsilon } (\tau )$.
% 
% Для проведення досліджень була складена програма моделювання, зокрема субблоки «DYCu» 
% і «Akselerometr», в яких у свою чергу формувалися субблоки «Model oshibok DUS» і 
% «Model oshibok akselerator», де моделювалися похибки датчиків. На рис. 6.2 як приклад 
% розкритий субблок  «Model oshibok akselerator».
% 
% \includegraphics[bb=0mm 0mm 208mm 296mm, width=170.1mm, height=120.5mm, viewport=3mm 
% 4mm 205mm 292mm]{image2.eps}
% 
% У верхньому правому куті субблока показаний ще один внутрішній блок -- блок моделювання 
% випадкових складові зсуву нулів датчиків (шумів вимірів). У нижній частині блоку 
% зібрані схеми формування складових похибок, обумовлених відцентровими прискоренням, 
% пропорційним зсувові центра мас акселерометра \textit{Lх, Ly, Lz, } від осі обертання 
% ЛА (центра мас ЛА). Величини зсуву  \textit{Lх, Ly, Lz  }задаються з блоку «Start
% \_napametru», там же передбачене переключення умов формування погрішностей датчиків 
% первинної інформації:
% 
% \begin{enumerate}
% \item - -ідеальні датчики (датчики без погрішностей) і датчики з похибками вимірів;
% 
% \item - -датчики 
% тільки із систематичної складової похибки і датчики із систематичної і динамічної 
% (випадкової) складової похибки;
% 
% \item - -датчики попередньо відкалібровані і не відкалібровані.
% \end{enumerate}
% 
% За принципом аналогічному субблоку «Model oshibok akselerator» побудований субблок 
% «Model oshibok DUS». На осцилограмах рис. 6.3 ілюструється характер зміни похибок 
% (град/сек) відкаліброваних і не відкаліброваних датчиків кутових швидкостей при прямолінійному 
% горизонтальному польоті. 

% 
