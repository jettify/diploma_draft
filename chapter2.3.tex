 
\subsection{Аналіз та вибір варіанта інерціальної навігаційної системи}

В інерціальній  навігаційній системи (ІНС)  інформацію про швидкість і 
координати одержують шляхом інтегрування сигналів, що відповідають прискоренням 
ЛА. Інформація про прискорення надходить від розташованих на борту ЛА 
акселерометрів. Процедура інтегрування векторних величин, якими є прискорення і 
швидкості ЛА, забезпечується шляхом відтворення (моделювання) на борті ЛА відповідної 
системи координат. З цією метою найчастіше використовують гіростабілізатори 
або гіроскопічні датчики кутової швидкості разом з обчислювачем. 

Наявність похибок датчиків ІНС у свою чергу приводить до похибок 
у визначенні навігаційних координат руху ЛА, от чому при створенні 
ІНС намагаються зменшити величину похибок первинних датчиків.
Перевагами інерціальних систем перед іншими системами навігації є їхня 
повна автономність, абсолютна перешкодозахищеність, а також висока інформативність.
У залежності від способів розташування акселерометрів на ЛА розрізняють платформні 
і безплатформні ІНС. У першому випадку акселерометри встановлюються на 
гіростабілізуючій платформі, у другому безпосередньо на корпусі ЛА або в 
спеціальному блоці чуттєвих елементів, при цьому осі чутливості акселерометрів 
не змінюють орієнтацію відносно напрямку осей, зв'язаних з ЛА.
Серед платформних ІНС розрізняють ІНС з некоректованою платформою та 
ІНС з горизонтальною платформою. 

У ІНС з некоректованою платформою осі платформи, а також акселерометри, що установлені на 
цій платформі, не обертаються в інерціальному просторі. 
ІНС з горизонтальною платформою у свою чергу класифікують як ІНС із вільною в 
азимуті платформою (платформа розташовується відносно точки світового простору – відносно зірки) 
та ІНС з корегованою в азимуті платформою (платформа стабілізується відносно меридіана – „направлена” на північ).
По ролі обчислювача у визначенні кутових і лінійних координат прийнято 
розрізняти геометричні, напіваналітичні та аналітичні ІНС. У геометричних ІНС 
основним елементом служить гіростабілізатор, що відтворює напрямок осей інерціальної 
системи відліку, і платформа з акселерометрами, осі чутливості яких відтворюють деякі 
напрямки в площині обрію і напрямок місцевої вертикалі. Роль обчислювача в такій ІНС 
мінімальна і зводиться до забезпечення корекції заданого положення платформи. Інформація 
про координати знімається з кутомірних пристроїв гіростабілізатора і платформи.

До напіваналітичних систем відносять системи з горизонтальною платформою. У 
цих системах гіроплатформа з акселерометрами відтворює напрямок нормальної (рухливої) 
системи відліку. З кутомірних пристроїв гіростабілізатора знімається інформація про 
кути крену, тангажу, курсу ЛА. Обчислювач ІНС вирішує задачу визначення кінематичних 
параметрів руху центра мас ЛА і видає сигнали для корекції гіростабілізатора.
До аналітичних ІНС відносять безплатформні ІНС та ІНС з акселерометрами на некоректованому 
або вільному гіростабілізаторі. Обчислювач ІНС у даному випадку виконує найбільший обсяг 
обчислень. Крім визначення кінематичних параметрів руху центра мас ЛА він визначає кутову 
орієнтацію нормальної рухливої системи координат відносно інерціальної і кутову орієнтацію 
зв'язаної рухливої системи координат щодо нормальної. 

Побудова прецизійних і одночасно надійних гіроплатформ являє собою складну технічну задачу. 
Тому останнім часом усе більше уваги приділяється розробці так званих безплатформних ІНС (БІНС), 
у яких датчики акселерометрів жорстко зв’язані з корпусом ЛА. Такі системи мають у своєму складі 
гіроскопічні прилади, але головною задачею цих пристроїв є забезпечення обчислювачів БІНС 
інформацією про кутове положення ЛА, а так само про положення осей чутливості акселерометрів 
відносно обраної навігаційної системи координат. Відсутність горизонтальної платформи вимагає 
виділяти з показань акселерометрів сигнали, що є прискореннями ЛА, тобто обчислювачі БІНС 
аналітично визначають напрямок вертикалі. При цьому точність зазначеного моделювання 
визначається точністю роботи обчислювача і, природно, точністю датчиків первинної навігаційної інформації.
До числа потенційних переваг безплатформних інерціальних навігаційних систем БІНС у 
порівнянні з платформними ІНС можна віднести:

\begin{itemize}
 \item менші розміри, вага й енергоємність;
 \item істотне спрощення механічної частини системи ; 
 \item відсутність обмежень по кутах розвороту;
 \item скорочення часу початкової виставки.
\end{itemize}

Тому, навіть за певних труднощів, що виникають при створенні БІНС, таких як:
\begin{itemize}
 \item 
 \item розробка датчиків інформації із широким діапазоном вимірів і прийнятною точністю в умовах їхнього твердого кріплення на борті ЛА;
 \item розробка БЦВМ, що мають достатню швидкодію.
\end{itemize}

У роботі розглядатиметься безплатформна інерціальна система.
В залежності від способу визначення кутового положення об'єкта в інерціальному просторі 
можливі наступні основні варіанти схеми БІНС:

Перший варіант передбачає наявність у БІНС шести акселерометрів  рознесених по осям об'єкта на відстань 
(для виміру кутових прискорень) і обчислювального пристрою (ОП);

Другий варіант включає три лінійних акселерометри  і три вимірники кутової 
швидкості руху об'єкта щодо центра мас, встановлених в центрі мас об'єкта, а також ОП.

Третій варіант передбачає наявність трьох лінійних акселерометрів, і вимірника 
кутового положення об'єкта в інерціальному просторі, встановлених у центрі мас об'єкта, і ОП.

Стосовно розглянутого класу ЛА використання БІНС першого варіанту зустрічає складності 
реалізації через малу вимірювальну базу (2l ) визначення кутових прискорень об'єкта за 
допомогою акселерометрів. До того ж, похибки БІНС цього варіанту у визначенні координати, 
обумовлені помилками виміру кутових прискорень, має три складових: одна з них постійна, 
інша наростає пропорційно квадратові часу руху, а третя змінюється з періодом Шулера. 
Звідси ясно, що цей варіант схеми може бути застосований тільки при досить точних 
акселерометрах і для об'єктів, що здійснюють політ протягом нетривалого часу.
 
Реалізація третього варіанта БІНС припускає наявність у складі навігаційної 
системи триступеневого гіроскопічного вимірника кутових положень (електростатичні 
гіроскопи, гіроскопи, що динамічно з’являються у великій кількості ) -- 
досить дорогі  прецизійні прилади. 

За результатами аналізу можна зробити висновок, що в даній роботі доцільно 
використовувати БІНС, що побудована на  трьох акселерометрах  і трьох вимірниках 
кутової швидкості, тобто  БІНС другого класу за вище приведеною класифікацією. 
Найбільш поширеними й перспективними у використанні в якості чутливих елементів є 
лазерні кільцеві гіроскопи. Так як БІНС може будуватись як середньоточна і навіть 
груба система, тому можна використати недорогі датчики інформації, такі як  
МЕМS-датчики. Відомо, що датчики даного класу у порівнянні з лазерними гіроскопами 
характеризуються меншою точністю, але з урахуванням вимого щодо невисокої 
собівартості системи керування в цілому, а також з огляду на те,що поставлена 
задача може бути вирішена із задовільною для нас точністю, використання недорогих 
датчиків на основі нанотехнологій є достатньою. Крім того, застосування МЕМS – 
датчиків є економічно – доцільними, для компенсації похибок, які виникають в 
процесі роботи БІНС з МЕМS-датчиками пропонується виконувати польотне калібрування.
Для реалізації польотного калібрування каналів БІНС запропоновано шляхом слідкування, 
за змінами оцінки помилки БІНС на виході фільтра схеми компенсації проводити 
апроксимацію з наступною екстраполяцією зміни цієї помилки. 

Під польотним калібруванням розуміють метод підвищення роботи БІНС шляхом оцінки у польоті 
систематичних складових похибок БІНС та їх компенсації. Для виконання такої оцінки необхідно 
порівнювати вихідну інформацію БІНС з еталонною навігаційною інформацією і, маючи модель 
помилок БІНС, виконати оцінку параметрів цією моделі за різницею між вихідною інформацією 
БІНС та еталонною інформацією.

   З урахуванням того, що БІНС працює у складі комплексної ІССН необхідно обрати спільну 
навігаційну систему  координат (СК) й для обраної СК розробити  алгоритми розв’язку 
кінематичних рівнянь числення навігаційних параметрів. З урахуванням того, 
що СНС частіше за все працює в географічній системі координат алгоритми 
роботи БІНС також слід формувати в цієї системі координат.  


% 
% На основі вище викладеного було проведене уточнення розроблених алгоритмів числення 
% навігаційних параметрів польоту, як у географічній так й в ортодромічній системі 
% координат. При цьому основний блок $-$ блок проведення навігаційних розрахунків залишається 
% незмінним.