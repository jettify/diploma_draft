% This document does not intend to be an everything-in-one document for
% creating the diploma paper. However it does the most meaningful work
% for you.
%
% You can get the full document list on your cathedra or use the following 
% (order preserved)
%
% 1.  Title page
% 2.  General task.
% 3.  Plan.
% 4.  Separate sections consults (for ex.: ecology and labour protection).
% 5.  Technical task.
% 6.  Annotation.
% 7.  Table of Contents.
% 8.  Nomenclature list.
% 9.  Diploma paper itself.
% 10. Conclusions.
% 11. Reference list.
% 12. Appendixes.
% Andriy Senkovych
%
% Many thanks to my friend Andriy, for this template, original comments saved and some my notes added.
% 
% Mykola Novik

% diploma.tex


\documentclass[ukrainian,utf8,simple,floatsubsection, hpadding=5mm,equationsubsection,]{eskdtext}
% hpadding=5mm  intend between frame and text, by default 3mm
% equationsubsection enumerate formulas based on number of subsection


%%%%%-------------[Math Tuning]--------------------------%%%%%%
%%%%%----------------------------------------------------%%%%%%
\usepackage[warn]{mathtext}
\usepackage[unicode]{hyperref}   % enable hyperlinks (активувати посилання)
% Formulae + cyrillic = wtf???
% You should use \text{} in formulae to write cyrillic characters. 
% Unfortunately mathtext package is not supported :(
% Example: $ \text{а} + \text{б} = \text{в} $

% explanation after formulas should be in next form:
% \begin{ESKDexplanation}
%     \item  [something]
%     \item  [something else]
% \end{ESKDexplanation}
\usepackage{amssymb}             % special math characters
\usepackage{amsmath}             % using cyrillic in formulae
\usepackage{amsfonts}            % special math fonts

%%%%%-------------[Graphics Tuning]----------------------%%%%%%
%%%%%----------------------------------------------------%%%%%%
\usepackage{graphicx,epstopdf}   % epstopdf-convert eps files to pdf
\usepackage{float}               % used to manage positon of float figures and etc.
\graphicspath{{algorithms/}{schemes/}{software/}{fig/}} % look up folders for figures

%%%%%-------------[Tables Tuning]------------------------%%%%%%
%%%%%----------------------------------------------------%%%%%%
\usepackage{longtable}           % multipage tables
\usepackage{multirow}            % using rowspan in tables

%%%%%-------------[Abbreviations support]----------------%%%%%%
%%%%%----------------------------------------------------%%%%%%
\usepackage{nomencl}             % support for abbreviations
\makenomenclature                % generate abbrevs index file


%%%%%-------------[Other stuff]--------------------------%%%%%%
%%%%%----------------------------------------------------%%%%%%
\usepackage{eskdtotal}          % calculate total number of pages, figures, tables and etc.
% You should use \ESKDtotal{page} to count number of pages in Diploma
% \ESKDtotal{ELEMENT}    return total number of elements (pages,figures and so on)
% ELEMENT could be: page, figure, table, appendix or bibitem
% \ESKDtotal{page}  if you wand to obtain number of pages in document
\usepackage{listings}            % to add source codes


%%%%%-------------[Class Parameters]---------------------%%%%%%
%%%%%----------------------------------------------------%%%%%%
% variables.tex
% This file contains information about author and other specific
% people for use in eskdx collection.

\title{\fontsize{12}{12} \selectfont Інтегрована інерціально-супутникова система навігації, що базується на принципах комплексної обробки інформації
з використанням калманівської фільтрації}
% smaller size of font set for the title in frame
\author{НовікМ.В.}

\ESKDchecker{ФіляшкінМ.К.}
\ESKDnormContr{КозловА.П.}
\ESKDapprovedBy{СинєглазовВ.М.}

\ESKDdepartment{Міністерство освіти і науки України}
\ESKDcompany{Національний авіаційний університет}

\ESKDsignature{НАУ 11 00 75 000 ПЗ}
% 11 year of defend
% 00 number of thesis
% 75 last two nambers of studens mark book
% 000 must stay 000

\ESKDgroup{ІАСУ 608}

\ESKDsectAlign{section}{Center}
\ESKDsectAlign{subsection}{Center}
\ESKDsectAlign{subsubsection}{Center}

             % class parameters tuning

\begin{document}
\newpage

%%%%--------------------[DELETE THIS]---------------------%%%%%
% \subsection*{TO DO}
% \begin{enumerate}
%  \item Посатавити номер наказу та дату видачи наказу;
%  \item Уточнити номер НДР;
%  \item Перевірити Технічне завдання;
%  \item Перевірити нумерацію сторінок;
%  \item Перевірити і додати джерела до списку літератури;
%  \item Можливо зменшити шрифт в прізвищах.
% \end{enumerate}

%% Empty page for title
\newpage
\ESKDthisStyle{empty}
\mbox{}

% 2 empty pages for task
\newpage
\ESKDthisStyle{empty}
\mbox{}
\newpage
\ESKDthisStyle{empty}
\mbox{}

%% Technical task
\newpage
\ESKDthisStyle{formIIab}
\documentclass[ukrainian,utf8,simple,floatsingle,hpadding=5mm]{eskdtext}
\usepackage[numberright]{eskdplain}
% variables.tex
% This file contains information about author and other specific
% people for use in eskdx collection.

\title{\fontsize{12}{12} \selectfont Інтегрована інерціально-супутникова система навігації, що базується на принципах комплексної обробки інформації
з використанням калманівської фільтрації}
% smaller size of font set for the title in frame
\author{НовікМ.В.}

\ESKDchecker{ФіляшкінМ.К.}
\ESKDnormContr{КозловА.П.}
\ESKDapprovedBy{СинєглазовВ.М.}

\ESKDdepartment{Міністерство освіти і науки України}
\ESKDcompany{Національний авіаційний університет}

\ESKDsignature{НАУ 11 00 75 000 ПЗ}
% 11 year of defend
% 00 number of thesis
% 75 last two nambers of studens mark book
% 000 must stay 000

\ESKDgroup{ІАСУ 608}

\ESKDsectAlign{section}{Center}
\ESKDsectAlign{subsection}{Center}
\ESKDsectAlign{subsubsection}{Center}


\include{textcomp}
\usepackage{longtable} % multipage tables
\usepackage{multirow} % using rowspan in tables

\ESKDstyle{formIIab}
\begin{document}

\ESKDthisStyle{formII}


\section*{Технічне завдання}

\subsubsection*{1. Найменування та галузь застосування}
Комітет ІКАО з перспективних навігаційних систем (FANS- Future Air Navigation System) прийняв 
рішення про обов'язкове використання систем супутникової навігації в сполученні з ІНС. Тому в 
даний час у всіх галузях авіації основним інформаційним ядром сучасного навігаційного комплексу 
є інтегрована інерціально-супутникова система навігації (ІССН).

В теперішній час визнано, що одним з основних шляхів вдосконалення навігаційного обладнання 
є створення комплексних навігаційних систем, що інтегрують  командні прибори в єдиний блок, 
здатний автоматично вирішувати задачу навігації ЛА. Сутність комплексування полягає у 
використанні інформаційної та структурної надмірності для підвищення точності, надійності 
та завадостійкості інформації при вимірюванні одних і тих же або функціонально зв’язаних 
навігаційних параметрів. Інформаційна надмірність полягає в тому, що забезпечується отримання 
однорідної інформації від декількох навігаційних датчиків різної фізичної природи з наступною 
сумісною обробкою цієї інформації в спеціалізованому обчислювачі. Надмірність структури 
комплексу забезпечує його працездатність при відмові, особливо короткотривалій, одного із датчиків. 


Найбільш привабливим для розв’язання цієї задачі є залучення калманівської фільтрації. 
Фільтр Калмана призначений для ідентифікації (оцінювання) змінних стану системи за даними 
вимірювання вихідних сигналів цієї системи, які містять похибки вимірювання (вимірювальний шум). 
Ідентифікація оптимальна в тому смислі, що сума квадратів похибок оцінювання змінних стану
в будь-який момент часу має найменше з можливих значень. Похибка оцінювання це різниця 
між оцінкою фільтра й дійсним значенням змінних стану системи при наявності в системі 
детермінованих і випадкових похибок вимірювань. Отже, фільтр Калмана призначений для 
найкращого  відновлення змінних стану, тобто для оптимального приглушення вимірювальних шумів.

\subsubsection*{Підстава до розробки}

Наказ по Національному авіаційному університету

\No1111/ст від <<20>> жовтня 2010 р.

\subsubsection*{Мета та призначення розробки}

Основною метою роботи є аналіз та вибір схеми комплексної інерціально-супутникової 
навігаційної системи та схем оцінювання та корекції в цій системі і, як наслідок, 
розробка слабко зв’язаної схеми інтеграції, що базується на принципах комплексної 
обробки інформації з використанням калманівської фільтраці, дослідження ступеню 
впливу похибок датчиків первинної інформації  безплатформної інерціальної системи 
(БІНС) та супутникової навігаційної системи (СНС) на стійкість фільтра Калмана, 
точнісні характеристики числення навігаційних параметрів і динаміку зміни похибок, 
впливу перерв у роботі СНС на траекторний рух ЛА, моделювання зміни похибок 
комплексної інерціально-супутникової навігаційної системи.

\subsubsection*{Технічні вимоги}

Основні технічні вимоги:
\begin{itemize}
      \item точність визначення навігаційних параметрів:\\
  координат (СКВ), м \dotfill 15\\
  висоти (СКО), м \dotfill $0\div20$
     \item характеристики повинні зберігатися при:\\
швидкості, м/с до \dotfill 300
     \item час визначення (холодний старт), хв \dotfill <2
     \item частота відновлення координат, c$^{-1}$ \dotfill <1
     \item масса, кг \dotfill <1
     \item автоматичне, безперервне, всепогодне визначення 
поточних 3D координат місця розташування, вектора шляхової 
швидкості і шляхового кута ЛА.
    \item автоматичний тестовий контроль функціонування блоків і 
вузлів апаратури, індикація блоків, що відмовили
    \item стійке визначення навігаційних параметрів при русі з 
лінійними прискореннями і при стрибкоподібних змінах прискорення
    \item БІНС повинна забезпечити визначення координат на протязі 60 с
    \item взаємна корекція СНС та БІНС
    \item підтримка СНС від БІНС для зменшення часу повторного 
запуску (“гарячого старту”) при короткочасних перервах у роботі СНС
\end{itemize}

Вимоги до засобів захисту
\begin{itemize}
    \item робоча температура, C   \dotfill -40...+60
    \item робоча вологість (25 C)   \dotfill 98 
\end{itemize}

Додаткові вимоги
\begin{itemize}
    \item швидкість польоту ЛА, м/с \dotfill 400 
    \item максимальний кут крену ЛА,град \dotfill 450
    \item максимальний кут тангажу ЛА,град \dotfill 200
\end{itemize}




\subsection*{Джерела розробки}

Хоздоговірна науково-дослідна робота № 201-Хд04 
“Ресурс”: “Розробка попередніх алгоритмів роботи 
інерціально-супутникової навігаційної системи та 
інформаційного зв'язку з літаком-носієм ”.


\subsection*{Стадії та етапи розробки}

Проведення аналізу та вибору навігаційного забезпечення ЛА, 
схеми комплексної інерціальної-супутникової системи навігації 
та застосування калманівської фільтрації для оцінки навігаційних 
даних, розробка слабко зв’язаної схеми інтеграції.

Розробка алгоритмів роботи комплексної навігаційної системи, 
дослідження ступеню впливу похибок датчиків первинної інформації  
безплатформної інерціальної системи (БІНС) та супутникової 
навігаційної системи (СНС) на точнісні характеристики числення 
навігаційних параметрів і динаміку зміни похибок, впливу перерв 
у роботі СНС на траекторний рух ЛА.

Розробка програми моделювання помилок комплексної 
інерціальної-супутникової системи навігації з використанням 
фільтра Калмана. Пропозиція щодо удосконалення запропонованої 
навігаційної системи, шляхом модифікації оптимального фільтра 
для поліпшення його стійкості.

\subsection*{Порядок контролю та приймання}
Контроль за ходом виконання календарного плану дипломної роботи протягом 
всього періоду дипломного проектування здійснює керівник дипломного 
проектування. Керівник визначає строки виконання та почерговість кожної 
стадії розробки дипломного проекту, проведення розрахункових та 
дослідницьких робіт, виконання графічних робіт, кінцевого оформлення 
дипломного проекту та подачі проекту до попереднього захисту  на провідній 
кафедрі. Допуск до захисту у державній екзаменаційній комісії відбувається 
з дозволу завідувача кафедри після попереднього захисту.
\footnotesize
\begin{longtable}{|p{5cm}|c|c|}

% \footnotesize


%\centering
%\begin{tabular}[c]{|p{5cm}|c|c|c|c|}

\hline
\bfseries Етапи виконання дипломного проекту (роботи) & 
\bfseries Термін виконання роботи& \bfseries Примітка  \\


\hline
Підбір літератури
& 01.11.10 – 03.11.10 &  \\ 

\hline
Технічне завдання
& 03.11.10 – 06.11.10 &   \\ 
\hline
Вступ
& 06.11.10 – 09.11.10 &   \\ 
\hline
1. Обґрунтування необхідності розробки
& 09.11.10 – 12.11.10&   \\ 

\hline
2. Аналіз та вибір навігаційного забезпечення БПЛА
2.1. Аналіз і вибір варіанта супутникової навігаційної системи
2.2. Аналіз та вибір варіанта інерціальної навігаційної системи
2.3. Аналіз та вибір схеми комплексної інерціально-супутникової 
навігаційної системи 
& 12.11.10 – 18.11.10&   \\ 

\hline
3. Постановка задачі 
& 8.11.10 – 21.11.10&   \\ 
\hline
4.  Аналіз та вибір схем оцінювання та корекції в комплексній інерціально-супутниковій системі
4.1. Аналіз та вибір методу сумісної обробки інформації
4.2. Аналіз та вибір схеми корекції
4.3. Розробка слабко в’язаної комплексної інерціально-супутникової системи навігації 

&21.11.10 – 28.11.10&   \\ 
\hline




%%\end{tabular}
%\caption{норми освітленості в кабінетах з ПК}
%\label{tab:labour protection}
\end{longtable} 


Приймання здійснюється на підставі захисту дипломної роботи ДЕК Інституту 
аерокосмічних систем управління.

Термін здачі дипломної роботи: <<09>> лютого 2011 р.

\end{document}


%% Annotation aka Referat
\newpage
\ESKDthisStyle{formIIab}
% \documentclass[ukrainian,utf8,simple]{eskdtext}
% \usepackage[numberright]{eskdplain}
% % variables.tex
% This file contains information about author and other specific
% people for use in eskdx collection.

\title{\fontsize{12}{12} \selectfont Інтегрована інерціально-супутникова система навігації, що базується на принципах комплексної обробки інформації
з використанням калманівської фільтрації}
% smaller size of font set for the title in frame
\author{НовікМ.В.}

\ESKDchecker{ФіляшкінМ.К.}
\ESKDnormContr{КозловА.П.}
\ESKDapprovedBy{СинєглазовВ.М.}

\ESKDdepartment{Міністерство освіти і науки України}
\ESKDcompany{Національний авіаційний університет}

\ESKDsignature{НАУ 11 00 75 000 ПЗ}
% 11 year of defend
% 00 number of thesis
% 75 last two nambers of studens mark book
% 000 must stay 000

\ESKDgroup{ІАСУ 608}

\ESKDsectAlign{section}{Center}
\ESKDsectAlign{subsection}{Center}
\ESKDsectAlign{subsubsection}{Center}


% \title{Реферат}
% \begin{document}

\section*{РЕФЕРАТ}
Пояснювальна записка до дипломного проекту <<Інтегрована інерціально-супутникова система навігації, що базується на принципах комплексної обробки інформації з використанням калманівської фільтрації>>: стор. --- \ESKDtotal{page}  , рис. --- \ESKDtotal{figure}, використаних джерел --- \ESKDtotal{bibitem}.



ІНЕРЦІАЛЬНА НАВІГАЦІЙНА СИСТЕМА, МЕТОДИ КОМПЛЕКСНОЇ ОБРОБКИ ІНФОРМАЦІЇ,ФІЛЬТР КАЛМАНА, КОМП’ЮТЕРНО ІНТЕГРОВАНИЙ КОМПЛЕКС.

Об’єкт дослідження --- методи та алгоритми комплексної обробки інформації, принципи побудови інтегрованих навігаційних комплексів, на базі процедури
оптимальної калманівської фільтрації.

Мета диплому --- аналіз та вибір схеми комплексної інерціально-супутникової навігаційної системи та схем оцінювання та корекції в цій системі і, як наслідок, розробка слабко зв’язаної схеми інтеграції, дослідження ступеню впливу похибок датчиків первинної інформації  безплатформної інерціальної системи та точнісні характеристики числення навігаційних параметрів і динаміку зміни похибок, впливу перерв у роботі СНС на траєкторний рух ЛА, моделювання зміни похибок комплексної інерціально-супутникової навігаційної системи.

Метод дослідження --- математичне моделювання.

Розробленей алгоритм авіаційного бортового навігаційного комплексу, що включає безплатформенну інерціальну навігаційну систему, супутникову навігаційну систему та баровисотомір, дозволяє ефективно оцінити навігаційні параметри, залишивши переваги кожної із підсистем і значно знизити вплив їх недоліків.

Матеріали дипломного проекту рекомендується використовувати при проведені наукових досліджень та у навчальному процесі.

%\end{document}


%% Table of Contents
\newpage
\tableofcontents

%% Nomenclature
\newpage
\ESKDthisStyle{formIIab}
\addcontentsline{toc}{section}{ПЕРЕЛІК УМОВНИХ СКОРОЧЕНЬ}
\printnomenclature

%% Intro pages
\newpage
\ESKDthisStyle{formIIab}
\section*{Вступ}
\addcontentsline{toc}{section}{Вступ}
Підвищення ефективності експлуатації 
\nomenclature{АТ}{авіаційна техніка}, рівня безпеки 
\nomenclature{ПС}{повітряне судно}, зниження затрат на 
\nomenclature{ТОіР}{технічне обслуговування та ремонт} і 
комплектуючих виробів є основними задачами в сфері цивільної 
авіації. їх розв'язок можливий за уови корінної перебудови 
всієї системи ТОіР ПС, основу якої складає напрацювання і 
ресурс виробів.

Планово-попереджувальна система ТОіР не відповідає підвищеним 
вимогам до АТ. Перспективною є система ТОіР ПС по стану, яка 
передбачає збільшення часу екстплуатації АТ, зниження 
експлуатаційних витрат і підвищення рівня безпеки польотів.

При розробці методів і засобів настройки окремих елементів і 
функціональних сиситем ПС, необхідних для впровадження системи 
ТОіР за станом окремих типів ПС та їх комплектуючих виробів, 
значну увагу приділяють системам автоматичного керування руху 
як системам, що суттєво впливають на безпеку та економічність 
польоту ПС.

Під час експлуатації конструкція повітряного судна, його 
агрегати й окремі частини знаходяться під дією різноманітних 
навантажень, що спричиняють поступову зміну параметрів 
математичної моделі ПС.  Характер дії цих сил може бути 
різноманітними. Постійним від завантаження літака, підіймальної 
сили, змінним від дії аеродинамічних сил, епізодичним при 
ударних навантаженнях при посадці, зіткненні з нерівностями на 
злітно-посадковій смузі та ін. В результаті дії цих сил 
накопичуються втомні деформації, змінюються параметри механічних 
з'єднань важелів управління з керуючими поверхнями, що спричинює 
до невідповідності реакцій еталонної математичної моделі та 
реальної системи.

На сьогоднішній день експлуатаційне обслуговування ПС, зокрема 
діагностика та настройка \nomenclature{САК}{система автоматичного керування} 
ПС ведеться згідно регламенту, в 
якому передбачається перевірка основних конструктивних параметрів 
складових елементів САК через визначений час експлуатації, ремонт 
чи заміна окремих вузлів та елементів конструкції згідно вимог 
експлуатаційної документації. Терміни, склад і порядок виконання 
регламентних робіт складають на основі результатів 
передексплуатаційних випробувань. При цьому основні способи 
настройки спрямовані на утримання основних конструктивних 
параметрах в заданому значенні з відповідними похибками, а 
дослідження всієї системи в цілому можливе лише під час її 
активної роботи, тобто під час польоту ПС.


%% Chapter 1 with subsections
\newpage
\ESKDthisStyle{formII}
\section{Обгрунтування необхідності розробки}

Для реалізації польотного завдання літальний апарат, повинен містити у складі 
бортового устаткування пілотажний та навігаційний комплекси. Під пілотажним 
комплексом у найпростішому випадку розуміється система автоматичного керування 
(автопілот), а під навігаційним комплексом (НК) \nomenclature{НК}{навігаційний комплекс} 
розуміють сукупність бортових систем і пристроїв, призначених для рішення задач 
навігації (навігаційна система). До складу НК і ПК входять датчики 
пілотажно-навігаційної інформації, навігаційні обчислювачі пристрою керування, 
індикації та сигналізації.

Датчики навігаційної інформації слугують для вимірювань параметрів різноманітних 
фізичних полів, на базі яких визначаються навігаційні елементи польоту. Їх 
можна поділити на дві групи: 1. датчики навігаційних параметрів положення, 
які визначають координати місцезнаходження літального апарата відносно опорних 
ліній і навігаційних точок ; 2. датчики навігаційних параметрів руху, які 
вимірюють параметри вектора швидкості літака та його складові: шляхову 
швидкість, вертикальну швидкість, напрямок польоту.

Датчики пілотажної інформації вимірюють параметри польоту, які характеризують кутовий 
рух ЛА : кути крену, тангажу, рискання і кутові швидкості.

Найважливішими з пілотажно-навігаційних датчиків є: інерціально-навігаційна 
система, інерціальна курсовертикаль, система курсу і вертикалі, допплерівський 
вимірник швидкості  і кута знесення типу ДВШЗ, інформаційний комплекс 
висотно-швидкісних параметрів типу ІК ВШП або система повітряних сигналів 
типу СПС \nomenclature{СПС}{система повітряних сигналів}.

Найбільш інформативною є інерціально – навігаційна  система (ІНС)\nomenclature{ІНС}{інерціальна навігаційна система}. 
Це така навігаційна система, у якій отримання інформації про швидкість і координати 
забезпечується шляхом інтегрування сигналів, що відповідають прискоренням ЛА. 
Інформація про прискорення надходить від розташованих на борту ЛА 
акселерометрів. Процедура інтегрування векторних величин, швидкості і 
прискорення, забезпечується шляхом відтворення на борту ЛА \nomenclature{ЛА}{літальний апарат} відповідної 
системи координат, для цього, частіше за все, використовують гіростабілізатори 
чи гіроскопічні датчики кутової швидкості з обчислювачем. 


В залежності від способу розташування акселерометрів  розрізняють платформні і 
безплатфомні ІНС. У першому випадку акселерометри  встановлюються на 
гіростабілізуючій платформі, у другому – безпосередньо на корпусі ЛА чи у 
спеціальному блоці чутливих елементів. Обидві системи мають свої переваги 
та недоліки. До переваг платформних ІНС відносять простоту алгоритмів обробки 
інформації про кутове положення і лінійні прискорення та високу точність, 
зумовлену сприятливими умовами роботи вимірювачів, оскільки вони розміщуються 
на гіростабілізаційній платформі, а не безпосередньо на корпусі об’єкта.

Зараз інтенсивно розвивається БІНС, перспективність яких визначається 
такими перевагами: висока надійність, низькі масогабаритні характеристики, 
зручність експлуатації. Характерна особливість таких ІНС, полягає у 
відсутності гіростабілізаційної платформи, яка являє собою складний 
електромеханічний пристрій та відкриває широкі можливості у плані 
зменшення масогабаритних характеристик й енергоспоживання.

До навігаційних датчиків, що визначають положення ЛА відносно навігаційних 
точок і базових ліній необхідно віднести радіотехнічні системи ближньої і 
дальньої навігації, літаковий далекомір, супутникову систему навігації (СНС), 
бортову радіолокаційну станцію, різні візирні пристрої, автоматичний компас, 
астрономічну навігаційну систему, кореляційно-екстремальну навігаційну систему. 
Найсучаснішими є супутникова навігаційна система і кореляційно-екстремальна 
навігаційна система.

СНС призначені для визначення місцеположення транспортних засобів, а також 
положення нерухомих об’єктів. Особливість дії СНС \nomenclature{СНС}{супутникова 
навігаційна система} – це використання штучних 
супутників Землі як радіонавігаційних точок, координати яких, на відміну від 
наземних радіолокаційних точок, змінні. 

Ці системи досить обґрунтовано довели високу експлуатаційну якість у 
різноманітних навігаційних галузях. Зокрема, вони визнані найбільш 
перспективними й економічно ефективними в більшості авіаційних сферах 
застосування. Поряд з цим, у зв’язку з можливою короткочасною втратою 
сигналів, які поступають із супутників, ці системи не можуть забезпечити 
необхідного рівня надійності навігаційних вимірів за такими показниками 
як цілісність, доступність і безперервність. Вирішити задачу підвищення 
цих показників можна шляхом комплексування супутникових навігаційних систем 
з іншими системами. Найбільш перспективним варіант полягає у інтеграції 
супутникових та інерціальних навігаційних систем. Така інтеграція дозволяє 
ефективно використовувати переваги кожної із систем. 

Інерціальні навігаційні системи, як найбільш інформативні системи, дають 
змогу одержувати всю сукупність необхідних параметрів для керування об'єктом, 
включаючи кутову орієнтацію. При цьому, такі системи цілком автономні, 
тобто для їхнього нормального функціонування не потрібно використання 
будь-якої інформації від інших систем. Ще одна з переваг цих систем полягає 
у високій швидкості надання інформації зовнішнім споживачам: швидкість 
відновлення кутів орієнтації складає до 100 Гц, навігаційної - від 10 до 
100 Гц. Цей показник для супутникових систем складає для кращих приймачів 
10 Гц, а для звичайних, як правило, 1 Гц. Разом з тим, інерціальним системам 
притаманні недоліки, що не дозволяють використовувати їх довгий час в 
автономному режимі. Вимірювальним елементам ІНС, насамперед, гіроскопам 
та акселерометрам, притаманні методичні й інструментальні помилки, 
вихідні данні не можуть бути введені абсолютно точно, обчислювач, 
що входить до складу ІНС, вносить свої похибки. Під впливом цих факторів 
ІНС працює в так званому «збуреному» режимі, і отримана від ІНС інформація, 
буде містити похибки, що викликані впливом цих збурень, і, головне, які з 
часом збільшуються. Для корекції ІНС застосовують різні методи і засоби. 

Корекція ІНС також може здійснюватися від радіотехнічних систем навігації 
(далекомірних, різницево-далекомірних), що складаються з наземної і бортової 
підсистем. Вони забезпечують одночасний вимір пеленга (азимута) і похилої 
дальності літального апарата щодо радіонавігаційної точки, і по цій інформації 
визначається місце розташування літака в заданій системі координат. 
До радіотехнічних систем варто віднести і супутникову систему навігації. 
Численні дослідження та практика експлуатації супутникових систем показують, 
що найбільш перспективним засобом корекції ІНС є супутникові системи, які 
володіють найбільш високою точністю і глобальністю застосування. При цьому 
можливо поліпшення характеристик автономних БІНС не тільки за координатами 
і швидкістю, але й за кутовою орієнтацією. 

Недоліком всіх радіотехнічних методів навігації, у тому числі і супутникових, 
є те, що на переданий і прийнятий радіосигнал можуть накладатися природні й 
штучно створювані радіозавади. Мала потужність сигналу, велика дальність джерел 
сигналу від приймачів (26000 км), мале відношення “сигнал-шум” приводить 
до слабкої перешкодозахищеності приймачів СРНС. Контури зрушення по фазі 
і за часом можуть легко “втратити” відповідний супутник при наявності активних 
перешкод. Особливо чуттєвим щодо цього є контур спостереження за фазою. 

До того ж, існує явище періодичного зникнення сигналу від СНС. При  збільшенні 
періоду “радіомовчання” супутника величина помилки навігаційних визначень 
збільшується аж до зриву керування (стабілізації на заданій траєкторії).  

Виникає потреба у автономних засобах навігації, які не вимагають зовнішніх 
сигналів, а тому й не зазнають впливу радіоелектронного придушення. Цим 
умовам відповідає так звана інерціальна навігація. Використання інтегрованих 
інерціально-супутникових систем обумовлюється наступним: інерціальна і 
супутникова навігаційні системи вимірюють різні параметри: СНС - лінійні 
параметри (вектор положення ЛА в деякій геоцентричній системі координат і 
вектор його швидкості), а ІНС - як лінійні, так і кутові параметри. 

Взагалі, СНС можна використовувати і для виміру кутових координат, але для 
цього необхідне використання декількох антен, установлених на визначеній 
відстані один від одного, і декількох приймачів, що різко ускладнюють й 
підвищують собівартість системи. Проте, використання корегованої від СНС, 
наприклад, за допомогою фільтра Калмана, ІНС дозволяє вимірювати кутове 
положення ЛА з досить малою похибкою. До того ж, ІНС дозволить екстраполювати 
сигнали СНС при значному періоді квантування сигналів. 

Використання інтегрованих інерціально-супутникових систем навігації (ІССН) 
компенсує недоліки окремих систем, і забезпечує високу точність і надійність 
виміру параметрів польоту. Це підтверджує необхідність включення до складу 
навігаційного забезпечення ЛА комплексної інерціально-супутникової системи 
навігації, а також,  розробки та дослідження працездатності алгоритмів її 
роботи, ступінь впливу похибок датчиків первинної інформації  безплатформної 
інерціальної системи (БІНС \nomenclature{БІНС}{безплатформенна інерціальна навігаційна система}) 
та супутникової навігаційної системи (СНС) на 
точнісні характеристики числення навігаційних параметрів і динаміку зміни 
похибок, впливу перерв у роботі СНС на траєкторний рух ЛА при польоті за 
складним маршрутом.

Саме тому тема  роботи є досить актуальною на сьогоднішній час.
%% Chapter 2
\newpage
\ESKDthisStyle{formII}
\section{Аналіз та вибір навігаційного забезпечення}

Задача створення комплексної навігаційної системи на базі супутникової та інерціальної 
систем навігації для визначення координат місцеположення рухомого об'єкта, передбачає 
попередній аналіз існуючих варіантів компонентів комплексної навігаційної системи, тобто 
варіантів побудови супутникової й  інерціальної систем навігації та вибір за певними критеріями найбільш оптимальних. 


\subsection{Аналіз і вибір варіанта супутникової навігаційної системи }

На сьогодні має сенс розглядати лише дві супутникові навігаційні системи : GPS (Global Positioning System), 
ГЛОНАСС (Глобальна Навігаційна Супутникова Система).

Двадцять чотири супутники системи GPS знаходяться на 12-годинних орбітах висотою 
20146 км із нахиленням орбіти, рівним 55. Таким чином, 
у будь-якій крапці земної кулі в межах прямої видимості мається не менш чотирьох супутників 
у конфігурації, сприятливої для місцевизначення.

Система заснована на обчисленні відстані від користувача до супутника за обмірюваним часом 
від передачі сигналу супутником до прийому цього сигналу користувачем.

Глобальна Навігаційна Супутникова Система (ГЛОНАСС) -- це технології російських конструкторів і вчених.
Вона складається 
з 24 супутників, що, знаходячись у заданих крапках на високих орбітах, безупинно випромінюють 
убік Землі спеціальні навігаційні сигнали. Люба людина або транспортний засіб, оснащені 
спеціальним приладом для прийому й обробки цих сигналів, можуть з високою точністю в 
будь-якій крапці Землі і навколоземного простору визначити власні координати і швидкість 
руху, а також здійснити прив'язку до точного часу.

У складі сучасної супутникової радіонавігаційної системи (СРНС) типу ГЛОНАСС і 
GPS функціонують три основні підсистеми:

\begin{enumerate}

\itemПідсистема космічних апаратів (ПКА), що складається з навігаційних супутників (НС) 
(мережа навігаційних супутників - космічний сегмент). ПКА СРНС складається з визначеного 
числа навігаційних супутників. Основні функції НС --- формування і випромінювання 
радіосигналів, необхідних для навігаційних визначень споживачів СРНС, контролю бортових 
систем супутника підсистемою контролю і керування СРНС. Відповідні характеристики сигналів 
НС і способи їхньої обробки дозволяють проводити навігаційні виміри з високою точністю.

 \item Підсистема контролю і керування (ПКК) (наземний командно-вимірювальний комплекс (КВК)) - 
сегмент керування. ПКК являє собою комплекс наземних засобів (КВК), що забезпечують 
спостереження і контроль за траєкторіями руху НС, якістю функціонування їхньої апаратури, 
керування режимами її роботи і параметрами супутникових радіосигналів, складом, обсягом і 
дискретністю переданої із супутників навігаційної інформації та ін.

 \itemАпаратура споживачів (АС) СРНС (прийомоіндикатори (ПІ)) - сегмент споживачів.
Апаратура споживачів призначена для визначення просторових координат, вектора швидкості, 
часу й інших навігаційних параметрів у результаті прийому й обробки радіосигналів багатьох 
навігаційних супутників (НС).

\end{enumerate}
На вхід ПІ надходять сигнали від НС, що знаходяться в зоні радіо видимості. Оскільки для 
рішення навігаційної задачі необхідно вимірити псевдодальності і псевдошвидкості відносно, 
як мінімум, чотирьох НС, то ПІ повинний бути багатоканальним (більш 24 у сполучених ГЛОНАСС і GPS ).

Сучасні ПІ є аналого-цифровими системами, що здійснюють аналогову і цифрову обробку 
сигналів. Перехід на цифрову обробку здійснюється на одній із проміжних частот, при 
цьому має місце тенденція до підвищення цієї проміжної частоти.

Основа типового варіанту ПІ -- два конструктивно роздільних блоків: антенний блок (АБ) та 
прийомообчислювач (ПО), які призначені для прийому й обробки навігаційних сигналів 
супутників з метою визначення необхідної споживачам інформації (просторово-тимчасових 
координат, напрямки і швидкості і т.п.).

В антенному блоці (рис. \ref{fig:ant_sns}) сукупність сигналів НС, прийнятих антеною, попередньо 
підсилюється і фільтрується по всій смузі несучих частот НС у попередньому підсилювачі 
(ПП) зі смуговим фільтром (СФ). 
\begin{figure}[here]
\centering
\includegraphics[scale=0.4]{ant_sns}
\caption{Схема антенного блоку СНС}
\label{fig:ant_sns}
\end{figure} 

Прийомообчислювач виконаний у вигляді блоку, у якому розташовані модулі вторинних 
джерел живлення і плати --- прийомокорелятора, навігаційного обчислювача та інтерфейсного 
пристрою (рис. \ref{fig:sns}). Вхід ПО через фідерну лінію з'єднаний з виходом антенного блоку. 
В аналоговому приймачі АП сигнали підсилюються, фільтруються і переносяться з несучої 
частоти на проміжну (зниження частоти). В аналого-цифровому перетворювачі АЦП аналоговий 
сигнал перетвориться в цифрову форму.
\begin{figure}[here]
\centering
\includegraphics[scale=0.9]{sns}
\caption{Схема прийомообчислювача}
\label{fig:sns}
\end{figure} 
В кореляторі (КОР) у цифровій формі формуються синфазні  і квадратурні  відліки, що є 
основою роботи алгоритмів пошуку сигналів по затримці і частоті спостереження за псевдодальністю, 
фазою сигналу і виділення навігаційного повідомлення.

Навігаційний обчислювач НО є цифровим процесором, у якому реалізується обчислювальний процес 
і керування роботою ПІ. Навігаційний обчислювач зручно представити у виді сигнального процесора 
СП, що реалізує алгоритми первинної обробки квадратурних складових, і навігаційного процесора 
НП, що реалізує алгоритми низькочастотної обробки, тобто рішення навігаційної задачі.

У прийнятого радіосигналу виміряються затримка $\tau$ або доплерівський зсув частоти $f_{\text{доп}}$, 
які є радіонавігаційними параметрами, а відповідні їм дальність до об'єкта $D=c*\tau$  
і радіальна швидкість зближення $V_{p}=f_{\text{доп}}\lambda$   служать навігаційними параметрами 
(\textit{с } -- швидкість світла;$\lambda$ -- довжина хвилі радіосигналу).

Просторове положення споживача визначається в прийомоіндикаторі в два етапи: спочатку визначаються 
поточні координати супутників і первинні навігаційні параметри (дальність, її похідні й ін.) щодо 
відповідних НС, а потім розраховуються вторинні --- географічна широта, довгота, висота споживача і т.д.

Вектор швидкості споживача обчислюють шляхом обробки результатів вимірів доплерівських зсувів 
частоти сигналів НС з урахуванням відомого вектора швидкості супутника. 

Інтерфейсний пристрій (ІП) призначений для забезпечення взаємодії прийомоіндикатора з зовнішніми 
пристроями такими, наприклад, як пульт керування й індикації (ПКІ). Додатково до складу ІП входять 
два підсилювачі П, що формують ознаку відмови ПІ і сигнали дискретного керування, а також 8-розрядний 
регістр Рг, що приймає сигнали дискретного керування. Цей регістр доступний для читання з боку НО. 
Останній, у залежності від інформації, що знаходиться в регістрі, вибирає той або інший режим роботи.

Таким чином, основною операцією, що виконуваної в СНС за допомогою космічного сегменту, сегменту 
керування та сегменту споживача, є визначення просторових координат місця розташування споживачів і 
часу, тобто просторово-тимчасових координат (ПТК). Як було показано, цю операцію здійснюють відповідно 
до концепції незалежної навігації, що передбачає обчислення шуканих навігаційних параметрів 
безпосередньо в апаратурі споживача. У рамках цієї концепції в СРНС обраний позиційний спосіб 
визначення місця розташування споживачів на основі беззапитних (пасивних) далекомірних вимірів по 
сигналах декількох навігаційних штучних супутників Землі з відомими координатами. Висока точність 
визначення місця розташування споживачів обумовлена багатьма факторами, включаючи взаємне розташування 
супутників і параметри їхніх навігаційних сигналів. Структура космічного сегмента забезпечує для 
споживача постійну видимість необхідного числа супутників.

Використання СНС в інтересах місцезнаходження і навігації рухливих об'єктів, а також у рішенні 
спеціальних задач (спостереження, аерофотознімання, пошук корисних копалин, пошук і порятунок 
транспортних засобів, що терплять нещастя, і людей) висуває високі вимоги.

Вимоги до точнісних характеристик, таких як середньоквадратичне відхилення помилки (СКП) визначення 
навігаційних параметрів, показників надійності навігаційного забезпечення, тощо наступні:
\begin{itemize}
  \itemдоступність (готовність),  мірою якої є імовірність працездатності СРНС перед виконанням 
тієї або іншої задачі та у процесі її виконання. Чисельні значення доступності складають 0,95...\dots 0,997;
 \itemцілісність, мірою якої є імовірність виявлення відмови протягом часу, рівному заданому 
або менше. Вимоги до цілісності для маршрутних польотів складає 0,999;
 \itemбезперервність обслуговування, мірою якої служить імовірність працездатності системи 
протягом найбільш відповідальних відрізків часу. На етапах заходу на посадку вимоги до безперервності 
обслуговування складають $10^{-5}$ \dots ... $10^{-4}$ для проміжків часу від 15 до 150 с.
\end{itemize}

Основні навігаційні параметри, що визначаються в СРНС -- дальність і радіальна швидкість. Відповідними 
їм радіонавігаційними параметрами (параметрами радіосигналу) служать затримка t сигналу і доплерівський 
зсув частоти $f_\text{доп}$. Оскільки головною вимогою до СРНС є висока точність виміру 
навігаційних параметрів, отже, й основною вимогою до радіосигналів так само є висока точність 
виміру затримки t сигналу і доплерівського зсуву частоти $f_\text{доп}$.

Вимоги до підвищення точності затримки сигналу і доплерівського зсуву частоти суперечливі. 
Для підвищення точності виміру затримки необхідно розширювати спектр сигналу, а для підвищення 
точності виміру  доплерівського зсуву частоти --  збільшувати тривалість сигналу.

Дане протиріччя вирішується при вирішенні задачі спільної оцінки t та  $f_\text{доп}$.

Підвищення точності спільних оцінок затримки сигналу і доплерівського зсуву частоти можна 
досягти за рахунок збільшення так званої  бази сигналу -- \textit{В}(добуток ефективної 
тривалості сигналу на ефективну ширину спектра сигналу) і основною вимогою до радіосигналів у 
СРНС є збільшення бази сигналу $В>>1$. Такі сигнали називають шумоподібними. 
Відомо, що стійкість до перешкод радіотехнічної системи визначається значенням бази сигналу, 
а для більшості ЛА скритність і перешкодозахищеність є одним з визначальних вимог. 

Інша істотна вимога --- забезпечення багатостанційного доступу. При визначенні навігаційних 
параметрів у споживача повинна бути можливість одночасного доступу до сигналів від різних 
супутників. Проблема багатостанційного доступу вирішується шляхом тимчасового, частотного 
або кодового поділу сигналів, наприклад, у супутниковій навігаційній системі GPS використовується 
кодовий поділ, у СРНС ГЛОНАСС - частотний.

З результатів аналізів стає очевидно, що не має принципової різниці між супутниковими 
навігаційними системами GPS та ГЛОНАСС.

В залежності від області використання апаратура споживача (АС) має свої особливості, 
тому виробники АС завжди вказують на область застосування відповідного зразка. Крім 
основних блоків, таких, як антена, приймач, індикатор, АС може містити допоміжні, що 
забезпечують виконання спеціальних сервісних функцій, наприклад, діагностику вузлів 
транспортного засобу, зв'язок з диспетчерським пунктом і т.п.

В табл. \ref{tb:ac} наведені коротка інформація про основні зразки АС, що працюють за сигналами 
СРНС ГЛОНАСС та GPS. Наведена інформація не претендує на повноту відомостей як про існуючі 
зразки АС, так і про іх характеристики, а дається для ілюстрації досягнутого рівня 
в розробці та виробництві АС СРНС.
Апаратура споживачів
\begin{table}[here]
\small
\caption{Апаратура споживачів}
\centering
\begin{tabular}{|p{30mm}|p{20mm}|p{20mm}|p{20mm}|p{20mm}|p{20mm}|p{10mm}|} \hline 
Найменування апаратури & Область використання & Виробник & Число каналів & 
\multicolumn{2}{|p{30mm}|}{Точність (в автономному режимі)} & Маса, кг \\ \hline 
 &  &  &  & координат, м & швидкості, м/с &  \\ \hline 
Станція моніторингу та формування ДП & Моніторинг & РНИИ КЛ & 24 & 1...3 & 1...2 & 6,0 \\ \hline 
„Гном-М'' & Авіація &  & 6...12 & 80...90 & 12...15 & 3,2 \\ \hline 
АСН-22 & Авіація & РИРВ & 18 & 25...30 &  & 0,4 \\ \hline 
НАВИС СН 3301 & Авіація &  & 14 & 15...20 & 8...10 & 2,4 \\ \hline 
„Интер-А'' & Авіація & МКБ КОМПАС & 12 & 25...30 & 10...30 & 3,5 \\ \hline 
А-744 & Авіація & Фирма „Кодтик'' & 6 & 30...35 & 15...20 & 2,0 \\ \hline 
\end{tabular}
\label{tb:ac}
\end{table}

З огляду на, те що  супутникова система навігації буде працювати в комплексі з 
інерціальною системою навігації, то навряд варто встановлювати  на борт ЛА повний 
комплект супутникової системи. Досить обмежитися  прийомоіндикатором і сигнальним 
процесором, думаючи, що алгоритми рішення навігаційної задачі будуть вирішуватися 
в спільному процесорі інерціально - супутникової системи навігації. 

Виходячи з вищенаведеного, а також враховуючи умови застосування ЛА та вимоги 
ТЗ можна сформулювати вимоги, яким повинний задовольняти обраний тип прийомоіндикатора 
СРНС. 

Розв'язувані задачі:
\begin{itemize}
\item автоматичне, безперервне, глобальне, всепогодне визначення поточних ЗD-координат 
місця розташування, вектора шляхової швидкості шляхового кута ЛА при роботі: по 
сигналу стандартної точності частотного діапазону L1 ГЛОНАСС; по сигналі З/А-коду 
GPS; при спільній обробці вищевказаних сигналів;
\item видача поточних ЗD-координат місця розташування ЛА, що є складовими вектора 
швидкості і шляхового кута в системі координат СК-42 або ПЗ-90 у географічному 
форматі, а також ознак режиму роботи апаратури;
\item стійке визначення навігаційних параметрів при русі з лінійними прискореннями 
і при стрибкоподібних змінах прискорення;
\item  можливість переключення з антени носія на антену ЛА; 
\item інтегральна оцінка очікуваної точності визначення поточних координат місця розташування;
\item автоматичний вибір оптимального з погляду очікуваної точності сузір'я НС ГЛОНАСС і GPS при роботі в сполученому режимі;
\item автоматичне рішення навігаційної задачі в географічній системі координат:  
\end{itemize}

\vspace{5mm}
\textbf{Джерела похибок СНС} \\ 
Визначення координат вимагає точний час, позицію супутників і затримки вимірів 
отриманого сигналу. Точність позиціонування переважно залежить від координат 
супутників і затримки сигналу.Загальним недоліком любої СНС є те, що сигнал при деяких умовах може не доходити 
до приймача, або приходити із значними затримками та спотвореннями. Далі розглянуто 
основні джерела похибок СНС.

\vspace{5mm}
\textit{Вибіркова доступність} 

Суттєвим недоліком є повна залежність умов отримання сигналу від міністерства 
оборони США у випадку GPS, методом додавання похибки елалону часу супутниками, що 
впливає на визначення координат для не авторизованих користувачів. В травні 2000  
року таке обмеження було знято, але немає гарантії, що це не станеться знову. 
Так, наприклад, під час бойових дій в Іраці, весь цивільний сектор був відключений.

\vspace{5mm}
\textit{Атмосферні явища}

Атмосферні ефекти представляються наступними помилками. Тропосфера знаходиться 
на висоті від 6 до 18 км. Вона електрично нейтральна і недисперсна для частот 
до 15 ГГЦ \cite{gps1,gps2}. Але через наявність водяного пару, атмосферної температури 
та тиску, спричиняє затримки.

Іоносфера знаходиться на вистоті від 50 до 1500 км і включає велику кількість  
вільних електронів і позитивно заряджених іонів. Це створює групову затримку 
сигналу, а також рефракційні та дифракційні ефекти[10]. Іоносферна 
активність значно залежить від кількості плям на Сонці. Використання деяких 
моделей та DGPS може значно поліпшити визначення координат.

\vspace{5mm}
\textit{Помилки ефемерид та еталону часу}

Інше джерело похибок -- це неточність визначення ефемерид. Хоча 
ефемериди і передаються кожні 30 секунд, сама інформація може бути вже 2 
години як застарілою. 

Атомні годинники в супутниках мають бути синхронізовані з часом 
всієї системи. Найменші відхилення моніторяться спеціальними станціями і 
помилка передається як коефіцієнти поліному другого порядку. Більші помилки 
утворються в приймачах і варіюється від мікро- до мілі- секунд.

\vspace{5mm}
\textit{Ефекти відбивання}

Сигнали СНС може спотворюватись ефектами не прямолінійності траекторії проходження  
сигналу, де радіосигнал відбивається від навколишнього ландшафту, будинків 
гірської поверхні. Ці затримки сигналу впливають на виміри псевдодальності та фази.

\begin{table}[here]
\small
\caption{Типові похибки, що впливають на вимірювання псевдодальностей}
\centering
\begin{tabular}{|p{100mm} p{20mm} p{20mm}|} \hline 
Тип похибки& & типові значення \\ \hline 
Похибка ефемерид & $\delta r_{orb} $ & 2.1 м \\ %\hline 
Похибка еталону часу супутника & $c \cdot \delta t_{s}$ & 2.1 м \\ %\hline 
Похибка еталону часу приймача  & $c \cdot \delta t_{r}$ & 0.5 м \\ %\hline 
Вибіркова доступність (задіяно/ незадіяно) & $\delta r_{sa}$ & 25/0 м \\ %\hline 
Іоносферні затримки & $\delta r_{ion} $  & 4 м \\ %\hline 
Тропосферні затримки & $\delta r_{trop} $  & 0.7 м \\ %\hline 
Похибки відбивання сигналу & $\delta r_{MP} $  & 1.4 м \\ %\hline 
Шум приймача & \textit{v} & 0.5 м \\ \hline 
Похибка обладнання користувача & $\sigma_{uere}$ & 5.3 м \\ \hline 
\end{tabular}
\label{tb:sns_main_errors}
\end{table}
\vspace{5mm}
\textit{Затримки сигналу}

Для виміру затримки, приймач порівнює послідовність бітів, отриманих з супутника, 
з генерованою версією. Через порівняння наростання і спадання імпульсів, сучасна 
електроніка може визначати зміщення сигналу імпульсу кожного біта в межах одного 
відсотку, або приблизно 10 нс для С/А коду. Так як сигнал СНС розповсюджується із 
швидкістю світла, виникає помилка приблизно 3м.
Точність може бути покращена приблизно в 10 разів, за рахунок викорисання 
більш високочастоного сисгналу, помилка зменшуєтья приблизно до 0.3 м.

Значення типових похибок зведено до таблиці \ref{sns_main_errors}. 

\vspace{5mm}
\textit{Зниження точності (DOP)}

DOP -  зниження точності (англ. Dilution of precision, DOP)
\nomenclature{DOP}{зниження точності (англ. Dilution of precision, DOP)} - термін, 
що використовується в області систем глобального позиціонування для 
параметричного опису геометричного розташування супутників щодо антени 
приймача. Коли супутники в області видимості знаходяться дуже близько 
один до одного говорять про «слабку» геометрії розташування (високе значення DOP), 
і, навпаки, при достатній віддаленості геометрію вважають «сильною» (низьке значення DOP). 

Фактори, що впливають на геометричне зниження точності.
\begin{itemize}
 \item орбіти супутників;
 \item присутність обєктів перешкод, що затіняють необхідну область неба;
 \item вплив атмосфери;
 \item відбивання радіохвиль.
\end{itemize}

Поимилки псевдодальностей $ \delta\rho $ може бути отримана з позиційних помилок 
 та помилок еталону часу $\delta e =[\delta x, \delta y, \delta z, c\cdot p\delta t]^T $
 за допомогою лінеаризованого рівняння:
\begin{equation}
\label{eq:sns_dop_err}
\delta \rho = H \delta e + \delta \epsilon_{\rho}
\end{equation}   
де H (m$\times$ 4) матриця частинних похідни по відповідним 4м змінним.
$\delta\epsilon_\rho$ білий шум з нульовим математичним сподіванням. Приймаючи
багато спрощень \cite{gps4}, можливо отримати оцінку вектора $\delta\hat{\epsilon_\rho}$ і
$E[\delta\hat{\epsilon_\rho} \delta\hat{\epsilon_\rho}^T] = \sigma_\epsilon^2(HH^T)^-1$.
Перемоноження матриць та інверсію можна представити наступним чином:  
\begin{equation}
\label{eq:sns_dop_matrix}
(H^{T}H)^{-1} = \left[
\begin{array}{cccc}
D_{11} & D_{12} & D_{13} & D_{14} \\
D_{21} & D_{22} & D_{23} & D_{24} \\
D_{31} & D_{32} & D_{33} & D_{34} \\
D_{41} & D_{42} & D_{43} & D_{44} \\
\end{array}\right]
\end{equation}   
де $D_{i,j}$ -- зображує масштаб для дисперсії $\epsilon^{2}_{e}$. Так як діагональні
елементи матриці показують похибки вимірювання для визначених координат та часу, можна знайти наступні
параметри:
Основні параметри:
\begin{itemize}
 \item $HDOP = \sqrt{D_{11}+D_{22}}$ (Horizontal Dilution of Precision) -- зниження точності в горизонтальній площині;
 \item $VDOP = \sqrt{D_{33}}$ (Vertical) -- зниження точності у вертикальній площині;
 \item $TDOP = \sqrt{D_{44}}$ (Time) -- зниження точності за часом;
 \item $PDOP = \sqrt{D_{11}+D_{22}+D_{33}}$ (Position) -- зниження точності за місцем розташування, більш комплексний показник;
 \item $GDOP = \sqrt{D_{11}+D_{22}+D_{33}+D_{44}}$ (Geometric) -- геометричне зниження точності.
\end{itemize}

Загальну позиційну, вертикльану помилку чи зниження точності еталону часу може бути оцінене як 
добуток СКВ $\sigma_{e}$ з бажаним DOP параметром.


%% Chapter 3
\newpage
\ESKDthisStyle{formII}
\section{Розробка автоматизованої системи діагностування}

Для розробки автоматизованої системи діагностування був використаний підхід до оптимізації параметрів за допомогою діаграм стійкості. Методика отримання таких діаграм в загальному вигляді описана в попередньому розділі. Тим не менше, дані методи не можуть бути використані для чисельного моделювання та дослідження. З цією метою для отримання діаграми стійкості САУ був розроблений чисельний ітеративний метод виділення країв. Для отримання ж конкретних кількісних і якісних характеристик САУ були використані вже існуючі алгебраїчні та частотні алгоритми дослідження.

Нижче наведені методи, що використані в ході розробки програмного забезпечення 
автоматизованої системи діагностування САУ ПС. Дані методи розбиті на підгрупи 
за своїм призначенням:

\begin{enumerate}
 \item Матричні методи
  \subitem Швидке обчислення визначників;
  \subitem Обчислення власних чисел матриці;
  \subitem Методи поліноміальної арифметики;
 \item Конверсійні методи
  \subitem Метод формування матриці Гурвіца-Раута;
  \subitem Метод перетворення передатної функції до форми Коші;
 \item Методи обробки даних
  \subitem Метод найменших квадратів;
  \subitem Методи пошуку екстремуму;
 \item Методи роботи з передатними функціями
 \subitem спрощення складних передатних функцій;
 \subitem приведення поліномів передатної функції до нормальної форми;
 \subitem метод побудови ЛАЧХ;
 \subitem метод побудови АФЧХ;
 \subitem критерій стійкості Гурвіца – Раута;
 \subitem метод побудови діаграми стійкості.
\end{enumerate}

Оскільки більшість алгоритмів, що наведені вище, носять прикладний характер, слід розглянути ті з них, які прямо впливають на ефективність роботи програмного забезпечення.

\subsection{Розробка алгоритмів алгебраїчного дослідження стійкості САК}
\subsubsection{Критерій стійкості Гурвіца-Раута}

<<Для стійкості системи n-ого порядку необхідно і достатньо, щоб n визначників, 
складених з коефіцієнтів характеристичного рівняння 

$$A(p) = a_n p^n + a_{n-1} p^{n-1} + \ldots + a_2 p^2 + a_1 p + a_0 = 0$$ 

були додатніми.>>

При цьому визначники беруться як головні мінори матриці вигляду:
\[
\begin{array}{ccccccccc}
a_{n-1} & a_{n-3} & a_{n-5} & a_{n-7} & \cdots & 0 & 0 & 0 & 0\\
a_{n} & a_{n-2} & a_{n-4} & a_{n-6} & \cdots & 0 & 0 & 0 & 0\\
0 & a_{n-1} & a_{n-3} & a_{n-5} & \cdots & 0 & 0 & 0 & 0\\
0 & a_{n} & a_{n-2} & a_{n-4} & \cdots & 0 & 0 & 0 & 0\\
\vdots & \vdots & \vdots & \vdots & \ddots & \vdots & \vdots & \vdots & \vdots\\
0 & 0 & 0 & 0 & \cdots & a_{3} & a_{1} & 0 & 0\\
0 & 0 & 0 & 0 & \cdots & a_{4} & a_{2} & a_{0} & 0\\
0 & 0 & 0 & 0 & \cdots & a_{5} & a_{3} & a_{1} & 0\\
0 & 0 & 0 & 0 & \cdots & a_{6} & a_{4} & a_{2} & a_{0}\end{array}\]


%%  Chapter 4
\newpage
\ESKDthisStyle{formII}
\section{Аналіз та вибір схем оцінюванн та корекції в комплексній інерціально-супутниковій системі}

Основними задачами пілотажно-навігаційних комплексів (ПНК) як постачальника 
інформаційного забезпечення польоту ЛА є сумісна обробка навігаційної інформації, 
яка надходить на борт ЛА та забезпечення високої надійності функціонування бортових 
систем та комплексів ЛА і взагалі безпеки польоту за рахунок резервування 
джерел інформації. Висока ефективність використання інформації, яка 
надходить на борт ЛА, забезпечується застосуванням різних методів її обробки. 

Найкращі результати підвищення якісних характеристик вимірювальних комплексів 
досягаються  в системах зі структурною надмірністю, коли існує можливість 
отримання пілотажно-навігаційної інформації паралельно декількома способами з 
використанням інформації від приладів та вимірювальних систем, що входять до 
складу ПНК. Отримана таким чином інформація комплексується.

В існуючих ПНК широке розповсюдження знайшли такі способи сумісної обробки 
інформації, що надходять від декількох вимірників, як взаємна компенсація і 
фільтрація похибок вимірювальних приладів, що вимірюють один і той самий 
навігаційний параметр та оптимальне оцінювання вектора стану з використанням 
апріорної інформації про контрольований процес та поточні вимірювання.

Методи оптимальної обробки інформації в ПНК використовуються з метою 
отримання оцінок вектора стану повітряного судна (або деякої частини 
цього вектора) в умовах впливу випадкових збурень і завад на процес 
вимірювання. При цьому оцінюються не самі параметри польоту, а їхні похибки. 
За оптимальної обробки пілотажно - навігаційної інформації в ПНК найважливішим 
процесом є процес отримання оптимальних оцінок. В основу алгоритмів отримання 
оптимальних оцінок можуть бути покладені такі методи обробки інформації:
\begin{itemize}
 \item метод найменших квадратів;
 \item метод максимуму правдоподібності;
 \item рекурентний неоптимальний фільтр;
 \item оптимальний фільтр Калмена.
\end{itemize}




%\textbf{11. 3. Методи оптимальної обробки інформації }

 Методи оптимальної обробки інформації в навігаційних комплексах використовуються  
з метою отримання оцінок вектора стану ПС (або деякої частини цього вектора) в умовах 
впливу випадкових збурень і завад на процес вимірювання. При цьому оцінюються не 
самі параметри польоту, а  їхні похибки.

Нехай вектор стану динамічної системи описується векторно-матричним рівнянням                            
\begin{equation}
\label{eq:__11_3_}
\dot{X}(t)=A(t)X(t)+B(t)V_{x}(t)
\end{equation}
\begin{ESKDexplanation}                
\item де $\dot{X}\left(t\right)$ -- \textit{n}-мірний вектор стану системи; 
\item \textbf{A(t)} -- квадратна матриця розмірності $n\times n$, яка являє собою матрицю коефіцієнтів 
системи; 
\item $V_{x} \left(t\right)$ -- \textit{k}-мірний вектор збурень, які діють 
на вході динамічної системи; 
\item \textbf{B}(\textit{t}) -- матриця збурень.
\end{ESKDexplanation}
Будемо вважати, що компоненти вектора $V_{x} \left(t\right)$ лінійно зв'язані з випадковими 
функціями типу білого шуму, мають нульові математичні сподівання  $M[V_{x}(t)]=0$ 
і характеризуються кореляційною матрицею $R_{x} \left(t\right) = M[V_{x}(t),V_{x}(t)^{T}]$. 

Отже,
\[M[V_{x}(t)]=0 \] 
\[R_{x} \left(t\right) = M[V_{x}(t),V_{x}(t)^{T}] \]
\begin{ESKDexplanation}
 \item де \textit{М} - символ математичного сподівання; 
 \item $\delta (t -\tau)$ -- дельта-функція.
\end{ESKDexplanation}

З  вектором стану системи $X(t)$ співвідношенням
\begin{equation} 
\label{eq:__11_4_} Y(t)=H(t)X(t) 
\end{equation} 

зв'язаний вектор спостережень \textbf{Y(t)} розмірності \textit{m}.
У рівнянні \eqref{eq:__11_4_} \textbf{H(t)} -- матриця зв'язку (матриця 
спостереження). Необхідною умовою оптимального оцінювання є повна спостережливість 
вектора стану \textbf{Х(t)} за вектором спостереження \textbf{Y(t)}.

Вважається, що процес \textbf{Х(t)} цілком спостережний на інтервалі $t \epsilon [t_{1},t_{2}]$,
 якщо за значенням вектора \textbf{Y(t)} при$t \epsilon [t_{1},t_{2}]$ 
можна вичислити значення вектора \textbf{Х(t)} при$t \epsilon [t_{1},t_{2}]$].
Умови повної спостережливості виконуються при \textit{m}$\leq$\textit{n}. Оскільки 
компоненти вектора \textbf{Y(t)} вимірюються з похибками, то як наслідок 
вимірювання отримують новий вектор, так званий вектор вимірювання  
\[ Z(t) = Y(t) + v_{z}(t),\]
де $V_{z}(t)$ --  вектор 
похибок вимірювання (припускається, що компоненти вектора $V_{z}(t)$ 
можна подати у вигляді білого шуму з нульовим математичним сподіванням). Вектор \textbf{V}\textit{z}(\textit{t}), 
аналогічно вектору \textbf{V}\textit{x}(\textit{t})\textit{,} характеризується кореляційною 
матрицею \textbf{R}\textit{z}(\textit{t}) розмірності \textit{m }$\times$\textit{ n}, 
тобто 

\textit{М }[\textbf{V}\textit{z}(\textit{t})]\textit{ =}0; \textit{}

\textit{М }[(\textbf{V}\textit{z}(\textit{t})\textit{,$V_{z}^{{\rm T}} (t)$}]\textit{ = }\textbf{R}\textit{z}(\textit{t})\textbf{$\delta$}(\textit{t} \textit{--} \textit{$\tau$}) \textit{.}

Припускаючи, 
що компоненти векторів \textbf{V}\textit{x}(\textit{t})\textit{ }і \textbf{V}\textit{z}(\textit{t}) 
некорельовані, на підставі викладеного математичну модель динамічної системи та рівняння 
спостереження можна записати у вигляді 

\begin{equation} \label{eq:__11_5_} \begin{array}{l} {\dot{{\rm X} }(t)=A(t)X(t)+B(t)V_{x} 
(t);} \\ {Z(t)=H(t)X(t)+V_{z} (t).} \end{array} \end{equation} 

При оптимальній обробці інформації в навігаційних комплексах (НК) найбільш важливим 
процесом є процес отримання оптимальних оцінок $\hat{{\rm X} }\left(t\right)$. В 
основу алгоритмів отримання оптимальних оцінок можуть бути покладені такі методи 
обробки інформації:

\begin{enumerate}
\item - -метод найменших квадратів (МНК);

\item - -метод максимуму правдоподібності;

\item - -рекурентний метод (оптимальний фільтр  Калмана ).
\end{enumerate}















%% Chapter 5
\newpage
\ESKDthisStyle{formII}
\section{Розробка алгоримів оптимального комплексування в інерціально-супутникових
систем навігації}

Загальною вимогою для організації процесу комплексування є наявність математичних 
моделей підсистем, що підлягають комплексуванню. Сучасний стан обчислювальної техніки, 
знань в області інерціальної та супутникової навігації дозволяють скласти досить 
повні й адекватні моделі цих систем. У комплексі системи описуються на рівні їхніх 
похибок. Таким чином, для нормальної роботи комплексу потрібний адекватний опис похибок 
підсистем, включаючи неконтрольовані джерела похибок. 

\subsection{Моделі похибок  інерціальних навігаційних систем }

Рівняння похибок БІНС описують збурений режим роботи системи і є основою при аналізі 
її точності, при організації корекції, при побудові оптимальних навігаційних алгоритмів.

Матриця переходу від зв'язаної СК до географічної  СК  $B(\psi ,\vartheta ,\gamma )$ має 
вид:
\begin{equation}
\label{eq:noname_1} 
\scriptstyle
B(\psi ,\vartheta ,\gamma )=\left(
\begin{array}{ccc} 
{\scriptstyle \sin \psi \cos \vartheta } & 
{\scriptstyle\cos \psi \sin \gamma -\sin \psi \cos \gamma \sin \vartheta } & 
{\scriptstyle\cos \psi \cos \gamma +\sin \psi \sin \gamma \sin \vartheta } \\ 
{\scriptstyle\cos \psi \cos \vartheta } & 
{\scriptstyle-\sin \psi \sin \gamma -\cos \psi \cos \gamma \sin \vartheta } & 
{\scriptstyle-\sin \psi \cos \gamma +\cos \psi \sin \gamma \sin \vartheta } \\ 
{\scriptstyle\sin \vartheta } & 
{\scriptstyle\cos \gamma \cos \vartheta } & 
{\scriptstyle-\sin \gamma \cos \vartheta } 
\end{array}\right),
\end{equation}

\begin{ESKDexplanation}
\item де $\psi \left(t\right),\vartheta \left(t\right),\gamma \left(t\right)$- кути курсу, 
тангажа та крену ЛА відповідно. 
\end{ESKDexplanation}
Матриця переходу від географічної  СК до  рухомої 
екваторіальної СК $Q\left(\varphi \right)$ має вигляд:

\[Q\left(\varphi \right)=\left(\begin{array}{ccc} {1} & {0} & {0} \\ 
{0} & {\cos\varphi } & {\sin \varphi } \\ 
{0} & {-\sin \varphi } & {\cos \varphi } \end{array}\right),\] 

де $\varphi $- географічна широта.

Матриця переходу від зв'язаної СК до рухомої екваторіальної СК $C(\psi ,\vartheta,\gamma ,\varphi)$ 
задовольняє співвідношенням виду:
\[C\left(\psi ,\vartheta ,\gamma ,\varphi \right)=
Q\left(\varphi \right)\cdot B\left(\psi ,\vartheta ,\gamma \right).\] 
При розв`язанні задач повітряної навігації як основні навігаційні параметри ЛА можна 
розглядати поточні географічні координаті ( довготу $\lambda $, широту $\varphi $ и 
висоту над поверхнею земного еліпсоїда \textit{Н}), проекції шляхової швидкості $V_{E} 
,V_{N} ,V_{h} ,$а також елементи матриці переходу $B\left(\psi ,\vartheta ,\gamma 
\right)$, що характеризує орієнтацію ЛА у просторі.

Вказані навігаційні параметри задовольняє таким диференціальним рівнянням:
\begin{equation}
\left .
\begin{array}{c} 
{\dot{\lambda }=
\frac{V_{E} \left(t\right)}{\left(R_{1}+h\right)\cos \varphi \left(t\right)} } \\ 
{\dot{\varphi }=\frac{V_{N} \left(t\right)}{\left(R_{2} +h\right)} } \\ $-$
{\dot{h}=V_{h} \left(t\right)} \end{array}\right\};
\label{eq:coordinates}
\end{equation}
\begin{equation}
\dot{B}=B\Omega_{c} -\Omega_{\Gamma}B ;               
\label{eq:dBmatrix}
\end{equation}
\begin{equation}
\dot{\bar{V}}=B\bar{a}_{c} -\Delta \bar{n}\left(t\right)+\bar{g}_{T} ,     
\label{eq:dVector}
\end{equation}
\begin{ESKDexplanation}
\item де \eqref{eq:coordinates} -- рівняння для числення географічних координат; 
\item \eqref{eq:dBmatrix} -- матричне рівняння Пуассона для визначення матриці 
направляючих косинусів $B\left(\psi ,\vartheta ,\gamma \right)$; 
\item \eqref{eq:dVector} -- векторне рівняння відновно 
проекцій шляхової швидкості ЛА 
$\bar{V}=\left(\begin{array}{ccc} {V_{E} ,} & {V_{N} 
,} & {V_{h} } \end{array}\right)^{T} $; $\bar{a}_{c} \left(t\right)=\left(\begin{array}{ccc} 
{a_{x1} \left(t\right),} & {a_{y1} \left(t\right),} & {a_{z1} \left(t\right)} \end{array}
\right)^{T} $-- вектор проекцій уявного прискорення початку зв'язаної СК на її осі;
\end{ESKDexplanation}
\[\Omega_{c} =\left(\begin{array}{ccc} 
{0} & {-\omega {}_{z1} } & {\omega {}_{y1} } \\ 
{\omega {}_{z1} } & {0} & {-\omega {}_{x1} } \\ 
{-\omega {}_{y1} } & {\omega {}_{x1}} & {0} 
\end{array}\right);\] 
\[\Omega _{\Gamma } =\left(\begin{array}{ccc} 
{0} & {-(\dot{\lambda }+u)\sin \varphi } & {(\dot{\lambda }+u)\cos \varphi } \\ 
{(\dot{\lambda}+u)\sin \varphi } & {0} & {\dot{\varphi }} \\
{-(\dot{\lambda }+u)\cos \varphi } & {-\dot{\varphi }} & {0} 
\end{array}\right);\] 
\begin{ESKDexplanation}
\item $\omega _{x1}$ ,$\omega _{y1}$ ,$\omega _{z1}$-- проекції абсолютної кутової швидкості 
зв'язаної з ЛА СК на її осі; $u$-- кутова швидкість обертання Землі; 
\item $R_{1} $ и $R_{2} $-- головні радіуси кривизни обраного земного еліпсоїда;
\end{ESKDexplanation}

\[\begin{array}{l} 
{R_{1} =a\left[1-e^{2} \sin ^{2} \varphi (t)\right]^{-\frac{1}{2}};} \\ 
{R_{2} =a\left(1-e^{2} \right)\left[1-e^{2} \sin ^{2} \varphi(t)\right]^{-\frac{3}{2}};} 
\end{array}\] 
\begin{ESKDexplanation}
\item $a$,$e$-- велика піввісь и ексцентриситет земного еліпсоїда;
\item $\bar{g}_{T} =(\begin{array}{ccc}{g_{TE},}&{g_{TN},}&{g_{Th} }\end{array})^{T} $
-- вектор проекцій прискорення сили ваги на оси географічної СК;
\item $\Delta \bar{n}=\begin{array}{ccc} {\Delta n_{E} ,} & {\Delta n_{N} ,} & {
\Delta n_{h} ,} \end{array})^{T} $-- вектор проекцій суми переносного и кориолісова 
прискорень на осі географічної СК;
\end{ESKDexplanation}
\[\begin{array}{l} 
{\Delta n_{E} =\frac{V_{E} V_{h} }{R_{1} +h} -\frac{V_{E} V_{N}}{R_{1} +h} tg\varphi +2u\left(V_{h} \cos \varphi -V_{N} \sin \varphi \right);} \\ 
{\Delta n_{N} =\frac{V_{N} V_{h} }{R_{2} +h} +\frac{V_{E}^{2} }{R_{1} +h} tg\varphi+2uV_{E} \sin \varphi ;} \\ 
{\Delta n_{h} =-\frac{V_{E}^{2} }{R_{1} +h} -\frac{V_{N}^{2}}{R_{2} +h} -2uV_{E} \cos \varphi ;} 
\end{array}\] 
\begin{ESKDexplanation}
\item $\bar{g}_{T} =\left[0,0,g_{e} \right]^{T} $-- вектор проекцій нормального 
прискорення сили ваги на осі географічної СК 
$g_{e}=\mu//a^{2}$, $\mu=398600,44\cdot 10^{9} \left[\text{м}^{3}/c^{2} \right]$
\end{ESKDexplanation}


Маючи інформацію  про вихідні координати та проекції шляхової швидкості ЛА, про вихідну 
матрицю орієнтації $B_{0}$ (її визначення є предметом задачі початкового виставлення  
БІНС ), а також про моделі прискорення сили ваги $g^{T}$($\varphi $, $\lambda $, \textit{h}), 
на основі рівнянь \eqref{eq:coordinates}$\div $\eqref{eq:dVector} с використанням  
поточних показів ДУС и акселерометрів можна отримати поточні значення  шуканих навігаційних 
параметрів ЛА.

При точному завдані вихідних умов и при точній  моделі прискорення сили ваги, а також 
при відсутності похибок інерціальних ДПІ и похибок обчислення в наслідок інтегрування 
рівнянь \eqref{eq:coordinates}$\div $\eqref{eq:dVector}  будуть отримані істинні 
значення основних навігаційних параметрів ЛА.

Похибки завдання вихідних координат и проекцій шляхової швидкості ЛА, похибки  початкового 
виставлення , аномальні варіації прискорення сили ваги, похибки інерціальних ДПІ, 
методичні похибки алгоритмів обчислення и похибки через  кінцеву довжину розрядній 
сітці обчислювача (похибки округлення) будуть приводити до похибок визначення шуканих 
навігаційних параметрів ЛА.

У лінійному наближенні еволюція похибок БІНС у визначенні основних навігаційних параметрів 
у часі може бути описана лінійними диференціальними рівняннями похибок.

Рівняння похибок БІНС у визначенні координат випливає з динамічних рівнянь числення 
координат, що наведені в алгоритмах БІНС і мають вигляд:

\begin{equation} 
\label{eq:dRsdins} 
\begin{array}{l} 
{\Delta \dot{R}_{E} =\Delta V_{E}(t)\cdot \frac{R_{\text{З}} }{R\cos \varphi (t)} 
+\Delta R_{N} (t)\frac{V_{E}^{}(t)\sin \varphi (t)}{R_{\text{З}} R\cos ^{2} \varphi (t)} 
-\Delta h(t)\frac{R_{} V_{E}^{}(t)}{R^{2} \cos \varphi (t)} ;} \\ 
{\Delta \dot{R}_{N} =\Delta V_{N}(t)\cdot \frac{R_{\text{З}}}{R} -\Delta h(t)\frac{R_{\text{З}} V_{N}(t)}{R^{2}};} \\ 
{\Delta \dot{h} =\Delta V_{h} (t);} \end{array} \end{equation} 
\begin{ESKDexplanation}
\item де $\Delta R_{E} (t)=\Delta \lambda (t)R_{{\rm }} ,\, \, \Delta R_{N} (t)=\Delta\varphi (t)R_{{\rm }} $
-- похибка БІНС у визначенні приведених координат місцезнаходженняЛА; 
\item $\Delta \lambda (t)$,$\Delta \varphi (t)$,$\Delta H(t)$-- похибки БІНС у визначенні 
географічних координат; $\Delta V_{E} (t),\Delta V_{N} (t),\Delta V_{H} (t)$-- похибки 
БІНС у визначенні проекції шляхової швидкості ЛА; 
\item $R=R_{\text{З}} +H$; $R_{\text{З}}$  -- радіус земної сфери; 
\end{ESKDexplanation}
Еволюція похибок БІНС у визначенні проекції шляхової швидкості ЛА $\Delta V_{E}^(t)$,
$\Delta V_{N}(t)$,$\Delta V_{h}(t)$, також може бути отримана з динамічних 
рівнянь числення шляхової швидкості в алгоритмах БІНС, і описується наступною системою 
рівнянь: 

\[\begin{array}{l} {\Delta \dot{V}_{E} =a_{N} \alpha _{h} -a_{h} \alpha _{N} +\sum 
_{i=1}^{3}b_{1,i}  \Delta a_{i} -\Delta V_{h} U(t)\cos \varphi +\Delta V_{N} U(t)
\sin \varphi +} \\ {+\frac{\Delta R_{N} }{R_{} } \left(U(t)(V_{h} \sin \varphi +V_{N} 
\cos \varphi \right))-(\frac{\Delta V_{E} }{R\cos \varphi } +\frac{V_{E} \sin \varphi 
}{R\cos ^{2} \varphi } \frac{\Delta R_{N} }{R_{} } )\times } \\ {\times (V_{h} \cos 
\varphi -V_{N} \sin \varphi )+\frac{\Delta hV_{E} }{R^{2} } (V_{h} -V_{N} tg\varphi 
);} \end{array}\] 

\begin{equation} \label{eq:__6_5_} \begin{array}{l} {\Delta \dot{V}_{N} =-a_{E} 
\alpha _{h} +a_{h} \alpha _{E} +\sum _{i=1}^{3}b_{2,i}  \Delta a_{i} -\Delta V_{E} 
U(t)\sin \varphi -\Delta V_{h} \dot{\varphi }(t)-} \\ {-\frac{\Delta R_{N} }{R_{} 
} V_{E} U(t)\cos \varphi -\frac{\Delta V_{N} }{R} V_{h} -(\frac{\Delta V_{E} }{R
\cos \varphi } +\frac{V_{E} \sin \varphi }{R\cos ^{2} \varphi } \frac{\Delta R_{N} 
}{R_{} } )V_{E} \sin \varphi +} \\ {+\frac{\Delta h}{R^{2} } (V_{E}^{2} tg\varphi 
+V_{N} V_{h} );} \end{array} \end{equation} 

\[\begin{array}{l} {\Delta \dot{V}_{h} =a_{E} \alpha _{N} -a_{N} \alpha _{E} +\sum 
_{i=1}^{3}b_{3,i}  \Delta a_{i} +\Delta V_{E} U(t)\cos \varphi +\Delta V_{N} \dot{
\varphi }(t)-} \\ {-\frac{\Delta R_{N} }{R_{} } V_{E} U(t)\sin \varphi +\frac{\Delta 
V_{N} }{R} V_{N} +(\frac{\Delta V_{E} }{R\cos \varphi } +\frac{V_{E} \sin \varphi 
}{R\cos ^{2} \varphi } \frac{\Delta R_{N} }{R_{} } )V_{E} \cos \varphi +} \\ {+g_{e} 
\left(-\frac{2\Delta h}{a} +\frac{3}{2} e^{2} \sin \varphi \cos \varphi \frac{\Delta 
R_{N} }{R_{} } \right)-\frac{\Delta h}{R^{2} } \left(V_{E}^{2} +V_{N}^{2} \right),} 
\end{array}\] 



де $b_{ij} \left(i,j=1,2,3\right)$ -- елементи матриці направляючих косинусів \textit{B}; $\Delta 
a_{i} \left(i=1,2,3\right)$ -- приведені похибки акселерометрів БІНС (з урахуванням 
похибок чисельного інтегрування рівняння  у бортовому обчислювачі); $a_{H} ,a_{E} 
,a_{N} $ -- поточні значення проекцій уявного прискорення початку зв'язаної СК на 
осі географічної СК; $\alpha _{H} ,\alpha _{E} ,\alpha _{N} $ -- похибки моделювання 
в БІНС орієнтації географічного координатного тригранника ($\alpha _{E} $ і $\alpha 
_{N} $-- похибки побудови вертикалі, $\alpha _{H} $-- азимутальна похибка); $R=R_{{
\rm }} +H$ -- поточна висота; 

\[U(t)=2\Omega _{{\rm }} +\dot{\lambda }(t);\, \, \dot{\varphi }(t)=\frac{V_{N} }{R} 
;\, \, \, \dot{\lambda }(t)=\frac{V_{E} }{R\cos \varphi } .\] 

  Аналіз показує, що еволюція параметрів $\alpha _{h} ,\, \, \alpha _{E} ,\, \, \alpha 
_{N} $ у часі описується наступною системою рівняннь:



\begin{equation} \label{eq:__6_6_} \begin{array}{l} {\dot{\alpha }_{E} =-\omega 
_{N} \alpha _{h} +\omega _{h} \alpha _{N} -\frac{\Delta V_{N} }{R} -\sum _{i=1}^{3}b_{1,i}  
\varepsilon _{i} ,} \\ {\dot{\alpha }_{N} =-\omega _{h} \alpha _{E} +\omega _{E} 
\alpha _{h} +\frac{\Delta V_{E} }{R} -u\sin \varphi \frac{\Delta R_{N} }{R_{7} } 
-\sum _{i=1}^{3}b_{2,i}  \varepsilon _{i} ,} \\ {\dot{\alpha }_{h} =-\omega _{E} 
\alpha _{N} +\omega _{N} \alpha _{E} +\frac{\Delta V_{E} }{R} tg\varphi +(u\cos \varphi 
+\frac{V_{E} }{R\cos ^{2} \varphi } )\frac{\Delta R_{N} }{R_{7} } -\sum _{i=1}^{3}b_{3,i}  
\varepsilon _{i} ,} \end{array} \end{equation} 

де  $\omega _{E} =-\dot{\varphi }(t),\omega _{N} =\left[u+\dot{\lambda }(t)\right]
\cos \varphi ,\omega _{h} =\left[u+\dot{\lambda }(t)\right]\sin \varphi ,$

$\dot{\lambda }=\frac{V_{E} }{R\cos \varphi } ;$ $\dot{\varphi }=\frac{V_{N} }{R} $; $\varepsilon 
_{i} (i=1,2,3)-$приведені похибкм ДУС БІНС;

Аналіз показує, що похибки моделювання географічного тригранника $\alpha _{h} $,$\alpha 
_{E} $ ,$\alpha _{N} $ зв'язані з похибками визначення координат $\Delta R_{N} $,$\Delta 
R_{} $ і похибками моделювання орієнтації рухливої екваторіальної СК $\delta _{\xi 
} $, $\delta _{\eta } $, $\delta _{\zeta } $ такими   співвідношеннями:

\[\begin{array}{l} {\alpha _{E} =\delta _{\xi } -\frac{\Delta R_{N} }{R_{{\rm }} 
} ;} \\ {\alpha _{N} =\delta _{\eta } \cos \varphi -\delta \sin \varphi +\frac{\Delta 
R_{E} }{R_{{\rm }} } \cos \varphi ;} \\ {\alpha _{h} =\delta _{\eta } \sin \varphi 
-\delta _{\zeta } \cos \varphi +\frac{\Delta R_{E} }{R_{{\rm }} } \sin \varphi .} 
\end{array}\] 

Еволюція в часі похибок моделювання рухливої екваторіальної СК $\delta _{\xi } $, $\delta 
_{\eta } $,$\delta _{\zeta } $ описується більш простими, ніж \eqref{eq:__6_6_}, 
рівняннями:

\[\begin{array}{l} {\dot{\delta }_{\xi } =-(u+\dot{\lambda })\delta _{\zeta } -\varepsilon 
_{\zeta } (t)} \\ {\dot{\delta }_{\eta } =-\varepsilon _{\eta } (t)} \\ {\dot{\delta 
}_{\zeta } =-(u+\dot{\lambda })\delta _{\xi } -\varepsilon _{\zeta } (t)} \end{array};\] 

де $\dot{
\lambda }=\frac{V_{E} (t)}{R\cos (t)} $.

Якщо ввести в розгляд  інерціальну прямокутну геоцентричну СК $\xi _{{\rm u}} \eta 
_{{\rm u}} \zeta _{{\rm u}} $, вісь $\eta _{{\rm u}} $ якої збігається з віссю $\zeta $, 
а вісь $\xi _{{\rm u}} $ у момент  \textit{t }= 0 лежить у площині Гринвіцького меридіана, 
то можна сказати, що похибки моделювання орієнтації такої СК $\delta _{\xi _{{\rm 
u}} } $,$\delta _{\eta _{{\rm u}} } $,$\delta _{\zeta _{{\rm u}} } $ зв'язані з параметрами $\delta 
_{\xi } $,$\delta _{\eta } $,$\delta _{\zeta } $ співвідношеннями виду:

\[\begin{array}{l} {\delta _{\xi } =\delta _{\xi _{{\rm u}} } Aos\lambda _{*} -\delta 
_{\xi _{u} } \sin \lambda _{*} } \\ {\delta _{\eta } =\delta _{\eta _{{\rm u}} } 
} \\ {\delta _{\zeta } =\delta _{\xi _{{\rm u}} } \sin \lambda _{*} -\delta _{\zeta 
_{{\rm u}} } Aos\lambda _{*} } \end{array};\] 

де  $\lambda _{*} =ut+\lambda (t)$ .

Рівняння, що описують еволюцію в часі похибок моделювання інерціальної СК $\delta 
_{\xi _{{\rm u}} } $,$\delta _{\eta _{{\rm u}} } $,$\delta _{\zeta _{{\rm u}} } $ виявляється 
досить простими:

\[\begin{array}{l} {\dot{\delta }_{\xi _{{\rm u}} } =-\varepsilon _{\xi _{{\rm u}} 
} (t);} \\ {\dot{\delta }_{\eta _{{\rm u}} } =-\varepsilon _{\eta _{{\rm u}} } (t);} 
\\ {\dot{\delta }_{\zeta _{{\rm u}} } =-\varepsilon _{\zeta _{{\rm u}} } (t),} \end{array}\] 

де $\left(
\begin{array}{l} {\varepsilon _{\xi _{{\rm u}} } } \\ {\varepsilon _{\eta _{{\rm 
u}} } } \\ {\varepsilon _{\zeta _{{\rm u}} } } \end{array}\right)=\Delta {\bf C}(t){
\bf C}(t)\left(\begin{array}{l} {\varepsilon _{1} } \\ {\varepsilon _{2} } \\ {\varepsilon 
_{3} } \end{array}\right)$;

$\Delta {\bf C}(t)=\left(\begin{array}{ccccc} {\cos \lambda _{*} } & {} & {-\sin 
\lambda _{*} } & {} & {0} \\ {\sin \lambda _{*} } & {} & {\cos \lambda _{*} } & {} 
& {0} \\ {0} & {} & {0} & {} & {1} \end{array}\right)$ -- матриця переходу від рухливої  
екваторіальної СК до  інерціальної СК.

Таким чином, у моделі похибок БІНС можливе використання принаймні трьох груп параметрів, 
що характеризують похибки моделювання орієнтації СК:

\{$\alpha _{E} $,$\alpha _{N} $,$\alpha _{h} $\},\{$\delta _{\xi } $,$\delta _{\eta 
} $,$\delta _{\zeta } $\}, \{$\delta _{\xi _{{\rm u}} } $,$\delta _{\eta _{{\rm u}} 
} $,$\delta _{\zeta _{{\rm u}} } $\}.

Надалі в роботі використовуються параметри $\alpha _{E} $,$\alpha _{N} $,$\alpha 
_{h} $, що характеризують похибки  моделювання географічної СК і мають найбільш наочну 
фізичну інтерпретацію. Цим параметрам відповідають рівняння еволюції \eqref{eq:__6_6_}.

Для 
замикання системи рівнянь похибок БІНС \eqref{eq:dRsdins}, \eqref{eq:__6_5_}, 
\eqref{eq:__6_6_} необхідно вказати моделі еволюції приведених похибок ДПІ. 

З 
урахуванням вигляду моделі еволюції похибок ДПІ \eqref{eq:__6_13_}, яка описується  
в п.п. 6.2, рівняння похибок БІНС \eqref{eq:dRsdins}, \eqref{eq:__6_5_}, \eqref{eq:__6_6_} 
можуть бути замкненні  наступними  рівняннями відносно $C_{\omega } $, $C_{a} $, $C_{
\varepsilon } $, $D_{a} $, $\bar{\varepsilon }_{A} $, $\Delta \bar{a}_{c} $:

\begin{equation} \label{eq:__6_9_} \begin{array}{l} {\dot{C}_{\omega }^{} =\xi 
_{A\omega } (t);} \\ {\dot{C}_{a}^{} =\xi _{Aa} (t);} \\ {\dot{C}_{\varepsilon }^{} 
=\xi _{A\varepsilon } (t);} \\ {\dot{D}_{a} =\xi _{Da} (t);} \\ {\dot{\bar{\varepsilon 
}}_{c} =\bar{\xi }_{A} (t);} \\ {\Delta \dot{\bar{a}}_{c} =\bar{\xi }_{\Delta a} 
(t),} \end{array} \end{equation} 

де $\xi _{A\omega } (t);$$\xi _{Aa} (t);$$\xi _{A\varepsilon } (t);$$\xi _{Da} (t);$$\bar{
\xi }_{A} (t);$$\bar{\xi }_{\Delta a} (t)$-- білошумні збурення відповідної розмірності, 
які характеризують дрейф квазістаціонарних  параметрів моделі ДПІ \eqref{eq:__6_13_}.

Повертаючись 
до моделей похибок БІНС відзначимо, що коли  вектор-стовпець похибок БІНС $\bar{X}(t)$ прийняти 
у вигляді:

\[\bar{X}=(\Delta R_{E} ,\Delta R_{N} ,\Delta h,\Delta V_{E} ,\Delta V_{N} ,\Delta 
V_{h} ,\alpha _{E} ,\alpha _{N} ,\alpha _{h} ,\varepsilon _{c1} ,\varepsilon _{c2} 
,\varepsilon _{c3} ,\Delta a_{c1} ,\Delta a_{c2} ,\Delta a_{c3} ,)^{T} ,\] 

то модель еволюції похибок БІНС може бути подана у компактній формі

\begin{equation} \label{eq:__6_10_} \dot{\bar{X}}=F\bar{X}\left(t\right)+G\bar{
\xi }(t), \end{equation} 

де F та G   --  матриці 15 $\times$ 15 і 15 $\times$ 21 відповідно; $\bar{\xi }(t)$ -- вектор-стовпець 
розмірності 21, компонентами якого є незалежні Гауссівські «білі» шуми з нульовими 
середніми значеннями и одиничними дисперсіями.

Відмінні від нуля елементи матриці $F$ мають вигляд:



\[f_{1,2} =\frac{\dot{\lambda }}{R_{} } tg\varphi ;f_{1,3} =\frac{-\dot{\lambda }R_{} 
}{R} ;f_{1,4} =\frac{R_{} }{R\cos \varphi } ;f_{2,3} =\frac{-\dot{\varphi }R_{} }{R} 
;f_{2,5} =\frac{R_{} }{R} ;f_{3,6} =1;\] 

\[f_{4,2} =\frac{2u+\dot{\lambda }}{R_{} } \left(V_{h} \sin \varphi +V_{N} \cos \varphi 
\right)-\frac{\dot{\lambda }}{R_{} } tg\varphi \left(V_{h} \cos \varphi -V_{N} \sin 
\varphi \right);\] 

\[f_{4,3} =\frac{V_{E} }{R^{2} } \left(V_{h} -V_{N} tg\varphi \right);f_{4,4} =\frac{V_{N} 
\sin \varphi -V_{h} \cos \varphi }{R\cos \varphi } ;\] 

\[f_{4,5} =\left(2u+\dot{\lambda }\right)\sin \varphi ;f_{4,6} =-\left(2u+\dot{\lambda 
}\right)\cos \varphi ;\] 

\[f_{4,8} =-a_{h} ;f_{4,9} =a_{N} ;f_{4,13} =b_{1,1} ;f_{4,14} =b_{1,2} ;f_{4,15} 
=b_{1,3} ;\] 

\begin{equation} \label{eq:__6_11_} f_{5,2} =-\frac{2u+\dot{\lambda }}{R_{} } 
V_{E} \cos \varphi -\frac{V_{E}^{2} }{RR_{} } tg^{2} \varphi ;f_{5,3} =\frac{V_{E}^{2} 
tg\varphi +V_{h} V_{N} }{R^{2} } ; \end{equation} 

\[f_{5,4} =-\left(2u+\dot{\lambda }\right)\sin \varphi ;f_{5,5} =-\frac{V_{h} }{R} 
;f_{5,6} =-\dot{\varphi }(t);\] 

\[f_{5,7} =a_{h} ;f_{5,9} =-a_{E} ;f_{5,13} =b_{2,1} ;f_{5,14} =b_{2,2} ;f_{5,15} 
=b_{2,3} ;\] 

\[\begin{array}{l} {f_{6,2} =-2u\frac{V_{E}^{} \sin \varphi }{R} +\frac{3g_{e} }{2R_{} 
} e^{2} \sin \varphi \cos \varphi ;f_{6,3} =-\frac{2g_{e} }{a} -\frac{V_{E}^{2} +V_{N}^{2} 
}{R^{2} } ;} \\ {f_{6,4} =\left(2u+\dot{\lambda }\right)\cos \varphi ;} \end{array}f_{6,5} 
=\dot{\varphi }(t)+\frac{V_{N} }{R} ;f_{6,7} =-a_{N} ;f_{6,8} =a_{E} ;f_{6,13} =b_{3,1} 
;f_{6,14} =b_{3,2} ;f_{6,15} =b_{3,3} ;\] 

\[f_{7,5} =-\frac{1}{R} ;f_{7,8} =\omega _{h} ;f_{7,9} =-\omega _{N} ;f_{7,10} =-b_{1,1} 
;f_{7,11} =-b_{1,2} ;f_{7,12} =-b_{1,3} ;\] 

\[\begin{array}{l} {f_{8,2} =-\frac{u}{R} \sin \varphi ;f_{8,4} =\frac{1}{R} ;f_{8,7} 
=-\omega _{h} ;f_{8,9} =\omega _{E} ;} \\ {f_{8,10} =-b_{2,1} ;f_{8,11} =-b_{2,2} 
;f_{8,12} =-b_{2,3} ;} \\ {f_{9,2} =\frac{1}{R} _{7} (u\cos \varphi +\frac{\dot{
\lambda }}{\cos \varphi } );f_{9,4} =\frac{tg\varphi }{R} ;f_{9,7} =\omega _{N} ;f_{9,8} 
=-\omega _{E} ;} \\ {f_{9,10} =-b_{3,1} ;f_{9,11} =-b_{3,2} ;f_{9,12} =-b_{3,3} .} 
\end{array}\] 

Відрізні від нуля елементи матриці \textit{G} (15$\times $21) задовольняють таким 
співвідношенням:

\begin{equation} \label{eq:__6_12_} \begin{array}{l} {g_{i,i} =\sigma _{i} ,
\, \, \, \, i=1,..,15;} \\ {g_{i+3,j+18} =b_{i,j} \sigma _{a} ,\, \, \, \, i=1,2,3,j=1,2,3;} 
\\ {g_{i+6,j+15} =-\sigma _{\omega } b_{i,j} ,\, \, \, \, \, i=1,2,3,j=1,2,3;} \end{array} \end{equation} 

де $\sigma 
_{1} \div \sigma _{15} $- середньоквадратичні значення (СКЗ) білошумних збурень, 
що характеризують вплив різних факторів ($\sigma _{1} \div \sigma _{3} $-- похибок 
численного інтегрування рівняння \eqref{eq:coordinates}; $\sigma _{4} \div \sigma 
_{6} $ --  підсумковий ефект аномалій гравітаційного поля и похибок численного інтегрування 
рівняння \eqref{eq:dVector}, $\sigma _{7} \div \sigma _{9} $-- похибок численного 
інтегрування рівняння для параметрів орієнтації \eqref{eq:dBmatrix};  $\sigma _{10} 
\div \sigma _{15} $ -- випадкового дрейфу квазістаціонарних зведених погрішностей 
ДПІ  $\bar{\varepsilon }_{A} $ и $\Delta \bar{0}_{A} $);

$\sigma _{a} $, $\sigma _{\omega } $ -- СКЗ білошумних складових погрішностей акселерометрів 
и ДКШ БІНС.

Елементи матриць \textit{F} и \textit{G, } що випливає з аналізу співвідношень \eqref{eq:__6_11_} 
и \eqref{eq:__6_12_}, залежать від поточних значень навігаційних параметрів польоту 
ЛА.

Безперервної моделі еволюції похибок БІНС  \eqref{eq:__6_10_} відповідає такий 
дискретний аналог:

\[\bar{\% }_{k+1} =\Phi _{k} \bar{\% }_{k} +G_{k} \bar{\xi }_{k} ,\] 

де $\Phi _{k} =E+F(t_{k} )\Delta t,$   $G_{k} =G(t_{k} )\cdot \Delta t;$ $\Delta 
t$--  крок дискретизації часу;

$E$ -- одинична матриця  $15\times 15$.





% \subsection{6.2  Математичні моделі похибок  МЕМC-датчиків }
% 
% Строго говорячи, кожен тип гіроскопа або акселерометра має свою модель з її характерними 
% компонентами і чисельними значеннями. Проте, можна задатися деякою узагальненою моделлю, 
% яка якісно враховує залежності похибок від того або іншого збурюючого фактора. Для 
% конкретного типу гіроскопів і акселерометрів коефіцієнти в цих моделях повинні одержати 
% відповідні чисельні значення, а частина членів, несуттєвих для приладів даного типу, 
% можуть прийняти нульові значення. Можна, однак, уявити собі й іншій ситуації, коли 
% така узагальнена модель для якогось типу приладу не буде мати істотної для нього 
% складової. У цьому випадку модель повинна бути доповнена відповідними компонентами.
% 
% Аналіз 
% характеристик ММГ показав, що в моделях похибок ММГ доцільно враховувати наступні 
% фактори:
% 
% - нестабільність масштабних коефіцієнтів;
% 
% - перекіс (неоктагональность) осей чутливості;
% 
% - систематичні складові дрейфів, які  характеризують
% 
%   зсув нулів від пуску до пуску;
% 
% -  випадкові складові дрейфів, які характеризують
% 
%     дрейф нулів у конкретному пуску;
% 
% - складові дрейфів через дію лінійних прискорень
% 
%   (перевантажень);
% 
% - флюктуационные складових дрейфів
% 
%   (шуми вихідних сигналів).
% 
% У свою чергу, у моделях похибок ММА доцільно враховувати:
% 
% - нестабільність масштабних коефіцієнтів;
% 
% - перекіс (неоктагональность) осей чутливості;
% 
% - систематичні зсуви нулів від пуску до пуску;
% 
% - випадкові дрейфи нулів у пуску;
% 
% - флюктуационные похибки.
% 
% Аналіз показує, що незалежно від типу конкретних ММГ і ММА похибки вихідної інформації 
% блоку мікромеханічних інерціальных датчиків можуть бути описані за допомогою наступної 
% узагальненої моделі:
% 
% \begin{equation} \label{eq:__6_13_} \begin{array}{l} {\Delta \bar{\omega }(t)=C_{
% \omega } \bar{\omega }(t)+C\bar{a}(t)+\Delta \bar{\omega }_{{\rm AB}} +\Delta \bar{
% \omega }_{{\rm A;}} (t{\rm )}+\bar{\eta }_{\omega } (t);} \\ {\Delta \bar{a}(t)=C_{a} 
% \bar{a}(t)+\Delta \bar{a}_{{\rm cB}} +\Delta \bar{a}_{{\rm A;}} (t{\rm )}+\bar{\eta 
% }_{a} (t),} \end{array} \end{equation} 
% 
% Де $C_{\omega } $ й$C_{a} $ -- матриці 3 $\times$ 3 коефіцієнтів систематичних похибок 
% датчиків (діагональні елементи цих матриць характеризують похибки масштабних коефіцієнтів, 
% а недіагональні елементи -- перекоси осей чутливості ММГ і ММА відносно осей ортогональної 
% системи координат, пов'язаної із блоком датчиків);
% 
% \textit{С} -- матриця 3 $\times$ 3 коефіцієнтів систематичних похибок ММГ, що залежать 
% від перевантажень;
% 
% $\bar{\omega }(t),\bar{a}(t)$-- вектори-стовпці поточної складової абсолютної кутової 
% швидкості й гаданого прискорення початку приладового координатного базису в його 
% осях;
% 
% $\Delta \bar{\omega }(t)$, $\Delta \bar{a}(t)$ - вектори-стовпці поточних похибок 
% виміру складової абсолютної кутової швидкості й уявного прискорення;
% 
% $\Delta \bar{\omega }_{{\rm AB}} $- вектор-стовпець систематичних дрейфів ММГ;
% 
% $\Delta \bar{\omega }_{{\rm A;}} (t{\rm )}$- вектор-стовпець випадкових дрейфів ММГ;
% 
% $\bar{
% \eta }_{\omega } (t)$ - вектор-стовпець вихідних шумів ММГ;
% 
% $\Delta \bar{a}_{{\rm cB}} $ - вектор-стовпець систематичних зсувів ММА;
% 
% $\Delta \bar{a}_{{\rm A;}} (t{\rm )}$ - вектор-стовпець випадкових дрейфів нулів 
% ММА;
% 
% $\bar{\eta }_{a} (t)$ - вектор-стовпець вихідних шумів ММА.
% 
% Параметри моделей: $C_{\omega } $ ,$C_{a} $ , \textit{С}, $\Delta \bar{\omega }_{{
% \rm AB}} $, $\Delta \bar{a}_{{\rm cB}} $ -- можна вважати квазістаціонарними, залежними 
% в основному від температурного режиму блоку датчиків. Флуктуаційні похибки $\bar{
% \eta }_{\omega } (t)$, $\bar{\eta }_{a} (t)$ можна розглядати як некорельовані «білі» 
% шуми. Випадкові дрейфи  $\Delta \bar{\omega }_{{\rm A;}} (t{\rm )}$й $\Delta \bar{a}_{{
% \rm A;}} (t{\rm )}$ можна інтерпретувати як марковські процеси першого порядку з 
% періодами кореляції \textit{Т}$\omega $ ≈ 600 с і \textit{Та}  ≈ 100 с;
% 
% \[\begin{array}{l} {\dot{\bar{\omega }}_{{\rm A;}} ({\rm t)}=-\frac{{\rm 1}}{T_{
% \omega } } \left[\bar{\omega }_{{\rm A;}} (t{\rm )}-\bar{\xi }_{\omega } {\rm (}t{
% \rm )}\right]\, \, {\rm ;}} \\ {\Delta \dot{\bar{a}}_{{\rm A;}} {\rm (t)}=-\frac{{
% \rm 1}}{T_{a} } \left[\Delta \bar{a}_{{\rm A;}} {\rm (}t{\rm )}-\bar{\xi }_{a} (t)
% \right]\, \, \, ,} \end{array}\] 
% 
% де $\bar{\xi }_{\omega } {\rm (}t{\rm )}$, $\bar{\xi }_{a} (t)$ - збурення типу «білого» 
% шуму із заданими інтенсивностями.
% 
% Основні підходи до підвищення точності блоків мікромеханічних інерціальних датчиків 
% у робочому діапазоні експлуатаційних температур наступні:
% 
% - калібрування елементів $C_{\omega } $ ,$C_{a} $ , \textit{С}, $\Delta \bar{\omega 
% }_{{\rm AB}} $, $\Delta \bar{a}_{{\rm cB}} $ у дискретних точках робочого діапазону 
% температур з наступною інтерполяцією за результатами вимірів температури й введення 
% відповідних виправлень (алгоритмічна компенсація погрішностей;
% 
% - термостабілізація блоків датчиків (виправлення вводяться тільки для розрахункової 
% температури блоку).
% 
% 
% \subsection{6.3 Математичні моделі похибок супутникової системи навігації}
% 
% Для опису  похибок СНС у визначенні координат і проекцій шляхової швидкості ЛА пропонується 
% використовувати математичні моделі, що містять  Марківські і гаусовські складові 
% похибок:
% 
% \begin{equation} \label{eq:__6_14_} \begin{array}{l} {\Delta R_{Es,k} =\Delta 
% R_{Ec,k} +\frac{\sigma _{Rs} }{\cos \varphi _{k} } \eta _{REs,k} +\frac{\sigma _{
% \delta Rs} }{\cos \varphi _{k} } \eta _{\delta RE,k} ;} \\ {\Delta R_{Ns,k} =\Delta 
% R_{Nc,k} +\sigma _{Rs} \eta _{RNs,k} +\sigma _{\delta Rs} \eta _{\delta RN,k} ;} 
% \\ {\Delta H_{s,k} =\Delta H_{c,k} +\sigma _{Hs} \eta _{Hs,k} +\sigma _{\delta Rs} 
% \eta _{\delta H,k} {\rm ,}\, \, } \\ {\Delta V_{ls,k} =\Delta V_{lc,k} +\sigma _{Vs} 
% \eta _{V\, ls,k} +\sigma _{\delta Vs} \eta _{\delta V\, ls,k} {\rm ,}\, \, \, \, 
% \, \, \, \, \, {\rm ?@8}\, \, \, \, l=E,N,H;} \end{array} \end{equation} 
% 
% 
% 
% де $\Delta R_{ls,k} \, (l=E,N);\, \, \, \Delta H_{s,k} \, ;\, \, \, \Delta V_{ls,k} 
% \, (l=E,N,H)\, \, $-- похибки СНС у визначенні приведених  координат, висоти і складових 
% шляхової швидкості ЛА;
% 
% $\Delta R_{lc,k} \, (l=E,N);\, \, \, \Delta H_{c,k} \, ;\, \, \, \Delta V_{lc,k} 
% \, (l=E,N,H)\, \, $-- корельовані (Марківські) складові  похибок СНС;
% 
% $\sigma _{Rs} ,\, \, \sigma _{Hs} ,\, \, \sigma _{Vs} $  --  СКЗ білошумових складових 
% похибок СНС;
% 
% $\sigma _{\delta Rs} $, $\sigma _{\delta Hs} $, $\sigma _{\delta Vs} $ -- СКЗ додаткових 
% білошумових складових похибок СНС, що виникають тільки за умови, що tk -- момент 
% зміни сузір'я навігаційних супутників; 
% 
% $\eta _{Rls,k} ,\, \, \, \, \eta _{\delta Rls,k} \, \, \, (l=E,N);\, \, \eta _{Hs,k} 
% ,\, \, \eta _{\delta Hs,k} {\rm ;}\, \, \, \eta _{V\, ls,k} ,\eta _{\delta V\, ls,k} 
% \, \, \, (l=E,N,H)$ -- стандартні білі дискретні шуми зі СКЗ.
% 
% Корельовані складові похибок СНС описуються наступними співвідношеннями:
% 
% \begin{equation} \label{eq:__6_15_} \begin{array}{l} {\Delta R_{Ec,k} =W_{R} 
% \Delta R_{Ec,k-1} +q_{R} \frac{\sigma _{Rc} }{\cos \varphi _{k} } \eta _{REc,k} +
% \frac{\sigma _{\delta RC} }{\cos \varphi _{k} } \eta _{\delta REc,k} ;} \\ {\Delta 
% R_{Nc,k} =W_{R} \Delta R_{Nc,k-1} +q_{R} \sigma _{Rc} \eta _{RNc,k} +\sigma _{\delta 
% RC} \eta _{\delta RNc,k} ;} \\ {\Delta H_{c,k} =W_{R} \Delta H_{c,k-1} +q_{R} \sigma 
% _{Hc} \eta _{Hc,k} +\sigma _{\delta Hc} \eta _{\delta Hc,k} ;} \\ {\Delta V_{lc,k} 
% =W_{V} \Delta V_{lc,k-1} +q_{V} \sigma _{Vc} \eta _{V\, lc,k} +\sigma _{\delta Vc} 
% \eta _{\delta V\, lc,k} \, \, \, \, \, \, \, {\rm ?@8}\, \, l=E,N,H,} \end{array} \end{equation} 
% 
% де  
% 
% \[W_{R} 
% =e^{-(\lambda _{s} V_{{\rm H}} +\lambda _{st} )\Delta t} ;\, \, \, \, \, \, q_{R} 
% =\left[1-\exp \left(-2\left(\lambda _{s} V_{{\rm H}} +\lambda _{st} \right)\Delta 
% t\right)\right]\, ^{0,5}   ;\] 
% 
% \[W_{V} =e^{-\lambda _{V} \Delta t} ;\, \, \, \, \, \, \, q_{V} =\left[1-\exp \left(-2
% \lambda _{V} \Delta t\right)\right]\, ^{0,5}  ;\] 
% 
% $\lambda _{s} $-- показник просторової кореляції похибки СНС за координатами; $\lambda 
% _{V} ,\lambda _{st} $-- показник часової кореляції похибок СНС за швидкістю та за 
% координатами; $V_{{\rm H}} $-- шляхова швидкість ЛА;
% 
% $\Delta t$-- дискрета оновлення вихідної інформації СНС у часі;
% 
% $\sigma _{Rc} ,\sigma _{Hc} ,\sigma _{Vc} $ -- СКЗ корельованих складових похибок 
% СНС;
% 
% $\sigma _{\delta Rc} $,  $\sigma _{\delta Hc} $, $\sigma _{\delta } _{Vc} $  -- СКЗ 
% додаткових гаусовських збурень у моменти зміни сузір'я навігаційних супутників;
% 
% $\eta _{Rlc,k} ,\, \, \, \, \eta _{\delta Rlc,k} \, \, \, (l=E,N),\, \, \eta _{Hc,k} 
% ,\, \, \eta _{\delta Hc,k} {\rm ,}\, \, \, \eta _{V\, lc,k} ,\eta _{\delta V\, lc,k} 
% \, \, (l=E,N,H)$ -- стандартні центровані дискретні білі шуми з одиничною інтенсивністю.
% 
% Для 
% стандартного режиму СНС типу GPS NAVSTAR можуть бути рекомендовані наступні значення 
% параметрів моделей \eqref{eq:__6_14_}, \eqref{eq:__6_15_}:
% 
% 
% 
% \[\lambda _{s} =4\cdot 10^{-6} {\rm <}^{-1} ; \lambda _{st} =5\cdot 10^{-4} {\rm 
% c}^{-1} ; \lambda _{V} =\left(0,0017\div 0,05\right)\, {\rm c}^{-1} ;\] 
% 
% \[\sigma _{Rs} =\left(1\div 3\right){\rm <};\sigma _{Hs} =\left(1,5\div 4\right)
% \, {\rm <};\sigma _{Vs} =\left(0,01\div 0,05\right){\rm <}/{\rm c};\] 
% 
% \[\sigma _{\delta Rs} =\left(1\div 4\right){\rm <}; \sigma _{\delta } _{Vs} =\left(0,02
% \div 0,2\right){\rm <}/{\rm c}; \sigma _{Rc} =\left(5\div 7\right){\rm <};\] 
% 
% \[\sigma _{Hc} =\left(7\div 10\right){\rm <};  \sigma _{Vc} =\left(0,02\div 0,3\right){
% \rm </c};\] 
% 
% \[\sigma _{\delta Rc} =\left(2\div 5\right){\rm <}; \sigma _{\delta } _{Vc} =\left(0,01
% \div 0,02\right){\rm <}/{\rm c}; \sigma _{\delta Hc} =\left(3\div 7\right){\rm <}.\] 
% 
% Головною 
% задачею СНС є визначення псевдодальностей $D_{sl}^{} $ і псевдошвидкостей $V_{sl}^{} $ (l 
% = 1, \dots , N  -- число видимих навігаційних супутників),  які задовольняють співвідношенням 
% виду: 
% 
% \[\begin{array}{l} {D_{sl,k} =\{ \left[x_{l} \left(t_{k} -\tau _{l} \right)-x\left(t_{k} 
% \right)\right]^{2} +\left[y_{l} \left(t_{k} -\tau _{l} \right)-y\left(t_{k} \right)
% \right]^{2} +} \\ {\, \, \, \, \, \, \, \, \, \, \, \, \, \, \, \, +\left[z_{l} \left(t_{k} 
% -\tau _{l} \right)-z\left(t_{k} \right)\right]^{2} \} ^{\frac{1}{2} } +c\Delta \tau 
% _{k} ;} \end{array}\] 
% 
% \[\begin{array}{l} {V_{sl,k} =\{ \left[V_{xl} \left(t_{k} -\tau _{l} \right)-V_{x} 
% \left(t_{k} \right)\right]\left[x_{l} \left(t_{k} -\tau _{l} \right)-x\left(t_{k} 
% \right)\right]+} \\ {\, \, \, \, \, \, \, \, \, \, \, \, \, +\left[V_{yl} \left(t_{k} 
% -\tau _{l} \right)-V_{y} \left(t_{k} \right)\right]\left[y_{l} \left(t_{k} -\tau 
% _{l} \right)-y\left(t_{k} \right)\right]+} \\ {\, \, \, \, \, \, \, \, \, \, \, \, 
% \, +\left[V_{zl} \left(t_{k} -\tau _{l} \right)-V_{z} \left(t_{k} \right)\right]
% \left[z_{l} \left(t_{k} -\tau _{l} \right)-z\left(t_{k} \right)\right]+} \\ {\, \, 
% \, \, \, \, \, \, \, \, \, \, \, +\Omega _{{\rm }} x_{l} \left(t_{k} -\tau _{l} \right)y
% \left(t_{k} \right)\, -\Omega _{{\rm }} y_{l} \left(t_{k} -\tau _{l} \right)x\left(t_{k} 
% \right)\} \tilde{D}_{sl,k}^{-1} +cV,} \end{array}\] 
% 
% де $x_{l} \left(t_{k} -\tau _{l} \right)$,$y_{l} \left(t_{k} -\tau _{l} \right)$,$z_{l} 
% \left(t_{k} -\tau _{l} \right)$,$V_{xl} \left(t_{k} -\tau _{l} \right),V_{yl} \left(t_{k} 
% -\tau _{l} \right),V_{zl} \left(t_{k} -\tau _{l} \right)$-- координати і проекції 
% абсолютної швидкості $l$-го навігаційного супутника в прямокутної гринвіцькій  геоцентричній 
% СК XYZ;
% 
% $x\left(t_{k} \right),y\left(t_{k} \right),z\left(t_{k} \right),V_{x} \left(t_{k} 
% \right),V_{y} \left(t_{k} \right),V_{z} \left(t_{k} \right)$ -- координати і проекції 
% шляхової швидкості ЛА в СК XYZ; 
% 
% $\tau _{l} $− час проходження радіосигналу від  l-го навігаційного супутника до ЛА;
% 
% \[\tilde{D}_{sl,k} 
% =D_{sl,k} -\Delta \tau ;\] 
% 
% $\Delta \tau _{k} =\Delta \tau _{k-1} +V_{\varepsilon _{k} } \Delta t+\sigma _{\zeta 
% \tau } \xi _{\tau ,k-1} \, {\rm B0}\, \, V_{\varepsilon _{k} } =V_{\varepsilon _{k-1} 
% } +\sigma _{\zeta V\tau } \xi _{V\tau ,k-1} $--зрушення та дрейф шкали часу в бортовій 
% апаратурі СНС;
% 
% $\Omega _{{\rm }} $ − кутова швидкість обертання Землі; с -- швидкість  поширення 
% світла.
% 
% Похибки СНС при визначенні псевдодальностей $D_{sl}^{} $ и псевдошвидкостей $V_{sl}^{} $ можуть 
% бути описані наступними співвідношеннями:
% 
% \begin{equation} \label{eq:__6_16_} \begin{array}{c} {\Delta D_{sl,k} =\Delta 
% D_{scl,k} +\sigma _{DS} \eta _{Dl,k} ;} \\ {\Delta V_{sl,k} =\Delta V_{scl,k} +\sigma 
% _{VS} \eta _{Vl,k} ;} \end{array}\, \, \, \, \, \, \, \, \, \, (l=1,...,N), \end{equation} 
% 
% 
% 
% тут 
% 
% $\begin{array}{l} 
% {\Delta D_{scl,k} =W_{R} \Delta D_{scl,k-1} +q_{R} \sigma _{Ds} \eta _{sl,k} } \\ 
% {\Delta V_{scl,k} =W_{V} \Delta V_{scl,k-1} +q_{V} \sigma _{Vc} \eta _{scl,k} } \end{array}$;                                                   
% \eqref{eq:__6_17_}
% 
% 
% 
% $\eta _{Dl,k} $,$\eta _{Vl,k} $,$\eta _{sl,k} $,$\eta _{scl,k} $,$\xi _{\tau ,k} $,$\xi 
% _{V\tau ,k} $ -- стандартні центровані дискретні білі шуми з одиничною інтенсивністю;
% 
% $\sigma 
% _{DS} $, $\sigma _{Vs} $ -- СКЗ гаусівських складових похибок СНС у визначенні псевдодальностей 
% і псевдошвидкостей;
% 
% $\sigma _{DC} $, $\sigma _{Vc} $ --  СКЗ корельованих складових похибок СНС у визначенні 
% псевдодальностей і псевдошвидкостей;
% 
%  Для стандартного режиму СНС GPS NAVSTAR можуть бути рекомендовані наступні значення 
% параметрів моделі \eqref{eq:__6_16_}, \eqref{eq:__6_17_}:
% 
% \[\begin{array}{l} {\sigma _{Ds} =(1\div 4){\rm <;}\, \, \, \, \, \sigma _{Vs} =(0,02
% \div 0,03){\rm </A};} \\ {\sigma _{Dc} =(5\div 9){\rm <;}\, \, \, \, \sigma _{Vc} 
% =(0,02\div 0,03){\rm </A};} \end{array}\] 
% 
% При застосуванні в навігаційних розрахунках комбінованих методів додаткову навігаційну 
% функцію дає вимірник висоти. Так, у далекомірному методі при наявності на борту ЛА 
% високоточної системи вимірювання висоти польоту Н, сфера з радіусом Rз + Н (де Rз 
% = 6371116 м -- радіус сфери, рівновеликої земному геоїду) може бути прийнята  за 
% додаткову поверхню положення. У цьому випадку можна замість вимірювань трьох дальностей 
% до НС обмежитися вимірюванням двох дальностей, тоді навігаційна функція буде включати 
% два рівняння сфери, а третє необхідне рівняння дає вимірник висоти 
% 
% (Rз + H)2 = x2 + y2 + z2.
% 
% Ось чому для реалізації процедур оптимального комплексування  інерціальної та супутникової 
% систем навігації необхідно мати додаткову модель похибок барометричного висотоміра.
% 
% 
% 
% 
% \subsection{6.4 Математичні 
% моделі похибок барометричного висотоміра}
% 
% Похибка барометричного висотоміра (БВ) у визначенні абсолютної висоти ЛА може бути 
% описана співвідношенням вигляду:
% 
% $\Delta h(t_{k} )=\Delta h_{{\rm 2A}} +\sigma _{h} \eta _{n,k} $ ,               \eqref{eq:__6_18_}
% 
% де $\Delta 
% h_{{\rm 2A}} $-- квазістаціонарна похибка виміру барометричної висоти, що обумовлена 
% неточністю початкової виставки, а також змінами температури та тиску атмосфери за 
% час польоту;
% 
% $\sigma _{h} _{} $-- СКЗ флюктуаційної складової похибки, що обумовлена пульсаціями 
% тиску й іншими факторами;
% 
% $\eta _{n,k} $-- дискретний білий шум з одиничною інтенсивністю.
% 
% У свою чергу дискретна модель еволюції квазістаціонарної похибки БВ може бути представлена 
% в наступному вигляді:
% 
% $\Delta h_{c,k} =\Delta h_{{\rm 2c},k-1} +\sigma _{\xi A} \xi _{k-1} $,                                       \eqref{eq:__6_19_}
% 
% де  $\sigma 
% _{\xi A} $-- заданий параметр; $\xi _{k-1} $ -- стандартний дискретний білий шум 
% і одинична інтенсивність.
% 
% Аналіз показує, що для моделі похибок БВ \eqref{eq:__6_18_}, \eqref{eq:__6_19_} 
% можна рекомендувати наступні значення параметрів:
% 
% \[\sigma _{h} =(0,5\div 1){\rm <;}\, \, \, \, \sigma _{\xi c} =(0,05\div 0,02){\rm 
% <;}\, \, \, \, \, \sigma _{\Delta hc,0} =(3\div 5){\rm <,}\] 
% 
% де $\sigma _{\Delta hc,0} $ --  СКЗ похибки $\Delta $\textit{hс} у початковий момент 
% часу.
% 
% 
% \subsection{6.5 Дослідження математичних моделей похибок MEMС- датчиків}
% 
% Математична модель похибок трикомпонентного МЕМС датчика кутової швидкості при проведені 
% моделюванні була представлена у більш розширеному вигляді ніж у п.п. 6.2.
% 
% Нижче приведена модель датчика кутової швидкості, яка використовувалася при проведенні 
% досліджень:
% 
% \[\begin{array}{l} {\omega _{x}^{m} =(1+K_{\omega x} )[\omega _{x} +K_{xz} \omega 
% _{y} -K_{xy} \omega _{z} +\varepsilon _{\omega x} ]+\varepsilon _{\omega x_{{\rm 
% c}} } ;} \\ {\omega _{y}^{m} =(1+K_{\omega y} )[\omega _{y} +K_{yx} \omega _{z} -K_{yz} 
% \omega _{x} +\varepsilon _{\omega y} ]+\varepsilon _{\omega y_{{\rm c}} } ;} \\ {
% \omega _{z}^{m} =(1+K_{\omega z} )[\omega _{z} +K_{zy} \omega _{z} -K_{yz} \omega 
% _{y} +\varepsilon _{\omega z} ]+\varepsilon _{\omega z_{{\rm c}} } ,} \end{array}\] 
% 
% де $\omega 
% _{x}^{m} ,{\rm \; }\omega _{y}^{m} ,{\rm \; }\omega _{z}^{m} $ --  обмірювані величини 
% кутових швидкостей; 
% 
% $\omega$\textit{x} , $\omega$\textit{y} , $\omega$\textit{z}  -- `` дійсні '' значення 
% кутових швидкостей; 
% 
% \textit{К}$\omega$\textit{x}, \textit{К} $\omega$\textit{y}, \textit{К} $\omega$\textit{z} -- 
% похибки масштабних лінійних коефіцієнтів; 
% 
% \textit{Кxy}, \textit{Кyz} \dots  \textit{Кzx}, \textit{Кzy} -- помилки невиставки 
% приладів у  відповідних площинах зв'язаної системи; 
% 
% $\epsilon$$\omega$\textit{x} , $\epsilon$$\omega$\textit{y} , $\epsilon$$\omega$\textit{z} -- 
% систематичні складові зсуви нулів датчиків,  $\epsilon$$\omega$\textit{xс} , $\epsilon$$\omega$\textit{yс} , $\epsilon$$\omega$\textit{zс} -- 
% випадкові складові зсуви нулів датчиків (шуми вимірів).
% 
% Похибки, зв'язані із систематичною і випадковою складовими зсувів нулів датчиків, 
% зі змінами масштабних лінійних коефіцієнтів, розподіляються таким чином, що при збільшенні 
% одного з них зростають всі інші. Цей факт ілюструється у табл. 6.1, 6.2, де показані 
% приклади зміни складових  похибок датчиків первинної інформації сучасних прецизійних 
% ІНС. 
% 
% 
% 
% 
% 
% 
% 
% 
% 
%   \textit{Таблиця }6.1\textit{}
% 
% \begin{tabular}{|p{2.5in}|p{0.7in}|p{0.7in}|p{0.7in}|} \hline 
% \multicolumn{4}{|p{1in}|}{Акселерометри} \\ \hline 
% Зсув показань, 10-3 & 0,01 & 0,05 & 0,1 \\ \hline 
% Масштабний коефіцієнт & 0,001 & 0,005 & 0,01 \\ \hline 
% Неортогональність, кут.с & 10 & 10 & 20 \\ \hline 
% Випадкова складова, м/с3/год & 0,009 & 0,01 & 0,02 \\ \hline 
% \end{tabular}
% 
% \textit{Таблиця }6.2\textit{}
% 
% \begin{tabular}{|p{2.8in}|p{0.6in}|p{0.6in}|p{0.6in}|} \hline 
% \multicolumn{4}{|p{1in}|}{Датчики кутової швидкості} \\ \hline 
% Дрейф, що не залежить від перевантаження, град/год & 0,005 & 0,01 & 0,1 \\ \hline 
% Дрейф, 
% що залежить від перевантаження, град/год & 0,0075 & 0,015 & 0,15 \\ \hline 
% Масштабний коефіцієнт & 0,0025 & 0,005 & 0,02 \\ \hline 
% Неортогональність, кут. сек & 20 & 60 & 120 \\ \hline 
% Випадкове блукання, град/год & 0,0005 & 0,001 & 0,01 \\ \hline 
% \end{tabular}
% 
% 
% 
% Для моделюванні похибок МЕМС датчика кутової швидкості були проаналізовані його джерела 
% похибок, кількісні оцінки яких приведені в табл. 6.3. 
% 
% \textit{Таблиця }6.3.\textit{}
% 
% \begin{tabular}{|p{3.3in}|p{1.3in}|} \hline 
% Систематична складова  зсуву нуля & 15 град/год \\ \hline 
% Випадкова складова  зсуву нуля (за час польоту) & 10 град/год \\ \hline 
% Похибка масштабного лінійного коефіцієнта & 1. 5 град/год \\ \hline 
% Невиставлення осі (похибки юстирування) за 1 віссю & 15 кутов. сек \\ \hline 
% Невиставлення осі (похибки юстирування) за 2 віссю & 15 кутов. сек \\ \hline 
% \end{tabular}
% 
% 
% 
% Зокрема величина систематичної складові зсуву нуля датчика кутової швидкості як величина 
% ($\pm$2$\sigma$)  = = $\pm$15\dots 20$\circ$/год була визначена в ході аналізу стабільності 
% швидкості дрейфу існуючих та перспективних (найближчі 10 років) МЕМС-гіроскопів.  
% 
% При 
% проведенні досліджень $\epsilon$$\omega$\textit{x} , $\epsilon$$\omega$\textit{y} , $\epsilon$$\omega$\textit{z} задавалися 
% в межах $\pm$0,01\dots 0,02$\circ$/с для невідкаліброваних датчиків кутової швидкості 
% і у два три рази менше для датчиків, що пройшли попереднє калібрування на етапі виставлення. 
% 
% Випадкова 
% складові зсуву нуля датчика кутової швидкості  за годину польоту  може досягати величини 
% того ж порядку, що і систематична складових невідкаліброваних датчиків кутової швидкості. 
% При проведенні досліджень випадкова складового зсуву нуля моделювалася за допомогою 
% формуючих фільтрів, на входи яких надходив випадковий процес типу білого шуму, а 
% на виході утворюється процес  $\epsilon$$\omega$\textit{x}с(\textit{t}), $\epsilon$$\omega$\textit{вус}(\textit{t}), $\epsilon$$\omega$\textit{z}с 
% (\textit{t}). Експериментально параметри формуючого фільтра минулого підібрані  таким 
% чином, щоб випадкові блукання нуля датчика кутової швидкості за годину польоту не 
% виходили за межі  $\pm$0,05\dots 0,01$\circ$/с.
% 
% Похибки масштабних лінійних коефіцієнтів датчиків кутової швидкості, також мають 
% величину того ж порядку, що і  систематична складових невідкаліброваних датчиків 
% кутової швидкості, але в два\dots три рази її перевищують. При проведенні досліджень 
% величини \textit{К}$\omega$\textit{x}, \textit{К}$\omega$\textit{y}, \textit{К}$\omega$\textit{z} задавалися 
% в межах $\pm$0,05\dots 0,008$\circ$/с. 
% 
% Помилки невиставки приладів у відповідних площинах зв'язаної системи  \textit{Кxy}, \textit{Кyz} \dots  \textit{Кzx}, \textit{Кzy}  залежать 
% від точності юстировки датчиків у блоці чутливих елементів (БЧЕ) і від точності юстировки 
% самого БЧЕ по осях ЛА. В даний час досягнута точність юстирування складає 5...10 
% кут. с., що в перерахуванні відповідає коефіцієнтам  \textit{Кxy}, \textit{Кyz} \dots  \textit{Кzx}, \textit{Кzy}  рівним  
% 2,5$.$10-5\dots 5$.$10-5$\circ$/с. При проведенні досліджень ці коефіцієнти були 
% збільшені на порядок, з метою урахування статичних деформацій конструкції ЛА в місці 
% установки БЧЕ. 
% 
% Як показали попередні дослідження моделей похибки датчиків кутової швидкості при 
% такому рівні систематичних і випадкових складових зсувів нулів датчиків мертва зона 
% і гістерезис практично не надають впливу на помилки обчислення навігаційних параметрів.  
% 
% Аналогічний 
% МЕМС датчику кутової швидкості мають вид моделі МЕМС акселерометрів: 
% 
% \[\begin{array}{l} {a_{x}^{m} =(1+K_{ax} )[a_{x} +K_{xy} a_{y} -K_{xz} a_{z} +Q_{x} 
% a_{x}^{2} +\varepsilon _{ax} +\omega _{z} L_{x} +\omega _{y} L_{x} ]+\varepsilon 
% _{ax_{{\rm c}} } ;} \\ {a_{y}^{m} =(1+K_{ay} )[a_{y} +K_{yx} a_{z} -K_{yz} a_{x} 
% +Q_{y} a_{y}^{2} +\varepsilon _{ay} +\omega _{x} L_{y} +\omega _{z} L_{y} ]+\varepsilon 
% _{ay_{{\rm c}} } ;} \\ {a_{z}^{m} =(1+K_{az} )[a_{z} +K_{zy} a_{x} -K_{zx} a_{y} 
% +Q_{z} a_{z}^{2} +\varepsilon _{az} +\omega _{x} L_{z} +\omega _{y} L_{z} ]+\varepsilon 
% _{az_{{\rm c}} } ,} \end{array}\] 
% 
% де $a_{x}^{m} ,{\rm \; }a_{y}^{m} ,{\rm \; }a_{z}^{m} $ -- обмірювані прискорення,  
% 
% \textit{ax , 
% ay , az} --  `` дійсні ''  прискорення,
% 
% $\omega$\textit{x , }$\omega$\textit{y , }$\omega$\textit{z} -- кутові швидкості 
% обертання ЛА,
% 
% \textit{Lx , Ly , Lz}  -- лінійні зсуви місця установки блоку чуттєвих елементів 
% від центра мас ЛА;
% 
% \textit{Кax} \dots  \textit{Кaz} -- похибки масштабних коефіцієнтів, 
% 
% \textit{Кxz} \dots  \textit{К}z\textit{x} -- похибки юстирування, 
% 
% \textit{Qx, Qy, Qz} -- коефіцієнти квадратичної похибки через нелінійність характеристики 
% приладу, 
% 
% $\epsilon$ах , $\epsilon$а\textit{y} , $\epsilon$а\textit{z} -- систематичні складові 
% зсувів нулів датчиків,  $\epsilon$а\textit{x}с , $\epsilon$а\textit{y}с , $\epsilon$а\textit{z}с 
% -- випадкові складові зсувів нулів датчиків (шуми вимірів).
% 
% Проаналізувавши розподіли погрішностей існуючих акселерометрів (табл. 6.4)  сформовані 
% значення складових погрішностей МЕМС акселерометрів. Зокрема  $\epsilon$ах , $\epsilon$а\textit{y} , $\epsilon$а\textit{z} задавалися 
% в діапазоні 5 $.$10-2 \textit{g}  для невідкаліброваних  датчиків і 2,5 $.$10-2 \textit{g}  для 
% датчиків, що пройшли передстартову виставку (калібрування).  Випадкові складові зсуви 
% нулів акселерометрів були підібрані експериментально так, щоб випадкові блукання 
% нуля датчика за годину польоту не виходили за межі  $\pm$(0.9...1,2) $.$10-2 \textit{g} . 
% Похибки масштабних коефіцієнтів \textit{Кax} \dots  \textit{Кaz}  задавалися в межах 
% 0,004\dots 0,005,а коефіцієнти квадратичної похибки \textit{Qx, Qy, Qz}  варіювалися 
% в межах 2,5 $.$10-4\dots 1,5 $.$10-4. Помилки невиставки приладів у відповідних площинах 
% зв'язаної системи формулювалися аналогічно датчикам кутових швидкостей.
% 
% \textit{Таблиця }6.4.\textit{}
% 
% \begin{tabular}{|p{3.3in}|p{1.3in}|} \hline 
% Мертва зона (гранична чутливість)\textbf{} & $10^{-3} g...10^{-2} g$ \\ \hline 
% Систематична складова  зсуву нуля & (5\dots 3) $.$10-2 \textit{g} \\ \hline 
% Випадкова складова  зсуву нуля (за час польоту) & 0.9 $.$10-2 \textit{g} \\ \hline 
% Похибка 
% масштабного лінійного коефіцієнта & 0.1\%  по всей шкале \\ \hline 
% Нелінійний коефіцієнт калібрування (коефіцієнт впливу  \textit{g}2) & 0,3 $.$10-2 \textit{g}/\textit{g}2 \\ \hline 
% Невиставлення 
% осі приладу (похибки юстирування) & 15 кутов. сек \\ \hline 
% Лінійний температурний коефіцієнт  & 10-2 \textit{g}/град \\ \hline 
% \end{tabular}
% 
% 
% 
% Випадкові складові і перекручування масштабного коефіцієнта моделювалися з використанням 
% генераторів "білого шуму" і формуючих фільтрів. При цьому вважалося, що кожен чуттєвий 
% елемент цілком визначається значеннями цих складових, а самі ці складові змінюються 
% таким чином, що при збільшенні одного з них зростають і всі інші.
% 
% При моделюванні білошумних погрішностей датчиків БІНС можна скористатися схемою  формування 
% випадкових  сигналів, приведеної на рис 6.1.
% 
% 
% 
% Складовими випадкової похибки датчиків БІНС є марковський процес і процес випадкового 
% блукання.
% 
% Марковський процес використовується для апроксимації високочастотного стаціонарного 
% випадкового процесу. Процес випадкового блукання є нестаціонарним.
% 
% Складові марковського процесу і випадкового блукання можуть збуджуватися окремими 
% білими шумами або спільним білим шумом, у цьому випадку виникає кореляція між процесами.  
% 
% Марковська 
% складова випадкової похибки датчиків БІНС $\varepsilon _{} $ має кореляційну функцію
% 
% \[K_{
% \varepsilon } (\tau )=A\cdot e^{-\mu \left|\tau \right|}  ,\] 
% 
% де    $A=\sigma ^{2} $ дисперсія випадкової похибки; $\sigma $ середньоквадратичне 
% відхилення; $\tilde{\mu }$ коефіцієнт згасання кореляційної функції; $T={1 \mathord{
% \left/{\vphantom{1 \mu }}\right.\kern-\nulldelimiterspace} \mu } $  стала часу кореляції, 
% що дорівнює 0.5...\dots 1год.
% 
% Зазвичай марковськую складову $\varepsilon _{} $ представляють у вигляді випадкового 
% процесу, зв'язаного з білим шумом диференціальним рівнянням першого порядку.
% 
% Так, кореляційної функції $_{\varepsilon } $ відповідає рівняння
% 
% \[\dot{\varepsilon }_{M} =-\mu \cdot \varepsilon _{M} +\sqrt{2\cdot A\cdot \mu } 
% \cdot w(t) ,\] 
% 
% де   \textit{w}(\textit{t})   вихідний білий шум одиничної інтенсивності з рівним 
% нулю математичним сподіванням  і кореляційною функцією $M\left[w(t)\cdot w(t)\right]=
% \delta (t-\tau )$.
% 
% \includegraphics[bb=0mm 0mm 208mm 296mm, width=170.6mm, height=183.8mm, viewport=3mm 
% 4mm 205mm 292mm]{image1.eps} Рисунок 6.2
% 
% Рівняння   такого виду   називають рівнянням формуючого фільтра, на вхід якого надходить 
% випадковий процес \textit{w}(\textit{t}) типу білого шуму, а на виході маємо процес $\varepsilon 
% (t)$з кореляційною функцією $K_{\varepsilon } (\tau )$.
% 
% Для проведення досліджень була складена програма моделювання, зокрема субблоки «DYCu» 
% і «Akselerometr», в яких у свою чергу формувалися субблоки «Model oshibok DUS» і 
% «Model oshibok akselerator», де моделювалися похибки датчиків. На рис. 6.2 як приклад 
% розкритий субблок  «Model oshibok akselerator».
% 
% \includegraphics[bb=0mm 0mm 208mm 296mm, width=170.1mm, height=120.5mm, viewport=3mm 
% 4mm 205mm 292mm]{image2.eps}
% 
% У верхньому правому куті субблока показаний ще один внутрішній блок -- блок моделювання 
% випадкових складові зсуву нулів датчиків (шумів вимірів). У нижній частині блоку 
% зібрані схеми формування складових похибок, обумовлених відцентровими прискоренням, 
% пропорційним зсувові центра мас акселерометра \textit{Lх, Ly, Lz, } від осі обертання 
% ЛА (центра мас ЛА). Величини зсуву  \textit{Lх, Ly, Lz  }задаються з блоку «Start
% \_napametru», там же передбачене переключення умов формування погрішностей датчиків 
% первинної інформації:
% 
% \begin{enumerate}
% \item - -ідеальні датчики (датчики без погрішностей) і датчики з похибками вимірів;
% 
% \item - -датчики 
% тільки із систематичної складової похибки і датчики із систематичної і динамічної 
% (випадкової) складової похибки;
% 
% \item - -датчики попередньо відкалібровані і не відкалібровані.
% \end{enumerate}
% 
% За принципом аналогічному субблоку «Model oshibok akselerator» побудований субблок 
% «Model oshibok DUS». На осцилограмах рис. 6.3 ілюструється характер зміни похибок 
% (град/сек) відкаліброваних і не відкаліброваних датчиків кутових швидкостей при прямолінійному 
% горизонтальному польоті. 

% 

% Chapter 6 % Health safety
\newpage
\ESKDthisStyle{formII}
% \documentclass[ukrainian,utf8,simple,floatsubsection, hpadding=5mm,equationsubsection,]{eskdtext}
% \usepackage[warn]{mathtext}
% \usepackage[unicode]{hyperref} % enable hyperlinks (активувати посилання)
% \usepackage{amssymb} % special math characters
% \usepackage{amsmath} % using cyrillic in formulae
% \usepackage{amsfonts} % special math fonts
% \usepackage{eskdtotal} 
% \usepackage{graphicx,epstopdf} % epstopdf-convert eps files to pdf
% \graphicspath{{algorithms/}{schemes/}{software/}{fig/}} % look up folders for figures
% \usepackage{listings} % to add source codes
% \usepackage{longtable} % multipage tables
% \usepackage{multirow} % using rowspan in tables
% \usepackage{nomencl} % support for abbreviations
% \makenomenclature % generate abbrevs index file
% \usepackage{float}
% % variables.tex
% This file contains information about author and other specific
% people for use in eskdx collection.

\title{\fontsize{12}{12} \selectfont Інтегрована інерціально-супутникова система навігації, що базується на принципах комплексної обробки інформації
з використанням калманівської фільтрації}
% smaller size of font set for the title in frame
\author{НовікМ.В.}

\ESKDchecker{ФіляшкінМ.К.}
\ESKDnormContr{КозловА.П.}
\ESKDapprovedBy{СинєглазовВ.М.}

\ESKDdepartment{Міністерство освіти і науки України}
\ESKDcompany{Національний авіаційний університет}

\ESKDsignature{НАУ 11 00 75 000 ПЗ}
% 11 year of defend
% 00 number of thesis
% 75 last two nambers of studens mark book
% 000 must stay 000

\ESKDgroup{ІАСУ 608}

\ESKDsectAlign{section}{Center}
\ESKDsectAlign{subsection}{Center}
\ESKDsectAlign{subsubsection}{Center}

 % class parameters tuning
% \ESKDcolumnXIfIV{РусаловськийА.В.}
% \ESKDstyle{formIIab}
% \renewcommand\labelenumi{\arabic{enumi}.} 
% \renewcommand\labelenumii{\theenumi.\arabic{enumii}.}
% \renewcommand\labelenumiii{\arabic{enumi}.\arabic{enumii}.\arabic{enumiii}.}
% \begin{document}
% \ESKDthisStyle{formII}
\section{Охорона праці}
\subsection{Вступ}

В дипломній роботі розробляється інерціально-супутникова навігаційна система, що базується на основі комплексної обробки інформації з використанням фільтра Калмана. Розробкою та відпрацюванням алгоритмів роботи навігаційної системи, налаштуванням обладнання та калібровкою датчиків займаються інженери програмісти та радіотехніки лабораторії. Отже суб’єктом є інженер програміст лабораторії, функціональним зобов'язаннями якого є програмування навігаційних алгоритмів для бортової обчислюваної машині, засобом праці є персональний комп'ютер та модулі навігаційного обладнання: датчики навігаційної системи (мікромеханічні чи лазерні акселерометри та гіроскопи), бортовий обчислювач навігаційної інформації.

До роботи з навігаційним обладнанням та ЕОМ допускаються працівники, що не мають медичних протипоказань, пройшли вчасно періодичний медичний огляд, інструктаж і навчання  правилам техніки безпеки і виробничої санітарії.

Основним місцем роботи інженера програміста є лабораторія навігаційного обладнання авіаційного підприємства чи науково-дослідного інституту. Періодично місцем роботи може бути літак, де встановлено навігаційне обладнання (налаштування, тестування, випробування) або ЗПС, у випадку, якщо розроблені пристрої встановлюються на БПЛА.

\subsection{Опис робочого місця}
Для приміщення лабораторії вибрана площа 30 $\text{м}^2$, з висотою стелі -- 3м. Виробничі будівлі та приміщення споруджуються згідно з вимогами будівельних і санітарних норм. Об’єм виробничих приміщень для програмістів, операторів відеотермінальних пристроїв на одного працівника складає 19,5 $\text{м}^2$, площа приміщень — 6 м2 з урахуванням максимального числа працівників в одну зміну. 

\begin{figure}[H]
\centering
\includegraphics[scale=0.5]{lab_plan}
\caption{План робочого приміщення}\label{fig:lab_plan}
\end{figure}

\begin{enumerate}
 \item Робоче місце №1;
 \item Координатна платформа;
 \item Робоче місце №2;
 \item Стіл з ПК приєднаний до мережі та до вимірювальної апаратури;
 \item Шафа з допоміжним обладнанням та витратними матеріалами;
 \item Вогнегасник;
 \item Датчик пожежний;
 \item Світильник.
\end{enumerate}

В приміщенні розташовані два робочих місця. Перше обладнане ПК з рідкокристалічним дисплеєм, приєднане до локальної мережі. На столі додатково встановлені програматори, осцилограф, лабораторний блок живлення та мультиметр, телефон, принтер. Друге робоче місце обладнане паяльною станцією, контрольно-вимірною апаратурою, лабораторним блоком живлення, та додатковими розетками з мережею живлення 115В 400Гц. ПК, приєднаний до локальної мережі Ethernet. Над столом знаходиться полиця з радіоелементами. Всі прилади мають бути заземлені.

Трудова діяльність інженера програміста відбувається у певному виробничому середовищі, де діють такі шкідливі фактори: 
\begin{enumerate}
  \item Понижений рівень штучного освітлення 
  \item Шум
  \item Електричний струм
  \item Пожежонебезпека
  \item Мікроклімат

\end{enumerate}

\subsection{Шум}

Допустимі рівні звукового тиску у октавних смугах частот, еквівалентні рівні звуку на робочому місці регламентовані ДСН 3.3.6.037-99, для інженера показані в таблиці \ref{tb:noise}.
\begin{table}[H]
\small
\caption{Нормовані рівні звукового тиску та рівень шуму на робочому місці}
\centering
\begin{tabular}{|p{0.4in}|p{0.4in}|p{0.4in}|p{0.4in}|p{0.4in}|p{0.4in}|p{0.4in}|p{0.4in}|p{0.4in}|p{0.4in}|} \hline 
\multicolumn{9}{|p{4in}|}{Рівні звукового тиску (дБ) в  октавних смугах з серединами геометричними частотами, Гц} & Рівень звуку, дБА \\ \hline 
31.5 & 63 & 125 & 250 & 500 & 1000 & 2000 & 4000 & 8000 & 50 \\ \hline 
86 & 71 & 61 & 54 & 49 & 45 & 42 & 40 & 38 &  \\ \hline 
\end{tabular}
\label{tb:noise}
\end{table}
Методи вимірювання шуму, інфразвуку та ультразвуку регламентовано ДСН 3.3.6.037-99. Вплив шуму на людину, його вимірювання та оцінювання, для присутньої обчислюваної та офісної техніки здійснюється відповідно до ДСТУ ISO 7779:2005, а присутні кондиціонер за ДСТУ 3010-95. 

Якщо лабораторія знаходиться не далеко від аеродрому, необхідна додаткова звукоізоляція. У якості звукоізолюючих матеріалів, які застосовують у конструкціях перекриттів для зниження передачі структурного (ударного) звуку переважно використовують мати та плити із скляного та мінерального волокна, м'які плити з деревних стружок, картон, гуму, утеплений лінолеум.

\subsection{Освітлення}
Особу увагу необхідно приділити важливому з точки зору виробничої санітарії питанню освітлення на робочому місці. Основним документом, який регламентує норми освітленості є ДБН.В.2.5-28-2006 <<Природне і штучне освітлення>>. 

Освітлення на робочому місці повинно бути поєднаним (штучне та природне світло). 
Природне освітлення повинно бути боковим. При виконанні робот з категорії високої 
зорової точності коефіцієнт природної освітленості повинен відповідати нормативним 
рівням по ДБН.В.2.5-28-2006 (не нижче 1,5), при зоровій роботі середньої точності – не нижче 1.

Освітлення повинно бути достатнім, щоб очі без зайвого напруження могли розрізняти деталі, що розглядаються; стабільним – для цього напруга в електричній мережі не повинна коливатися більше ніж на 4 \%; рівномірно розподіленим по робочих поверхнях, щоб очам не доводилося потрапляти з дуже темного місця у світле і навпаки; таким, що не викликає сліпучої дії на око людини як самого джерела світла, так і від відбиваючих поверхонь, що знаходяться в полі зору робітника. 
Зменшення віддзеркалювання джерел світла досягається шляхом застосування світильників; 
таким, щоб не викликати різкі тіні на робочих місцях (цього можна досягти при правильному розташуванні світильників); 
безпечним – не призводити до вибуху, пожежі у виробничих приміщеннях.

Для створення сприятливих умов зорової роботи освітлення робочих приміщень задовольняються наступні умови:
\begin{enumerate}
 \item рівень освітленості робочих місць відповідає гігієнічним нормам для даного виду роботи;
 \item забезпечена рівномірність та часова стабільність рівня освітленості у приміщенні, відсутні різкі контрасти між освітленою робочою зоною та навколишнім простором;
 \item відсутні різкі та рухомі тіні;
 \item у полі зору предмета немає сліпучого блиску
 \item штучне світло за спектральним складом наближається до природного
\end{enumerate}

Розрахунок виробничого освітлення зроблено за методом використання 
світлового потоку. За цім методом світловий потік однієї лампи (у люменах) визначається за формулою:
\begin{equation}
\label{eq:fnop}
 F_n = \frac{E_{n}SkZ}{n \eta}
\end{equation}
\begin{ESKDexplanation}
\item де $E_n$ -- нормована освітленість для проектованих ділянок, цехів, лабораторій;
\item S – площа приміщення, у якому проектується виробниче освітлення, м2;
\item k – коефіцієнт запасу світлового потоку. Він приймається: для люмінесцентних ламп при малому виділенні пилу, диму, кіптяви - 1,5, при середньому і великому виділенні відповідно -1,8 й 2,0; для ламп накалювання при малому виділенні пилу, диму, кіптяви - 1,3, при середньому й великому, відповідно - 1,5 й 1,7;
\item Z – поправочний коефіцієнт, що відбиває відношення  , приймається при найвигіднішому розташуванні світильників, коли світловий потік використається для освітлення робочої зони найбільш раціонально, рівним 1,1-1,2; 
\item n – число ламп в приміщенні;
\item  $\eta$ – коефіцієнт використання світового потоку від світильника, що показує, яка частина світлового потоку лампи   досягає освітлюваної поверхні, у тому числі завдяки відбиттю світлового потоку від стін, стелі й робочої поверхні.
Коефіцієнт $\eta$, що залежить від показника геометричних розмірів приміщення   і коефіцієнтів відбиття стін  , стелі   і приміщення  , обчислений для різних типів світильників.	  
\end{ESKDexplanation}

Показник  приміщення:
\begin{equation}
\label{eq:iop}
 i = \frac{a \times b}{H(a+b)}= \frac{5 \times 6}{1,925(6+5)}
\end{equation}
\begin{ESKDexplanation}
  \item де a та b – довжина й ширина освітлюваного приміщення, м;  
  \item H – висота підвісу світильників над робочою поверхнею, м.
\end{ESKDexplanation}

Висота підвісу знаходить з наступної формули:
\begin{equation}
\label{eq:hop}
 H =h_{b} - N - h_{c} 
\end{equation}   
\begin{ESKDexplanation}
\item де $h_{b}$ – висота приміщення, м;
\item $h_{с}$ – висота світильника, м;
\item N – висота робочої поверхні, м.
\end{ESKDexplanation}
Приміщення лабораторії має наступні геометричні формули: довжина робочого кабінету складає 6 м;
ширина - 5 м; висота - 3 м. Визначимо висоту підвісу світильників, підставив вихідні значення в формулу \ref{eq:hop}:
\begin{equation}
 H =   3,0 - 0,275 - 0,8 = 1,925(m).
\end{equation}
тоді індекс приміщення:
\begin{equation*}
 i =  \frac{5 \times 6}{1,925(6+5)}= 1.41676
\end{equation*}            

Далі визначимо значення показника приміщення, підставляючи в формулу значення :       
По показнику приміщення та коефіцієнтам світлового потоку від підлоги – 10\% (0,1), від стін – 30\% (0,3) та від стелі – 80\% (0,8)
визначаємо для світлодіодної лампи ML-T8-13W/0.6-SMD значення коефіцієнта використання світлового потоку ($\eta$). $\eta$ = 0,69; коефіцієнт запасу світлового потоку k = 1,25; поправочний коефіцієнт Z = 1,2. 

Норма (мінімум) освітленості при проведенні середньо точних робіт складає 400 лк.Світловий потік лампи ML-T8-13W/0.6-SMD складає 1950 лм. З формули \ref{eq:fnop} виразимо число ламп в приміщенні та підставляючи відомі значення в вираз одержимо
 
\begin{equation}
 n = \frac{400 \times 30 \times 1.25 \times 1.2 }{1750 \times 0.69} = 18
\end{equation}

Округляючи значення до більшої цілої цифри, отримуємо, що вимагається 18 ламп. Якщо в світильник 2 лампи то нам необхідно 9 світильників, які необхідно розмістити рівномірно у три ряди по три світильника.

Приміщення, призначені для роботи ПК, повинні мати природне освітлення. Орієнтація вікон повинна бути на північ або на північний схід, вікна повинні мати жалюзі, які можна регулювати, або штори.

\subsection{Електробезпека}

Відповідно до НАОП 0.00-1.28-10 Правила охорони праці під час експлуатації електронно-обчислювальних машин, ЕОМ з ВДТ і ПП, інше устаткування (апарати управління, контрольно-вимірювальні прилади, світильники), електропроводи та кабелі за виконанням і ступенем захисту мають відповідати класу зони за НПАОП 40.1-1.01-97, мати апаратуру захисту від струму короткого замикання та інших аварійних режимів.

Під час монтажу та експлуатації ліній електромережі необхідно повністю унеможливити виникнення електричного джерела загоряння внаслідок короткого замикання та перевантаження проводів, обмежувати застосування проводів з легкозаймистою ізоляцією і, за можливості, застосовувати негорючу ізоляцію.

Під час ремонту ліній електромережі шляхом зварювання, паяння та з використанням відкритого вогню необхідно дотримуватися НАПБ А.01.001-2004.

Лінія електромережі для живлення ЕОМ з ВДТ і ПП виконується як окрема групова трипровідна мережа шляхом прокладання фазового, нульового робочого та нульового захисного провідників. Нульовий захисний провідник використовується для заземлення (занулення) електроприймачів.
Не допускається використовувати нульовий робочий провідник як нульовий захисний провідник.

Нульовий захисний провідник прокладається від стійки групового розподільного щита, розподільного пункту до розеток електроживлення.

Не допускається підключати на щиті до одного контактного затискача нульовий робочий та нульовий захисний провідники.

Площа перерізу нульового робочого та нульового захисного провідника в груповій трипровідній мережі має бути не менше площі перерізу фазового провідника. Усі провідники мають відповідати номінальним параметрам мережі та навантаження, умовам навколишнього середовища, умовам розподілу провідників, температурному режиму та типам апаратури захисту, вимогам НПАОП 40.1-1.01-97.

У приміщенні, де одночасно експлуатуються понад п'ять ЕОМ з ВДТ і ПП, на помітному та доступному місці встановлюється аварійний резервний вимикач, який може повністю вимкнути електричне живлення приміщення, крім освітлення.

ЕОМ з ВДТ і ПП повинні підключатися до електромережі тільки за допомогою справних штепсельних з'єднань і електророзеток заводського виготовлення.
У штепсельних з'єднаннях та електророзетках, крім контактів фазового та нульового робочого провідників, мають бути спеціальні контакти для підключення нульового захисного провідника. Їхня конструкція має бути такою, щоб приєднання нульового захисного провідника відбувалося раніше, ніж приєднання фазового та нульового робочого провідників. Порядок роз'єднання при відключенні має бути зворотним.

Не допускається підключати ЕОМ з ВДТ і ПП до звичайної двопровідної електромережі, в тому числі - з використанням перехідних пристроїв.

Електромережі штепсельних з'єднань та електророзеток для живлення ЕОМ з ВДТ і ПП потрібно виконувати за магістральною схемою, по 3-6 з'єднань або електророзеток в одному колі.

Штепсельні з'єднання та електророзетки для напруги 12 В та 42 В за своєю конструкцією мають відрізнятися від штепсельних з'єднань для напруги 127 В та 220 В.
Штепсельні з'єднання та електророзетки, розраховані на напругу 12 В та 42 В, мають візуально (за кольором) відрізнятися від кольору штепсельних з'єднань, розрахованих на напругу 127 В та 220 В.

Індивідуальні та групові штепсельні з'єднання та електророзетки необхідно монтувати на негорючих або важкогорючих пластинах з урахуванням вимог НПАОП 40.1-1.01-97 та НАПБ А.01.001-2004.

Електромережу штепсельних розеток для живлення ЕОМ з ВДТ і ПП при розташуванні їх уздовж стін приміщення прокладають по підлозі поруч зі стінами приміщення, як правило, в металевих трубах і гнучких металевих рукавах, а також у пластикових коробах і пластмасових рукавах з відводами відповідно до затвердженого плану розміщення обладнання та технічних характеристик обладнання.
При розміщенні в приміщенні до п'яти ЕОМ з ВДТ і ПП допускається прокладання трипровідникового захищеного проводу або кабелю в оболонці з негорючого чи важкогорючого матеріалу по периметру приміщення без металевих труб та гнучких металевих рукавів.

При організації робочих місць операторів електромережу штепсельних розеток для живлення ЕОМ з ВДТ і ПП у центрі приміщення прокладають у каналах або під знімною підлогою в металевих трубах або гнучких металевих рукавах. При цьому не допускається застосовувати провід і кабель в ізоляції з вулканізованої гуми та інші матеріали, які містять сірку.

\subsection{Забезпечення пожежної безпеки в розроблювальному проекті}
Пожежна безпека забезпечена у відповідності з НАПБ А.01.001-2004 <<Правила пожежної безпеки в Україні>>, який є обов'язковим для виконання всіма підприємствами не залежно від форми власності. Правила встановлюють загальні вимоги з пожежної безпеки. Забезпечуючи пожежну безпеку, слід також керуватись ПУЕ та НПАОП 40.1-1.32-01 <<Правила побудови електроустановок.Електрообладнання спеціальних установок>> та інших нормативних документів, що стосуються штучного освітлення і електротехнічних пристроїв, а також вимог нормативно-технічної експлуатаційної документації заводу-виробника.

Робоче приміщення лабораторії за класифікацією пожежонебезпечності має відноситься до категорії Д.

Забезпечення пожежної безпеки в лабораторії досягається за рахунок застосування мір пожежної профілактики 
й активного пожежного захисту, тобто комплексу мір попередження виникнення пожеж або зменшення їх наслідків. Причинами виникнення пожежі електроустаткування можуть бути:
\begin{enumerate}
 \item перевантаження проводів;
 \item неякісне виконання з'єднань електропроводки;
 \item перевантаження різних електричних пристроїв;
 \item коротке замикання
 \item контакт горючих речовин з нагрівальними пристроями.
\end{enumerate}

Джерела електричної енергії (розподільчі пристрої, трансформатори) розташовувані у відокремлених приміщеннях.

Освітлювальну електричну мережу виконано згідно вимог ПЕУ – правилам устрою електроустановок для пожежонебезпечних зон.
Прокладання кабелю через перекриття, стіни, фальшпідлогу здійснено в стальних трубах з наповнювачем з негорючих матеріалів. Аварійні мережі освітлення, дистанційного та автоматичного пуску протипожежних систем та сигналізації 
прокладено окремо від силових та інших електричних комунікацій, а при сумісному прокладанні їх 
розділено перегородками з негорючих матеріалів (метал, гетинакс).


Повітропроводи виконані з негорючих матеріалів. Система вентиляції обладнана пристроєм, що 
забезпечує автоматичне її відключення, а також перекриття повітропроводів лабораторії 
автоматичними заслінками в разі виникнення пожежі. Кабельні вертикальні шахти розділені 
по поверхах діафрагмами з негорючих матеріалів.

Ефективність застосування вогнегасника, у першу чергу пов’язана з правильним вибором його типу залежно від класу пожежі, яку не необхідно погасити. Основні вимоги до оснащення об’єктів вогнегасниками регламентуються НАПБ Б.03.001-2004 Типові норми належності вогнегасників. При експлуатації вогнегасників слід керуватись НАПБ Б.01.008-2004 Правила експлуатації вогнегасників.

Для гасіння та локалізації пожежі до прибуття пожежних підрозділів використовуються ручні вогнегасники.  У приміщенні необхідний 1  вуглекислотний вогнегасник типу ВВ (ВВ-2, ВВ-5, ВВ-8). Застосування вуглекислотних вогнегасників зумовлено тим, що вони можуть використовуватися для гасіння дорогого обладнання, яке знаходиться під напругою до 1000 В. 

Для виявлення пожежі використовують пожежно-охоронну сигналізацію, у відповідності до ДСТУ EN54-2:2003. Пожежні сповіщувачі використовуються для формування командного імпульсу автоматичного пуску системи автоматичного пожежегасіння. Кількість теплових пожежних сповіщувачів визначається за таблицею і для приміщення розмірами 6 х 5 х 3 м становить 2. 
Температура спрацювання сповіщувачів встановлюється не менше ніж на $20^o$С вище 
максимальної припустимої температури в приміщенні.

Для виявлення пожежі, замість старик точкових пристроїв пропонується встановити датчики Honeywell Notifier SFAPT-453(A)
(Acclimate PlusTM Multi-Sensor Low-Profile Smoke Intelligent Detector). Датчик використовує аналізатор диму та комбінацію фотоелектричних та температурних сенсорів з вмонтованим для підвищення імунітету до фальшивого спрацювання. Пристрій обладнаний мікропроцесором для обробки інформації, в результаті він налаштовує чутливість автоматично не залежно від оператора контрольної панелі та проводить самотестування. Іншою перевагою, даного типу датчика є, його безпосередня зв'язаність з системою кондиціювання та вентиляції. В разі пожежі вимикається та блокується кондиціонер,
і закриваються заслонки вентиляційної системи.

Дані з сенсорів подаються на загальну панель керування пожежно-охоронної системи сигналізації, а далі на пост чергового пожежної частини. Оператор або панель керування автоматично приймають рішення, щодо вимкення постачання електроенергії
до приміщення лабораторії

Евакуація здійснюється відповідно до НАПБ А.01.003-2009 "Правила улаштування та експлуатації систем оповіщення про пожежу та управління евакуацією людей в будинках та спорудах". Комплекс виробничих приміщень має два евакуаційних виходи. Двері на шляхах евакуації мають відчинятися у напрямку виходу зі споруди, ширина шляхів -- не менше 1м, а ширина дверей -- 0.8м. Шляхи евакуації показані на рисунку \ref{fig:lab_evac}.

\begin{figure}[H]
\centering
\includegraphics[scale=0.4]{lab_plan3}
\caption{План евакуації}\label{fig:lab_evac}
\end{figure}


\subsection{Висновок}
В роботі проаналізовано основні небезпечні чинники, можна відзначити, що при дотриманні правил безпеки і виробничої санітарії обладнання навігаційної системи не є пожежонебезпечним. Запропоновані світлодіодні світильники мають строк служби 50 тисяч годин, що значно краще ніж у люмінесцентних ламп, де строк рівний 10 - 20 тисяч годин, і крім того залежить від кількості переключень. З іншого боку світильники є економічнішими на 44.4 \% (світлодіодна лампа 20 +/- 1 Вт, люмінісцентна 36 +/-1Вт), більш ударостійкі, не містять токсичних речовин і не мають спеціальних вимог щодо утилізації. Ці лампи створюють оптимальні умови для зорової роботи інженера програміста, а порівняно не висока температура нагрівання підвищує рівень пожежної безпеки.

% \end{document}

% Chapter 7 % Ecology
\newpage
\ESKDthisStyle{formII}
\include{chapter7}
% 
%% Last chapter %General Conclusions
\newpage
\ESKDthisStyle{formIIab}
\section*{Висновки} \addcontentsline{toc}{section}{Висновки}
% В результаті роботи:
% \begin{enumerate}
%  \item проаналізовано основні схеми комплексування БІНС та СНС, запропоновано слабко зв’язану схему;
%  \item опрацьовані та обґрунтовані основні алгоритми роботи БІНС, на основі наступних датчиків: лазерних гіроскопів та кварцевих акселерометрів, отримані і запропоновані рівняння еволюції похибок БІНС, СНС та БВ;
%  \item проведено дослідження еволюцій похибок стаціонарно закріпленої БІНС, з метою перевірки адекватності моделі;
%  \item розроблено модель ІСНС, для оцінки похибок системи застосовано оптимальний лінійний Фільтр Калмана;
%  \item змодельовано роботу комплексної ІСНС,  перевірені похибки оцінки навігаційних параметрів: приведених координат, швидкості, кутової орієнтації;
%  \item перевірено еволюцію похибок оцінок у випадку радіомовчання СНС протягом 200с.
% \end{enumerate}
% 
% Аналіз результатів моделювання доказує працездатність розроблених алгоритмів комплексування. Отримані дані відповідають критеріям заданим в технічному завданні. Помилка по координаті не перевищує 6 метрів, по швидкості не більше 0.03 м/с, що відповідає рівню точності СНС.
% 
% Для покращення роботи в майбутньому запропоновано наступні вдосконалення:
% \begin{enumerate}
%  \item провести дослідження залежності спостережності навігаційних параметрів в залежності від маневрів ЛА;
%  \item в розробленому програмному забезпеченні реалізувати основні види фільтра Калмана: Шмідта, Карлсона, Бірмана та модифікацій Тронтона;
%  \item додати можливість моделювання в реальному часі.
% \end{enumerate}


\begin{enumerate}
\item  Розроблені лінійні моделі еволюції похибок БІНС, а також новітні моделі похибок супутникової системи навігації, які включають неконтрольовані джерела похибок у тому числі, ті що виникають у момент зміни сузір'я навігаційних супутників, підвищують адекватність опису цих помилок, що, у свою чергу, забезпечує коректність розв`язання задачі оптимальної калмановської фільтрації. 

\item  Розроблені для інтегрованих інерціально-супутникових навігаційних систем алгоритми комплексної обробки навігаційної інформації на основі процедур оптимальної лінійної калмановської фільтрації дозволяють розв`язувати задачі оцінювання похибок однієї підсистеми на фоні похибок іншої підсистеми.

\item  Аналіз результатів моделювання доводить працездатність розроблених алгоритмів комплексування. Отримані дані відповідають критеріям заданим в технічному завданні.
\item  Перевірено еволюцію похибок оцінок у випадку радіомовчання СНС протягом 200с, фільтр Калмана в цьому випадку продовжує можливий час автономної роботи БІНС.
\item  Розроблений моделюючий комплекс, який містить не тільки моделі алгоритмів комплексної системи, але й взаємозв`язані моделі підсистем, моделі датчиків, а також модель динаміки польоту літака разом із системою керування траєкторним і кутовим рухом дозволяє проводити комплексні дослідження отриманих за запропонованими методиками алгоритмів інерціально-супутникової навігаційної системи шляхом математичного моделювання. 

\item  Аналіз результатів моделювання доказує працездатність розроблених алгоритмів комплексування. Отримані дані відповідають критеріям заданим в технічному завданні. Помилка по координаті не перевищує 6 метрів, по швидкості не більше 0.03 м/с, що відповідає рівню точності СНС.
\end{enumerate}

% 
% % Print reference page
\newpage
\ESKDthisStyle{formIIab}
\begin{thebibliography}{99}
\bibitem{bib:1} М.К. Філяшкін В.О. Рогожин, А.В. Скрипець, Т.І. Лукінова Інерціально-супутникові навігаційні  системи. - К.: Вид-во НАУ, 2009. - 306 с.
\bibitem{bib:7} Ільін О.Ю., Філяшкін М.К., Черних Ю.О. Пілотажно-навігаційні системи та комплекси. - К.: Вид-во КІ ВПС, 1999. - 335 с.
\bibitem{bib:8} Интегрированные инерциально-спутниковые навигационные системы // Под ред. В.А. Пешехонова. -  С.-Петербург: 2001. - 235 с.
\bibitem{bib:9} Интегрированные комплексы на базе ИНС и приемника «Навстар» // Новости зарубежной науки и техники, Серия «авиационные системы». ГосНИИАС, 1995, №10-12.
\bibitem{bib:10} Кузовков Н.Т., Салычев О.С. Инерциальная навигация и оптимальная фильтрация. - М.: Машиностроение, 1988.  - 216 с.
\bibitem{bib:pnk} Рогожин В.О., Синєглазов В.М., Філяшкін М.К. Пілотажно-навігаційні комплекси повітряних суден -К.: Вид-во НАУ, 2005. - 316с.

\bibitem{bib:gps_ins} Grewal M. S. Global Positioning Systems, Inertial Navigation and Integration / M. S. Grewal, L.R.Weill, A. P. Andrews. - A John Wiley and Sons, Inc., Publication, 2007. - 525 p.
\bibitem{bib:12} Соловьев Ю.А. Системы спутниковой навигации - М.: ЭКО-ТРЕНДЗ, 2000.  - 270 с. 
\bibitem{bib:gps1} J. Zander, B. Slimane, and L. Ahlin, Principles of Wireless Communications.// Stockholm: Royal Institute of Technology, 2005.
\bibitem{bib:gps2} A. El-Rabbany, Introduction to GPS - The Global Positioning System // London: Artech House, 2002
\bibitem{bib:gps3} USCG Navigation Center, GPS Standard Positioning Service - Performance Standard, // USCG Navigation Center, Oct. 2001.
\bibitem{bib:gps4} B. W. Parkinson and J. J. Spilker, Global Positioning System: Theory and
 Applications. Progress in Astronautics and Aeronautics, 1996, vol. 163.
\bibitem{bib:joseph} R. S. Bucy and P. D. Joseph, Filtering for Stochastic Processes, with Applications to Guidance, Wiley, New York, 1968.
\bibitem{bib:kalman_1} R. E. Kalman, A new approach to linear filtering and prediction problems, ASME Journal of Basic Engineering, Vol. 82, pp. 34-45, 1960.
\bibitem{bib:kalman_2} R. E. Kalman, New methods in Wiener filtering,  in Proceeding of the First Symposium on Engineering Applications of 
Random Function Theory and Probability, Wiley, New York, 1963.
\bibitem{bib:2} Авиационные приборы и навигационные системы // Под  ред. Бабича О.А. - М.: Изд-во ВВИА им. проф. Н.Е. Жуковского, 1981. - 648 с.
\bibitem{bib:3} Бабич О.А. Обработка информации в навигационных комплексах. - М.: Машиностроение, 1988.  - 212 с.
\bibitem{bib:4} Власенко А. В. Интегральные гироскопы iMEMS - датчики угловой скорости фирмы Analog Devices (Интернет-издание), 2006.
\bibitem{bib:5} Воробьев В.Г., Глухов В.В., Кадышев И.К. Авиационные приборы, информационно-измерительные системы и комплексы. - М.: Транспорт, 1992.  - 399 с.
\bibitem{bib:6} Глобальная спутниковая навигационная система ГЛОНАСС // под ред. В.Н. Харисова, А.И.Петрова, В.А.Болдина. - М.: ИПРЖР, 1998.  - 400 с. 
\end{thebibliography}

% \bibitem{bib:13} E. Anderson, Z. Bai, C. Bischoff. LAPACK: A portable linear algebra package for high-performance computers. //In Proceedings of Supercomputing '90, pages 1-10. IEEE Press, 1990. 
% \bibitem{bib:14} Netlib BLAS --- http://www.netlib.org/blas/index.html.
% \bibitem{bib:15} Blitz++ C++ Class Library for Scientific Computing --- http://oonumerics.org/blitz, 1996.

% \bibitem{bib:14) A. El-Rabbany, Introduction to GPS - The Global Positioning System.// London: Artech House, 2002.
% \bibitem{bib:15) USCG Navigation Center, GPS Standard Positioning Service - Performance Standard,// USCG Navigation Center, Oct. 2001.


% \bibitem{bib:13} Li P. GNSS/Pseudolite Signal Re-Acquisition with the aid of INS in Short Signal Blockage Scenarios http://www.gmat.unsw.edu.au/snap/publications/lip_etal2008a.pdf
% \bibitem{bib:16} Wang J. Integration of GPS/INS/Vision sensors to navigate Unmanned Aerial Vehicles /Wang J., Garratt M., Lambert A. and other // Proceedings of XXI Congress of the Int. Society of Photogrammetry and Remote Sensing. - Beijing, China, 2008. - P. 963 - 970.
% \bibitem{bib:18} Bhatti U. I. Improved integrity algorithms for integrated GPS/INS systems in the presence of slowly growing errors: PhD thesis / U. I. Bhatti. - Department of Civil and Environmental Engineering. - Imperial College London, United Kingdom, 2007. - P. 363.
% \bibitem{bib:19} Titterton, D. H. Strapdown Inertial Navigation Technology, Peter Peregrinus Press. - London, 2004. - 549 p.16. 
% \bibitem{bib:20} Grewal M. S. Global Positioning Systems, Inertial Navigation and Integration / M. S. Grewal,L. R. Weill, A. P. Andrews. - A John Wiley & Sons, Inc., Publication, 2007. - 525 p.

% Appendix A
\newpage
\ESKDthisStyle{formIIab}
\ESKDappendix{}{Принцип роботи автоматизованої системи діагностування}
\centering 
\includegraphics[height=0.8\textheight]{base_algorithm}
\label{general diagnostics sysem structure}
% Appendix B
% \newpage
% \ESKDthisStyle{formIIab}
% \ESKDappendix{}{Код програми в середовищі Mathlab}
\lstset{ %
basicstyle=\tiny,               % print whole listing small
language=Matlab,                % choose the language of the code
tabsize=4,			% sets default tabsize to 2 spaces
captionpos=t,			% sets the caption-position to bottom
breaklines=true,		% sets automatic line breaking
breakatwhitespace=true, 	% sets if automatic breaks should only happen at whitespace
numbers=left,                   % where to put the line-numbers
escapeinside={\%*}{*)}         % if you want to add a comment within your code
}

\subsection*{Тестування алгоритмів в середовищі Mathlab}
\mcode{
\lstinputlisting[caption=flight_path.m]{/home/phenom/kalman/src/flight_path.m}
}


\mcode{/home/phenom/kalman/src/flight_path.m}
% \lstinputlisting[caption=model_isns.m]{/home/phenom/kalman/src/model_isns.m}

% /home/phenom/kalman/src
% flight_path.m
% model_isns.m
% init_Fins.m



\end{document}

