\section*{Висновки} \addcontentsline{toc}{section}{Висновки}
% В результаті роботи:
% \begin{enumerate}
%  \item проаналізовано основні схеми комплексування БІНС та СНС, запропоновано слабко зв’язану схему;
%  \item опрацьовані та обґрунтовані основні алгоритми роботи БІНС, на основі наступних датчиків: лазерних гіроскопів та кварцевих акселерометрів, отримані і запропоновані рівняння еволюції похибок БІНС, СНС та БВ;
%  \item проведено дослідження еволюцій похибок стаціонарно закріпленої БІНС, з метою перевірки адекватності моделі;
%  \item розроблено модель ІСНС, для оцінки похибок системи застосовано оптимальний лінійний Фільтр Калмана;
%  \item змодельовано роботу комплексної ІСНС,  перевірені похибки оцінки навігаційних параметрів: приведених координат, швидкості, кутової орієнтації;
%  \item перевірено еволюцію похибок оцінок у випадку радіомовчання СНС протягом 200с.
% \end{enumerate}
% 
% Аналіз результатів моделювання доказує працездатність розроблених алгоритмів комплексування. Отримані дані відповідають критеріям заданим в технічному завданні. Помилка по координаті не перевищує 6 метрів, по швидкості не більше 0.03 м/с, що відповідає рівню точності СНС.
% 
% Для покращення роботи в майбутньому запропоновано наступні вдосконалення:
% \begin{enumerate}
%  \item провести дослідження залежності спостережності навігаційних параметрів в залежності від маневрів ЛА;
%  \item в розробленому програмному забезпеченні реалізувати основні види фільтра Калмана: Шмідта, Карлсона, Бірмана та модифікацій Тронтона;
%  \item додати можливість моделювання в реальному часі.
% \end{enumerate}


\begin{enumerate}
\item  Розроблені лінійні моделі еволюції похибок БІНС, а також новітні моделі похибок супутникової системи навігації, які включають неконтрольовані джерела похибок у тому числі, ті що виникають у момент зміни сузір'я навігаційних супутників, підвищують адекватність опису цих помилок, що, у свою чергу, забезпечує коректність розв`язання задачі оптимальної калмановської фільтрації. 

\item  Розроблені для інтегрованих інерціально-супутникових навігаційних систем алгоритми комплексної обробки навігаційної інформації на основі процедур оптимальної лінійної калмановської фільтрації дозволяють розв`язувати задачі оцінювання похибок однієї підсистеми на фоні похибок іншої підсистеми.

\item  Аналіз результатів моделювання доводить працездатність розроблених алгоритмів комплексування. Отримані дані відповідають критеріям заданим в технічному завданні.
\item  Перевірено еволюцію похибок оцінок у випадку радіомовчання СНС протягом 200с, фільтр Калмана в цьому випадку продовжує можливий час автономної роботи БІНС.
\item  Розроблений моделюючий комплекс, який містить не тільки моделі алгоритмів комплексної системи, але й взаємозв`язані моделі підсистем, моделі датчиків, а також модель динаміки польоту літака разом із системою керування траєкторним і кутовим рухом дозволяє проводити комплексні дослідження отриманих за запропонованими методиками алгоритмів інерціально-супутникової навігаційної системи шляхом математичного моделювання. 

\item  Аналіз результатів моделювання доказує працездатність розроблених алгоритмів комплексування. Отримані дані відповідають критеріям заданим в технічному завданні. Помилка по координаті не перевищує 6 метрів, по швидкості не більше 0.03 м/с, що відповідає рівню точності СНС.
\end{enumerate}
