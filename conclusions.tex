\section*{Висновки} \addcontentsline{toc}{section}{Висновки}
В результаті роботи:
\begin{enumerate}
 \item проаналізовано основні схеми комплексування БІНС та СНС, запропоновано слабко зв’язану схему;
 \item опрацьовані та обґрунтовані основні алгоритми роботи БІНС, на основі наступних датчиків: лазерних гіроскопів та кварцевих акселерометрів, отримані і запропоновані рівняння еволюції похибок БІНС, СНС та БВ;
 \item проведено дослідження еволюцій похибок стаціонарно закріпленої БІНС, з метою перевірки адекватності моделі;
 \item розроблено модель ІСНС, для оцінки похибок системи застосовано оптимальний лінійний Фільтр Калмана;
 \item змодельовано роботу комплексної ІСНС,  перевірені похибки оцінки навігаційних параметрів: приведених координат, швидкості, кутової орієнтації;
 \item перевірено еволюцію похибок оцінок у випадку радіомовчання СНС протягом 200с.
\end{enumerate}

Аналіз результатів моделювання доказує працездатність розроблених алгоритмів комплексування. Отримані дані відповідають критеріям заданим в технічному завданні. Помилка по координаті не перевищує 6 метрів, по швидкості не більше 0.03 м/с, що відповідає рівню точності СНС.

Для покращення роботи в майбутньому запропоновано наступні вдосконалення:
\begin{enumerate}
 \item провести дослідження залежності спостережності навігаційних параметрів в залежності від маневрів ЛА;
 \item в розробленому програмному забезпеченні реалізувати основні види фільтра Калмана: Шмідта, Карлсона, Бірмана та модифікацій Тронтона;
 \item додати можливість моделювання в реальному часі.
\end{enumerate}