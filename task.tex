% \documentclass[ukrainian,utf8,simple,floatsingle,hpadding=5mm]{eskdtext}
% \usepackage[numberright]{eskdplain}
% % variables.tex
% This file contains information about author and other specific
% people for use in eskdx collection.

\title{\fontsize{12}{12} \selectfont Інтегрована інерціально-супутникова система навігації, що базується на принципах комплексної обробки інформації
з використанням калманівської фільтрації}
% smaller size of font set for the title in frame
\author{НовікМ.В.}

\ESKDchecker{ФіляшкінМ.К.}
\ESKDnormContr{КозловА.П.}
\ESKDapprovedBy{СинєглазовВ.М.}

\ESKDdepartment{Міністерство освіти і науки України}
\ESKDcompany{Національний авіаційний університет}

\ESKDsignature{НАУ 11 09 02 000 ПЗ}
\ESKDgroup{ІАСУ 608}

\ESKDsectAlign{section}{Center}
\ESKDsectAlign{subsection}{Center}
\ESKDsectAlign{subsubsection}{Center}


% \include{textcomp}
% \usepackage{longtable} % multipage tables
% \usepackage{multirow} % using rowspan in tables
% % Enumerations in listings use arabic numbers insted of letters
% \renewcommand\labelenumi{\arabic{enumi}.} 
% \renewcommand\labelenumii{\theenumi.\arabic{enumii}.}
% \renewcommand\labelenumiii{\arabic{enumi}.\arabic{enumii}.\arabic{enumiii}.}
% \ESKDstyle{formIIab}
% \begin{document}
% 
% \ESKDthisStyle{formII}


\subsection*{Технічне завдання}

\subsubsection*{1. Найменування та галузь застосування}

% В теперішній час визнано, що одним з основних шляхів вдосконалення навігаційного обладнання 
% є створення комплексних навігаційних систем, що інтегрують  командні прибори в єдиний блок, 
% здатний автоматично вирішувати задачу навігації ЛА. Сутність комплексування полягає у 
% використанні інформаційної та структурної надмірності для підвищення точності, надійності 
% та завадостійкості інформації при вимірюванні одних і тих же або функціонально зв’язаних 
% навігаційних параметрів. Інформаційна надмірність полягає в тому, що забезпечується отримання 
% однорідної інформації від декількох навігаційних датчиків різної фізичної природи з наступною 
% сумісною обробкою цієї інформації в спеціалізованому обчислювачі. Надмірність структури 
% комплексу забезпечує його працездатність при відмові, особливо короткотривалій, одного із датчиків. 
% 


Досить актуальною на даний час є задача створення комплексної навігаційної системи на базі супутникової та інерціальної систем навігації для визначення координат місця розташування рухливого об’єкта, у тому числі ЛА. Однією з центральних ідей розвитку навігаційного обладнання літальних апаратів  є функціональне, інформаційне й апаратурне об'єднання навігаційних вимірників в інтегрований навігаційний комплекс. Використання інтегрованих інерціально-супутникових систем компенсує недоліки окремих систем, і забезпечує високу точність і надійність виміру параметрів польоту.

Задача створення комплексної навігаційної системи на базі супутникової та інерціальної систем навігації длявизначення координат місце-положення рухомого об'єкта, передбачає попередній аналіз існуючих варіантів компонентів комплексної навігаційної системи, тобто варіантів побудови супутникової й  інерціальної систем навігації та вибір за певними критеріями найбільш оптимальних. Природно, що за головний критерій повинно бути обрано вартість системи та задана точність визначення координат рухомого об’єкта.

Найбільш привабливим для розв’язання цієї задачі є залучення калманівської фільтрації. 
Фільтр Калмана призначений для ідентифікації (оцінювання) змінних стану системи за даними 
вимірювання вихідних сигналів цієї системи, які містять похибки вимірювання (вимірювальний шум). 
Ідентифікація оптимальна в тому смислі, що сума квадратів похибок оцінювання змінних стану
в будь-який момент часу має найменше з можливих значень. Похибка оцінювання це різниця 
між оцінкою фільтра й дійсним значенням змінних стану системи при наявності в системі 
детермінованих і випадкових похибок вимірювань. Отже, фільтр Калмана призначений для 
найкращого  відновлення змінних стану, тобто для оптимального приглушення вимірювальних шумів.

\subsubsection*{2. Підстава до розробки}

Наказ по Національному авіаційному університету \No2583/ст від <<19>> жовтня 2010 р.

\subsubsection*{3. Мета та призначення розробки}

Основною метою роботи є аналіз та вибір схеми комплексної інерціально-супутникової 
навігаційної системи та схем оцінювання та корекції в цій системі і, як наслідок, 
розробка слабко зв’язаної схеми інтеграції, що базується на принципах комплексної 
обробки інформації з використанням калманівської фільтраці, дослідження ступеню 
впливу похибок датчиків первинної інформації  безплатформної інерціальної системи 
(БІНС) та супутникової навігаційної системи (СНС) на стійкість фільтра Калмана, 
точнісні характеристики числення навігаційних параметрів і динаміку зміни похибок, 
впливу перерв у роботі СНС на траекторний рух ЛА, моделювання зміни похибок 
комплексної інерціально-супутникової навігаційної системи.

\subsubsection*{4. Технічні вимоги}

Основні технічні вимоги:
\begin{enumerate}
 \item точність визначення навігаційних параметрів:\\
 координат (СКВ), м \dotfill 6\\
  висоти (СКО), м \dotfill $10$
 \item характеристики повинні зберігатися при:\\
швидкості, м/с до \dotfill 400
 \item час визначення (холодний старт), хв \dotfill <2
 \item частота відновлення координат, c$^{-1}$ \dotfill <1
 \item масса, кг \dotfill <5
 \item автоматичне, безперервне, всепогодне визначення поточних 3D координат місця розташування, вектора шляхової швидкості і шляхового кута ЛА.
 \item автоматичний тестовий контроль функціонування блоків і вузлів апаратури, індикація блоків, що відмовили
 \item стійке визначення навігаційних параметрів при русі з лінійними прискореннями і при стрибкоподібних змінах прискорення
 \item БІНС повинна забезпечити визначення координат на протязі 200 с
 \item взаємна корекція СНС та БІНС
 \item підтримка СНС від БІНС для зменшення часу повторного запуску (“гарячого старту”) при короткочасних перервах у роботі СНС.
\end{enumerate}

Вимоги до засобів захисту:
\begin{enumerate}
 \item робоча температура, C     \dotfill -40...+60
 \item робоча вологість (25 C)   \dotfill 98 
\end{enumerate}

Додаткові вимоги:
\begin{enumerate}
 \item швидкість польоту ЛА, м/с        \dotfill 40.0 
 \item максимальний кут крену ЛА,град   \dotfill 45.0
 \item максимальний кут тангажу ЛА,град \dotfill 20.0
\end{enumerate}

\subsubsection*{5. Джерела розробки}
\begin{enumerate}
 \item Хоздоговірна науково-дослідна робота № 201-Хд04 “Ресурс”: “Розробка 
попередніх алгоритмів роботи інерціально-супутникової навігаційної 
системи та інформаційного зв'язку з літаком-носієм ”.
 \item М.К. Філяшкін В.О. Рогожин, А.В. Скрипець, Т.І. Лукінова 
Інерціально-супутникові навігаційні  системи. – К.: Вид-во НАУ, 2009. – 306 с.
 \item Науково-дослідна робота  № 396 ДБ-07 : Методика побудови 
комплексної навігаційної системи на основі спрощеного варіанту 
безплатформної інерціальної та високоточної супутникової навігаційних систем
\end{enumerate}
\subsubsection*{6. Стадії та етапи розробки}
Проведення аналізу та вибору навігаційного забезпечення ЛА, схеми комплексної інерціальної-супутникової системи навігації та застосування калманівської фільтрації для оцінки навігаційних даних, розробка слабко зв’язаної схеми інтеграції.

Розробка алгоритмів роботи комплексної навігаційної системи, дослідження ступеню впливу похибок датчиків первинної інформації безплатформної інерціальної системи (БІНС) та супутникової навігаційної системи (СНС) на точнісні характеристики числення навігаційних параметрів і динаміку зміни похибок, впливу перерв у роботі СНС на траекторний рух ЛА.

Розробка програми моделювання помилок комплексної інерціальної-супутникової системи навігації з використанням фільтра Калмана. Пропозиція щодо удосконалення запропонованої навігаційної системи, шляхом модифікації оптимального фільтра для поліпшення його стійкості.

\subsubsection*{7. Порядок контролю та приймання}
Контроль за ходом виконання календарного плану дипломної роботи протягом всього періоду дипломного проектування здійснює керівник дипломного проектування. Керівник визначає строки виконання та почерговість кожної стадії розробки дипломного проекту, проведення розрахункових та дослідницьких робіт, виконання графічних робіт, кінцевого оформлення дипломного проекту та подачі проекту до попереднього захисту  на провідній кафедрі. Допуск до захисту у державній екзаменаційній комісії відбувається з дозволу завідувача кафедри після попереднього захисту.

Приймання здійснюється на підставі захисту дипломної роботи ДЕК Інституту аерокосмічних систем управління.\\

Термін здачі дипломної роботи: <<   >> лютого 2011 р.




% 
% Для реалізації польотного завдання літальний апарат, повинен містити у складі бортового устаткування пілотажний та навігаційний комплекси. Під пілотажним комплексом у найпростішому випадку розуміється система автоматичного керування (автопілот), а під навігаційним комплексом (НК) розуміють сукупність бортових систем і пристроїв, призначених для рішення задач навігації (навігаційна система). До складу НК і ПК входять датчики пілотажно-навігаційної інформації, навігаційні обчислювачі пристрою керування, індикації та сигналізації.
% 
% Датчики навігаційної інформації слугують для вимірювань параметрів різноманітних фізичних полів, на базі яких визначаються навігаційні елементи польоту. Їх можна поділити на дві групи: 1. датчики навігаційних параметрів положення, які визначають координати місцезнаходження літального апарата відносно опорних ліній і навігаційних точок ; 2. датчики навігаційних параметрів руху, які вимірюють параметри вектора швидкості літака та його складові: шляхову швидкість, вертикальну швидкість, напрямок польоту.
% Датчики пілотажної інформації вимірюють параметри польоту, які характеризують кутовий рух ЛА : кути крену, тангажу, рискання і кутові швидкості.
% 
% Найважливішими з пілотажно-навігаційних датчиків є: інерціально-навігаційна система, інерціальна курсовертикаль, система курсу і вертикалі, допплерівський вимірник швидкості і кута знесення типу ДВШЗ, інформаційний комплекс висотно-швидкісних параметрів типу ІК ВШП або система повітряних сигналів типу СПС СПСсистема повітряних сигналів.
% Найбільш інформативною є інерціально – навігаційна система (ІНС)ІНСінерціальна навігаційна система. Це така навігаційна система, у якій отримання інформації про швидкість і координати забезпечується шляхом інтегрування сигналів, що відповідають прискоренням ЛА. Інформація про прискорення надходить від розташованих на борту ЛА акселерометрів. Процедура інтегрування векторних величин, швидкості і прискорення, забезпечується шляхом відтворення на борту ЛА ЛАлітальний апарат відповідної системи координат, для цього, частіше за все, використовують гіростабілізатори чи гіроскопічні датчики кутової швидкості з обчислювачем.
% 
% В залежності від способу розташування акселерометрів розрізняють платформні і безплатфомні ІНС. У першому випадку акселерометри встановлюються на гіростабілізуючій платформі, у другому – безпосередньо на корпусі ЛА чи у спеціальному блоці чутливих елементів. Обидві системи мають свої переваги та недоліки. До переваг платформних ІНС відносять простоту алгоритмів обробки інформації про кутове положення і лінійні прискорення та високу точність, зумовлену сприятливими умовами роботи вимірювачів, оскільки вони розміщуються на гіростабілізаційній платформі, а не безпосередньо на корпусі об’єкта.
% 
% Зараз інтенсивно розвивається БІНС, перспективність яких визначається такими перевагами: висока надійність, низькі масогабаритні характеристики, зручність експлуатації. Характерна особливість таких ІНС, полягає у відсутності гіростабілізаційної платформи, яка являє собою складний електромеханічний пристрій та відкриває широкі можливості у плані зменшення масогабаритних характеристик й енергоспоживання.
% 
% До навігаційних датчиків, що визначають положення ЛА відносно навігаційних точок і базових ліній необхідно віднести радіотехнічні системи ближньої і дальньої навігації, літаковий далекомір, супутникову систему навігації (СНС), бортову радіолокаційну станцію, різні візирні пристрої, автоматичний компас, астрономічну навігаційну систему, кореляційно-екстремальну навігаційну систему. Найсучаснішими є супутникова навігаційна система і кореляційно-екстремальна навігаційна система.
% 
% СНС призначені для визначення місцеположення транспортних засобів, а також положення нерухомих об’єктів. Особливість дії СНС СНСсупутникова навігаційна система – це використання штучних супутників Землі як радіонавігаційних точок, координати яких, на відміну від наземних радіолокаційних точок, змінні.
% 
% Ці системи досить обґрунтовано довели високу експлуатаційну якість у різноманітних навігаційних галузях. Зокрема, вони визнані найбільш перспективними й економічно ефективними в більшості авіаційних сферах застосування. Поряд з цим, у зв’язку з можливою короткочасною втратою сигналів, які поступають із супутників, ці системи не можуть забезпечити необхідного рівня надійності навігаційних вимірів за такими показниками як цілісність, доступність і безперервність. Вирішити задачу підвищення цих показників можна шляхом комплексування супутникових навігаційних систем з іншими системами. Найбільш перспективним варіант полягає у інтеграції супутникових та інерціальних навігаційних систем. Така інтеграція дозволяє ефективно використовувати переваги кожної із систем.
% 
% Інерціальні навігаційні системи, як найбільш інформативні системи, дають змогу одержувати всю сукупність необхідних параметрів для керування об'єктом, включаючи кутову орієнтацію. При цьому, такі системи цілком автономні, тобто для їхнього нормального функціонування не потрібно використання будь-якої інформації від інших систем. Ще одна з переваг цих систем полягає у високій швидкості надання інформації зовнішнім споживачам: швидкість відновлення кутів орієнтації складає до 100 Гц, навігаційної - від 10 до 100 Гц. Цей показник для супутникових систем складає для кращих приймачів 10 Гц, а для звичайних, як правило, 1 Гц. Разом з тим, інерціальним системам притаманні недоліки, що не дозволяють використовувати їх довгий час в автономному режимі. Вимірювальним елементам ІНС, насамперед, гіроскопам та акселерометрам, притаманні методичні й інструментальні помилки, вихідні данні не можуть бути введені абсолютно точно, обчислювач, що входить до складу ІНС, вносить свої похибки. Під впливом цих факторів ІНС працює в так званому «збуреному» режимі, і отримана від ІНС інформація, буде містити похибки, що викликані впливом цих збурень, і, головне, які з часом збільшуються. Для корекції ІНС застосовують різні методи і засоби.
% 
% Корекція ІНС також може здійснюватися від радіотехнічних систем навігації (далекомірних, різницево-далекомірних), що складаються з наземної і бортової підсистем. Вони забезпечують одночасний вимір пеленга (азимута) і похилої дальності літального апарата щодо радіонавігаційної точки, і по цій інформації визначається місце розташування літака в заданій системі координат. До радіотехнічних систем варто віднести і супутникову систему навігації. Численні дослідження та практика експлуатації супутникових систем показують, що найбільш перспективним засобом корекції ІНС є супутникові системи, які володіють найбільш високою точністю і глобальністю застосування. При цьому можливо поліпшення характеристик автономних БІНС не тільки за координатами і швидкістю, але й за кутовою орієнтацією.
% 
% Недоліком всіх радіотехнічних методів навігації, у тому числі і супутникових, є те, що на переданий і прийнятий радіосигнал можуть накладатися природні й штучно створювані радіозавади. Мала потужність сигналу, велика дальність джерел сигналу від приймачів (26000 км), мале відношення “сигнал-шум” приводить до слабкої перешкодозахищеності приймачів СРНС. Контури зрушення по фазі і за часом можуть легко “втратити” відповідний супутник при наявності активних перешкод. Особливо чуттєвим щодо цього є контур спостереження за фазою.
% 
% До того ж, існує явище періодичного зникнення сигналу від СНС. При збільшенні періоду “радіомовчання” супутника величина помилки навігаційних визначень збільшується аж до зриву керування (стабілізації на заданій траєкторії).
% Виникає потреба у автономних засобах навігації, які не вимагають зовнішніх сигналів, а тому й не зазнають впливу радіоелектронного придушення. Цим умовам відповідає так звана інерціальна навігація. Використання інтегрованих інерціально-супутникових систем обумовлюється наступним: інерціальна і супутникова навігаційні системи вимірюють різні параметри: СНС - лінійні параметри (вектор положення ЛА в деякій геоцентричній системі координат і вектор його швидкості), а ІНС - як лінійні, так і кутові параметри.
% 
% Взагалі, СНС можна використовувати і для виміру кутових координат, але для цього необхідне використання декількох антен, установлених на визначеній відстані один від одного, і декількох приймачів, що різко ускладнюють й підвищують собівартість системи. 
%  Однією з центральних ідей розвитку навігаційного обладнання літальних апаратів (ЛА) є функціональне, інформаційне й апаратурне об'єднання навігаційних вимірників в інтегрований навігаційний комплекс. Більшість ЛА мають у складі свого бортового обладнання ряд навігаційних систем, серед яких найбільш поширеними є радіотехнічні системи: апаратура радіотехнічних систем ближньої (РСБН) та дальньої (РСДН) навігації, радіолокаційні станції, курсо-доплерівські системи, супутникові системи навігації (СНС), а також автономні нерадіотехнічні навігаційні системи.
% Основними автономними засобами навігації ЛА є інерціальні навігаційні системи (ІНС), які використовують на ЛА різного приз-начення. Курсо-повітряні системи застосовують на літаках і вертольотах, обладнаних курсовими системами та засобами визначення повітряної швидкості. Всі ЛА мають також засоби виміру барометричної та геометричної висоти польоту. На деяких літаках, крім цього, є банк даних про висоту рельєфу місцевості. До складу багатьох навігаційних комплексів рухомих об'єктів входять датчики часу (бортові еталони точного часу).
% 
% Об'єднання (інтеграція) такого обладнання в єдиний функціонально, структурно і конструктивно взаємозалежний навігаційний комплекс дозволяє повніше використовувати наявну на борту ЛА надмірну інформацію, завдяки цьому з'являється можливість розширити коло розв'язуваних задач і поліпшити якість їх виконання. Метою комплексування навігаційного обладнання є об'єднання різних вимірників у єдиний навігаційний комплекс (НК), який має більш високі характеристики точності, завадостійкості та надійності навігаційних визначень у порівнянні з окремими вимірниками. 

% \end{document}
