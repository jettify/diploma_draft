\documentclass[ukrainian,utf8,simple,floatsingle,hpadding=5mm]{eskdtext}
\usepackage[numberright]{eskdplain}
% variables.tex
% This file contains information about author and other specific
% people for use in eskdx collection.

\title{\fontsize{12}{12} \selectfont Інтегрована інерціально-супутникова система навігації, що базується на принципах комплексної обробки інформації
з використанням калманівської фільтрації}
% smaller size of font set for the title in frame
\author{НовікМ.В.}

\ESKDchecker{ФіляшкінМ.К.}
\ESKDnormContr{КозловА.П.}
\ESKDapprovedBy{СинєглазовВ.М.}

\ESKDdepartment{Міністерство освіти і науки України}
\ESKDcompany{Національний авіаційний університет}

\ESKDsignature{НАУ 11 09 02 000 ПЗ}
\ESKDgroup{ІАСУ 608}

\ESKDsectAlign{section}{Center}
\ESKDsectAlign{subsection}{Center}
\ESKDsectAlign{subsubsection}{Center}


\include{textcomp}
\usepackage{longtable} % multipage tables
\usepackage{multirow} % using rowspan in tables

\ESKDstyle{formIIab}
\begin{document}

\ESKDthisStyle{formII}


\section*{Технічне завдання}

\subsubsection*{1. Найменування та галузь застосування}
Комітет ІКАО з перспективних навігаційних систем (FANS- Future Air Navigation System) прийняв 
рішення про обов'язкове використання систем супутникової навігації в сполученні з ІНС. Тому в 
даний час у всіх галузях авіації основним інформаційним ядром сучасного навігаційного комплексу 
є інтегрована інерціально-супутникова система навігації (ІССН).

В теперішній час визнано, що одним з основних шляхів вдосконалення навігаційного обладнання 
є створення комплексних навігаційних систем, що інтегрують  командні прибори в єдиний блок, 
здатний автоматично вирішувати задачу навігації ЛА. Сутність комплексування полягає у 
використанні інформаційної та структурної надмірності для підвищення точності, надійності 
та завадостійкості інформації при вимірюванні одних і тих же або функціонально зв’язаних 
навігаційних параметрів. Інформаційна надмірність полягає в тому, що забезпечується отримання 
однорідної інформації від декількох навігаційних датчиків різної фізичної природи з наступною 
сумісною обробкою цієї інформації в спеціалізованому обчислювачі. Надмірність структури 
комплексу забезпечує його працездатність при відмові, особливо короткотривалій, одного із датчиків. 


Найбільш привабливим для розв’язання цієї задачі є залучення калманівської фільтрації. 
Фільтр Калмана призначений для ідентифікації (оцінювання) змінних стану системи за даними 
вимірювання вихідних сигналів цієї системи, які містять похибки вимірювання (вимірювальний шум). 
Ідентифікація оптимальна в тому смислі, що сума квадратів похибок оцінювання змінних стану
в будь-який момент часу має найменше з можливих значень. Похибка оцінювання це різниця 
між оцінкою фільтра й дійсним значенням змінних стану системи при наявності в системі 
детермінованих і випадкових похибок вимірювань. Отже, фільтр Калмана призначений для 
найкращого  відновлення змінних стану, тобто для оптимального приглушення вимірювальних шумів.

\subsubsection*{2. Підстава до розробки}

Наказ по Національному авіаційному університету

\No1111/ст від <<20>> жовтня 2010 р.

\subsubsection*{3. Мета та призначення розробки}

Основною метою роботи є аналіз та вибір схеми комплексної інерціально-супутникової 
навігаційної системи та схем оцінювання та корекції в цій системі і, як наслідок, 
розробка слабко зв’язаної схеми інтеграції, що базується на принципах комплексної 
обробки інформації з використанням калманівської фільтраці, дослідження ступеню 
впливу похибок датчиків первинної інформації  безплатформної інерціальної системи 
(БІНС) та супутникової навігаційної системи (СНС) на стійкість фільтра Калмана, 
точнісні характеристики числення навігаційних параметрів і динаміку зміни похибок, 
впливу перерв у роботі СНС на траекторний рух ЛА, моделювання зміни похибок 
комплексної інерціально-супутникової навігаційної системи.

\subsubsection*{4. Технічні вимоги}

Основні технічні вимоги:
\begin{itemize}
      \item точність визначення навігаційних параметрів:\\
  координат (СКВ), м \dotfill 15\\
  висоти (СКО), м \dotfill $0\div20$
     \item характеристики повинні зберігатися при:\\
швидкості, м/с до \dotfill 300
     \item час визначення (холодний старт), хв \dotfill <2
     \item частота відновлення координат, c$^{-1}$ \dotfill <1
     \item масса, кг \dotfill <1
     \item автоматичне, безперервне, всепогодне визначення 
поточних 3D координат місця розташування, вектора шляхової 
швидкості і шляхового кута ЛА.
    \item автоматичний тестовий контроль функціонування блоків і 
вузлів апаратури, індикація блоків, що відмовили
    \item стійке визначення навігаційних параметрів при русі з 
лінійними прискореннями і при стрибкоподібних змінах прискорення
    \item БІНС повинна забезпечити визначення координат на протязі 60 с
    \item взаємна корекція СНС та БІНС
    \item підтримка СНС від БІНС для зменшення часу повторного 
запуску (“гарячого старту”) при короткочасних перервах у роботі СНС
\end{itemize}

Вимоги до засобів захисту
\begin{itemize}
    \item робоча температура, C   \dotfill -40...+60
    \item робоча вологість (25 C)   \dotfill 98 
\end{itemize}

Додаткові вимоги
\begin{itemize}
    \item швидкість польоту ЛА, м/с \dotfill 400 
    \item максимальний кут крену ЛА,град \dotfill 450
    \item максимальний кут тангажу ЛА,град \dotfill 200
\end{itemize}




\subsection*{5. Джерела розробки}

\begin{enumerate}
 \item Хоздоговірна науково-дослідна робота № 201-Хд04 “Ресурс”: “Розробка 
попередніх алгоритмів роботи інерціально-супутникової навігаційної 
системи та інформаційного зв'язку з літаком-носієм ”.
 \item М.К. Філяшкін В.О. Рогожин, А.В. Скрипець, Т.І. Лукінова 
Інерціально-супутникові навігаційні  системи. – К.: Вид-во НАУ, 2009. – 306 с.
 \item Науково-дослідна робота  № 396 ДБ-07 : Методика побудови 
комплексної навігаційної системи на основі спрощеного варіанту 
безплатформної інерціальної та високоточної супутникової навігаційних систем
\end{enumerate}
\subsection*{6. Стадії та етапи розробки}

Проведення аналізу та вибору навігаційного забезпечення ЛА, 
схеми комплексної інерціальної-супутникової системи навігації 
та застосування калманівської фільтрації для оцінки навігаційних 
даних, розробка слабко зв’язаної схеми інтеграції.

Розробка алгоритмів роботи комплексної навігаційної системи, 
дослідження ступеню впливу похибок датчиків первинної інформації  
безплатформної інерціальної системи (БІНС) та супутникової 
навігаційної системи (СНС) на точнісні характеристики числення 
навігаційних параметрів і динаміку зміни похибок, впливу перерв 
у роботі СНС на траекторний рух ЛА.

Розробка програми моделювання помилок комплексної 
інерціальної-супутникової системи навігації з використанням 
фільтра Калмана. Пропозиція щодо удосконалення запропонованої 
навігаційної системи, шляхом модифікації оптимального фільтра 
для поліпшення його стійкості.

\subsection*{7. Порядок контролю та приймання}
Контроль за ходом виконання календарного плану дипломної роботи протягом 
всього періоду дипломного проектування здійснює керівник дипломного 
проектування. Керівник визначає строки виконання та почерговість кожної 
стадії розробки дипломного проекту, проведення розрахункових та 
дослідницьких робіт, виконання графічних робіт, кінцевого оформлення 
дипломного проекту та подачі проекту до попереднього захисту  на провідній 
кафедрі. Допуск до захисту у державній екзаменаційній комісії відбувається 
з дозволу завідувача кафедри після попереднього захисту.
\footnotesize
\begin{longtable}{|p{5cm}|c|c|}

% \footnotesize


%\centering
%\begin{tabular}[c]{|p{5cm}|c|c|c|c|}

\hline
\bfseries Етапи виконання дипломного проекту (роботи) & 
\bfseries Термін виконання роботи& \bfseries Примітка  \\


\hline
Підбір літератури
& 01.11.10 – 03.11.10 &  \\ 

\hline
Технічне завдання
& 03.11.10 – 06.11.10 &   \\ 
\hline
Вступ
& 06.11.10 – 09.11.10 &   \\ 
\hline
1. Обґрунтування необхідності розробки
& 09.11.10 – 12.11.10&   \\ 

\hline
2. Аналіз та вибір навігаційного забезпечення БПЛА
2.1. Аналіз і вибір варіанта супутникової навігаційної системи
2.2. Аналіз та вибір варіанта інерціальної навігаційної системи
2.3. Аналіз та вибір схеми комплексної інерціально-супутникової 
навігаційної системи 
& 12.11.10 – 18.11.10&   \\ 

\hline
3. Постановка задачі 
& 8.11.10 – 21.11.10&   \\ 
\hline
4.  Аналіз та вибір схем оцінювання та корекції в комплексній інерціально-супутниковій системі
4.1. Аналіз та вибір методу сумісної обробки інформації
4.2. Аналіз та вибір схеми корекції
4.3. Розробка слабко в’язаної комплексної інерціально-супутникової системи навігації 

&21.11.10 – 28.11.10&   \\ 
\hline




%%\end{tabular}
%\caption{норми освітленості в кабінетах з ПК}
%\label{tab:labour protection}
\end{longtable} 


Приймання здійснюється на підставі захисту дипломної роботи ДЕК Інституту 
аерокосмічних систем управління.

Термін здачі дипломної роботи: <<09>> лютого 2011 р.

\end{document}
