\documentclass[ukrainian,utf8]{eskdtext}
\usepackage[numberright]{eskdplain}
% variables.tex
% This file contains information about author and other specific
% people for use in eskdx collection.

\title{\fontsize{12}{12} \selectfont Інтегрована інерціально-супутникова система навігації, що базується на принципах комплексної обробки інформації
з використанням калманівської фільтрації}
% smaller size of font set for the title in frame
\author{НовікМ.В.}

\ESKDchecker{ФіляшкінМ.К.}
\ESKDnormContr{КозловА.П.}
\ESKDapprovedBy{СинєглазовВ.М.}

\ESKDdepartment{Міністерство освіти і науки України}
\ESKDcompany{Національний авіаційний університет}

\ESKDsignature{НАУ 11 00 75 000 ПЗ}
% 11 year of defend
% 00 number of thesis
% 75 last two nambers of studens mark book
% 000 must stay 000

\ESKDgroup{ІАСУ 608}

\ESKDsectAlign{section}{Center}
\ESKDsectAlign{subsection}{Center}
\ESKDsectAlign{subsubsection}{Center}



\begin{document}
 
\section*{Технічне завдання}

\subsection*{Найменування та галузь застосування}
Підвищення ефективності експлуатації авіаційної техніки(АТ), 
рівня безпеки повітряних суден(ПС), зниження затрат на технічне 
обслуговування та ремонт(ТОіР) і комплектуючих виробів є 
основними задачами в сфері цивільної авіації. їх розв'язок 
можливий за уови корінної перебудови всієї системи ТОіР ПС, 
основу якої складає напрацювання і ресурс виробів.

Планово-попереджувальна система ТОіР не відповідає підвищеним 
вимогам до АТ. Перспективною є система ТОіР ПС по стану, яка 
передбачає збільшення часу екстплуатації АТ, зниження 
експлуатаційних витрат і підвищення рівня безпеки польотів.

При розробці методів і засобів настройки окремих елементів і 
функціональних сиситем ПС, необхідних для впровадження системи 
ТОіР за станом окремих типів ПС та їх комплектуючих виробів, 
значну увагу приділяють системам автоматичного керування руху 
як системам, що суттєво впливають на безпеку та економічність 
польоту ПС.

Під час експлуатації конструкція повітряного судна, його 
агрегати й окремі частини знаходяться під дією різноманітних 
навантажень, що спричиняють поступову зміну параметрів 
математичної моделі ПС.  Характер дії цих сил може бути 
різноманітними. Постійним від завантаження літака, підіймальної 
сили, змінним від дії аеродинамічних сил, епізодичним при 
ударних навантаженнях при посадці, зіткненні з нерівностями на 
злітно-посадковій смузі та ін. В результаті дії цих сил 
накопичуються втомні деформації, змінюються параметри механічних 
з'єднань важелів управління з керуючими поверхнями, що спричинює 
до невідповідності реакцій еталонної математичної моделі та 
реальної системи.

На сьогоднішній день експлуатаційне обслуговування ПС, зокрема 
діагностика та настройка САК ПС ведеться згідно регламенту, в 
якому передбачається перевірка основних конструктивних параметрів 
складових елементів САУ через визначений час експлуатації, ремонт 
чи заміна окремих вузлів та елементів конструкції згідно вимог 
експлуатаційної документації. Терміни, склад і порядок виконання 
регламентних робіт складають на основі результатів 
передексплуатаційних випробувань. При цьому основні способи 
настройки спрямовані на утримання основних конструктивних 
параметрах в заданому значенні з відповідними похибками, а 
дослідження всієї системи в цілому можливе лише під час її 
активної роботи, тобто під час польоту ПС.

У таких випадках застосовуються методи настройки, побудовані на 
математичному моделюванні САК ПС виходячи з конструктивних 
параметрів її окремих елементів. Дані методи характеризуються високим 
ступенем достовірності, універсальністю та інваріантністю стосовно 
досліджуваної системи, значною тривалістю дослідження. 

Для скорочення часу дослідження та підвищення наочності вихідних даних 
запропоновано розробити програмний комплекс системи настройки САК ПС, 
яка може бути використана в якості керуючого елементу настройки САК в 
реальному часі згідно попередньо заданих критеріїв.

\subsection*{Підстава до розробки}

Наказ по Національному авіаційному університету

\No 2748/ст від <<20>> жовтня 2008 р.

\subsection*{Мета та призначення розробки}

Метою даної роботи є аналіз існуючих засобів настройки та діагностування 
САК ПС, дослідження принципів математичного моделювання динамічних систем, 
створення узагальненої математичної моделі (ММ) САК, розробка та 
покращення алгоритмів дослідження САК та наступне дослідження роботи 
розроблених алгоритмів, їх ефективності та достовірності.

\subsection*{Технічні вимоги}

Основні технічні вимоги:
\begin{enumerate}
 \item точність розрахунку параметрів ММ, знаків, не менше \hfill 10
 \item діапазон допустимих значень параметрів ММ \hfill $\pm2.7\cdot10^{20}$
\end{enumerate}

\subsection*{Стадії та етапи розробки}

Класифікація й області застосування систем автоматичного керування ЛА. 
Виділення критерію площинності систем стабілізації та дослідження їх 
функціонального виконання даних систем.

Аналіз існуючих засобів настройки та діагностування САК ПС. Розробка 
вимог до систем настройки САК ЛА. Дослідження принципів математичного 
моделювання динамічних систем, розробка узагальненої математичної моделі 
(ММ) САК. 

Розрахунок діаграми якості САК ЛА. Розробка алгоритму виділення області 
стійкості.

Розробка програмного забезпечення процесів моделювання систем стабілізації 
ЛА та проведення досліджень синтезованих алгоритмів на достовірність та 
ефективність шляхом багатократного моделювання.

Розробка системи реального часу для демонстрації роботи алгоритмів в 
якості керуючого елемента самонастроюваних систем.

\subsection*{Порядок контролю та приймання}
Контроль за ходом виконання календарного плану дипломної роботи протягом 
всього періоду дипломного проектування здійснює керівник дипломного 
проектування. Керівник визначає строки виконання та почерговість кожної 
стадії розробки дипломного проекту, проведення розрахункових та 
дослідницьких робіт, виконання графічних робіт, кінцевого оформлення 
дипломного проекту та подачі проекту до попереднього захисту  на провідній 
кафедрі. Допуск до захисту у державній екзаменаційній комісії відбувається 
з дозволу завідувача кафедри після попереднього захисту.

Приймання здійснюється на підставі захисту дипломної роботи ДЕК факультету 
аерокосмічних систем управління Інституту електроніки й систем управління.

Термін здачі дипломної роботи: <<09>> лютого 2008 р.

\end{document}
