\section{Розробка автоматизованої системи діагностування}

Для розробки автоматизованої системи діагностування був використаний підхід до оптимізації параметрів за допомогою діаграм стійкості. Методика отримання таких діаграм в загальному вигляді описана в попередньому розділі. Тим не менше, дані методи не можуть бути використані для чисельного моделювання та дослідження. З цією метою для отримання діаграми стійкості САУ був розроблений чисельний ітеративний метод виділення країв. Для отримання ж конкретних кількісних і якісних характеристик САУ були використані вже існуючі алгебраїчні та частотні алгоритми дослідження.

Нижче наведені методи, що використані в ході розробки програмного забезпечення 
автоматизованої системи діагностування САУ ПС. Дані методи розбиті на підгрупи 
за своїм призначенням:

\begin{enumerate}
 \item Матричні методи
  \subitem Швидке обчислення визначників;
  \subitem Обчислення власних чисел матриці;
  \subitem Методи поліноміальної арифметики;
 \item Конверсійні методи
  \subitem Метод формування матриці Гурвіца-Раута;
  \subitem Метод перетворення передатної функції до форми Коші;
 \item Методи обробки даних
  \subitem Метод найменших квадратів;
  \subitem Методи пошуку екстремуму;
 \item Методи роботи з передатними функціями
 \subitem спрощення складних передатних функцій;
 \subitem приведення поліномів передатної функції до нормальної форми;
 \subitem метод побудови ЛАЧХ;
 \subitem метод побудови АФЧХ;
 \subitem критерій стійкості Гурвіца – Раута;
 \subitem метод побудови діаграми стійкості.
\end{enumerate}

Оскільки більшість алгоритмів, що наведені вище, носять прикладний характер, слід розглянути ті з них, які прямо впливають на ефективність роботи програмного забезпечення.

\subsection{Розробка алгоритмів алгебраїчного дослідження стійкості САК}
\subsubsection{Критерій стійкості Гурвіца-Раута}

<<Для стійкості системи n-ого порядку необхідно і достатньо, щоб n визначників, 
складених з коефіцієнтів характеристичного рівняння 

$$A(p) = a_n p^n + a_{n-1} p^{n-1} + \ldots + a_2 p^2 + a_1 p + a_0 = 0$$ 

були додатніми.>>

При цьому визначники беруться як головні мінори матриці вигляду:
\[
\begin{array}{ccccccccc}
a_{n-1} & a_{n-3} & a_{n-5} & a_{n-7} & \cdots & 0 & 0 & 0 & 0\\
a_{n} & a_{n-2} & a_{n-4} & a_{n-6} & \cdots & 0 & 0 & 0 & 0\\
0 & a_{n-1} & a_{n-3} & a_{n-5} & \cdots & 0 & 0 & 0 & 0\\
0 & a_{n} & a_{n-2} & a_{n-4} & \cdots & 0 & 0 & 0 & 0\\
\vdots & \vdots & \vdots & \vdots & \ddots & \vdots & \vdots & \vdots & \vdots\\
0 & 0 & 0 & 0 & \cdots & a_{3} & a_{1} & 0 & 0\\
0 & 0 & 0 & 0 & \cdots & a_{4} & a_{2} & a_{0} & 0\\
0 & 0 & 0 & 0 & \cdots & a_{5} & a_{3} & a_{1} & 0\\
0 & 0 & 0 & 0 & \cdots & a_{6} & a_{4} & a_{2} & a_{0}\end{array}\]

