\section{Постановка задачі}

Задачу досліджень сформулюємо як дослідження можливостей комплексування 
навігаційної інформації двох систем, що є на борту сучасного літака: 
однієї - невисокої точності, а значить дешевої безплатформенної інерціальної 
навігаційної системи (БІНС) і іншої супутникової високоточної навігаційної системи (СНС).

Розглядається спосіб підвищення точності роботи інерціально-супутникової 
навігаційної системи на основі фільтра Калмана. Цей підхід дає можливість 
оптимально оцінити спостерігаємі параметри, і з їх допомогою покращити оцінки 
не спостерігаємих (на приклад орієнтацію географічного тригранника). А непрямий 
підхід до оцінки отриманих даних дає можливість використовувати лінійні моделі 
похибок БІНС, і добре освоєні процедури лінійного оптимального фільтра Калмана 
(наприклад модифіікацї Джозефа, Поттера чи Карлсона)

Пропонується слабкозв’язана схема комплексування, оскільки вихідна інформація 
двох систем може піддаватися комплексної обробці з використанням тих чи інших 
алгоритмів оптимальної фільтрації.  Окрім цього, для створення  архітектури 
такої інтегрованої ІССН потрібні мінімальні зміни в апаратних засобах і 
програмному забезпеченні вже існуючого обладнання ЛА.

Наводяться результати математичного моделювання, що підтверджують 
доцільність такого підходу до підвищення точності автономної роботи ІСН.



