\section{Аналіз та вибір схем оцінюванн та корекції в комплексній інерціально-супутниковій системі}

Основними задачами пілотажно-навігаційних комплексів (ПНК) як постачальника 
інформаційного забезпечення польоту ЛА є сумісна обробка навігаційної інформації, 
яка надходить на борт ЛА та забезпечення високої надійності функціонування бортових 
систем та комплексів ЛА і взагалі безпеки польоту за рахунок резервування 
джерел інформації. Висока ефективність використання інформації, яка 
надходить на борт ЛА, забезпечується застосуванням різних методів її обробки. 

Найкращі результати підвищення якісних характеристик вимірювальних комплексів 
досягаються  в системах зі структурною надмірністю, коли існує можливість 
отримання пілотажно-навігаційної інформації паралельно декількома способами з 
використанням інформації від приладів та вимірювальних систем, що входять до 
складу ПНК. Отримана таким чином інформація комплексується.

В існуючих ПНК широке розповсюдження знайшли такі способи сумісної обробки 
інформації, що надходять від декількох вимірників, як взаємна компенсація і 
фільтрація похибок вимірювальних приладів, що вимірюють один і той самий 
навігаційний параметр та оптимальне оцінювання вектора стану з використанням 
апріорної інформації про контрольований процес та поточні вимірювання.

Методи оптимальної обробки інформації в ПНК використовуються з метою 
отримання оцінок вектора стану повітряного судна (або деякої частини 
цього вектора) в умовах впливу випадкових збурень і завад на процес 
вимірювання. При цьому оцінюються не самі параметри польоту, а їхні похибки. 
За оптимальної обробки пілотажно - навігаційної інформації в ПНК найважливішим 
процесом є процес отримання оптимальних оцінок. В основу алгоритмів отримання 
оптимальних оцінок можуть бути покладені такі методи обробки інформації:
\begin{itemize}
  \itemметод найменших квадратів;
 \itemметод максимуму правдоподібності;
 \itemрекурентний неоптимальний фільтр;
 \itemоптимальний фільтр Калмена.
\end{itemize}


% При цьому визначники беруться як головні мінори матриці вигляду:
% \[
% \begin{array}{ccccccccc}
% a_{n-1} & a_{n-3} & a_{n-5} & a_{n-7} & \cdots & 0 & 0 & 0 & 0\\
% a_{n} & a_{n-2} & a_{n-4} & a_{n-6} & \cdots & 0 & 0 & 0 & 0\\
% 0 & a_{n-1} & a_{n-3} & a_{n-5} & \cdots & 0 & 0 & 0 & 0\\
% 0 & a_{n} & a_{n-2} & a_{n-4} & \cdots & 0 & 0 & 0 & 0\\
% \vdots & \vdots & \vdots & \vdots & \ddots & \vdots & \vdots & \vdots & \vdots\\
% 0 & 0 & 0 & 0 & \cdots & a_{3} & a_{1} & 0 & 0\\
% 0 & 0 & 0 & 0 & \cdots & a_{4} & a_{2} & a_{0} & 0\\
% 0 & 0 & 0 & 0 & \cdots & a_{5} & a_{3} & a_{1} & 0\\
% 0 & 0 & 0 & 0 & \cdots & a_{6} & a_{4} & a_{2} & a_{0}\end{array}\]