\section{Аналіз та вибір схем оцінюванн та корекції в комплексній інерціально-супутниковій системі}

Основними задачами пілотажно-навігаційних комплексів (ПНК) як постачальника 
інформаційного забезпечення польоту ЛА є сумісна обробка навігаційної інформації, 
яка надходить на борт ЛА та забезпечення високої надійності функціонування бортових 
систем та комплексів ЛА і взагалі безпеки польоту за рахунок резервування 
джерел інформації. Висока ефективність використання інформації, яка 
надходить на борт ЛА, забезпечується застосуванням різних методів її обробки. 

Найкращі результати підвищення якісних характеристик вимірювальних комплексів 
досягаються  в системах зі структурною надмірністю, коли існує можливість 
отримання пілотажно-навігаційної інформації паралельно декількома способами з 
використанням інформації від приладів та вимірювальних систем, що входять до 
складу ПНК. Отримана таким чином інформація комплексується.

В існуючих ПНК широке розповсюдження знайшли такі способи сумісної обробки 
інформації, що надходять від декількох вимірників, як взаємна компенсація і 
фільтрація похибок вимірювальних приладів, що вимірюють один і той самий 
навігаційний параметр та оптимальне оцінювання вектора стану з використанням 
апріорної інформації про контрольований процес та поточні вимірювання.

Методи оптимальної обробки інформації в ПНК використовуються з метою 
отримання оцінок вектора стану повітряного судна (або деякої частини 
цього вектора) в умовах впливу випадкових збурень і завад на процес 
вимірювання. При цьому оцінюються не самі параметри польоту, а їхні похибки. 
За оптимальної обробки пілотажно - навігаційної інформації в ПНК найважливішим 
процесом є процес отримання оптимальних оцінок. В основу алгоритмів отримання 
оптимальних оцінок можуть бути покладені такі методи обробки інформації:
\begin{itemize}
 \item метод найменших квадратів;
 \item метод максимуму правдоподібності;
 \item рекурентний неоптимальний фільтр;
 \item оптимальний фільтр Калмена.
\end{itemize}




%\textbf{11. 3. Методи оптимальної обробки інформації }

 Методи оптимальної обробки інформації в навігаційних комплексах використовуються  
з метою отримання оцінок вектора стану ПС (або деякої частини цього вектора) в умовах 
впливу випадкових збурень і завад на процес вимірювання. При цьому оцінюються не 
самі параметри польоту, а  їхні похибки.

Нехай вектор стану динамічної системи описується векторно-матричним рівнянням                            
\begin{equation}
\label{eq:__11_3_}
\dot{X}(t)=A(t)X(t)+B(t)V_{x}(t)
\end{equation}
\begin{ESKDexplanation}                
\item де $\dot{X}\left(t\right)$ -- \textit{n}-мірний вектор стану системи; 
\item \textbf{A(t)} -- квадратна матриця розмірності $n\times n$, яка являє собою матрицю коефіцієнтів 
системи; 
\item $V_{x} \left(t\right)$ -- \textit{k}-мірний вектор збурень, які діють 
на вході динамічної системи; 
\item \textbf{B}(\textit{t}) -- матриця збурень.
\end{ESKDexplanation}
Будемо вважати, що компоненти вектора $V_{x} \left(t\right)$ лінійно зв'язані з випадковими 
функціями типу білого шуму, мають нульові математичні сподівання  $M[V_{x}(t)]=0$ 
і характеризуються кореляційною матрицею $R_{x} \left(t\right) = M[V_{x}(t),V_{x}(t)^{T}]$. 

Отже,
\[M[V_{x}(t)]=0 \] 
\[R_{x} \left(t\right) = M[V_{x}(t),V_{x}(t)^{T}] \]
\begin{ESKDexplanation}
 \item де \textit{М} - символ математичного сподівання; 
 \item $\delta (t -\tau)$ -- дельта-функція.
\end{ESKDexplanation}

З  вектором стану системи $X(t)$ співвідношенням
\begin{equation} 
\label{eq:__11_4_} Y(t)=H(t)X(t) 
\end{equation} 

зв'язаний вектор спостережень \textbf{Y(t)} розмірності \textit{m}.
У рівнянні \eqref{eq:__11_4_} \textbf{H(t)} -- матриця зв'язку (матриця 
спостереження). Необхідною умовою оптимального оцінювання є повна спостережливість 
вектора стану \textbf{Х(t)} за вектором спостереження \textbf{Y(t)}.

Вважається, що процес \textbf{Х(t)} цілком спостережний на інтервалі $t \epsilon [t_{1},t_{2}]$,
 якщо за значенням вектора \textbf{Y(t)} при$t \epsilon [t_{1},t_{2}]$ 
можна вичислити значення вектора \textbf{Х(t)} при$t \epsilon [t_{1},t_{2}]$].
Умови повної спостережливості виконуються при \textit{m}$\leq$\textit{n}. Оскільки 
компоненти вектора \textbf{Y(t)} вимірюються з похибками, то як наслідок 
вимірювання отримують новий вектор, так званий вектор вимірювання  
\[ Z(t) = Y(t) + v_{z}(t),\]
де $V_{z}(t)$ --  вектор 
похибок вимірювання (припускається, що компоненти вектора $V_{z}(t)$ 
можна подати у вигляді білого шуму з нульовим математичним сподіванням). Вектор \textbf{V}\textit{z}(\textit{t}), 
аналогічно вектору \textbf{V}\textit{x}(\textit{t})\textit{,} характеризується кореляційною 
матрицею \textbf{R}\textit{z}(\textit{t}) розмірності \textit{m }$\times$\textit{ n}, 
тобто 

\textit{М }[\textbf{V}\textit{z}(\textit{t})]\textit{ =}0; \textit{}

\textit{М }[(\textbf{V}\textit{z}(\textit{t})\textit{,$V_{z}^{{\rm T}} (t)$}]\textit{ = }\textbf{R}\textit{z}(\textit{t})\textbf{$\delta$}(\textit{t} \textit{--} \textit{$\tau$}) \textit{.}

Припускаючи, 
що компоненти векторів \textbf{V}\textit{x}(\textit{t})\textit{ }і \textbf{V}\textit{z}(\textit{t}) 
некорельовані, на підставі викладеного математичну модель динамічної системи та рівняння 
спостереження можна записати у вигляді 

\begin{equation} \label{eq:__11_5_} \begin{array}{l} {\dot{{\rm X} }(t)=A(t)X(t)+B(t)V_{x} 
(t);} \\ {Z(t)=H(t)X(t)+V_{z} (t).} \end{array} \end{equation} 

При оптимальній обробці інформації в навігаційних комплексах (НК) найбільш важливим 
процесом є процес отримання оптимальних оцінок $\hat{{\rm X} }\left(t\right)$. В 
основу алгоритмів отримання оптимальних оцінок можуть бути покладені такі методи 
обробки інформації:

\begin{enumerate}
\item - -метод найменших квадратів (МНК);

\item - -метод максимуму правдоподібності;

\item - -рекурентний метод (оптимальний фільтр  Калмана ).
\end{enumerate}



















\textbf{Алгоритм оцінювання за методом найменших квадратів}

Метод найменших квадратів застосовується для одержання оптимальних оцінок при обробці 
накопичених вимірювань. Якщо виконано \textit{m} вимірювань координат \textbf{Х}\textit{ }(параметрів) 
системи \eqref{eq:__11_3_}, тоді

\textbf{Z} = \textbf{HX }+ \textbf{V}\textit{z },\textit{                                             }\eqref{eq:__11_6_}

де \textbf{H -- }матриця 
спостереження; \textbf{Z}, \textbf{X} і \textbf{V}z  -- вектори, компонентами яких 
є реалізація вектора вимірювання \textbf{Z}\textit{i}, вектора стану системи \textbf{X}\textit{i }і 
вектора похибок вимірювання \textbf{V}\textit{zi}, відповідно, причому $i=\overline{1,m}$.$ $

Необхідно 
за спостереженнями \textbf{Z}  і заданою матрицею спостереження \textbf{Н} щонайкраще 
оцінити стан вектора \textbf{Х}\textit{.} Критерієм такої оцінки за МНК беремо функціонал    

\[J=
\sum _{i=1}^{m}V_{{\rm z}_{i} }^{{\rm T}}  V_{z_{i} } ,\] 

який  мінімізує  суму  квадратів  похибок  вимірювання   \textbf{V}\textit{zi}.

У матричному вигляді цей вираз запишеться так:

\[J=|V_{z_{1} } V_{z_{2} } \ldots V_{z_{m} } |\left|\begin{array}{c} {V_{z_{1} } 
} \\ {V_{z_{2} } } \\ {\vdots } \\ {V_{z_{m} } } \end{array}\right|\] 

або з урахуванням \eqref{eq:__11_6_} 

\textit{J = }(\textbf{Z}\textit{ -- }\textbf{HX})T(\textbf{Z}\textit{ -- }\textbf{HX}).\textit{                             }\eqref{eq:__11_7_}\textit{}

Оцінку$\hat{{
\rm X} }$  вектора стану системи \textbf{Х }можна одержати шляхом розв'язання  рівняння  $\frac{
\partial J}{\partial X} =0$.

З урахуванням рівняння \eqref{eq:__11_7_} маємо

$H^{{\rm T}} {\rm (}Z-H\hat{X}{\rm )}+{\rm (}Z-H\hat{X}{\rm )}^{{\rm T}} H={\rm 0}$.                 
\eqref{eq:__11_8_}

Доданки  виразу \eqref{eq:__11_8_} рівні між собою, оскільки є добутками транспонованих 
відносно один до одного однакових співмножників \textbf{H}\textit{ і }(\textbf{Z}\textit{ -- }\textbf{H}\textit{$\hat{{
\rm X} }$}). Отже, тільки рівність  нулю кожного з цих двох доданків забезпечує рівність 
нулю виразу \eqref{eq:__11_8_}.

Нехай 

\textbf{H}Т(\textbf{Z}\textit{ -- }\textbf{H}\textit{$\hat{{\rm X} }$}) \textit{= }0,

тоді  \textbf{H}Т\textbf{Z}\textit{ -- }\textbf{H}Т\textbf{H}\textit{$\hat{{
\rm X} }$}= 0,  або  \textit{$\hat{{\rm X} }$} = (\textbf{H}Т\textbf{H)--}1\textbf{H}Т\textbf{Z}.

Тепер 
можна сформулювати необхідні і достатні умови одержання оптимальних оцінок \textit{$\hat{{
\rm X} }$} вектора стану системи \textbf{Х} за методом найменших квадратів  у вигляді 
основних положень, виконання яких передбачає:

\begin{enumerate}
\item - -наявність накопичених спостережень \textit{Zi},\textit{ }$i=\overline{1,m}$;

\item - -знання 
матриці спостережень \textbf{Н} ;

\item - -неособливість матриці \textbf{H}Т\textbf{H}, тобто \textbar \textbf{H}Т\textbf{H}\textbar  $\ne $0.
\end{enumerate}

Структурна 
схема одержання оптимальних оцінок за методом найменших квадратів показана на рис.11.6.    

$\dot{
\% }(t)$$ $$ $$\frac{1}{p} $$\hat{X}_{m} (t)$$\dot{\% }(t)$$ $$ $$\frac{1}{p} $$\hat{X}_{m} 
(t)$Метод найменших квадратів також застосовується для одержання оптимальних оцінок 
при обробці надмірної кількості вимірювань. Зрозуміло, що обробка більшої кількості 
вимірювань підвищує точність. Приклад розв'язання задачі оцінювання за методом найменших 
квадратів докладно був розглянутий в алгоритмах супутникових навігаційних систем. 
Такий алгоритм застосовується  тоді, коли споживач працює в умовах наявності в зоні 
видимості більш чотирьох навігаційних супутників, і в приймальній апаратурі можна 
отримати більшу кількість вимірювань.

\textbf{Алгоритм оцінювання за методом максимуму правдоподібності} 

Алгоритм оцінювання за методом максимуму правдоподібності,  як і алгоритм оцінювання 
за МНК, потребує накопичення вимірювань, тобто наявності вектора спостережень. 

Передбачається, що похибки вимірювання розподілені за нормальним законом. Тоді щільність 
розподілу ймовірностей вектора  \textbf{V}\textit{zm} має вигляд:

$P\left(V_{z_{m} } \right)=\frac{1}{\sqrt{(2{\rm \pi })^{m} \left|R_{z} \right|} 
} {\rm exp}\left[-\frac{1}{2} V_{z_{m} }^{{\rm T}} R_{z}^{-1} V_{z_{m} } \right]$,           
\eqref{eq:__11_9_}

де\textit{ }\textbf{R}z -- кореляційна матриця похибок вимірювання ; \textbar \textbf{R}\textit{z} \textbar  
-- визначник матриці \textbf{R}\textit{z}.                                                   

Використання 
алгоритму оцінок за методом максимуму правдоподібності передбачає виконання умови 
\textbar \textbf{R}\textit{z} \textbar  $\neq$ 0, тобто матриця \textbf{R}\textit{z }не 
повинна бути особливою. Підставивши \eqref{eq:__11_6_} у \eqref{eq:__11_14_}, 
отримаємо вираз для функції правдоподібності 

\[\psi (X)=\frac{1}{\sqrt{(2\pi )^{m} \left|R_{z} \right|} } {\rm exp}\left[-\frac{1}{2} 
(Z_{m} -HX_{m} )^{{\rm T}} R_{z}^{-1} (Z_{m} -HX_{m} )\right],\] 

яка являє собою щільність розподілу похибок  вимірювання.

Необхідно вибрати таку оцінку \textit{ }$\hat{{\rm X} }_{m} $\textit{,} при якій 
функція правдоподібності ${\rm \psi }$(\textbf{Х}) перетворюється в максимум, що 
відповідає мінімуму квадратів відхилень виміряних координат вектора \textbf{Х} від 
їхнього дійсного значення. Для цього необхідно, щоб           

\[\frac{\partial \psi (X)}{\partial X} =0.\] 

На практиці зручніше обчислювати максимум не самої функції правдоподібності, a її 
логарифма, тобто

${\rm Ln}{\rm \psi }(X){\rm \; \; }={\rm Ln}\frac{1}{\sqrt{(2{\rm \pi })^{m} \left|R_{z} 
\right|} } -\frac{1}{2} (Z_{m} -HX_{m} )^{{\rm T}} R_{z}^{-1} (Z_{m} -HX_{m} )$.   
\eqref{eq:__11_10_}     

Узявши в рівнянні \eqref{eq:__11_10_} похідні за компонентами вектора \textbf{X}\textit{m} і 
прирівнявши\textbf{ }їхню суму до нуля, одержимо\textbf{:}

\textit{$\frac{1}{2} H^{{\rm T}} R_{z}^{-1} (Z_{m} -H\hat{X}_{m} )+\frac{1}{2} HR_{z}^{-1} 
(Z_{m} -H\hat{X}_{m} )^{{\rm T}} =0$}.        \eqref{eq:__11_11_}

Зауважимо, що як і для формули оцінки вимірювань за методом найменших квадратів, 
один із доданків виразу \eqref{eq:__11_11_} є транспонованим відносно іншого. 
Отже, доданки цього виразу рівні між собою, вони не можуть бути від'ємні, тому кожний 
з них дорівнює нулю. Припустимо, що                    

\[{\rm H} ^{{\rm T}} R_{z}^{-1} (Z_{m} -H\hat{{\rm X} }_{m} )=0,\] 

тоді        

   $\hat{{\rm X} }_{m} =(H^{{\rm T}} R_{z}^{-1} H)^{-1} H^{{\rm T}} R_{z}^{-1} Z_{m} $.                      
\eqref{eq:__11_12_}

Вираз \eqref{eq:__11_12_} стає вихідним для розробки алгоритму отримання оптимальних 
оцінок  за методом максимуму правдоподібності.

Для визначення цих оцінок необхідно:

\begin{enumerate}
\item - -накопичити \textit{m}  спостережень -- \textbf{Z}\textit{m};

\item - -знати кореляційну матрицю \textbf{R }\textit{z} похибок вимірника;

\item - -знати матрицю зв'язків спостереження \textbf{H}.
\end{enumerate}

Структурна схема отримання оптимальних оцінок за методом максимуму правдоподібності 
показана на рис.11.7.

Як і для алгоритму оцінок за МНК отримання оцінки $\hat{X}_{m{\rm \; }} $  пов'язано 
з накопиченням вимірювань \textbf{Z}\textit{m}, тому цей метод, як і МНК, можна використовувати 
лише при вимірюванні одного параметра декількома системами. В іншому випадку нова 
оцінка похибок  не буде співпадати з поточним значенням похибок на час, який дорівнює 
часу накопичення спостережень.

$ $\textbf{Рекурентний метод обробки інформації}

 Рекурентний метод обробки інформації\textbf{ }дозволяє отримати оцінку параметра 
після кожного досліду. Для пояснення рекурентного (послідовного) методу отримання 
оцінок будемо вважати, що зроблено \textit{m} вимірювань параметра \textit{x}, тобто

\[z_{i} 
=x+v_{z_{i} } ;{\rm \; \; \; \; \; \; \; \; \; }i=\overline{1,{\rm \; }m},\] 

де  \textit{x} -- вимірюваний параметр; $v_{z_{i} } $-- похибка \textit{i}-го спостереження.

За 
результат оцінки вимірюваного параметра \textit{хm }при проведенні \textit{m} спостережень 
візьмемо:

\[\hat{x}_{m} =\frac{1}{m} \sum _{i=1}^{m}z_{i,}  \] 

Тоді на черговому (\textit{i }+ 1)-му кроці вимірювань значення оцінки $\hat{x}_{m+1} $ має 
вигляд:

\[\dot{\% }(t)  \frac{1}{p} \hat{X}_{m} (t)(H^{{\rm T}} R_{z}^{-1} H)^{-1} H^{{\rm 
T}} R_{z}^{-1} \dot{\% }(t)  \frac{1}{p} \hat{X}_{m} (t)(H^{{\rm T}} R_{z}^{-1} H)^{-1} 
H^{{\rm T}} R_{z}^{-1} \hat{x}_{m+1} =\frac{\left(\sum _{i=1}^{m}z_{i}  \right)+z_{m+1} 
}{m+1} =\frac{m}{m+1} \left(\frac{1}{m} \sum _{i=1}^{m}z_{i}  \right)+\frac{1}{m+1} 
z_{m+1} \] 

або              

\[\hat{x}_{m+1} =\frac{m}{m+1} \hat{x}_{m} +\frac{1}{m+1} z_{m+1} ,\] 

де  \textit{zm}+1 -- останнє (\textit{m }+1)-\textit{ше} спостереження.

Додамо та віднімемо від правої частини виразу \eqref{eq:__11_11_} значення $\hat{x}_{m} $, 
внаслідок чого отримаємо

\[\hat{x}_{m+1} =\hat{x}_{m} +\left(\frac{m}{m+1} -1\right)\hat{x}_{m} +\frac{1}{m+1} 
z_{m+1} =\hat{x}_{m} -\frac{1}{m+1} \hat{x}_{m} +\frac{1}{m+1} z_{m+1} .\] 

Остаточно можна записати:

\[\hat{x}_{m+1} =\hat{x}_{m} +\frac{1}{m+1} \left(z_{m+1} -\hat{x}_{m} \right)\] 

або, 
позначивши  $\frac{1}{m+1} =k$,    

\begin{equation} \label{GrindEQ__11_11_} \hat{x}_{m+1} =\hat{x}_{m} +k\left(z_{m+1} 
-\hat{x}_{m} \right). \end{equation} 

Отже, оцінку $\hat{x}_{m+1} $ можна отримати з попередньої оцінки$\hat{x}_{m} $\textbf{ }шляхом 
складання її з різницею між новим  спостереженням \textit{zm}+1 та попередньою оцінкою, 
помноженою  на коефіцієнт ваги \textit{k}. У цьому випадку зникає необхідність зберігати \textit{m} спостережень, 
отриманих на попередніх кроках вимірювання, оскільки вся попередня інформація об'єднана 
в апріорній оцінці $\hat{x}_{m} $.\textbf{}

$\hat{x}_{m+1} (t)$$ $$ $$\hat{x}_{m} (t)$$ $$ $$\hat{x}_{m} (t)$$ $$ $$\dot{x}_{m+1} $$ $$ $$\frac{1}{p} $$\frac{1}{m+1} $$\hat{x}_{m+1} 
(t)$$ $$ $$\hat{x}_{m} (t)$$ $$ $$\hat{x}_{m} (t)$$ $$ $$\dot{x}_{m+1} $$ $$ $$\frac{1}{p} $$\frac{1}{m+1} $Математична 
модель рекурентного методу обробки інформації показана на рис. 11.8.

Рекурентний алгоритм \eqref{eq:__11_11_} зв'язує поточне значення оцінки $\hat{x}_{m+1} $\textbf{ }з 
її попереднім значенням $\hat{x}_{m} $.

Різниця (\textit{zm}+1 --$\hat{x}_{m} $) стає показником  цінності інформації, яку 
отримують при проведенні (\textit{zm}+1)-го спостереження. Дійсно, якщо ця різниця 
близька до нуля, то зафіксоване спостереження \textit{zm}+1 не несе будь-якої нової 
інформації у порівнянні з апріорною, і в цьому випадку $\hat{x}_{m+1} \cong \hat{x}_{m} $. 
Навпаки, при великій різниці  (\textit{zm}+1 --$\hat{x}_{m} $) з урахуванням вагового 
коефіцієнта здійснюється суттєве уточнення оцінки$\hat{x}_{m} $, отриманої на попередньому 
кроці розрахунків. 

Але коефіцієнт  $k=(m+1)^{-1} $отримано без використання критерію  оптимальності, 
тому оцінка$\hat{x}_{m+1} $ також не є оптимальною, що знижує цінність даного методу 
обробки інформації.


\textbf{Алгоритм неперервного оптимального фільтра Калмана}


Алгоритм неперервного оптимального фільтра Калмана (ОФК) об'єднує розв'язання 
двох задач: спостереження та фільтрації. Принцип побудови ОФК розглянемо для випадку 
лінійної системи, описаної системою рівнянь \eqref{eq:__11_5_}. Ставиться задача 
знайти такі оцінки $\hat{X}$, щоб похибки оцінювання 


$\Delta (t)=X(t)-\hat{X}(t)$


були мінімальними. Іншими словами, необхідно забезпечити мінімум суми діагональних 
елементів матриці

\[P(t)=\left[\begin{array}{cccc} {p_{11} } & {p_{12} } & {\cdots } & {p_{1n} } \\ 
{p_{21} } & {p_{22} } & {\cdots } & {p_{2n} } \\ {\vdots } & {\vdots } & {\vdots 
} & {\vdots } \\ {p_{n1} } & {p_{n2} } & {\cdots } & {p_{nn} } \end{array}\right]=M
\left[\Delta {\rm \; }(t)\Delta ^{{\rm T}} (t)\right] .\] 

У коваріаційній матриці \textbf{P}(\textit{t}) члени $p_{ii} =D_{i} =\sigma ^{2} 
_{i} $ характеризують дисперсії похибок оцінювання співпадаючих координат \textit{хi},  
а члени $p_{ij} $ -- їхню взаємну кореляцію.

Відомо, що для динамічної системи \eqref{eq:__11_5_} оптимальним фільтром, що 
забезпечує мінімум сліду матриці \textbf{P}, стає алгоритм, який складається з трьох 
блоків:

1) основного блока

\[\hat{\dot{X}}(t)=A(t)\hat{X}\left(t\right)+K_{{\rm D}} (t)\left[Z\left(t\right)-H(t)
\hat{X}\left(t\right)\right];\] 

2) блока обчислення коефіцієнтів корекції 

$K_{{\rm D}} (t)=P\left(t\right)H^{{\rm T}} (t)R_{Z} ^{-1} (t)$;                                    \eqref{eq:__11_12_}

3) 
блока розв'язання рівняння коваріацій

$\dot{P}(t)=A(t)P\left(t\right)+P\left(t\right)A^{{\rm T}} (t)-P\left(t\right)H^{{
\rm T}} (t)R_{Z}^{-1} (t)H(t)P\left(t\right)+B(t)R_{X} B^{{\rm T}} (t)$.\eqref{eq:__11_13_}

Цей 
алгоритм забезпечує оцінювання всіх координат вектора \textbf{X} при виконанні умов 
спостережливості динамічної системи. Його особливість полягає у виборі коефіцієнтів 
матриці \textbf{К}ф. Вихідні значення оцінок  $\hat{X}\left(0\right)$ визначаються 
за попередньою інформацією про координати вектора \textbf{Х}. Коваріаційну матрицю 
поточних похибок оцінювання \textbf{Р}(\textit{t}) необхідно обчислювати безперервно 
в процесі оцінювання або попередньо, якщо це можливо.

Зміст рівняння коваріацій, за яким розраховується матриця \textbf{P}(\textit{t})\textit{,} пояснюється 
таким чином: якщо модель динамічної системи ідеально точна, то реальні похибки оцінювання 
для фільтра, побудованого за вказаним рівнянням, будуть співпадати з коваріаціями \textbf{P}(\textit{t})\textit{ }і 
при цьому вони будуть мінімальні. Отже, оптимальність фільтра буде досягнута тільки 
у випадку точного опису структури моделі системи та правильно обраних параметрів: \textbf{A}, \textbf{B}, \textbf{H}, \textbf{R}\textit{X, }\textbf{R}\textit{Z}.

Ось 
чому при синтезі систем оцінювання особливу увагу приділяють процесу складання математичних 
моделей вимірників та досліджуваної системи.

Пояснимо фізичний зміст деяких елементів ОФК.    

У рівнянні коваріацій \eqref{eq:__11_13_} член \textbf{A}(\textit{t})\textbf{P}(\textit{t})\textit{+}\textbf{P}(\textit{t})\textbf{A}T(\textit{t}) 
характеризує зміну похибок в результаті власного руху системи. Для стійкої системи 
елементи матриці \textbf{А}(\textit{t}) (\textit{аіj }$<$ 0), що визначає зменшення 
швидкості зміни  дисперсій похибок оцінювання співпадаючих координат ($\dot{p}_{ii} 
<0$) і призводить до зменшення похибок, оскільки в стійкій\textbf{ }системі й $X(t)
\to 0$ і $\hat{X}(t)\to 0$ (початкова невизначеність із часом зменшується). 

Множник \textbf{P}(\textit{t})\textbf{H}Т(\textit{t})\textbf{R}\textit{z-}1(\textit{t})\textbf{H}(\textit{t})\textbf{P}(\textit{t}) 
характеризує ефект корекції оцінок від вимірників і завжди визначає $\dot{p}_{ii} 
<0$, тобто вказує на зменшення похибок оцінювання. 

Множник \textbf{B}(\textit{t})\textbf{R}\textit{X}(\textit{t})\textbf{B}T(\textit{t}), 
що характеризує величину невизначеності  контрольованого процесу, завжди визначає, 
що $\dot{p}_{ii} >0$, а значить похибки оцінювання не можуть бути зведені до нуля.

Зауважимо, 
що, оскільки рівняння коваріацій \eqref{eq:__11_13_} автономне і не зв'язане 
з реальними вимірюваннями \textbf{Z}(\textit{t})\textit{,} то його розв'язання може 
бути виконано окремо від основного алгоритму фільтра й навіть заздалегідь. Перед 
розв'язанням рівняння коваріацій необхідно оцінити та задати початкові значення \textbf{P}\eqref{eq:__0_} 
на основі аналізу точності інформації про координати вектора \textbf{Х(t)} 
і початкові значення $\hat{X}\eqref{eq:__0_}$, яка існує до вмикання фільтра. 
На практиці, зазвичай, обмежуються завданням головної діагоналі \textit{ріі }$<$ 
0, тобто вихідних дисперсій похибок оцінювання, вважаючи, що  $@_{ij} =0,{\rm \; 
\; \; (\; }i\ne j).$

З часом  процес розв'язання рівняння коваріацій (рис.11.9) приходить до усталеного  
значення \textbf{P}уст, при якому виконується рівність

\[P_{{\rm CAB}} ^{{\rm "}} R_{Z}^{-1} HP_{{\rm CAB}} =AP_{{\rm CAB}} +P_{{\rm CAB}} 
A^{{\rm T}} +BR_{X} B^{{\rm T}} .\] 

$PH^{{\rm T}} R_{Z}^{-1} $$ $$ $$\frac{1}{p} $$\dot{P}=AP+PA^{{\rm T}} +BR_{X}^{} 
B^{{\rm T}} -PH^{{\rm T}} R_{Z}^{-1} HP$$ $$ $$\stackrel{\frown}{Z}$$ $$ $$\stackrel{
\frown}{X}$$ $$ $$\stackrel{\frown}{X}\eqref{eq:__0_}$$ $$ $$\stackrel{\frown}{
\dot{X}}$$ $$ $$PH^{{\rm T}} R_{Z}^{-1} $$ $$ $$\frac{1}{p} $$\dot{P}=AP+PA^{{\rm 
T}} +BR_{X}^{} B^{{\rm T}} -PH^{{\rm T}} R_{Z}^{-1} HP$$ $$ $$\stackrel{\frown}{Z}$$ $$ $$\stackrel{
\frown}{X}$$ $$ $$\stackrel{\frown}{X}\eqref{eq:__0_}$$ $$ $$\stackrel{\frown}{
\dot{X}}$$ $$ $


Задача оцінки параметрів полягає в знаходженні компромісу при фільтрації 
завади, спектр якої визначений \textbf{R}\textit{Z}, і корисного сигналу, спектр 
якого визначений як \textbf{R}\textit{X}, з урахуванням динамічних характеристик 
процесу, тобто матриці коефіцієнтів системи \textbf{А}.

Зауважимо, що для стійкої та детермінованої системи, тобто при відсутності випадкових 
збурень (\textbf{R}\textit{X }= 0), завжди   $\dot{p}_{ii} <0$   та   $\mathop{\lim 
}\limits_{t\to \infty } P_{{\rm CAB}} =0$.

Таким чином, із часом  похибка оцінювання наближається до нуля. При цьому шуми вимірників 
не впливатимуть на похибку оцінювання, оскільки самі  вимірники  відключаються від 
схеми фільтра при \textbf{Р }= 0 і, як наслідок, відповідно до  рівняння коефіцієнтів 
корекції \textbf{К}Ф = 0\textit{.} При синтезі фільтрів цей ефект необхідно враховувати 
і до збурень \textbf{V}\textit{X} відносити не тільки реальні фізичні збурення, наприклад, 
вітрові збурення, що діють на ЛА, але й можливі неточності завдання моделей процесу 
і вимірювання керувань.

У рівнянні коефіцієнтів корекції \eqref{eq:__11_12_} матриця \textbf{К}ф має 
вигляд:

\[_{{\rm \$ }} =\left[\begin{array}{cccc} {k_{{\rm \$ }11} } & {k_{{\rm \$ }12} } 
& {\cdots } & {k_{{\rm \$ }1m} } \\ {k_{{\rm \$ }21} } & {_{{\rm \$ }22} } & {\cdots 
} & {k_{{\rm \$ }2m} } \\ {\vdots } & {\vdots } & {\vdots } & {\vdots } \\ {k_{{
\rm \$ }n1} } & {k_{{\rm \$ }n2} } & {\cdots } & {k_{{\rm \$ }nm} } \end{array}\right].\] 

Елементи 
цієї матриці $k_{{\rm D}_{ij} } $характеризують зв'язок сигналу вимірника \textit{zj }з 
оцінкою координати \textit{$\hat{x}_{i} $.} Матриця спостережень \textbf{Н}Т у рівнянні 
коефіцієнтів корекції визначає такий факт: якщо координата \textit{xi} не бере участь 
у формуванні сигналу \textit{zj}, то коефіцієнт $k_{{\rm D}_{ij} } $= 0. Між коефіцієнтами $k_{{
\rm D}_{ij} } $ і  \textit{рij}  відповідно до  рівняння корекції існує пряма залежність, 
що свідчить про те, що підсилення корекції оцінок відбувається сильніше в тих координатах, 
у яких більші передбачувані похибки оцінювання, а обернена залежність коефіцієнтів $k_{{
\rm D}_{ij} } $ і  \textit{Rzij} (у рівнянні корекції $_{{\rm D}} =f(R_{Z}^{-1} )$) 
визначає більш значні сигнали корекції від більш точних вимірників.

Як висновок, зобразимо структурну схему алгоритму ОФК, що відповідає наведеним вище 
рівнянням (рис. 11.10).

Зауважимо, що оскільки при  обчисленні \textbf{К}Ф необхідно обчислювати обернену 
матрицю $R_{Z}^{-1} $, то ні для одного з вимірників не можна задати \textit{Rzij} = 
0, а якщо такий вимірник існує, то необхідно штучно задати малий розмір \textit{Rzij} .

Саме калманівська фільтрація є найбільш привабливою для застосування в жорстко зв'язаних 
і глибоко інтегрованих інерціально-супутникових системах навігації. 




