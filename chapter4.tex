\section{Програмне забезпечення автоматизації діагностування САУ ПС}

Для забезпечення автоматизації процесу діагностування САУ ПС в ході 
роботи була розроблена програма, що визначає основні показники якості 
процесів регулювання та вказує на оптимальні параметри системи виходячи 
з заданих експлуатаційних показників моделі. 

Програмне забезпечення було розроблене за допомогою мови програмування C++ в 
середовищі KDevelop для операційної системи Linux. Для розрахункової частини 
була розроблена система класів, що забезпечувала програмний інтерфейс для 
зручної роботи з математичними бібліотеками LAPACK++ та BLAS, що дасть 
значний приріст обчислення при використанні програми на базі кластерної 
платформи. В своїй роботі програма використовує бібліотеку STL для зберігання 
як математичних, розрахункових даних, так і параметрів інтерфейсу.

Інтерфейсна частина програми була розроблена з використанням бібліотеки 
віджетів Qt 4, яка є кросплатформенною, а отже, програма може бути 
скомпільована і для інших операційних систем, зокрема Windows, Mac OS X або 
Solaris. Для відображення графічної інформації додатково були розроблені з 
використанням технології подвійної буферизації конструктори графіків з їх 
похідними для коректного відображення необхідної інформації.


\subsection{Опис вхідних та вихідних даних}
Вхідними даними для системи є математична модель системи, записана у 
формі передатної функції, з вказаними в ній параметрами блоків. 
Вихідними даними програми є показники якості процесів регулювання, 
оптимальні значення двох змінюваних параметрів, та графічна інформація, 
представлена діаграмою стійкості відносно двох контрольованих параметрів, 
амплітудофазочастотна характеристика замкнутого контуру та логарифмічна 
амплітудочастотна характеристика. Додатково на діаграмі стійкості 
виводиться область оптимальних регулювань для заданих критеріїв жорсткості 
системи та демпфування.

\subsection{Огляд графчного інтерфейсу та елементів управління}
Інтерфейс програми побудований на базових принципах конструювання 
багатодокументних інтерфейсів, який матиме програма в подальшій розробці. 
Використання динамічного меню, панелей інструментів та плаваючих вікон додає 
їй гнучкості у відображенні результатів для систем з різною роздільною 
здатністю монітора. Найбільш вживані операції винесені на панелі інструментів.

Після запуску програми вона попросить за допомогою майстра ввести основні дані,
 необхідні для роботи системи(рис. \ref{fig:wizard} )

\begin{figure}
\centering
\includegraphics[scale=0.5]{wizardPage5}
\caption{Налаштування параметрів системи}
\label{fig:wizard}
\end{figure} 

