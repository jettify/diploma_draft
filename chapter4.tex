\section{Аналіз та вибір схем оцінюванн та корекції в комплексній інерціально-супутниковій системі}

Основними задачами пілотажно-навігаційних комплексів (ПНК) як постачальника 
інформаційного забезпечення польоту ЛА є сумісна обробка навігаційної інформації, 
яка надходить на борт ЛА та забезпечення високої надійності функціонування бортових 
систем та комплексів ЛА і взагалі безпеки польоту за рахунок резервування 
джерел інформації. Висока ефективність використання інформації, яка 
надходить на борт ЛА, забезпечується застосуванням різних методів її обробки. 

Найкращі результати підвищення якісних характеристик вимірювальних комплексів 
досягаються  в системах зі структурною надмірністю, коли існує можливість 
отримання пілотажно-навігаційної інформації паралельно декількома способами з 
використанням інформації від приладів та вимірювальних систем, що входять до 
складу ПНК. Отримана таким чином інформація комплексується.

В існуючих ПНК широке розповсюдження знайшли такі способи сумісної обробки 
інформації, що надходять від декількох вимірників, як взаємна компенсація і 
фільтрація похибок вимірювальних приладів, що вимірюють один і той самий 
навігаційний параметр та оптимальне оцінювання вектора стану з використанням 
апріорної інформації про контрольований процес та поточні вимірювання.

Методи оптимальної обробки інформації в ПНК використовуються з метою 
отримання оцінок вектора стану повітряного судна (або деякої частини 
цього вектора) в умовах впливу випадкових збурень і завад на процес 
вимірювання. При цьому оцінюються не самі параметри польоту, а їхні похибки. 
За оптимальної обробки пілотажно - навігаційної інформації в ПНК найважливішим 
процесом є процес отримання оптимальних оцінок. В основу алгоритмів отримання 
оптимальних оцінок можуть бути покладені такі методи обробки інформації:
\begin{itemize}
 \item метод найменших квадратів;
 \item метод максимуму правдоподібності;
 \item рекурентний неоптимальний фільтр;
 \item оптимальний фільтр Калмена.
\end{itemize}




%\textbf{11. 3. Методи оптимальної обробки інформації }

 Методи оптимальної обробки інформації в навігаційних комплексах використовуються  
з метою отримання оцінок вектора стану ПС (або деякої частини цього вектора) в умовах 
впливу випадкових збурень і завад на процес вимірювання. При цьому оцінюються не 
самі параметри польоту, а  їхні похибки.

Нехай вектор стану динамічної системи описується векторно-матричним рівнянням                            
\begin{equation}
\label{eq:__11_3_}
\dot{X}(t)=A(t)X(t)+B(t)V_{x}(t)
\end{equation}
\begin{ESKDexplanation}                
\item де $\dot{X}\left(t\right)$ -- \textit{n}-мірний вектор стану системи; 
\item \textbf{A(t)} -- квадратна матриця розмірності $n\times n$, яка являє собою матрицю коефіцієнтів 
системи; 
\item $V_{x} \left(t\right)$ -- \textit{k}-мірний вектор збурень, які діють 
на вході динамічної системи; 
\item \textbf{B}(\textit{t}) -- матриця збурень.
\end{ESKDexplanation}
Будемо вважати, що компоненти вектора $V_{x} \left(t\right)$ лінійно зв'язані з випадковими 
функціями типу білого шуму, мають нульові математичні сподівання  $M[V_{x}(t)]=0$ 
і характеризуються кореляційною матрицею $R_{x} \left(t\right) = M[V_{x}(t),V_{x}(t)^{T}]$. 

Отже,
\[M[V_{x}(t)]=0 \] 
\[R_{x} \left(t\right) = M[V_{x}(t),V_{x}(t)^{T}] \]
\begin{ESKDexplanation}
 \item де \textit{М} - символ математичного сподівання; 
 \item $\delta (t -\tau)$ -- дельта-функція.
\end{ESKDexplanation}

З  вектором стану системи $X(t)$ співвідношенням
\begin{equation} 
\label{eq:__11_4_} Y(t)=H(t)X(t) 
\end{equation} 

зв'язаний вектор спостережень \textbf{Y(t)} розмірності \textit{m}.
У рівнянні \eqref{eq:__11_4_} \textbf{H(t)} -- матриця зв'язку (матриця 
спостереження). Необхідною умовою оптимального оцінювання є повна спостережливість 
вектора стану \textbf{Х(t)} за вектором спостереження \textbf{Y(t)}.

Вважається, що процес \textbf{Х(t)} цілком спостережний на інтервалі $t \epsilon [t_{1},t_{2}]$,
 якщо за значенням вектора \textbf{Y(t)} при$t \epsilon [t_{1},t_{2}]$ 
можна вичислити значення вектора \textbf{Х(t)} при$t \epsilon [t_{1},t_{2}]$].
Умови повної спостережливості виконуються при \textit{m}$\leq$\textit{n}. Оскільки 
компоненти вектора \textbf{Y(t)} вимірюються з похибками, то як наслідок 
вимірювання отримують новий вектор, так званий вектор вимірювання  
\[ Z(t) = Y(t) + v_{z}(t),\]
де $V_{z}(t)$ --  вектор 
похибок вимірювання (припускається, що компоненти вектора $V_{z}(t)$ 
можна подати у вигляді білого шуму з нульовим математичним сподіванням). Вектор \textbf{V}\textit{z}(\textit{t}), 
аналогічно вектору \textbf{V}\textit{x}(\textit{t})\textit{,} характеризується кореляційною 
матрицею \textbf{R}\textit{z}(\textit{t}) розмірності \textit{m }$\times$\textit{ n}, 
тобто 

\textit{М }[\textbf{V}\textit{z}(\textit{t})]\textit{ =}0; \textit{}

\textit{М }[(\textbf{V}\textit{z}(\textit{t})\textit{,$V_{z}^{{\rm T}} (t)$}]\textit{ = }\textbf{R}\textit{z}(\textit{t})\textbf{$\delta$}(\textit{t} \textit{--} \textit{$\tau$}) \textit{.}

Припускаючи, 
що компоненти векторів \textbf{V}\textit{x}(\textit{t})\textit{ }і \textbf{V}\textit{z}(\textit{t}) 
некорельовані, на підставі викладеного математичну модель динамічної системи та рівняння 
спостереження можна записати у вигляді 

\begin{equation} \label{eq:__11_5_} \begin{array}{l} {\dot{{\rm X} }(t)=A(t)X(t)+B(t)V_{x} 
(t);} \\ {Z(t)=H(t)X(t)+V_{z} (t).} \end{array} \end{equation} 

При оптимальній обробці інформації в навігаційних комплексах (НК) найбільш важливим 
процесом є процес отримання оптимальних оцінок $\hat{{\rm X} }\left(t\right)$. В 
основу алгоритмів отримання оптимальних оцінок можуть бути покладені такі методи 
обробки інформації:

\begin{enumerate}
\item - -метод найменших квадратів (МНК);

\item - -метод максимуму правдоподібності;

\item - -рекурентний метод (оптимальний фільтр  Калмана ).
\end{enumerate}














