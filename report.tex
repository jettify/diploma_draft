\documentclass[ukrainian,utf8,simple,floatsubsection, hpadding=1mm,equationsubsection,]{eskdtext}
\usepackage[warn]{mathtext}
\usepackage[unicode]{hyperref} % enable hyperlinks (активувати посилання)
\usepackage{amssymb} % special math characters
\usepackage{amsmath} % using cyrillic in formulae
\usepackage{amsfonts} % special math fonts
\usepackage{eskdtotal} 
\usepackage{graphicx,epstopdf} % epstopdf-convert eps files to pdf
\graphicspath{{algorithms/}{schemes/}{software/}{fig/}} % look up folders for figures
\usepackage{listings} % to add source codes
\usepackage{longtable} % multipage tables
\usepackage{multirow} % using rowspan in tables
\usepackage{nomencl} % support for abbreviations
\makenomenclature % generate abbrevs index file
\usepackage{float}
% variables.tex
% This file contains information about author and other specific
% people for use in eskdx collection.

\title{\fontsize{12}{12} \selectfont Інтегрована інерціально-супутникова система навігації, що базується на принципах комплексної обробки інформації
з використанням калманівської фільтрації}
% smaller size of font set for the title in frame
\author{НовікМ.В.}

\ESKDchecker{ФіляшкінМ.К.}
\ESKDnormContr{КозловА.П.}
\ESKDapprovedBy{СинєглазовВ.М.}

\ESKDdepartment{Міністерство освіти і науки України}
\ESKDcompany{Національний авіаційний університет}

\ESKDsignature{НАУ 11 09 02 000 ПЗ}
\ESKDgroup{ІАСУ 608}

\ESKDsectAlign{section}{Center}
\ESKDsectAlign{subsection}{Center}
\ESKDsectAlign{subsubsection}{Center}

 % class parameters tuning
\ESKDcolumnXIfIV{РусаловськийА.В.}
\ESKDstyle{empty}
\renewcommand\labelenumi{\arabic{enumi}.} 
\renewcommand\labelenumii{\theenumi.\arabic{enumii}.}
\renewcommand\labelenumiii{\arabic{enumi}.\arabic{enumii}.\arabic{enumiii}.}
\begin{document}
\ESKDthisStyle{empty}
\footnotesize
\section*{Доповідь}
\subsection*{Слайд 3}

Комiтет IКАО з перспективних навiгацiйних систем (FANS- Future Air Navigation System) прийняв рiшення про обов’язкове використання систем супутникової навiгацiї в сполученнi з IНС. Тому в даний час у всiх галузях авiацiї
основним iнформацiйним ядром сучасного навiгацiйного комплексу є iнтегрована iнерцiально-супутникова система навiгацiї (IССН).

На літаку Ан-148 на даний момент встановлено високоточну безплатформенну інерціальну систему навігації французького виробництва та відносно грубу інерціальну курсовертикаль російського виробництва. Французька БІНС є високоточною і не вимагає корекції в загальному випадку, то ді як для курсовертикалі корекція необхідна. Тому перед українським підприємством Орізон Навігація поставлена задача інтегрувати супутникову навігаційну систему з курсовертикаллю.

Ця задача вирішувалась у рамках держбюджетної НДР 598ДБ09 ``Методики створення інерціально супутникової навігаційної системи на основі нанотехнологічної БІНС”

\begin{enumerate}
\item  пiдвищення точностi визначення координат, висоти, швидкостi i часу споживача;
\item  уточнення кутiв орiєнтацiї (курсу, крену i тангажа);
\item  оцiнка й уточнення параметрiв калiбрування навiгацiйних датчикiв,
таких, як дрейфи гiроскопiв, масштабнi коефiцiєнти, зсуви акселерометрiв тощо;
\item  забезпечення на цiй основi безперервностi навiгацiйних визначень на всiх етапах руху, у тому числi i при тимчасовiй непрацездатностi приймача СНС у випадках впливу завад або енергiйних маневрiв ЛА.
\end{enumerate}

\subsection*{Слайд 4}
На даний момент умовно розділяють 4 схеми інтегрування БІНС та СНС:


\textbf{Роздільна}\\
Надмірність, обмеженість похибок оцінок місця розташування і швидкості, наявність інформації про орієнтацію і кутову швидкість, висока швидкість видачі інформації, мінімальні зміни в бортовій апаратурі

\textbf{Слабко зв'язана}\\
Усі перераховані особливості роздільних систем, плюс більш швидке відновлення слідкування за кодом і фазою сигналів СНС, виставлення та калібрування БІНС у польоті, як наслідок -- підвищена точність під час відсутності сигналу СНС

\textbf{Жорстко зв'язана}\\
Подальше поліпшення точності і калібрування, підвищена стійкість слідкування за сигналами СНС при маневрах ЛА, підвищена завадостійкість 

\textbf{Глибоко інтегрована}\\
Єдиний фільтр усуває проблему ``каскадного'' включення 
фільтрів, компактність, знижені вимоги з енергозабезпечення. Недоліки: вектор стану 
містить до 40 компонентів, тому фільтр складно реалізувати; необхідність розробки спеціальних датчиків.
\subsection*{Слайд 5}

Об'єктом, на який передбачається встановлювати інтегровану навігаційну систему є пасажирський середньомагістральний літак українського виробництва, через це обираємо  слабкозв’язану схему комплексування, оскільки архітектура такої інтегрованої КІССН потрібує мінімальної зміни в апаратних засобах і програмному забезпеченні складових систем комплексної системи. Це дає можливість використовувати надійні, покупні і уніфіковані блоки системи і легко розширяти навігаційне забезпечення додаючи нове обладнання. До того ж вихідна інформація двох систем може просто піддаватися комплексної обробці з використанням тих чи інших алгоритмів оптимальної фільтрації. Окрім цього структурна надмірність надає більшу надійність системи: вихід однієї підсистеми з ладу не впливає на роботу іншої (на відміну з жорстко зв’язаною схемою).

Інформація про вимірювані псевдодальності \textit{R} і псевдошвидкості $\dot{R}_{i} $ використовується 
в алгоритмах розв'язання навігаційних задач для отримання координат і швидкості споживача, 
а також виправлень до еталона часу та частоти приймача СНС. При наявності надмірності 
з метою підвищення точності зчислення навігаційних параметрів здійснюється їхнє спільне 
оцінювання, зокрема з використанням оптимальної калманівської  фільтрації. 

Робота супутникової системи коригується від ІНС на етапі ``холодного'' і ``гарячого'' 
стартів. Тут приймач СНС використовує інформацію від ІНС тільки з метою 
більш надійного та швидкого відновлення захоплення сигналу у випадку його втрати. Передана по цьому 
каналу інформація про обчислені координати та швидкість ЛА у випадку втрати слідкування 
дозволяє розрахувати оцінки передбачуваної затримки сигналу $\tau$ та доплерівського 
зсуву частоти несучої $f_{\text{доп}}$, що суттєво знижує час пошуку та захоплення сигналу. 
В результаті значно знижується час відновлення роботи приймача після втрати сигналу, тобто тут в деякому смислі реалізоване об'єднання ІНС і СНС не тільки на рівні вторинної обробки інформації, а й на рівні 
первинної обробки радіосигналів. 

\subsection*{Слайд 6}

При побудові інтегрованих  навігаційних систем широке поширення одержав прийом, 
заснований на формуванні різницевих вимірів, зі складу яких виключаються шукані 
параметри. З використанням різницевих вимірів  вирішується задача оцінювання 
похибок однієї підсистеми на фоні похибок іншої підсистеми, Цей прийом  найчастіше 
називають методом одержання інваріантних оцінок. При реалізації такого методу 
використовуються лінійні моделі еволюції похибок підсистем  і не потрібно введення 
в загальному випадку нелінійних моделей еволюції самих шуканих навігаційних параметрів, 
що істотно спрощує побудову алгоритмів комплексної обробки навігаційної інформації і 
дає можливість застосування добре освоєних процедур оптимальної  лінійної калманівської 
фільтрації. 

На слайді представлена блок-схема, що зображує роботу лінійного фільтра Калмана.

Схему можна розділити на 2 частини:

- коло оцінки навігаційних параметрів, що має зворотній звязок

- коло вирішення рівняння Рікатті без зворотнього звязку 

Далі по схемі описуються вхідні та вихідні величини.

\subsection*{Слайд 7}
Еталонні параметри руху ЛА у функції часу задавалися аналітично, що дозволило уникнути методичних похибок у процесі моделювання.Кути тангажа, курса та крена (вважається, що поздовжня ось об’єкта спів-падає з вектором відносної швидкості, а кут крена пропорційний боковому прискоренню) визначаються з вказаних співвідношень.

Така траєкторія вибрана, з міркувань спостережності. Відомо, що тільки за допомогою лише СНС всі параметри системи не є повністю спостережні. Цю ситуацію можна покращити, виконуючи спеціальні маневри. Так для наземних транспортних засобів це є зупинка, в цьому випадку оцінюються і спостерігаються дрейфи ДПІ. Для літального апарату це є просторовий рух, з обертанням навколо всіх осей з прискоренням. Саме при такому маневрі є можливість оцінити дрейфи гіроскопів та зміщення нуля акселерометрiв.
\subsection*{Слайд 8-9-10}

ІСНС складається з 3 х систем: БІНС, СНС та БВ. Розглянемо кожну систему окремо. На плакаті представлено алгоритми роботи БІНС в географічній системі координат. Вирішення рівнянь орієнтації може здійснюватсь за допомогою метода Ейлера чи за допомогою кватерніонів і формул Родрига-Гамільтона.

На основі цих рівнянь в НДР виведено рівняння похибок БІНС, що представлені на наступному слайді. 
Ці рівняння є зручними для використання, так як є лінійними відносно відповідних похибок БІНС:
\begin{enumerate}
 \item Помилка приведеної координати в м.
 \item Помилка по швидкісті м/с
 \item Помилка орієнтації координатного тригранника, радіани.
 \item Дрейфи гіроскопів і акселерометрів.
\end{enumerate}
Матриця динаміки системи показана на наступному слайді, має розмірність 15х15.

\subsection*{Слайд 11-12}
Рівняння похибок БІНС перевірено на адекватність. Для цього було досліджено вплив похибок з різних джерел на стаціонарно закріплену БІНС. З графіків чітко прослідковується коливання з частотою Шулера. Наприклд: при похибці гіроскопа 0.01 градус в годину та при наявності нахилу координатного тригранника 10Е-3 радіан, зображено на слайді.

На наступному плакаті зображено як на координату впливають окремо дрейф гіроскопа, зміщення нуля акселерометра, нахил координатного тригранника. Кінцева похибка БІНС буде рівна сумі похибок по кожному з цих джерел. З графіка можна зробити висновок, що найбільший внесок в похибку відбувається саме із за дрейфу гіроскопа.

\subsection*{Слайд 13}
Інша не менш важлива система СНС має наступні рівняння. В загальному випадку похибка СНС складається з випадкової та корельованої складової. Також в рівняння входять складові, що діють одноразово приз зміні сузір’я супутників, але ця можливість не враховувалась під час моделювання.

У свою чергу дискретна модель еволюцiї квазiстацiонарної похибки БВ може бути представлена в наступному виглядi.
\subsection*{Слайд 14}

Враховуючи вище зазначені рівняння трьох систем, пропонується наступний загальний вектор стану системи. Що складається з 22-х змінних, в тому числі з 15 компонент БІНС, 1 компоненти БВ, та 6 компонент СНС.
Вектор вимірювань склає 8 елементів і реалізується як різниця між відповідними складовими БІНС та СНС, та різниці між виміріваннями висоти БВ та БІНС, БВ та СНС.

Матриця шумів динаміки системи має розмірність 22х34.

\subsection*{Слайд 15}
Для записаної системи в просторі станів, пропонується наступний вигляд лінійного ФК. Рівняння розділені на 2 частини, прогноз і корекцію, при чому не важливо з якого кроку починати. Представлені співвідношення відрізняються від канонічних рівнянь, ці рівняння ще називають стабілізовані рівняння Джозефа, або фільтр Калмана-Джозефа. 


В практичній реалізації фільтра, коваріація вимірюваного шуму $R$ вимірюється
звичайно до використання фільтра. Знаходження цієї матриці в загальному випадку
практична задача, виміряні величини дають можливість визначити дисперсію шуму, що
діє на данні датчики.

Визначення коваріації шуму системи $Q$ в загальному більш складна задача,
просто не має можливості безпосередньо спостерігати процес який оцінюється.
Інколи відносно проста модель процесу може дати прийнятний результат, якщо
введена достатня величина невизначеності процесу через вибір матриці $Q$.

В іншому випадку, є чи немає раціональної бази для вибору параметрів, часто
краща продуктивність (з точки зору статистики) може бути отримана за допомогою
налаштування параметрів фільтра $Q$ та $R$. 

Асиметрія коваріаційної матриці $P$ один з факторів, що впливає на чисельну
нестійкість рівняння Рікатті. 


При проектуванні фільтра, результат корекції коваріаційної матриці  
має бути перевірений не тільки на симетрію але й на додатно визначеність.
Якщо ці умови не будуть виконуються, це свідчить про помилки в програмі або
матриця погано обумовлена. Для усунення проблеми обумовленості використовується
інше рівняння для, яке називається формою Джозефа, яка показна на
наступному рівнянні.

З рівняння видно, що права частина є сумою двох симетричних матриць.
Перша додатно визначена інша не від'ємно визначена, що робить $P_{k}(+)$ 
додатно визначеною матрицею.
\subsection*{Слайд 16}
Розроблено програмне забезпечення. Дослідження проведені шляхом моделювання еволюцій похибок БІНС та СНС, та їх оцінки за допомогою оптимального рекурентного фільтра Калмана. Програмне розроблене на об'єктно-орієнтовній мові програмування Java, основною перевагою якої є незалежність від архітектури.

Розробка проведена в середовищі Java SE 6 (1.6.0). У офіційній реалізації, Java програми компілюються у байткод, який при виконанні інтерпретується віртуальною машиною для конкретної платформи. Sun Microsystems надає компілятор Java та віртуальну машину Java, які задовольняють специфікації Java Community Process, під ліцезією GNU General Public License. 

Під «незалежністю від архітектури» мається на увазі те, що програма, написана на мові Java, працюватиме на будь-якій підтримуваній апаратній чи системній платформі без змін у початковому коді та перекомпіляції.

Цього можна досягти, компілюючи початковий Java код у байт-код, який являє собою спрощені машинні команди. Потім програму можна виконати на будь-якій платформі, що має встановлену віртуальну машину Java, яка інтерпретує байткод у код, пристосований до специфіки конкретної операційної системи і процесора. Зараз віртуальні машини Java існують для більшості процесорів і операційних систем, в тому числі різноманітних версій GNU/Linux, Microsoft Windows, Apple Mac OS, мобільних платформ, наприклад Google Android та багато інших.

Математичні обчислення проводяться за допомогою спеціалізованої бібліотеки JAMA. Це програмна бібліотека для розв'язання задач лінійної алгебри. Ця бібліотека створена Національним інститутом стандартів і технологій США і схожа по функціональності з LAPACK. Існують версії JAMA для мов програмування C++ та Java.Іншою перевагою є те, що ця бібліотека розповсюджується вільно, з багатою документацією та джерельними кодами.

\subsection*{Слайд 17-22}
Нижче представлені результати імітаційного моделювання алгоритмів калманівської фільтрації при побудові інваріантного алгоритму комплексної обробки інформації на прикладі польоту ЛА за заданою траєкторією. На рис  представлено еволюції похибок оцінювання:

\begin{enumerate}
 \item Координати
 \item Швидкості
 \item Орієнтації
 \item Дрейфи ДКШ
 \item Акселерометрів
 \item Помилки оцінки кутового положення ЛА: курс, крен, тангажа
 \item Траєкторія руху Ла та оцінка при повному мовчанні СНС. Помітно значне покращення в оцінці навігаційних параметрів, похибка по координаті наростає значно повільніше.
\end{enumerate}

Похибки оцінювання навігаційних параметрів, що спостерігаються, -- координат та складових швидкості сходяться до рівня корельованих складових похибок СНС, при цьому наявність корекції від барометричного висотоміра забезпечує стійкість вертикального каналу інтегрованої системи навігації. Одночасно оцінюються параметри, що не спостерігаються -- параметри кутової орієнтації, а також квазістаціонарні складові похибок ДПІ, які можна використовувати для польотного калібрування ДКШ та акселерометрів. 

\end{document}