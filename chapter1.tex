\section{Обгрунтування необхідності розробки}

Для реалізації польотного завдання літальний апарат, повинен містити у складі 
бортового устаткування пілотажний та навігаційний комплекси. Під пілотажним 
комплексом у найпростішому випадку розуміється система автоматичного керування 
(автопілот), а під навігаційним комплексом (НК) \nomenclature{НК}{навігаційний комплекс} 
розуміють сукупність бортових систем і пристроїв, призначених для рішення задач 
навігації (навігаційна система). До складу НК і ПК входять датчики 
пілотажно-навігаційної інформації, навігаційні обчислювачі пристрою керування, 
індикації та сигналізації.

Датчики навігаційної інформації слугують для вимірювань параметрів різноманітних 
фізичних полів, на базі яких визначаються навігаційні елементи польоту. Їх 
можна поділити на дві групи: 1. датчики навігаційних параметрів положення, 
які визначають координати місцезнаходження літального апарата відносно опорних 
ліній і навігаційних точок ; 2. датчики навігаційних параметрів руху, які 
вимірюють параметри вектора швидкості літака та його складові: шляхову 
швидкість, вертикальну швидкість, напрямок польоту.

Датчики пілотажної інформації вимірюють параметри польоту, які характеризують кутовий 
рух ЛА : кути крену, тангажу, рискання і кутові швидкості.

Найважливішими з пілотажно-навігаційних датчиків є: інерціально-навігаційна 
система, інерціальна курсовертикаль, система курсу і вертикалі, допплерівський 
вимірник швидкості  і кута знесення типу ДВШЗ, інформаційний комплекс 
висотно-швидкісних параметрів типу ІК ВШП або система повітряних сигналів 
типу СПС \nomenclature{СПС}{система повітряних сигналів}.

Найбільш інформативною є інерціально – навігаційна  система (ІНС)\nomenclature{ІНС}{інерціальна навігаційна система}. 
Це така навігаційна система, у якій отримання інформації про швидкість і координати 
забезпечується шляхом інтегрування сигналів, що відповідають прискоренням ЛА. 
Інформація про прискорення надходить від розташованих на борту ЛА 
акселерометрів. Процедура інтегрування векторних величин, швидкості і 
прискорення, забезпечується шляхом відтворення на борту ЛА \nomenclature{ЛА}{літальний апарат} відповідної 
системи координат, для цього, частіше за все, використовують гіростабілізатори 
чи гіроскопічні датчики кутової швидкості з обчислювачем. 


В залежності від способу розташування акселерометрів  розрізняють платформні і 
безплатфомні ІНС. У першому випадку акселерометри  встановлюються на 
гіростабілізуючій платформі, у другому – безпосередньо на корпусі ЛА чи у 
спеціальному блоці чутливих елементів. Обидві системи мають свої переваги 
та недоліки. До переваг платформних ІНС відносять простоту алгоритмів обробки 
інформації про кутове положення і лінійні прискорення та високу точність, 
зумовлену сприятливими умовами роботи вимірювачів, оскільки вони розміщуються 
на гіростабілізаційній платформі, а не безпосередньо на корпусі об’єкта.

Зараз інтенсивно розвивається БІНС, перспективність яких визначається 
такими перевагами: висока надійність, низькі масогабаритні характеристики, 
зручність експлуатації. Характерна особливість таких ІНС, полягає у 
відсутності гіростабілізаційної платформи, яка являє собою складний 
електромеханічний пристрій та відкриває широкі можливості у плані 
зменшення масогабаритних характеристик й енергоспоживання.

До навігаційних датчиків, що визначають положення ЛА відносно навігаційних 
точок і базових ліній необхідно віднести радіотехнічні системи ближньої і 
дальньої навігації, літаковий далекомір, супутникову систему навігації (СНС), 
бортову радіолокаційну станцію, різні візирні пристрої, автоматичний компас, 
астрономічну навігаційну систему, кореляційно-екстремальну навігаційну систему. 
Найсучаснішими є супутникова навігаційна система і кореляційно-екстремальна 
навігаційна система.

СНС призначені для визначення місцеположення транспортних засобів, а також 
положення нерухомих об’єктів. Особливість дії СНС \nomenclature{СНС}{супутникова 
навігаційна система} – це використання штучних 
супутників Землі як радіонавігаційних точок, координати яких, на відміну від 
наземних радіолокаційних точок, змінні. 

Ці системи досить обґрунтовано довели високу експлуатаційну якість у 
різноманітних навігаційних галузях. Зокрема, вони визнані найбільш 
перспективними й економічно ефективними в більшості авіаційних сферах 
застосування. Поряд з цим, у зв’язку з можливою короткочасною втратою 
сигналів, які поступають із супутників, ці системи не можуть забезпечити 
необхідного рівня надійності навігаційних вимірів за такими показниками 
як цілісність, доступність і безперервність. Вирішити задачу підвищення 
цих показників можна шляхом комплексування супутникових навігаційних систем 
з іншими системами. Найбільш перспективним варіант полягає у інтеграції 
супутникових та інерціальних навігаційних систем. Така інтеграція дозволяє 
ефективно використовувати переваги кожної із систем. 

Інерціальні навігаційні системи, як найбільш інформативні системи, дають 
змогу одержувати всю сукупність необхідних параметрів для керування об'єктом, 
включаючи кутову орієнтацію. При цьому, такі системи цілком автономні, 
тобто для їхнього нормального функціонування не потрібно використання 
будь-якої інформації від інших систем. Ще одна з переваг цих систем полягає 
у високій швидкості надання інформації зовнішнім споживачам: швидкість 
відновлення кутів орієнтації складає до 100 Гц, навігаційної - від 10 до 
100 Гц. Цей показник для супутникових систем складає для кращих приймачів 
10 Гц, а для звичайних, як правило, 1 Гц. Разом з тим, інерціальним системам 
притаманні недоліки, що не дозволяють використовувати їх довгий час в 
автономному режимі. Вимірювальним елементам ІНС, насамперед, гіроскопам 
та акселерометрам, притаманні методичні й інструментальні помилки, 
вихідні данні не можуть бути введені абсолютно точно, обчислювач, 
що входить до складу ІНС, вносить свої похибки. Під впливом цих факторів 
ІНС працює в так званому «збуреному» режимі, і отримана від ІНС інформація, 
буде містити похибки, що викликані впливом цих збурень, і, головне, які з 
часом збільшуються. Для корекції ІНС застосовують різні методи і засоби. 

Корекція ІНС також може здійснюватися від радіотехнічних систем навігації 
(далекомірних, різницево-далекомірних), що складаються з наземної і бортової 
підсистем. Вони забезпечують одночасний вимір пеленга (азимута) і похилої 
дальності літального апарата щодо радіонавігаційної точки, і по цій інформації 
визначається місце розташування літака в заданій системі координат. 
До радіотехнічних систем варто віднести і супутникову систему навігації. 
Численні дослідження та практика експлуатації супутникових систем показують, 
що найбільш перспективним засобом корекції ІНС є супутникові системи, які 
володіють найбільш високою точністю і глобальністю застосування. При цьому 
можливо поліпшення характеристик автономних БІНС не тільки за координатами 
і швидкістю, але й за кутовою орієнтацією. 

Недоліком всіх радіотехнічних методів навігації, у тому числі і супутникових, 
є те, що на переданий і прийнятий радіосигнал можуть накладатися природні й 
штучно створювані радіозавади. Мала потужність сигналу, велика дальність джерел 
сигналу від приймачів (26000 км), мале відношення “сигнал-шум” приводить 
до слабкої перешкодозахищеності приймачів СРНС. Контури зрушення по фазі 
і за часом можуть легко “втратити” відповідний супутник при наявності активних 
перешкод. Особливо чуттєвим щодо цього є контур спостереження за фазою. 

До того ж, існує явище періодичного зникнення сигналу від СНС. При  збільшенні 
періоду “радіомовчання” супутника величина помилки навігаційних визначень 
збільшується аж до зриву керування (стабілізації на заданій траєкторії).  

Виникає потреба у автономних засобах навігації, які не вимагають зовнішніх 
сигналів, а тому й не зазнають впливу радіоелектронного придушення. Цим 
умовам відповідає так звана інерціальна навігація. Використання інтегрованих 
інерціально-супутникових систем обумовлюється наступним: інерціальна і 
супутникова навігаційні системи вимірюють різні параметри: СНС - лінійні 
параметри (вектор положення ЛА в деякій геоцентричній системі координат і 
вектор його швидкості), а ІНС - як лінійні, так і кутові параметри. 

Взагалі, СНС можна використовувати і для виміру кутових координат, але для 
цього необхідне використання декількох антен, установлених на визначеній 
відстані один від одного, і декількох приймачів, що різко ускладнюють й 
підвищують собівартість системи. Проте, використання корегованої від СНС, 
наприклад, за допомогою фільтра Калмана, ІНС дозволяє вимірювати кутове 
положення ЛА з досить малою похибкою. До того ж, ІНС дозволить екстраполювати 
сигнали СНС при значному періоді квантування сигналів. 

Використання інтегрованих інерціально-супутникових систем навігації (ІССН) 
компенсує недоліки окремих систем, і забезпечує високу точність і надійність 
виміру параметрів польоту. Це підтверджує необхідність включення до складу 
навігаційного забезпечення ЛА комплексної інерціально-супутникової системи 
навігації, а також,  розробки та дослідження працездатності алгоритмів її 
роботи, ступінь впливу похибок датчиків первинної інформації  безплатформної 
інерціальної системи (БІНС \nomenclature{БІНС}{безплатформенна інерціальна навігаційна система}) 
та супутникової навігаційної системи (СНС) на 
точнісні характеристики числення навігаційних параметрів і динаміку зміни 
похибок, впливу перерв у роботі СНС на траєкторний рух ЛА при польоті за 
складним маршрутом.

Саме тому тема  роботи є досить актуальною на сьогоднішній час.