\section{Аналіз та вибір навігаційного забезпечення}

Задача створення комплексної навігаційної системи на базі супутникової та інерціальної 
систем навігації для визначення координат місцеположення рухомого об'єкта, передбачає 
попередній аналіз існуючих варіантів компонентів комплексної навігаційної системи, тобто 
варіантів побудови супутникової й  інерціальної систем навігації та вибір за певними критеріями найбільш оптимальних. 

\subsection{Оцінка орієнтовних значень похибок вимірників первинної інформації БІНС }

Датчики первинної інформації БІНС -- датчики кутової швидкості й акселерометри встановлюються жорстко на ЛА. 
Тяжкі умови роботи датчиків інформації призводять до появи значних похибок, тому в алгоритмах роботи БІНС бажано 
здійснити аналітичну компенсацію похибок вимірників (здійснювати їх польотне калібрування), перш ніж ці сигнали 
будуть використані для розрахунку параметрів орієнтації і для визначення складових уявного прискорення уздовж навігаційних осей.

Інструментальні похибки ІНС визначаються погрішностями аmкселерометрів, вимірників кутової швидкості або кута, 
а також погрішностями обчислювального пристрою. Очевидно, при застосуванні обчислювального пристрою досить високої 
точності похибки, ІНС будуть визначатися головним чином погрішностями первинних вимірювальних датчиків, що входять у систему.

Якщо акселерометри ІНС вимірюють прискорення $a_{x} $ і $a_{y} $ з погрішностями $\Delta a_{x} $ і $\Delta a_{y} $, то,  
це приведе до помилки у визначенні координати $\Delta \lambda _{y} $.

Приладові значення зазначених параметрів (зі значком «*»)

\begin{equation} 
\label{eq:err} 
\left. 
\begin{array}{l} 
{a_{\xi }^{*} =a_{\xi } +\Delta a_{\xi } ;{\rm \; \; }a_{x}^{*} =a_{x} +\Delta a_{x} ;{\rm \; \; \; }a_{y}^{*} =a_{y} +\Delta a_{y} ;} 
\\ {\dot{\lambda }_{y}^{*} =\dot{\lambda }_{y} +\Delta \dot{\lambda }_{y} ;{\rm \; \; \; }\lambda _{y}^{*} =\lambda _{y} 
+\Delta \lambda _{y} ;{\rm \; \; \; }}
\\ {\ddot{\vartheta }'^{*} =\ddot{\vartheta }'+\Delta \ddot{\vartheta }'; \dot{\vartheta }'^{*} =\dot{\vartheta }'+\Delta \dot{\vartheta }';
{\rm \; \; \; }\vartheta '^{*} =\vartheta '+\Delta \vartheta '.} \end{array}\right\} 
\end{equation} 

Підставивши значення цих параметрів у перші рівняння систем і зробивши відповідні перетворення наступне рівняння похибок:

\begin{equation} 
\label{eq:lam_err} 
\Delta \ddot{\lambda }_{y} +\frac{(a_{\eta } +g_{0} )}{R_{{\text{З}}} } 
\Delta \lambda _{y} =\frac{1}{R_{{\text{З}}} } \left[a_{x} \cos (\lambda _{y} -\vartheta ')+a_{y} \sin (\lambda _{y} -
\vartheta ')\right] 
\end{equation} 

Як видно, ліва частина рівняння \eqref{eq:lam_err} є (при $a_{\eta } =0$) рівнянням маятника Шулера, а права -- збурюючим впливом.

Координата $\lambda _{y} $ і кут $\vartheta '$ у процесі руху безупинно змінюються, тому права частина рівняння \eqref{eq:lam_err} 
буде теж змінною в часі.

З огляду на вираз і те, що при автоматичному керуванні рухом кут відхилення об'єкта від площини горизонту досить малий, а також вважаючи

\[\Delta a_{x} =\Delta a_{y} =\Delta a\] 

у першому наближенні одержимо

\begin{equation} 
\label{eq:2_51} 
\Delta \ddot{\lambda }_{y} +\frac{1}{R_{{\text{З}}} } (a_{\eta } +g_{0} )\Delta \lambda _{y} \cong \frac{\Delta a}{R_{{\text{З}}} }  
\end{equation} 

При $a_{\eta } =0$, $\Delta a={\rm const}$ рішення рівняння \eqref{eq:2_51} буде наступним:

\begin{equation} 
\label{eq:2_52_} 
\Delta \lambda _{y} \cong \frac{\Delta a}{g_{0} } \left(1-\cos \left(\sqrt{\frac{g_{0} }{R_{{\text{З}}} } } \cdot t\right)\right) 
\end{equation} 

З виразу \eqref{eq:2_52_} видно, що помилка ІНС у визначенні; координати $\lambda _{y} $, обумовлена похибкою акселерометрів, 
буде мати як постійну, так і змінну складові.Найбільше значення похибки не перевищить  $\Delta \lambda _{y} \le 2\frac{\Delta a}{g_{0} } $. 

%Графік залежності $\Delta \lambda \left(t\right)$, отриманий шляхом моделювання однокомпонентної БІНС, при наявності 
%постійних похибок акселерометрів представлений на мал. 2.5, \textit{а}. 

%\includegraphics[bb=0mm 0mm 208mm 296mm, width=86.2mm, height=65.5mm, viewport=3mm 4mm 205mm 292mm]{image1.ps}\includegraphics[bb=0mm 0mm 208mm 296mm, width=84.4mm, height=65.3mm, viewport=3mm 4mm 205mm 292mm]{image2.ps}                   \textit{а)}                                                                          \textit{б)}


\textbf{Оцінка помилки акселерометрів}

За допомогою \eqref{eq:2_52_} можуть бути отримані орієнтовані формули для розрахунку точнісних вимог пропонованих до датчиків первинної 
інформації -- акселерометрам.

\begin{equation} 
\label{eq:acc_err} 
\Delta a\cong \frac{\Delta \lambda _{y} g_{0} }{\left(1-\cos \left(\sqrt{\frac{g_{0} }{R_{{\text{З}}} } } \cdot t\right)\right)}.    
\end{equation} 

Як випливає з \eqref{eq:acc_err} вимоги до точнісних характеристик акселерометрів залежать від проміжків часу 
автономної роботи БІНС у складі комплексної інерціально-супутникової системи навігації. Виходячи з вимог до 
точності визначення координат (СКВ  5 м) отримані орієнтовані значення похибок акселерометра, у залежності 
від очікуваних перерв у роботі супутникової системи навігації. Розрахункові значення точнісних вимоги 
пропонованих до датчиків первинної інформації, зокрема акселерометрів відображені на графіку рис. \ref{fig:acc_err} 

\begin{figure}
\centering
\includegraphics[scale=0.8]{acc_err}
\caption{Графік залежності значень похибок акселерометра від часу}
\label{fig:acc_err}
\end{figure} 




\textbf{Оцінка помилки датчика кутової швидкості}

Якщо вимірник кутової швидкості об'єкта має погрішність $\Delta \vartheta '$, то приладове значення кутової швидкості

\[\dot{\vartheta }'^{*} =\dot{\vartheta }'-\Delta \dot{\vartheta }'.\] 

При цьому, мабуть, будуть мати місце помилки у визначенні інших параметрів.

Підставляючи значення параметрів $\dot{\vartheta }'^{*} $ і  $\lambda _{y}^{*} $ рівняння \eqref{eq:2_51},  
після перетворень з врахуванням другого рівняння системи \eqref{eq:err} одержимо

\begin{equation} 
\label{eq:2_54_} 
\Delta \ddot{\lambda }_{y} +\frac{a_{\eta } +g_{0} }{R_{\text{З}} } \Delta \lambda _{y} =-\frac{a_{\eta } +g_{0} }{R_{\text{З}} } \Delta \vartheta ' 
\end{equation} 

Як видно ліва частина рівняння \eqref{eq:2_54_} і в цьому випадку (при $a_{\eta } =0$) представляється рівняння 
маятника Шулера, а права частина -- фактор, що викликається, обумовленими погрішностями у вимірі $\vartheta '$кута .

Якщо вважати погрішність $\Delta \dot{\vartheta }'=\Delta \dot{\vartheta }'_{0} =const$, то $\Delta \vartheta '=\Delta \dot{\vartheta }'_{0} t$, 
при цьому рішення рівняння 
\eqref{eq:2_54_} буде (якщо $a_{\eta } =0$) наступної:

\begin{equation} 
\label{eq:2_55} 
\Delta \lambda _{y} =\Delta \dot{\vartheta }'_{0} \left(\sqrt{\frac{R_{\text{З}} }{g_{0} } } \sin \sqrt{\frac{g_{0} }{R_{\text{З}} } } \cdot t-t\right)
\end{equation} 


Як видно з виразу \eqref{eq:2_55},  погрішність у визначенні координати  $\lambda _{y} $, обумовлена 
постійною  помилкою  вимірника кутової швидкості, у першому наближенні має дві складові (рис. 2.5,\textit{б)}, 
одна  з яких  росте пропорційно  часу польоту

\[\Delta \lambda _{y0} =\Delta \dot{\vartheta }'_{0} t,\] 

а інша  змінюється з періодом маятника Шулера

\[\Delta \lambda _{y} =\Delta \dot{\vartheta }'_{0} \sqrt{\frac{R_{\text{З}} }{g_{0} } } \sin \sqrt{\frac{g_{0} }{R_{\text{З}} } } \cdot t\] 

Графік залежності $\Delta \lambda \left(t\right)$, представлений на рис. 2.5, \textit{б }отриманий 
шляхом моделювання однокомпонентної БІНС, при наявності постійної похибки вимірника кутової швидкості. 

Аналогічно \eqref{eq:acc_err} можуть бути отримані орієнтовані формули для розрахунку точносних вимог 
пропонованих до вимірників кутових швидкостей.

\[\Delta \dot{\vartheta }'_{0} =\frac{\Delta \lambda _{y} }{\left(\sqrt{\frac{R_{{\text{З}}} }{g_{0} } } \sin \left(\sqrt{\frac{g_{0} }{R_{{\text{З}}} } } \cdot t\right)-t\right)} \] 

Виходячи з вимог пропонованих до точносних характеристик визначення координат (СКО $\approx$ 5м) 
отримані орієнтовані значення похибок вимірникам кутових швидкостей, у залежності від очікуваних 
перерв у роботі супутникової системи навігації. Розрахункові значення точнісних вимоги пропонованих 
до датчиків первинної інформації, зокрема вимірникам кутових швидкостей приведені в табл. 2.6 і відображені на графіку рис.\ref{fig:gyro_err}.

\begin{figure}
\centering
\includegraphics[scale=0.8]{gyro_err}
\caption{Графік залежності значень похибок ДКШ від часу}
\label{fig:gyro_err}
\end{figure} 






Вихідні похибки БІНС визначаються в основному наступними складовими;

похибками Дусів і акселерометрів;

методичними похибками;

похибками обчислень і похибками моделі використовуваної для обліку впливу гравітаційного поля на поводження інерціальних чуттєвих елементів.





Рисунок 2.6 -- графік орієнтованих значень похибок%\includegraphics[bb=0mm 0mm 208mm 296mm, width=116.2mm, height=91.8mm, viewport=3mm 4mm 205mm 292mm]{image3.ps} вимірника кутової швидкості

Для БІНС розглянутого класу основний внесок у похибки визначення координат вносять датчики первинної 
інформації. Необхідно відзначити, що методичні похибки, у тому числі похибки, зв'язані зі спрощеннями 
кінематичних рівнянь БІНС, похибками моделювання форми Землі і похибками моделі гравітаційного поля, 
повинні бути не більше  похибок, внесених датчиками первинної інформації.

Багато складові вихідні похибки залежать від параметрів траєкторії й умов роботи, коефіцієнти моделі 
похибок істотно залежать від рівня вібрації і температури. Тому для більш детального дослідження 
точністних характеристик  БІНС необхідна вихідна інформація про аеродинамічні й інерційно масові 
характеристиках літака, а також параметри траєкторії. У цьому випадку можна буде провести детальні 
статистичні дослідження точністних характеристик з урахуванням впливу динамічних похибок датчиків первинної інформації.



\textit{Таблиця 2.7.}

\begin{tabular}{|p{0.9in}|p{2.2in}|p{1.4in}|} \hline 
Джерела & Характеристика джерел похибок & Значення характеристик джерел похибок \\ \hline 
Гіроскопічні датчики & Систематичний дрейф, град/год & 0,0003 \\ \hline 
 & Похибка масштабного коефіціента & $2\cdot 10^{-6} $ \\ \hline 
 & Не виставка осі 1, кут.с & 1 \\ \hline 
 & Не виставка оси 2, кут.с & 1 \\ \hline 
 & Випадкове блукання, град/год & 0,0005 \\ \hline 
Акселерометри & Систематична помилка g $10^{-6} $ & 10 \\ \hline 
 & Похибка масштабного коефіціента & $10^{-5} $ \\ \hline 
 & Не виставка осі 1, кут.с & 0,5 \\ \hline 
 & Не виставка оси 2, кут.с & 0,5 \\ \hline 
 & Коефіцієнт впливу $g^{2} ,10^{-6} g/g^{2} $ & 1 \\ \hline 
 & Випадкова складова м/с3/год & 0,00015 \\ \hline 
Методична помилка & Відхилення гравітаційної вертикалі, кут.с & 3 \\ \hline 
Обчислювальні похибки, еквівалентні похибки ЧЭ & Систематичні похибки гіроскопів, град/год & 0,0001 \\ \hline 
 & Похибки масштабних коефіцієнтів гіроскопів & $0,1\cdot 10^{-6} $ \\ \hline 
 & Випадкове блукання гіроскопів, град/год & 0,0001 \\ \hline 
 & Систематичні похибки акселерометрів, g & $10^{-6} $ \\ \hline 
 & Похибки масштабного коефіцієнта акселерометрів & $5\cdot 10^{-6} $ \\ \hline 
\end{tabular}



Однак при моделюванні враховувалися тільки деякі складові:

систематичні;

перекручування масштабного коефіцієнта;

випадкові складові;

зони нечутливості

 Випадкові складові і перекручування масштабного коефіцієнта моделювалися з використанням генераторів "білого шуму" 
і формуючих фільтрів. При цьому вважалося, що кожен чуттєвий елемент цілком визначається значеннями цих складових, 
а самі ці складові змінюються таким чином, що при збільшенні одного з них зростають і всі інші.

У табл. 2.8, 2.9 показані приклади зміни складових  похибок датчиків первинної інформації.

У роботі на моделі трикомпонентної БІНС були проведені статистичні дослідження впливу похибок датчиків 
первинної інформації на точність визначення координат. Дослідження проводилися за багатофакторним планом 
з оцінкою одночасної зміни факторів. При дослідженнях варіювалися знаки систематичних і випадкових складових 
дев'яти датчиків первинної інформації і виявлялися погрішності визначення координат. У кожнім циклі статистичних 
іспитів для визначеного часу автономної роботи БІНС варіювалося 25 можливих комбінацій факторів. За результатами 
статистичних іспитів розраховувалися СКО у визначенні координат.    



\textit{Таблиця 2.8}

\begin{tabular}{|p{2.3in}|p{0.7in}|p{0.7in}|p{0.7in}|} \hline 
\multicolumn{4}{|p{1in}|}{Акселерометри} \\ \hline 
Зсув показань, 10-3 & 0,01 & 0,05 & 0,1 \\ \hline 
Масштабний коефіцієнт & 0,001 & 0,005 & 0,01 \\ \hline 
Неортогональність кут.с & 10 & 10 & 20 \\ \hline 
Випадкова складова, м/с3/год & 0,009 & 0,01 & 0,02 \\ \hline 
\end{tabular}



\textit{Таблиця 2.9}

\begin{tabular}{|p{2.3in}|p{0.7in}|p{0.7in}|p{0.7in}|} \hline 
\multicolumn{4}{|p{1in}|}{Датчики кутової швидкості} \\ \hline 
Дрейф, що не залежить від перевантаження, град/год & 0,005 & 0,01 & 0,1 \\ \hline 
Дрейф, що залежить від перевантаження, град/год & 0,0075 & 0,015 & 0,15 \\ \hline 
Масштабний коефіцієнт & 0,0025 & 0,005 & 0,02 \\ \hline 
Неортогональність кут.с & 20 & 60 & 120 \\ \hline 
Випадкове блукання, град/год & 0,0005 & 0,001 & 0,01 \\ \hline 
\end{tabular}



У наступному циклі абсолютні складових похибок датчиків первинної інформації змінювалися таким чином, 
що при збільшенні одного з них зростали і всі інші. Після проведення ряду циклів іспитів одержували 
залежності СКО у визначенні координат від складових похибок датчиків первинної інформації, і проводився 
контрольний цикл іспитів для визначення величини складових похибок датчиків для заданої в технічному завданні величини СКО. 

Аналогічні дослідження були проведені для різних значень передбачуваного часу автономної роботи БІНС. 
Результати досліджень приведені в табл. 2.10, а уточнені за результатами статистичних іспитів значення 
погрішностей акселерометрів і датчиків кутових швидкостей - у табл. 2.5, 2.6. Залежності  погрішностей 
акселерометрів і датчиків кутових швидкостей від часу автономної роботи БІНС проілюстровані графіками на 
рис. 2.11,2.12, де як порівняння приведені залежності, отримані розрахунковими методами.








