\section{Охорона навколишнього середовища}

% \subsection{Екологічний аналіз і раціональне природокористування}
% 
% Науково-технічний прогрес, разом з розвитком технічної бази, погіршує стан  навколишнього середовища. Більшість науково-технічних досягнень вражає своєю масштабністю. Разом з тим, така зміна негативно впливає по відношенню до навколишнього середовища. Так, для виготовлення авіаційного обладнання з будь-якого матеріалу, потрібно його добування з надр Землі, обробка  на промислових підприємствах для виробництва деталей. Все це наносить шкоду навколишньому середовищу, порушує рівновагу у природі, при цьому, шкода тим більша, чим більші параметри будь-якого виробу.
% 
% Однією з найважливіших проблем, яка виникає на етапах виробництва та експлуатації устаткування, а також при утилізації блоків, що відробили чи вийшли з ладу, є нанесення шкоди навколишньому середовищу.
% Оцінюючи серйозність проблеми охорони навколишнього середовища, суспільство бачить її рішення в необхідності збереження життя на планеті, а вирішення природоохоронних задач сьогодні розглядається як фактор, що визначає стан здоров’я людей.
% 
% Збиток, заподіяний антропогенним забрудненням навколишньому середовищу, складає приблизно 1 млрд. гривень у рік.
% Авіація, у числі інших галузей народного господарства, також впливає на біосферу в наслідок дії акустичного забруднення, емісій авіаційних двигунів, забруднення електромагнітними полями, відторгнення значних земельних ділянок і т. ін.
% 
% Літаки ЦА забруднюють атмосферне повітря шкідливими речовинами і здійснюють  шумовий  вплив  на  навколишнє  середовище.  Частка  забруднення атмосферного повітря в результаті емісії авіаційних двигунів близько 75\% 
% від усіх викидів підприємствами ЦА. Це пов'язано зі спалюванням великої кількості авіаційного палива сучасними літаками.  При проектуванні різноманітних  систем варто приділяти цьому увагу. 

% \subsection{Еколого-економічне обґрунтування}
% 
% Останнім часом усе більша увага приділяється економічним механізмам керування охороною навколишнього середовища, 
% що дозволяє уникнути безвідповідального використання природних ресурсів і, таким чином, виключити можливі екологічні збитки.
% 
% Основними напрямками економічного і соціального розвитку сучасних країн є наступні завдання:
% 
% \begin{enumerate}
%  \item підвищити ефективність заходів для охорони природи;
%  \item ширше впроваджувати прогресивні безвідхідні технології і процеси;
%  \item розвивати комбіновані виробництва, що забезпечують комплексне і повне використання природних ресурсів, 
% сировини і матеріалів, істотне зниження шкідливого впливу на навколишнє середовище.
% \end{enumerate}
% 
% В даний час усе частіше виникає проблема утилізації, навіть при наявності багатьох організацій, що 
% стежать за дотриманням законів про охорону навколишнього середовища. Часто зустрічаються випадки, коли підприємства закопують, чи скидають відходи виробництва в ґрунт або в водойми. Цими протизаконними діями наноситься величезний, найчастіше невиправний збиток природі.
% 
% Одним з можливих способів рішення проблеми утилізації є безвідхідне виробництво. При цьому відходи, що виникають у процесі виробництва на одному підприємстві, можуть бути сировиною для іншого. Налагоджена мережа таких взаємин може створювати повну чи часткову структуру безвідхідного виробництва. З метою зменшення збитку навколишньому середовищу при утилізації відходів традиційними методами, необхідно встановити систему постійного контролю за технологією утилізації відходів і місцем її здійснення. При цьому, у першу чергу, повинні враховуватися не економічні, а екологічні передумови і фактори. Серед галузей народного господарства, діяльність яких зв'язана з несприятливим впливом на навколишнє середовище, знаходитися цивільна авіація. Польоти літальних апаратів, їхня технічна експлуатація, супроводжуються забрудненням навколишнього середовища.
% 
% З екологічної точки зору, доцільно не використовувати екологічно-небезпечні матеріали, напівфабрикати, вироби, що тим самим забезпечує зменшення екологічних збитків.

\subsection{Дослідження екологічного впливу аіаційного транспортного комплексу}

У результаті авіатранспортних перевезень відбувається забруднення ґрунтів, водних об’єктів та атмосфери, а сама специфіка впливу повітряного транспорту на довкілля виявлена в значній шумовій дії та значних викидах різноманітних забруднюючих речовин.

Негативна дія різних авіаційних джерел шуму, в першу чергу, здійснюється на операторів, інженерів та техніків виробничих підрозділів. Так історично склалося, що аеропорти розташовані поблизу густозаселених районів міста. Тому з ростом міст та інтенсифікацією авіатранспортних процесів постає серйозна проблема співіснування міста та аеропорту. Населення авіаміста та розташованих поблизу селищ відчувають шум від літаків, що пролітають. У меншій мірі відчувають шум персонал аеропортів, авіапасажири та відвідувачі.

Крім шуму авіація призводить до електромагнітного забруднення середовища. Його викликає радіолокаційна та радіонавігаційна техніка. Аеропорти України здійснюють вплив на довкілля через стаціонарні джерела прямої та непрямої дії на навколишнє середовище, які розташовані в авіатехнічній базі, аеровокзальному комплексі з привокзальною площею, складах паливно-мастильних матеріалів, котельних, сміттєспалювальних станціях. Кількість шкідливих речовин, які потрапили у 2000 році в атмосферу від стаціонарних джерел в аеропортах, склала 23,1 тисяч тон. Разом з викидами забруднюючих речовин парк літаків споживає у великій кількості кисень. 

В аеропортах накопичуються тверді та рідкі відходи споживання та виробництва. У багатьох випадках ці відходи безпечні у санітарно-гігієнічному співвідношенні. Об’єми накопичення твердих відходів у 2000 році склали: виробничі відходи --- 43 тис. т; побутові відходи --- 79,9 тис. т; відходи, які видаляються з літаків міжнародних авіаліній, --- 2,1 тис. т. Відходами у аеропортах зайнято спеціальні приміщення площею до 3,3 тис. кв.м, а площа відкритих сховищ (звалищ) складає 118,7 тис. кв.м, з них тільки 18\% спеціально підготовлені для зберігання та накопичення відходів.

У цивільній авіації авіаремонтні заводи та аеропорти із спецавтотранспортом є найбільш інтенсивними джерелами забруднення природної води. Стічні води авіаремонтних підприємств та аеропортів складаються з виробничих і господарсько-побутових стічних вод та поверхневих стоків. 

Кількість стічних вод і їх склад змінюються протягом доби, тижня, місяця. Для ряду виробничих процесів характерний залповий скид сильно концентрованих стічних вод. Найбільшу небезпеку для водних об’єктів становлять стоки з території аеропорту: передангарного та доводневого майданчиків, складів паливо-мастильних матеріалів, майданчиків для миття. 

Поверхневі стоки з територій транспортних підприємств містять рідкі нафтопродукти, залишки миючих, дезінфікуючих, антиобмерзаючих і протиожеледних реагентів, формувальних сумішей, розчинів, використовуваних у металообробці, відпрацьовані електроліти акумуляторних батарей, продукти руйнування штучних покрівель і зносу шин. 

Атмосферні опади, потоки дощових та талих вод також поглинають частину димових газів котелень, шкідливих викидів авто- та авіатранспорту, які осідають на аеродромі. 

У пришляховому просторі при зльоті літака приблизно 50\% викидів у вигляді мікрочастинок відразу розсіюється на прилеглих до аеропорту територіях. Нагромадження забруднюючих речовин у пришляховій смузі призводить до забруднення екосистем і робить ґрунти на прилеглих територіях непридатними до сільськогосподарського використання. 

Токсичні забруднюючі речовини з пересувних і стаціонарних джерел поділяються за ступенями небезпеки на 4 класи: 
\begin{enumerate}
 \item надзвичайно небезпечні (тетраетилсвинець, свинець, ртуть та ін.); 
 \item високо небезпечні (марганець, мідь, сірчана кислота, хлор та ін.); 
 \item помірно небезпечні (ксилол, метиловий спирт та ін.); 
 \item малонебезпечні (аміак, бензин паливний, газ, оксид вуглецю, скипидар, ацетон та ін.).
\end{enumerate}

Таким чином, авіація є джерелом досить широкого спектру факторів негативного впливу на довкілля. У зв’язку з цим своєчасною і актуальною задачею є розробка і впровадження державних нормативних актів, що регламентували б розташування населених пунктів поблизу аеропортів, а також є доцільною розробка заходів та рекомендацій щодо зниження негативного впливу авіатранспортних процесів на довкілля.

\subsection{Аналіз впливу шуму повітряних суден на навколишнє середовище}

Людина завжди жила в світі звуків і шуму. Звуком називають такі механічні коливання зовнішнього середовища, які сприймаються слуховим апаратом людини (від 16 до 20 000 коливань в секунду). Коливання більшої частоти називають ультразвуком , меншою, --- інфразвуком . Шум --- це набір зукових коливань приблизно однакової амплітуди широкого спектру. 

Авіаційний шум в силу своїх особливостей займає окреме місце серед транспортних джерел шуму внаслідок підвищених рівнів звуку (95-100 дБ поблизу кордону аеропорту), широкосмугового спектрального складу. 

Авіаційний шум несприятливо впливає на широке коло осіб, які безпосередньо пов’язані з діяльністю цивільної авіації: льотно-технічний склад, працівників підприємств цивільної авіації та авіапасажирів, а також населення, що проживає поблизу аеропортів. Несприятливий вплив шуму на людину пов’язаний з загальним роздратуванням, перешкодами розмові, неможливістю заснути, неможливістю зосередитись для виконання конкретної роботи, а при тривалому впливі шуму – втратою слуху та здоров’я. Такий вплив залежить від реакції людини на шум та фізичних характеристик шуму – інтенсивності та спектру, а також тривалості впливу.

Розрізняють три типи критеріїв оцінки подразнюючого впливу шуму:
\begin{enumerate}
 \item максимальні рівні шуму з урахуванням психофізіологічної реакції людини на шум;
 \item ефективні рівні шуму, що характеризуються впливом шуму при польоті літака з урахуванням часу його звучання, 
 \item критерії сумарного впливу шуму, що враховують не тільки максимальні рівні шуму при кожному прольоті, а також їх кількість за певний час доби.
\end{enumerate}

Були проведені дослідження, по даним яких слід очікувати, що максимальні зони зашумлення будуть спостерігатись при зльотах та прольотах по трасам літаків Ту-154 та Іл-86. 

Для зниження шуму використовується обладнання бар`єру (екрану) на шляху розповсюдження шуму. Для цього використовуються спеціальні конструкції, земляні відкопи, будівлі нежитлового призначення, а також смуги зелених насаджень.

\subsection{Аналіз впливу радіохвиль на навколишнє середовище}

З того часу, коли почалося практичне використання радіо, люди почали спостерігати шкідливий вплив радіохвиль на організми живих істот, у тому числі й людей.

Радіохвилі -- це електромагнітні коливання, що розповсюджуються в просторі із швидкістю світла (300 000 км/сек).

Радіохвилі переносять через простір енергію, що випромінюється генератором електромагнітних коливань. Електромагнітне випромінювання характеризується частотою, довжиною хвилі і потужністю переносної енергії. Частота електромагнітних хвиль показує, скільки разів в секунду змінюється у випромінювачі напрям електричного струму і, отже, скільки разів в секунду змінюється в кожній точці простору величина електричного і магнітного полів.

Оточуюче нас середовище завжди перебувало під впливом електромагнітних полів. Ці поля називаються фоновим випромінюванням та спричинені природою. З розвитком науки й техніки фонове випромінювання значно підсилилося. Тому електромагнітні поля, які можна віднести до антропогенних, значно перевищують природний фон і останнім часом перетворилися на небезпечний екологічний чинник.

Як відомо, основний принцип роботи нервової системи людини - передача електромагнітних імпульсів від однієї клітки до іншої. Але ж людина живе в світі, насиченому електромагнітними полями, постійно піддаючись їх шкідливій дії, їх створюють будь-які електричні прилади, теле- і радіоантени, тролейбуси і трамваї. Але найбільшу частину шкідливої дії людина отримує у себе удома або на своєму робочому місці.

\subsection{Характеристика ПК як джерела забруднення}

Усі елементи, які є складовими частинами персонального комп’ютера (ПК), такі, як системний блок, різні пристрої введеня/виведення інформації, засіб візуального відображення інформації, формують складний електромагнітний стан на робочому місці користувача, що вносить свій негативний внесок на навколишнє середовище.

Основними факторами  несприятливого  впливу  роботи  з  ПК є ергономічні параметри екрана монітора (зниження   контрасту   зображення   в   умовах   інтенсивного   зовнішнього освітлення, дзеркальні відблиски від передньої поверхні екранів моніторів, наявність мерехтіння зображення на екрані монітора). Випромінювальні характеристики монітора:

\begin{enumerate}
\item електромагнітне поле монітора в діапазоні частот 20 Гц--1000 МГц;
\item статичний електричний заряд на екрані монітора;
\item ультрафіолетове випромінювання в діапазоні 200--400 нм;
\item інфрачервоне випромінювання в діапазоні 1050 нм -- 1 мм;
\item рентгенівське випромінювання > 1,2 КеВ.
\end{enumerate}


\subsection{Вплив на здоров'я користувача електромагнітних полів ПК}

Вплив електромагнітних полів на людину має негативні наслідки для життєво важливих систем людини і може стати причиною важких захворювань. Адже на біологічну реакцію людини впливають такі параметри електромагнітних полів ЕОМ, як інтенсивність і частота випромінювання, тривалість опромінення і модуляція сигналу, частотний спектр і періодичність дії.

Так, деякі дослідження показали, що навіть при короткочасній роботі (45 хвилин), в організмі користувача, під впливом електромагнітного випромінювання монітора відбуваються значні зміни гормонального стану і специфічні зміни біострумів мозку. А збільшення часу користування ПК стає причиною різних важких захворювань. Згідно статистики, у працюючих за монітором від 2 до 6 годин на добу функціональні порушення центральної нервової системи відбуваються в середньому в 4,6 рази частіше, ніж у контрольних групах, хворобі серцевосудинної системи --- у 2 рази частіше, хвороби верхніх дихальних шляхів --- у 1,9 рази частіше, хвороби опорно-рухового апарата - у 3,1 рази частіше. Як результат --- при восьмигодинній роботі на протязі 4 місяців спостерігається зниження імунітету на 95\%.


\subsection{Комп'ютер як джерело електростатичного поля}

Кожен персональний комп'ютер включає засіб візуального відображення інформації - монітор. Як правило, це пристрій на основі електронно-променевої трубки. Люди, що працюють з монітором, здобувають електростатичний потенціал. Електростатичне поле (Естп) створюється накопиченням електростатичного заряду на екрані кінескопа при роботі монітора. Розкид електростатичних потенціалів користувачів коливається в діапазоні від -3 до +5 КВ.
Крім того, внеском у загальне електростатичне поле являються клавіатури, що електризуються від тертя поверхні, і миші. Експерименти показують, що навіть після роботи з клавіатурою, електростатичне поле швидке зростає з 2 до 12 КВ/м. На окремих робочих місцях в області рук реєструвалися напруженості статичних електричних полів більш 20 КВ/м.

\subsection{Комп'ютер як джерело рентгенівського випромінювання}

Крім причиною створення електростатичного поля, є джерелом рентгенівського, бета - і гамма-випромінювань. Таке випромінювання виникає при роботі монітора за рахунок гальмування пучка електронів і як характеристичне випромінювання атомів матеріалів кінескопа. Спектр рентгенівського випромінювання є безперервним, максимальна енергія якого - 20 КеВ. Джерелом бета-, гамма-випромінювання, які присутні і при включеному і при виключеному моніторі, є радіоактивний розпад ядер сімейств урану і торію, а також ядер калію-40. Спектральний склад гамма-випромінювання переважно складається з набору моноенергетичних ліній. спектральний склад бета- випромінювання безперервний, а його максимальна енергія -1.3 MeB. 

Шкідливий вплив на людину дії іонізуючих випромінювань може призвести до помутніння кришталика ока. Для запобігання такої шкоди здоров’ю людини, у моніторах була знижена анодна напруга, а в скло моніторів доданий свинець.
Серед вказаних вище негативних впливів ПК на здоров’я людини, можна назвати ще й шум в приміщеннях, обладнаних комп'ютерами, рівень якого в таких приміщеннях іноді досягає 85 дБ. Одними з джерел шуму є принтери, техніка й обладнання для кондиціонування повітря, у самих ПЕОМ - вентилятори систем охолодження і трансформатори. 


\subsection{Рекомендації щодо зменшення негативного впливу ПК на здоров'я
 людини та навколишнє середовище}

Перш за все, необхідно    скорочувати   час    роботи   за   комп'ютером, або якнайчастіше робити перерву в роботі. 

Серед основних правил, які слід пам’ятати при роботі з ПК, є те, що не слід залишати комп'ютер включеним на тривалий час, якщо він не використовується, рекомендується використовувати "сплячий режим"     для     монітора.     У    зв'язку     з    тим,     що     електромагнітне випромінювання максимальне збоку монітора,   необхідно розташовувати монітор таким чином, щоб він не випромінював на сусідні робочі місця. Оптимальною відстанню розташування монітора від користувача є більш 1,2 м, критичною - 1,2 м. На даний час, широкого розповсюдження набувають рідинно кристалічні монітори, випромінювання яких значно	менше,	ніж у моніторів з    електроннопроменевою трубкою. Також, комп'ютер повинен бути заземлений, при наявності захисного екрана, його теж варто заземлити.

В Україні безпека рівнів іонізуючих випромінювань комп'ютерних моніторів регламентується нормами НРБУ-97. Стандарти обмежують потужність дози рентгенівського випромінювання величиною 100 мкР/год на відстані 5 см від поверхні екрана монітора і встановлюють для населення межа річної еквівалентної дози випромінювань на кришталик ока рівний 15 мЗв. Потужність дози гамма-випромінювання на відстані 5 см від екрана монітора незначна(0.03-0.1 мкр/год) і складає 0.5\% від потужності дози тіла, щільність потоку бета-випромінювання на відстані 5 см від екрана монітора може складати 0.2-0.5 част/см2, максимальна потужність дози рентгенівського випромінювання на відстані 5 см від екрана монітора порівнянна з фоном і не перевищує 5-15 мкР/год. 

Звідси випливає, що дана відстань від екрану монітора є оптимальною для людини при роботі з ПК і потужність еквівалентної дози випромінювань за такої відстані складе 0.3-0.4 мкЗв/год. І оскільки накопичена хрусталиком ока річна еквівалентна доза (~0.7 мЗв) у 20 разів менше припустимого нормами НРБУ-97 значення, це свідчить про радіаційну безпеку комп'ютерних моніторів.


% \subsection{Розробка заходів щодо охорони навколишнього середовища}
% 
% Заради мінімізації забруднення навколишнього середовища через шкідливий вплив виробництва на нього, в даний час широко використовуються такі методи вирішення проблеми екологічності виробництва, суть яких зводиться до обмеження кількості забруднюючих речовин. Вирішення проблеми екологічної безпеки експлуатації мікросхем від використання електроенергії на сьогоднішній день складається в раціональному використанні енергії, застосуванні нетрадиційних методів її вироблення. Для запобігання радіозабруднення, апаратуру розміщають в екранованому корпусі.
