% \documentclass[ukrainian,utf8,simple,floatsingle,hpadding=5mm]{eskdtext}
% \usepackage[numberright]{eskdplain}
% % variables.tex
% This file contains information about author and other specific
% people for use in eskdx collection.

\title{\fontsize{12}{12} \selectfont Інтегрована інерціально-супутникова система навігації, що базується на принципах комплексної обробки інформації
з використанням калманівської фільтрації}
% smaller size of font set for the title in frame
\author{НовікМ.В.}

\ESKDchecker{ФіляшкінМ.К.}
\ESKDnormContr{КозловА.П.}
\ESKDapprovedBy{СинєглазовВ.М.}

\ESKDdepartment{Міністерство освіти і науки України}
\ESKDcompany{Національний авіаційний університет}

\ESKDsignature{НАУ 11 09 02 000 ПЗ}
\ESKDgroup{ІАСУ 608}

\ESKDsectAlign{section}{Center}
\ESKDsectAlign{subsection}{Center}
\ESKDsectAlign{subsubsection}{Center}


% \include{textcomp}
% \usepackage{longtable} % multipage tables
% \usepackage{multirow} % using rowspan in tables
% 
% \ESKDcolumnXIfIV{}
% \ESKDstyle{formIIab}
% \begin{document}

\ESKDthisStyle{formII}\section{Охорона праці}

В дипломній роботі розробляється інерціально-супутникова навігаційна система, що базується на
основі комплексної обробки інформації з використанням фільтра Калмана. Розробкою алгоритмів роботи
навігаційної системи, налаштуванням обладнання та калібровкою датчиків займаються інженери та 
працівники лабораторії. Робочий процес відбувається в лабораторії навігаційного забезпечення, 
в якій міститься наступне обладнання: 4 ЕОМ, 3D принтер, осцилографи, блоки живлення та паяльне 
обладнання, координатна платформа. Отже субєктом дипломної роботи є інженер лабораторії.

Трудова діяльність користувачів комп’ютерів відбувається у певному виробничому середовищі, де 
діють такі шкідливі фактори ,зазначені в ГОСТ 12.0.003-74: 
\begin{itemize}
  \item Освітлення 
  \item Пожежна безпека
  \item Електробезпека
  \item Шум
  \item Мікроклімат
\end{itemize}




\subsection{Освітлення}
Особу увагу необхідно приділити важливому з точки зору виробничої санітарії 
питанню освітлення на робочому місці.

Виробниче освітлення регулюється нормативно-технічним документом  СНтаП 11-4-79. 
Освітлення на робочому місці повинно бути поєднаним (штучне та природне світло). 
Природне освітлення повинно бути боковим. При виконанні робот з категорії високої 
зорової точності коефіцієнт природної освітленості повинен відповідати нормативним 
рівням по СНтаП 11-4-79 (не нижче 1,5), при зоровій роботі середньої точності – не нижче 1.


Освітлення повинно бути достатнім, щоб очі без зайвого напруження могли розрізняти деталі, що розглядаються; стабільним – 
для цього напруга в електричній мережі не повинна коливатися більше ніж на 4 \%; 
рівномірно розподіленим по робочих поверхнях, щоб очам не доводилося 
потрапляти з дуже темного місця у світле і навпаки; 
таким, що не викликає сліпучої дії на око людини як самого джерела 
світла, так і від відбиваючих поверхонь, що знаходяться в полі зору робітника. 
Зменшення віддзеркалювання джерел світла досягається шляхом застосування світильників; т
аким, щоб не викликати різкі тіні на робочих місцях (цього можна досягти при правильному 
розташуванні світильників); безпечним – не призводити до вибуху, пожежі у виробничих приміщеннях.


Штучне освітлення здійснено у вигляді комбінованої системи освітлення з 
використанням люмінесцентних джерел світла в світильниках загального освітлення. 
Це забезпечує рівномірне освітлення за допомогою відображеного або розсіяного 
світла розподілення.

Розрахунок виробничого освітлення зроблено за методом використання 
світлового потоку. За цім методом світловий потік однієї лампи (у люменах) визначається за формулою:
\begin{equation}
\label{eq:fnop}
 F_n = \frac{E_{n}SkZ}{n \eta}
\end{equation}
\begin{ESKDexplanation}
\item де $E_n$ -- нормована освітленість для проектованих ділянок, цехів, лабораторій;
\item S – площа приміщення, у якому проектується виробниче освітлення, м2;
\item k – коефіцієнт запасу світлового потоку. Він приймається: для люмінесцентних ламп при малому виділенні пилу, диму, кіптяви - 1,5, при середньому і великому виділенні відповідно -1,8 й 2,0; для ламп накалювання при малому виділенні пилу, диму, кіптяви - 1,3, при середньому й великому, відповідно - 1,5 й 1,7;
\item Z – поправочний коефіцієнт, що відбиває відношення  , приймається при найвигіднішому розташуванні світильників, коли світловий потік використається для освітлення робочої зони найбільш раціонально, рівним 1,1...1…1,2; 
\item n – число ламп в приміщенні;
\item  $\eta$ – коефіцієнт використання світового потоку від світильника, що показує, яка частина світлового потоку лампи   досягає освітлюваної поверхні, у тому числі завдяки відбиттю світлового потоку від стін, стелі й робочої поверхні.
Коефіцієнт $\eta$, що залежить від показника геометричних розмірів приміщення   і коефіцієнтів відбиття стін  , стелі   і приміщення  , обчислений для різних типів світильників і приводиться в роботі [13].	  
\end{ESKDexplanation}

Показник  приміщення
\begin{equation}
\label{eq:iop}
 i = \frac{a \times b}{H(a+b)}
\end{equation}
\begin{ESKDexplanation}
  \item де a та b – довжина й ширина освітлюваного приміщення, м;  
  \item H – висота підвісу світильників над робочою поверхнею, м.
\end{ESKDexplanation}

Висота підвісу знаходить з наступної формули:
\begin{equation}
\label{eq:hop}
 H =h_{b} - N - h_{c} 
\end{equation}   
\begin{ESKDexplanation}
\item де $h_{b}$ – висота приміщення, м;
\item $h_{с}$ – висота світильника, м;
\item N – висота робочої поверхні, м.
\end{ESKDexplanation}
Приміщення лабораторії має наступні геометричні формули: довжина робочого кабінету складає 6 м;
ширина - 5 м; висота - 3 м.
Приміщення освітлюється світильниками типа , в кожному з котрих знаходиться по дві лампи ЛБ. 
Висота робочої поверхні складе 0,8 м, висота світильника – 0,275 м.
Визначимо висоту підвісу світильників, підставив вихідні значення в формулу \ref{eq:hop}:
\begin{equation}
 H =   3,0 – 0,275 – 0,8 = 1,925(m).
\end{equation}            

Далі визначимо значення показника приміщення, підставляючи в формулу значення :       
По показнику приміщення та коефіцієнтам світлового потоку від підлоги – 10\% (0,1), від стін – 30\% (0,3) та від стелі – 80\% (0,8)
визначаємо для світильника ML-T8-13W/0.6-SMD значення коефіцієнта використання світлового потоку ($\eta$). $\eta$ = 0,69; 
коефіцієнт запасу світлового потоку k = 1,25; поправочний коефіцієнт Z = 1,2.
Згідно вимогам СНтаП 11-4-79 норма (мінімум) освітленості при проведенні середньо точних робіт складає 400 лк.
Світловий потік лампи ML-T8-13W/0.6-SMD складає 1950 лм.
З формули \ref{eq:fnop} виразимо число ламп в приміщенні та підставляючи відомі значення в вираз одержимо
 
\begin{equation}
 n = \frac{400 \times 30 \times 1.25 \times 1.2 }{1950 \times 069} = 14
\end{equation}

Округляючи значення до більшої цілої цифри, отримуємо, що вимагається 14 ламп. Якщо в світильник 2 лампи то нам необхідно
7 ламп.

\subsection{Пожежна безпека}
Пожежна безпека забезпечена у відповідності з ГОСТ 12.1.004-91 ”Пожежна безпека”. 
Забезпечення пожежної безпеки в 
лабораторії досягається за рахунок застосування мір пожежної профілактики 
й активного пожежного захисту, тобто комплексу мір попередження виникнення 
пожеж або зменшення їх наслідків. Причинами виникнення пожежі 
електроустаткування можуть бути:
\begin{itemize}
 \item перевантаження проводів;
 \item неякісне виконання з'єднань електропроводки;
 \item перевантаження різних електричних пристроїв;
 \item коротке замикання
 \item контакт горючих речовин з нагрівальними пристроями.
\end{itemize}

Відповідно до ДНАОП 0.00-1.31-99, під час проектування систем електропостачання, монтажу 
основного електрообладнання та електричного освітлення приміщень для ЕОМ дотримано вимог 
Правил влаштування електроустановок (ПВЕ), ГОСТ 12.1.006-84, ГОСТ 12.1.019-79, ГОСТ 12.1.030-81, 
ГОСТ 12.1.045-84, ПТЕ, ПБЕ, ВСН 59-88,  СН 357-77, Правил пожежної безпеки в Україні та 
інших нормативних документів, що стосуються штучного освітлення і електротехнічних пристроїв, 
а також вимог нормативно-технічної експлуатаційної документації заводу-виробника.

Приміщення лабораторії відповідно до СніП 11-2-80, дисплейні зали - відносяться до категорії Д 
по пожежній небезпеці і вибуховій небезпеці. Комплекс виробничих приміщень має два евакуаційних виходи.
Приміщення де зберігається інформація відокремлене і  обладнане шафами з негорючих матеріалів (метал).
Для запобігання загоряння від світильників у приміщенні, використовують лампи з температурою нагрівання зовнішніх стекол закритих плафонах.
Загальні вимоги пожежної та вибухової безпеки наведені відповідно в ГОСТ12.1.004-91 та ГОСТ 12.1.010-76.

Для акустичної ізоляції стін використані негорючі та важкогорючі матеріали (гіпсові, гіпсоволокнисті 
плити, із вмістом органічної маси до 8\%, бетонні матеріали, окремі види конструкційних пластмас).

Джерела електричної енергії (розподільчі пристрої, трансформатори) розташовувані у відокремлених приміщеннях.

Освітлювальну електричну мережу виконано згідно вимог ПЕУ – правилам устрою електроустановок для пожежонебезпечних зон.
Прокладання кабелю через перекриття, стіни, фальшпідлогу здійснено в стальних трубах з наповнювачем з негорючих матеріалів.
Аварійні мережі освітлення, дистанційного та автоматичного пуску протипожежних систем та сигналізації 
прокладено окремо від силових та інших електричних комунікацій, а при сумісному прокладанні їх 
розділено перегородками з негорючих матеріалів (метал, гетинакс).

Повітропроводи виконані з негорючих матеріалів. Система вентиляції обладнана пристроєм, що 
забезпечує автоматичне її відключення, а також перекриття повітропроводів лабораторії 
автоматичними заслінками в разі виникнення пожежі. Кабельні вертикальні шахти розділені 
по поверхах діафрагмами з негорючих матеріалів.

Для гасіння та локалізації пожежі до прибуття пожежних підрозділів використовуються 
ручні вогнегасники.  У приміщенні необхідне знаходження двох вуглекислотних вогнегасників 
типу ВВ (ВВ-2, ВВ-5, ВВ-8). Застосування вуглекислотних вогнегасників зумовлено тим, що 
вони можуть використовуватися для гасіння електроустаткування, яке знаходиться під напругою до 1000 В. 

Для виявлення пожежі використовують пожежно-охоронну сигналізацію з датчиками 

Пожежні сповіщувачи використовуються для формування командного імпульсу автоматичного 
пуску системи автоматичного пожежегасіння. Кількість теплових пожежних сповіщувачів 
визначається за таблицею і для приміщення розмірами 15 х 8 х 4 м становить 6. 
Температура спрацювання сповіщувачів встановлюється не менше ніж на 20 0С вище 
максимальної припустимої температури в приміщенні.

Для виявлення пожежі, замість старик точкових пристроїв пропонується
встановити датчики Honeywell Notifier FAPT-851(A)
(Acclimate PlusTM Multi-Sensor Low-Profile Intelligent Detector). Датчик 
використовує комбінацію фотоелектричних та температурних сенсорів для
підвищення імунітету до фальшивого спрацювання. Пристрій обладнаний мікропроцесором
для обробки інформації, в результаті він налаштовує чутливість автоматично
не залежно від оператора контрольної панелі та проводить самотестування.
Іншою перевагою, даного типу датчика є, його безпосередня зв'язаність з системою
кондиціювання та вентиляції. В разі пожежі вимикається та блокується кондиціонер,
та закриваються заслонки вентиляційної системи.

Дані з сенсорів подаються на загальну панель керування пожежно-охоронної системи
сигналізації, а далі на пост чергового пожежної частини. Оператор або панель
керування автоматично приймають рішення, щодо вимкення постачання електроенергії
до приміщення лабораторії

\subsection{Електробезпека}
Електробезпека при експлуатації електроустановок в робочій зоні забезпечена у 
відповідності з ГОСТ 12.1.004-91 “Електробезпека”. Для забезпечення електробезпеки, 
безаварійної і високопродуктивної роботи електроустановок (дисплеїв, друкувальних 
пристроїв, графобудівників) наряду з оснащенням засобами захисту (автоматичне захисне 
відключення, плавкі запобіжники й ін.) організована така їх експлуатація, що виключає 
всяку можливість помилок з боку обслуговуючого персоналу.

В ЕОМ застосовується конструктивний захист людини від дотику до струмоведучих частин: 
ізоляція провідників, виконання корпусів з діелектричних матеріалів. Крім цього, 
обов'язковим є використання захисного заземлення.

В відповідності з вимогами, гранично допустимі рівні напруги дотику і 
струмів при експлуатації і ремонті обладнання забезпечуються: застосуванням малих напруг; ізоляцією струмопровідних мереж; 
обґрунтуванням і оптимальним вибором елементної бази, що виключає передумови 
поразки електричним струмом; правильної компонування, монтажем приладів і елементів; 
дотриманням умов безпеки при постанові і заміні приладів і ін.

Одним із засобів, що забезпечують безпеку людини, що працює з ЕОМ, є --- захисне заземлення. 
Захисним є заземлення, що полягає в надійному з'єднанні  корпуса чи  металевих неструмоведучих 
частин електроустановки з землею. Принцип дії захисного заземлення --- зниження до безпечних значень 
напруги дотику і кроку, обумовлених замиканням на корпус. Для зниження напруги необхідно зменшити 
опір заземлення. Таким чином, з'єднуючи корпус електроустановки з землею, можна знизити напругу, 
що прикладається до тіла людини, до такого значення, при якому струм, що протікає через нього, не 
представляє  смертельної небезпеки .



























% 
% 
% При реалізації проекту на працівників можуть впливати шкідливі і небезпечні виробничі чинники: підвищений 
% рівень рентгенівських випромінювань, недостатнє  освітлення робочої зони, підвищене значення напруги електричного 
% струму, який може проходити крізь людину при коротких замиканнях електричної мережі, підвищений рівень шуму внаслідок
% роботи оргтехніки.
% 
% Електронно-променеві трубки, працюючи при напругах понад 6 кВ є джерелами "м'якого" рентгенівського випромінювання. 
% При напругах понад  10 кВ рентгенівське випромінювання виходить за межі скляного балону і розсіюється в 
% навколишньому просторі виробничого приміщення.
% 
% В силу тісного взаємозв'язку зору людини з роботою мозку освітлення виявляє істотний вплив на центральну нервову 
% систему, керуючу всією життєдіяльністю людини. Раціональне освітлення сприяє підвищенню продуктивності і безпеки 
% праці і збереженню здоров’я працюючих.
% 
% Недостатнє освітлення робочих місць може виникати з таких причин: невірне розташування сусідніх будівель, які 
% можуть створювати затемнення робочої зони; забруднення та недостатня кількість або непрацездатність деяких чи 
% всіх освітлювальних приладів; невірно підібрані чи замінені лампи в світильниках та інші.
% 
% При технічній експлуатації електричного обладнання  можуть виникати електротравми з таких причин: безпосереднє 
% доторкання чи доторкання інструментом до струмопровідних частин електроустановок під напругою, внаслідок невірних 
% дій персоналу, недотримання правил техніки безпеки або внаслідок помилок при монтажі схем і елементів; ураження 
% шаговою напругою при дотику до стін, підлоги, які опинились під напругою по причині погіршення ізоляції чи падінні дротів.  
% 
% \subsection{Технічні заходи щодо ліквідації і зниження дії небезпечних
%  і шкідливих виробничих факторів}
% 
% Приміщення, призначені для роботи ПК, повинні мати природне освітлення. Орієнтація вікон повинна бути на північ або на північний схід, вікна повинні мати жалюзі, які можна регулювати, або штори.
% Не дозволяється розміщувати кабінети обчислювальної техніки у підвальних приміщеннях будинку. Кабінети, обладнанні комп’ютерною технікою, повинні розміщуватись в окремих приміщеннях з природнім освітленням і організованим обміном повітря. Площа на одного працюючого за ПК повинна складати неменше $6 m^2$, об’єм  - не менше $20 m^3$. Стіни , стеля і підлога та обладнання кабінетів комп’ютерної техніки повинні мати покриття із матеріалів з матовою структурою з коефіцієнтом відбиття: стін --- 40--50\%, стелі --- 70--80\%, підлоги --- 20--30\%, предметів обладнання --- 40--50\% (робочого столу  --- 40--50\%, корпуса дисплею та  клавіатури --- 30--50\%, шаф та стелажів --- 40--60\%). Поверхня підлоги повинна  мати антистатичне покриття та бути зручною для вологого прибирання. Забороняється використовувати для оздоблення інтер’єру  комп’ютерних приміщень полімерні матеріали (дерев’яно-стружкові плити, шпалери,що придатні для миття, плівкові та рулонні синтетичні матеріали, шаровий пластик та ін.), що виділяють у повітря шкідливі хімічні речовини, які перевищують гранично допустимі концентрації.
% 
% Вміст шкідливих хімічних речовин в повітрі з комп’ютерною технікою не повинен перевищувати середньодобової концентрації, що наводяться в <<Переліку гранично допустимих концентрацій забруднюючих речовин в атмосферному повітрі населених пунктів> \No 3086-84 від 27.08.84 р. та доповненнях до нього, які затвердженні Міністерствам охорони здоров’я>>.
% Приміщення з ПК повинні мати природне та штучне освітлення. Природне освітлення повинно відповідати вимогам ДБНВ 2.2-3-97  <<Будинки та споруди навчальних закладів>>. Штучне освітлення в приміщеннях з ПК повинно здійснюватись системою загального освітлення. Як джерела світла при штучному освітленні повинні застосовуватись люмінесцентні лампи. Штучне освітлення повинно забезпечувати на робочих місцях в кабінетах з ПК освітленість не нижчу, а на екранах дисплеїв – не вище приведених в таб. \ref{tab:labour protection}
% 
% \begin{table}
% \centering
% \begin{tabular}[c]{|p{5cm}|c|c|c|c|}
% \hline
% \bfseries Характеристика & \bfseries Робоча & \bfseries Площина & \bfseries Освітленість& \bfseries Примітка \\ 
% \bfseries роботи         & \bfseries поверхня & &\bfseries , лк  & \\
% 
% \hline
% Робота переважно з екранами дисплеїв ПК (50\% робочого часу)
% & Екран & В & 200 & не вище \\ 
% & Клавіатура & Г & 400 & не нижче \\ 
% & Стіл & Г & 400 & не нижче \\ 
% 
% \hline
% Робота переважно з документами (з екранами дисплеїв ПК менше 50\% робочого часу)
% & Екран & В & 200 & не вище \\ 
% & Клавіатура & Г & 400 & не нижче \\ 
% & Стіл & Г & 500 & не нижче \\ 
% & Стенд & В & 500 & не нижче \\ 
% 
% \hline
% Проходи основні & Підлога & Г & 100 & \\
% \hline
% \end{tabular}
% \caption{норми освітленості в кабінетах з ПК}
% \label{tab:labour protection}
% \end{table} 
% 
% Загальне освітлення повинно бути виконано у вигляді суцільних або переривчатих  ліній світильників. Для загального 
% освітлення припустимо застосування світильників наступних класів світлорозподілу П (прямого світла), В 
% (переважно  відбитого с вітла).  Застосування   світильників   без   розсіюваів   та екрануючих гратів заборонено. 
% Яскравість світильників загального освітлення в зоні кутів випромінювання від 50$^{\circ}$ до 90$^{\circ}$ з вертикаллю в поздовжній та поперечних площинах повинна складати не быльше 200 кд/кв.м, захисний кут світильників повинен бути не менше 40$^{\circ}$. Коефіціент запаса (КЗ) для освітлювальних установок загального освітлення приймається рівним 1,4. Співвідношення яскравості між робочим екраном і близьким оточенням (стіл, книжки та ін.) не повинно перевищувати 5:1, між поверхнями робочого екрану і оточенням (стіл, обладнання) – 10:1. Величина коефіціенту пульсації освітлення не повиннв перевищувати 5\%. Необхідно передбачити обмеження прямої блискості від джерел природного та штучного освітлення. Яскравість великих поверхонь (вікна, світильники), що знаходяться у полі зору, не повинна перевищувати 200 кд/кв. м. Показник освітленості для джерел штучного освітлення у кабінетах з ВДТ не повинен бути більшим 20.
% 
% Мірою захисту від прямої блискості має бути зниження яскравості видимої частини джерел світла шляхом застосування розсіювачів, відбивачів та інших світлозахисних пристроїв, а також правильне розміщення робочих місць відносно джерел світла.
% 
% У робочій зоні виробничих приміщень ДЕСТ 12.1.005-88 ССБТ «Загальні санітарно-гігієнічні вимоги до повітря робочої зони» установлює норми температури, відносній вологості і швидкості руху повітря в теплий, холодний і перехідний періоди року, виходячи з категорії роботи по складності, призначенню приміщень, надлишкам тепла. Оптимальні параметри повітряного середовища забезпечуються застосуванням опалення, вентиляції і кондиціонування повітря відповідно до вимог БНіП 2.04.05-92 <<Опалення, вентиляція і кондиціонування повітря>>.
% 
% Забезпечення нормальних метеорологічних умов у робочій зоні виробничих приміщень домагаються 
% постійним контролем за ними і проведенням спеціальних заходів. Контроль за станом повітряного 
% середовища повинний виконуватися  з використанням термометрів і термографів (термографи 
% автоматично записують поточну температуру), психрометрів і гігрометрів (для виміру вологості), 
% актинометрів  (для  виміру  інтенсивності  теплових  випромінювань.  У  холодні  і  теплі 
% періоди року температура повітря, швидкість його руху і відносна вологість повітря повинні 
% відповідно складати: 18--20 $^{\circ}$С; 0,1--0,2 м/с; 60--40\%; температура повітря може 
% коливатися  від 16 до 23 $^{\circ}$С при збереженні інших параметрів мікроклімату  в зазначених вище межах. 
% У теплі періоди року температура повітря, його рухливість і відносна вологість повинні 
% відповідно складати: 21--23 $^{\circ}$С; 0,2--0,3 м/с; 60--40\%; температура повітря може 
% коливатися від 18 до 25 $^{\circ}$С при збереженні інших параметрів мікроклімату в зазначених вище межах.
% 
% Заходи, що забезпечують нормальні метеорологічні умови : 
% \begin{enumerate}
%  \item ізоляція джерел надлишкового тепла, їхнє екранування і раціональне розташування, що зменшує схрещування променистих потоків тепла на робочому місці;
%  \item пристрій приточно-витяжної вентиляції, що забезпечує видалення надлишкового тепла і вологи з приміщення, багаторазову зміну повітря й охолодження  організму чи  нагрівання у випадку кондиціонування повітря;
%  \item застосування повітряного душу при трудових процесах, коли інтенсивність теплового випромінювання велика або тепловіддача в навколишнє середовище утруднена.
% \end{enumerate}
% 
% В відповідності з вимогами, гранично допустимі рівні напруги дотику і 
% струмів при експлуатації і ремонті обладнання забезпечуються: застосуванням малих напруг; ізоляцією струмопровідних мереж; 
% обґрунтуванням і оптимальним вибором елементної бази, що виключає передумови 
% поразки електричним струмом; правильної компонування, монтажем приладів і елементів; 
% дотриманням умов безпеки при постанові і заміні приладів і ін.
% 
% Одним із засобів, що забезпечують безпеку людини, що працює з ЕОМ, є --- захисне заземлення. 
% Захисним є заземлення, що полягає в надійному з'єднанні  корпуса чи  металевих неструмоведучих 
% частин електроустановки з землею. Принцип дії захисного заземлення --- зниження до безпечних значень 
% напруги дотику і кроку, обумовлених замиканням на корпус. Напруга, під якою опиняється людина під час  
% дотику до корпуса електроустановки – Uч, rзаз - опір заземлення. Для зниження напруги необхідно зменшити 
% опір заземлення. Таким чином, з'єднуючи корпус електроустановки з землею, можна знизити напругу, 
% що прикладається до тіла людини, до такого значення, при якому струм, що протікає через нього, не 
% представляє  смертельної небезпеки .
% 
% Приклад: Розрахувати загальне освітлення ділянки дефектації вузлів авіаційних двигунів, де норма освітленості при застосуванні люмінесцентних ламп (розряд ) – 400 лк. Розміри приміщення: А = 40 м; В = 20 м; H = 3,0 м. Передбачається використовувати світильники типу ШОД з лампами ЛД, висота підвісу над робочою поверхнею hр = 2,5 м, коефіцієнт запасу приймаємо рівним 1,5 аналогічно приміщенням з малим виділенням пилу, диму і кіптяви.
% Визначимо показник приміщення:
% 
% Задавшись значеннями коефіцієнтів відбиття стелі $p = 0,7$; стін $\rho_z = 0,1$ і освітлюваної поверхні $\rho_p = 0,1$; за спеціальними таблицями знаходимо коефіцієнт використання світлового потоку світильника $\eta = 0,59$. Поправочний коефіцієнт Z приймаємо рівним 1,1.
% 
% Подальший розрахунок може зводитися до визначення необхідного світлового потоку однієї лампи, якщо відома кількість світильників і ламп у них, або до визначення кількості світильників і ламп, якщо відомий тип і потужність ламп.
% 
% Якщо в нашому прикладі передбачається використовувати світильники ШОД з лампами ЛД 2x80, $F = 13200$ лм, то кількість ламп знайдемо з виразу
% 
% Отже, світильники слід розташовувати рівномірно в трьох рядах по одинадцять штук.
% 
% Даний приклад показує нам як необхідно розраховувати виробниче освітлення. Маючи усі данні ми можемо розрахувати виробниче освітлення в необхідному нам приміщенні.
% 
% \subsection{Забезпечення пожежної і вибухової безпеки спроектованого 
%  об'єкта}
% 
% Загальні вимоги по забезпеченню пожежної та вибухової безпеки об’єктів виробничого призначення визначені відповідно у ДЕСТ 12.1.004-91 та ДЕСТ 12.1.010-76.
% 
% Пожежна небезпека може бути обумовлена утворенням електричної дуги, іскор, перегріву струмопровідних елементів.
% Вибухова небезпека відсутня згідно ДЕСТ 12.1.010-76, тому що відсутні джерела їх виникнення.
% 
% Заходи по забезпеченню пожежної безпеки :
% 
% \begin{enumerate}
%  \item лабораторії розміщаються в будівлях не нижче П ступіні вогнетривкості;
%  \item комплекс виробничих приміщень лабораторій має не менш двох самостійних евакуаційних виходів;
%  \item для акустичного оздоблення стін використані негорючі матеріали, які під впливом вогню або високої температури не загоряються, не тліють та не обвуглюються: до негорючих матеріалів відносять усі природні або штучні неорганічні матеріали, а також метали, які використовуються в будівництві.
%  \item джерела електричної енергії (розподільчі пристрої, трансформатори) знаходяться у відокремлених приміщеннях;
%  \item прокладка кабелів через перекриття, стіни здійснюється в сталевих трубах з ущільненням із негорючих матеріалів;
%  \item система вентиляції обладнана пристроєм, який забезпечує її автоматичне вимкнення у випадку пожежі.
% \end{enumerate}
% 
% При виникненні пожежі потрібно вивести людей і матеріальні цінності з небезпечної зони, викликати пожежну охорону, вжити міри по локалізації пожежі, по можливості, вжити міри по гасінню пожежі.
% 
% В приміщеннях є установки гасіння пожеж газовими вогнегасники засобами, в яких вогнегасною речовиною є вуглекислота. Можна також застосовувати для гасіння повітряно-механічну піну, завчасно знеструмив установки, тому що піна є електропроводною. 
% 
% Для гасіння пожеж в лабораторії застосовують переносні вуглекислотні вогнегасники, які установляються з розрахунку: 1 вогнегасник на 40-50 м2 полу.
% 
% Для виявлення пожежі в приміщеннях установлені датчики Honeywell Notifier FAPT-851(A)
% (Acclimate PlusTM Multi-Sensor Low-Profile Intelligent Detector). Датчик 
% використовує комбінацію фотоелектричних та температурних сенсорів для
% підвищення імунітету до фальшивого спрацювання. Датчик обладнаний мікропроцесором
% для обробки інформації, в результаті він налаштовує чутливість автоматично
% не залежно від оператора контрольної панелі та проводить самотестування.
% 
% Дані з сенсорів подаються на загальну панель керування пожежно-охоронної системи
% сигналізації та на
% 
% 
% 
% 
% \subsection{Інструкція з охорони праці під час виконання робіт зі 
%  спроектованим об'єктом}
% Інструкція складена відповідно до вимог ДНАОП 0.00-4.15-98  <<Положення про розробку інструкцій з охорони праці>>.
% 
% До роботи з проектованим об'єктом допускаються обличчя інженерно-технічного складу, що вивчили проектований пристрій, інструкцію з технічної експлуатації, дійсну інструкцію і що склали залік по техніці безпеки і пожежної безпеки.
% \begin{enumerate}
%  \item Упорядкувати робоче місце. 
%  \item Перевірити  справність роз’ємів кабелів електроживлення і блоків пристроїв, відсутність зламів і ушкоджень ізоляції живильних проводів, відсутність відкритих струмоведучих частин у пристроях ПК;
%  \item Відрегулювати сидіння робочого стільця (крісла) на оптимально зручну висоту      (кут нахилу спинки стільця повинний змінюватися в межах 90-11-град. до площини сидіння).
%  \item Розташувати крісло і дисплей так, щоб кут зору на екрані складав 15 град., а відстань до екрана 400-800 мм;
%  \item Вжити заходів, щоб при нормальній освітленості робочого місця пряме світло не падало на екрани моніторів.
%  \item Перед включенням штепсельної вилки кабелю електроживлення в розетку 220 В переконайтеся в тому, що усі вимикачі мережі на всіх пристроях ПК знаходяться в положенні «заземлені» (занулені).
%  \item Після підключення пристроїв ПКдо електромережі установіть яскравість і фокус зображення ВДТ ручками регулювання відповідно до особливості свого зору.
%  \item Не залишати свого робочого місця без повідомлення керівника робіт.
%  \item Не залишати працюючий ПКі його пристрої без спостереження.
%  \item Підключати і відключати роз’єми кабелів пристроїв ПК тільки при відключеній напрузі електричної мережі.
%  \item Подавати напругу на пристрої й окремі блоки ПК тільки після ретельної перевірки надійності кріплення провідників заземлення, справності кабелів і роз’ємів мережі електроживлення.
%  \item Для операторів ПК повинні бути додатково введені дві-три регламентованих перерви тривалістю 10 хвилин кожна, дві перерви при 8-мигодинному робочому дні, три перерви при 12-тигодинному робочому дні. 
%  \item Кількість оброблюваних символів (чи знаків ВДТ) не повинна перевищувати 30 тис. за 4 години роботи.
%  \item Установити в положення «виключено» усі тумблера (вимикачі) пристроїв, з якими ви працювали, а також перемикачі (рубильники) на електрощитах.
%  \item Відключити штепсельні вилки від розеток електроживлення.
%  \item Про всі несправності, виявлені під час роботи і про вжиті заходи з їхнього усунення, доповісти керівнику з відповідним записом у журналі обліку робіт.
%  \item Виключити загальний вимикач електроживлення всіх робочих місць.
%  \item Виключити світло на робочому місці й у приміщенні.
%  \item В аварійних ситуаціях:
%   \subitem Негайно припиніть роботу.
%   \subitem Залиште небезпечну зону і вживіть заходів з попередження подальшого розвитку аварії.
%   \subitem Повідомте про те, що трапилося, свого керівника, чи керівника ділянки, на якій відбулася аварія. 
%   \subitem При нещасних випадках забезпечте  долікарську допомогу потерпілому.
% \end{enumerate}
% 
% За порушення чи невиконання цих вимог винні несуть відповідальність відповідно до чинного законодавства.
\end{document}
