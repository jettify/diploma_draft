
\subsection{Розробка алгоритмів роботи БІНС}

\textbf{Алгоритм роботи з використанням кутів Ейлера}
В якості навігаційної системи координат застосуємо умовну географічну систему координат \textit{O}$\xi$$\eta$$\zeta$.Положення 
ЛА відносно Землі визначається географічною довготою $\lambda$у, широтою $\varphi$у 
і відстанню від центра Землі \textit{R}. Зразу відмітимо, що при застосування географічної 
координатної системи для БІНС виникають певні складності при роботі на високих широтах 
(складності зв'язані з визначенням довготи). Тому доцільно в якості навігаційної 
системи координат використовувати ортодромічну. Але на цьому етапі рішення навігаційних 
задач БІНС саме в географічній координатній системі.   

В якості вихідних параметрів будемо використовувати умовні географічні координати, 
а навігаційним трикутником буде служити трикутник, який будемо позначати $\xi$$\eta$$\zeta$.

Складання кінематичних рівнянь БІНС зводиться до виводу рівнянь руху акселерометричного тригранника $xyz$(рухомої 
системи координат \textit{В}) щодо земної поверхні (навігаційної   системи координат \textit{N}). 
Складання кінематичних рівнянь БІНС передбачає безпосереднє перерахування прискорень $a_{x} $, $a_{y} $, $a_{z} $,  
виміряних акселерометрами, на осі $\xi $, $\eta $, $\zeta $ навігаційної   системи 
координат \textit{N. }Для цього визначаються (обчислюються) кути відхилення акселерометричного 
тригранника щодо навігаційного $\psi_{a} $, $\vartheta_{a} $,  $\gamma_{a} $.

Нижче наведений один з варіантів запису кінематичних рівнянь БІНС \eqref{GrindEQ__2_1_1_}, 
у яких перерахування параметрів руху ЛА з рухомої системи координат у навігаційну 
здійснюється рішенням трансцедентних рівнянь, тобто за допомогою ейлерівських кутів.  



\[\begin{array}{l} 
{\dot{V}_{\xi } =a_{\xi } -\frac{V_{\xi } V_{\eta } \cos \varphi_{y} }{R_{{\rm C.?}} 
+H\cos \varphi_{y} } -\frac{V_{\xi } V_{\zeta } \cos \varphi_{y} }{R_{{\rm C.<}} 
+H\cos \varphi_{y} } -2(\Omega_{\eta } V_{\zeta } -\Omega_{\zeta } V_{\eta } )+g_{
\xi } ;} \\ {\dot{V}_{\eta } =a_{\eta } +\frac{V_{\xi }^{2} \cos \varphi_{y} }{R_{{
\rm C.?}} +H\cos \varphi_{y} } -\frac{V_{\zeta }^{2} }{R_{{\rm C.<}} +H} -2(\Omega 
_{\zeta } V_{\xi } -\Omega_{\xi } V_{\zeta } )+g_{\eta } ;} \\ {\dot{V}_{\zeta } 
=a_{\zeta } +\frac{V_{\xi }^{2} \cos \varphi_{y} }{R_{{\rm C.?}} +H\cos \varphi 
_{y} } -\frac{V_{\eta } V_{\zeta } }{R_{{\rm C.<}} +H} -2(\Omega_{\xi } V_{\eta 
} -\Omega_{\eta } V_{\xi } )+g_{\zeta } ;} \end{array}\] 

\[\begin{array}{l} {a_{\xi } =a_{x} \cos \psi_{a} \cos \vartheta_{a} +a_{y} (\sin 
\psi_{a} \sin \gamma_{a} -\cos \psi_{a} \sin \vartheta_{a} \cos \gamma_{a} )+} 
\\ {{\rm \; \; \; \; \; \; \; \; }+a_{z} (\sin \psi_{a} \cos \gamma_{a} +\cos \psi 
_{a} \sin \vartheta_{a} \sin \gamma_{a} );} \\ {a_{\eta } =a_{x} \sin \vartheta 
_{a} +a_{y} \cos \vartheta_{a} \cos \gamma_{a} -a_{z} \cos \vartheta_{a} \sin 
\gamma_{a} ;} \\ {a_{\zeta } =-a_{x} \sin \psi_{a} \cos \vartheta_{a} +a_{y} (
\cos \psi_{a} \sin \gamma_{a} +\sin \psi_{a} \sin \vartheta_{a} \cos \gamma_{a} 
)+} \\ {{\rm \; \; \; \; \; \; \; \; }+a_{z} (\cos \psi_{a} \cos \gamma_{a} -\sin 
\psi_{a} \sin \vartheta_{a} \sin \gamma_{a} );} \end{array}\] 

\[\begin{array}{l} {\dot{\lambda }_{y} =\frac{V_{\xi } }{R_{{\rm y.?}} +H\cos \varphi 
_{y} } ;} \\ {\dot{H}=V_{\eta } ;} \\ {\dot{\varphi }_{y} =-\frac{V_{\zeta } }{R_{{
\rm C.<}} +H} } \end{array}\] 

\[\begin{array}{l} {\dot{\psi }_{0} =\frac{1}{\cos \vartheta_{a} } (\omega_{ay} 
\cos \gamma_{a} -\omega_{az} \sin \gamma_{a} +\omega_{\xi } \cos \psi_{a} \sin 
\vartheta_{a} -} \\ {{\rm \; \; \; \; \; \; \; \; \; }-\omega_{\eta } \cos \vartheta 
_{a} -\omega_{\zeta } \sin \psi_{a} \sin \vartheta_{a} );} \\ {\dot{\vartheta 
}_{a} =\omega_{ay} \sin \gamma_{a} +\omega_{az} \cos \gamma_{a} -\omega_{\xi 
} \sin \psi_{a} -\omega_{\zeta } \cos \psi_{a} ;} \\ {\dot{\gamma }_{a} =\omega 
_{ax} -(\omega_{ay} \cos \gamma_{a} -\omega_{az} \sin \gamma_{a} )tg\vartheta 
_{a} -\omega_{\xi } \cos \psi_{a} \sec \vartheta_{a} +} \\ {{\rm \; \; \; \; \; 
\; \; }+\omega_{\zeta } \sin \psi_{a} \sec \vartheta_{a} ;} \end{array}\] 

\[\begin{array}{l} {\omega_{\xi } =\Omega_{\xi } -\dot{\varphi }_{y} ;} \\ {\omega 
_{\eta } =\Omega_{\eta } +\dot{\lambda }_{y} \sin \varphi_{y} ;} \\ {\omega_{
\zeta } =\Omega_{\zeta } -\dot{\lambda }_{y} \cos \varphi_{y} ;} \end{array}\] 

\[\begin{array}{l} 
{\Omega_{\xi } =\Omega_{\xi 0} \cos \lambda_{y} -\Omega_{\eta 0} \sin \lambda 
_{y} ;} \\ {\Omega_{\eta } =(\Omega_{\xi 0} \sin \lambda_{y} +\Omega_{\eta 0} 
\cos \lambda_{y} )\cos \varphi_{y} -\Omega_{\xi 0} \sin \varphi_{y} } \\ {\Omega 
_{\zeta } =(\Omega_{\xi 0} \sin \lambda_{y} +\Omega_{\eta 0} \cos \lambda_{y} 
)\sin \varphi_{y} +\Omega_{\zeta 0} \sin \varphi_{y} } \end{array}\] 



Нижче зображений алгоритм роботи БІНС, що розглядається (рис.2.1).

\eject 





Рис. 2.1. Алгоритм роботи БІНС з використанням кутів Ейлера



\textbf{\underbar{Алгоритм роботи БІНС  з використанням кватерніонів}}

\underbar{Запис кінематичних рівнянь БІНС у географічній системі координат має певні 
недоліки, тому в роботі запропоновано розв'язувати кінематичні рівняння БІНС в ортодромічної 
системі координат, з одночасним спрощенням математичної моделі Землі і використанням 
при отриманні матриці направляючих косинусів сучасну теорію кватерніонів. }

При використанні ортодромічних систем координат можна позбутися від недоліків  алгоритмів 
автономного рішення навігаційних завдань у географічній системі координат при польотах 
на високих широтах (складності пов'язані з визначенням довготи). У якості ортодромічної 
системи запропоновано використати праву ортодромічну систему координат. У цьому випадку, 
по-перше, спрощуються перемикання режимів роботи навігаційного обчислювача. По-друге, 
основне ядро алгоритмів числення координат у географічної і в ортодромічній системах 
не змінюється. 

Отримано кінематичні рівняння автономного рішення навігаційних завдань в ортодромічній 
системі координат по даним бесплатформенної курсовертикалі. 

Як навігаційний тригранник у даних кінематичних рівняннях служить раніше обраний 
тригранник \textit{LR$\Phi $}, а \textit{ }вихідними параметрами розрахунків уважаються 
умовні координати, які надалі перетворяться  або в географічні, або в ортодромічні. 

\[\begin{array}{l} 
{\dot{\lambda }=\frac{V_{L} }{R_{{\rm ?}} } ;} \\ {\dot{}=V_{R} ;} \\ {\dot{\varphi 
}_{{\rm C}} =\frac{V_{\Phi } }{R_{{\rm <}} } ;} \end{array}\] 

Складові шляхової швидкості ЛА  \textit{VL , VR , VФ} у навігаційній системі координат 
визначаються  в обох алгоритмах шляхом проектування векторного рівняння на відповідні 
осі навігаційного тригранника \textit{LR$\Phi $} у вигляді:

\[\begin{array}{l} {\dot{V}_{L} =a_{L} +V_{R} \omega_{\$_{\Sigma } } -V_{\Phi } 
\omega_{R_{\Sigma } } ;} \\ {\dot{V}_{R} =a_{R} +V_{\Phi } \omega_{L_{\Sigma } 
} -V_{L} \omega_{\$_{\Sigma } } +g_{R} ;} \\ {\dot{V}_{\Phi } =a_{\Phi } +V_{L} 
\omega_{R_{\Sigma } } -V_{R} \omega_{L_{\Sigma } } ,} \end{array}\] 

а сама орієнтація навігаційного тригранника залежить від обраної для навігаційних 
розрахунків системи координат (географічної або ортодромічної). 

Співвідношення для розрахунку проекцій кутової швидкості навігаційного тригранника 
відносно інерціального простору так само не змінюються

\[\begin{array}{l} {\omega_{\$_{\Sigma } } =\omega_{\$_{V} } +2\Omega_{\$ } 
;} \\ {\omega_{R_{\Sigma } } =\omega_{R_{V} } +2\Omega_{R} ;} \\ {\omega_{L_{
\Sigma } } =\omega_{L_{V} } +2\Omega_{L} .} \end{array}\] 

Не змінюються й формули розрахунку складової відносної кутової швидкості навігаційного 
тригранника $\omega_{R_{V} } ,{\rm \; }\omega_{L_{V} } ,{\rm \; }\omega_{\$_{V} 
} $.

\[
\begin{array}{l} 
{\omega_{L_{V} } =\frac{V_{\$ } }{R_{{\rm <}} } =\dot{\varphi }_{{\rm C}} ;} \\ 
{\omega_{R_{V}} =-\frac{V_{L} }{R_{{\rm ?}} } {\rm sin}\varphi 
_{{\rm C}} =-\dot{\lambda }{\rm sin}\varphi_{{\rm C}} ;} \\ 
{\omega_{\$_{V} } =-\frac{V_{L} }{R_{{\rm ?}} } =-\dot{\lambda }.} \end{array}\] 

При записі рівнянь БІНС в ортодромічній системі координат (з урахуванням малих відхилень 
ЛА від ортодромії) можна використати навіть сферичну модель Землі, при цьому відповідність 
між еліпсоїдом і сферою може бути встановлена за способом Рачковского, що по заданому 
ортодромічному напрямку забезпечує таку ж точність, з якої вирішуються геодезичні 
завдання на поверхні еліпсоїда. Радіус сфери в цьому випадку визначається вираженням $R_{0} 
=a_{e} \left(1-\frac{\alpha_{e} }{2} \sin {\rm \varphi }_{} \right)$ ($a_{e} $ = 
63783888 $-$ більша піввісь еліпсоїда; $\alpha_{e} $= 0,003352 $-$ стиснення еліпсоїда, $\varphi $\textit{В} $-$ геодезична 
широта вертекса ортодромії $\varphi $\textit{В  }= 90 $-$ $\Psi$0). 

де  \textit{е} -- ексцентриситет еліпсоїда, \textit{е}2\textit{ }= 6,73$.$10-3,     

\textit{a}  $-$ більша 
піввісь  еліпсоїда  рівна  \textit{a }=  metricconverterProductID6378388 м6378388 
м.

$\Psi$0 $-$ шляховий кут ортодромії в точці початку відліку.

\underbar{Однак можна, використати й спрощенні описання двох радіусів кривизни для 
сфероїда (еліпсоїда обертання).}

%\includegraphics[bb=0mm 0mm 208mm 296mm, width=119.9mm, height=49.1mm, viewport=3mm 4mm 205mm 292mm]{image2.eps}

$\begin{array}{l} {\Omega_{L} =\Omega_{L_{0} } \cos 
\lambda_{{\rm >@B}} -\Omega_{R_{0} } \sin \lambda_{{\rm >@B}} ;} \\ {\Omega_{R} 
=(\Omega_{L_{0} } \sin \lambda_{{\rm >@B}} +\Omega_{R_{0} } \cos \lambda_{{\rm 
>@B}} )\cos \varphi_{{\rm >@B}} -\Omega_{\$_{0} } \sin \varphi_{{\rm >@B}} ;} 
\\ {\Omega_{\$ } =(\Omega_{L_{0} } \sin \lambda_{{\rm >@B}} +\Omega_{R_{0} } 
\cos \lambda_{{\rm >@B}} )\sin \varphi_{{\rm >@B}} +\Omega_{\$_{0} } \cos \varphi 
_{{\rm >@B}} ;} \\ {\Omega_{L_{0} } =\Omega_{{\rm 7}} \cos \varphi_{0} \cos \Psi 
_{0} ;} \\ {\Omega_{R_{0} } =\Omega_{{\rm 7}} \sin \varphi_{0} ;} \\ {\Omega_{
\$_{0} } =-\Omega_{{\rm 7}} \cos \varphi_{0} \sin \Psi_{0} ,} \end{array}$$\begin{array}{l} 
{\Omega_{L} =\Omega_{L_{0} } \cos \lambda_{{\rm >@B}} -\Omega_{R_{0} } \sin \lambda 
_{{\rm >@B}} ;} \\ {\Omega_{R} =(\Omega_{L_{0} } \sin \lambda_{{\rm >@B}} +\Omega 
_{R_{0} } \cos \lambda_{{\rm >@B}} )\cos \varphi_{{\rm >@B}} -\Omega_{\$_{0} 
} \sin \varphi_{{\rm >@B}} ;} \\ {\Omega_{\$ } =(\Omega_{L_{0} } \sin \lambda 
_{{\rm >@B}} +\Omega_{R_{0} } \cos \lambda_{{\rm >@B}} )\sin \varphi_{{\rm >@B}} 
+\Omega_{\$_{0} } \cos \varphi_{{\rm >@B}} ;} \\ {\Omega_{L_{0} } =\Omega_{{
\rm 7}} \cos \varphi_{0} \cos \Psi_{0} ;} \\ {\Omega_{R_{0} } =\Omega_{{\rm 7}} 
\sin \varphi_{0} ;} \\ {\Omega_{\$_{0} } =-\Omega_{{\rm 7}} \cos \varphi_{0} 
\sin \Psi_{0} ,} \end{array}$При роботі навігаційного обчислювача в ортодромічної 
системі координат складові швидкості обертання Землі $\Omega_{R} ,{\rm \; }\Omega 
_{L} ,{\rm \; }\Omega_{\$ } $ визначаються вже по іншим співвідношеннями

тут  $\Omega_{{\rm 7}} $ -- кутова швидкість обертання Землі. $\Omega_{{\rm 7}} $= 
7,27$.$10-5 рад/с.

$\Psi$0, $\varphi$0 $-$ відповідно географічна широта й шляховий кут ортодромії в 
точці початку відліку. 

При польоті по окремих ортодроміям, коли маршрут задається в окремо-ортодромічній 
(етапно-ортодромічних) системі координат, $\Psi$0, $\varphi$0 $-$ географічна широта 
й шляховий кут окремої ортодромії початкового ППМ цієї ортодромії.

Проекції \textit{$0_{L,R,\$ } $ }уявного прискорення ЛА на осі навігаційного тригранника \textit{LR$\Phi $ }перераховуються 
за показниками акселерометрів зі зв'язаної з ЛА системи координат \textit{XYZ  }з 
використанням матриці напрямних косинусів \textit{В.}

\[\left[\begin{array}{c} {0_{L} } \\ {0_{R} } \\ {0_{\$ } } \end{array}\right]=\left[
\begin{array}{c} {0_{x_{{\rm }} } } \\ {0_{y_{{\rm }} } } \\ {0_{z_{{\rm }} } } \end{array}
\right].\] 

У роботі запропоновано для визначення орієнтації ЛА використати не тільки напрямні 
косинуси, але й параметри Родріго-Гамільтона у формі кватерніонів.  Із цією метою 
розроблені алгоритми формування матриці \textit{В}, що визначає перетворення векторів 
із системи координат, зв'язаної з ЛА в навігаційну, використовують кватерніонний 
метод її обчислення. Достоїнством методу кватерніонів полягає в тому, що він дозволяє 
описувати перехід від однієї системи координат до іншої за допомогою всього лише 
чотирьох чисел, а не 9 напрямних косинусів.

Кватерніонний метод ґрунтується  на теоремі Ейлера, яка говорить, що будь-який дійсний 
поворот однієї системи координат щодо іншої можна представити, як поворот на деякий 
кут навколо однієї нерухомої осі.

Кватерніон являє собою компактну форму запису орієнтації зазначеної осі (векторна 
частина кватерніона $\lambda_{1} ,\lambda_{2} ,\lambda_{3} $) і кута повороту 
(скалярна частина кватерніона $\lambda_{0} $) відповідно до теореми Ейлера.

Використання кватерніонів дозволяє представити ортогональні перетворення у формі 
множення кватерніонів. Дія над кватерніонами допускає матричне подання з використанням 
симетризованних матриць, що дуже зручно для створення програм бортових обчислювачів. 

Відповідно 
до теореми Ейлера-Шаля, усяке переміщення твердого тіла, що має нерухому крапку можна 
представити як результат повороту навколо незмінного напрямку (ейлеровой осі) на 
певний кут $\varphi $. Якщо зв'язати з розглянутим твердим тілом правий ортогональний 
координатний тригранник, то параметри Родріго-Гамільтона $\lambda_{0} ,\lambda_{1} 
,\lambda_{2} ,\lambda_{3} ,$ однозначно характеризують згадані переміщення, можна 
задати виразами 

\[\begin{array}{l} {\lambda_{1} =\frac{l_{1} \sin \varphi }{2} ,\, \, \, \, \, \, 
\lambda_{2} =\frac{l_{2} \sin \varphi }{2} ,\, \, \, \, \, } \\ {\lambda_{3} =
\frac{l_{3} \sin \varphi }{2} ,\, \, \, \, \, \lambda_{0} =\frac{\cos \varphi }{2} 
,} \end{array}\] 

де $l_{1} ,l_{2} ,l_{3} -$косинуси кутів, що утворюються ейлеровою віссю з осями 
тригранника в його початковому й кінцевому положенні. Зв'яжемо з ЛА, що несе бесплатформенну 
інерціальну систему, ортонормований базис \textit{Е} $-$ праву трійку взаємно ортогональних 
одиничних векторів $e_{1} ,e_{2} ,e_{3} .$ Орієнтацію базису \textit{Е} відносно 
ортонормованого інерціального базису \textit{І}, складеного з ортів $i_{1} ,i_{2} 
,i_{3} ,$ охарактеризуємо параметрами Родріго-Гамільтона $\lambda_{0} ,\lambda_{1} 
,\lambda_{2} ,\lambda_{3} .$ Матриця напрямних косинусів, обчислена по параметрах 
Родріго-Гамільтона (кватерніонам) має такий вигляд:

\[B=\left[\begin{array}{ccc} {\lambda_{0}^{2} +\lambda_{1}^{2} -\lambda_{2}^{2} 
-\lambda_{3}^{2} } & {2(\lambda_{1} \lambda_{2} -\lambda_{0} \lambda_{3} )} 
& {2(\lambda_{0} \lambda_{2} +\lambda_{1} \lambda_{3} )} \\ {2(\lambda_{0} \lambda 
_{3} +\lambda_{1} \lambda_{2} )} & {\lambda_{0}^{2} -\lambda_{1}^{2} +\lambda 
_{2}^{2} -\lambda_{3}^{2} } & {2(\lambda_{2} \lambda_{3} -\lambda_{0} \lambda 
_{1} )} \\ {2(\lambda_{1} \lambda_{3} -\lambda_{0} \lambda_{2} )} & {2(\lambda 
_{2} \lambda_{3} +\lambda_{0} \lambda_{1} )} & {\lambda_{0}^{2} -\lambda_{1}^{2} 
-\lambda_{2}^{2} +\lambda_{3}^{2} } \end{array}\right].\] 

Вимірювачі кутової швидкості, що входять до складу БІНС, вимірюють координати $\omega 
_{x} ,\omega_{y} ,\omega_{z} $ вектора $\bar{{\rm \Omega }}$ абсолютної кутової 
швидкості базису \textit{Е}, задані в цьому базисі. Необхідно знаючи значення параметрів 
Родріго-Гамільтона в початковий момент часу $t=t_{0} $ й використовуючи сигнали Вимірювачів 
кутової швидкості, обчислювати параметри Родріго-Гамільтона при $t>t_{0} $. У початковий 
момент часу за інформацією СБКВ про кути крену, тангажа й курсу можна розрахувати 
початкові значення параметрів  Родріго-Гамільтона 

\[\begin{array}{l} {\lambda_{0_{{\rm 0}} } =\sin \, \left({\gamma_{0}  \mathord{
\left/{\vphantom{\gamma_{0}  2}}\right.\kern-\nulldelimiterspace} 2} \right)\sin 
\, \left({\vartheta_{0}  \mathord{\left/{\vphantom{\vartheta_{0}  2}}\right.\kern-
\nulldelimiterspace} 2} \right)\sin \, \left({\psi_{0}  \mathord{\left/{\vphantom{
\psi_{0}  2}}\right.\kern-\nulldelimiterspace} 2} \right)+\cos \, \left({\gamma 
_{0}  \mathord{\left/{\vphantom{\gamma_{0}  2}}\right.\kern-\nulldelimiterspace} 
2} \right)\cos \, \left({\vartheta_{0}  \mathord{\left/{\vphantom{\vartheta_{0}  
2}}\right.\kern-\nulldelimiterspace} 2} \right)\cos \, \left({\psi_{0}  \mathord{
\left/{\vphantom{\psi_{0}  2}}\right.\kern-\nulldelimiterspace} 2} \right);} \\ 
{\lambda_{1_{{\rm 0}} } =-\sin \, \left({\vartheta_{0}  \mathord{\left/{\vphantom{
\vartheta_{0}  2}}\right.\kern-\nulldelimiterspace} 2} \right)\sin \, \left({\psi 
_{0}  \mathord{\left/{\vphantom{\psi_{0}  2}}\right.\kern-\nulldelimiterspace} 2} 
\right)\cos \, \left({\gamma_{0}  \mathord{\left/{\vphantom{\gamma_{0}  2}}\right.
\kern-\nulldelimiterspace} 2} \right)+\sin \left({\gamma_{0}  \mathord{\left/{\vphantom{
\gamma_{0}  2}}\right.\kern-\nulldelimiterspace} 2} \right)\cos \, \left({\vartheta 
_{0}  \mathord{\left/{\vphantom{\vartheta_{0}  2}}\right.\kern-\nulldelimiterspace} 
2} \right)\cos \, \left({\psi_{0}  \mathord{\left/{\vphantom{\psi_{0}  2}}\right.
\kern-\nulldelimiterspace} 2} \right);} \\ {\lambda_{2_{{\rm 0}} } =\sin \left({
\gamma_{0}  \mathord{\left/{\vphantom{\gamma_{0}  2}}\right.\kern-\nulldelimiterspace} 
2} \right)\cos \, \left({\vartheta_{0}  \mathord{\left/{\vphantom{\vartheta_{0}  
2}}\right.\kern-\nulldelimiterspace} 2} \right)\sin \, \left({\psi_{0}  \mathord{
\left/{\vphantom{\psi_{0}  2}}\right.\kern-\nulldelimiterspace} 2} \right)+\sin 
\, \left({\vartheta_{0}  \mathord{\left/{\vphantom{\vartheta_{0}  2}}\right.\kern-
\nulldelimiterspace} 2} \right)\cos \, \left({\gamma_{0}  \mathord{\left/{\vphantom{
\gamma_{0}  2}}\right.\kern-\nulldelimiterspace} 2} \right)\cos \, \left({\psi_{0}  
\mathord{\left/{\vphantom{\psi_{0}  2}}\right.\kern-\nulldelimiterspace} 2} \right);} 
\\ {\lambda_{3_{{\rm 0}} } =\sin \, \left({\psi_{0}  \mathord{\left/{\vphantom{
\psi_{0}  2}}\right.\kern-\nulldelimiterspace} 2} \right)\cos \, \left({\gamma_{0}  
\mathord{\left/{\vphantom{\gamma_{0}  2}}\right.\kern-\nulldelimiterspace} 2} \right)
\cos \, \left({\vartheta_{0}  \mathord{\left/{\vphantom{\vartheta_{0}  2}}\right.
\kern-\nulldelimiterspace} 2} \right)-\sin \, \left({\gamma_{0}  \mathord{\left/{
\vphantom{\gamma_{0}  2}}\right.\kern-\nulldelimiterspace} 2} \right)\sin \, \left({
\vartheta_{0}  \mathord{\left/{\vphantom{\vartheta_{0}  2}}\right.\kern-\nulldelimiterspace} 
2} \right)\cos \, \left({\psi_{0}  \mathord{\left/{\vphantom{\psi_{0}  2}}\right.
\kern-\nulldelimiterspace} 2} \right)} \end{array}\] 

Поточні значення параметрів $\lambda_{0} ,\lambda_{1} ,\lambda_{2} ,\lambda_{3} 
,$ можна визначити, знаючи проекції кутової швидкості ЛА $\omega_{x} ,\omega_{y} 
,\omega_{z} $ на зв'язаній осі $XYZ$, шляхом рішення лінійного диференціального 
рівняння зі змінними коефіцієнтами. У цьому випадки параметри $\lambda_{0} ,\lambda 
_{1} ,\lambda_{2} ,\lambda_{3} ,$ кватерніона описують положення осей ЛА $XYZ$ відносно 
інерціального простору:

\[\dot{\lambda }=\frac{1}{2} \Omega \left(t\right)\cdot \lambda \left(t\right).\] 

де $\Omega 
\left(t\right)$$-$ кососиметрична $\left(4\times 4\right)$-матриця, що ставиться 
у відповідність вектору $\omega =\left[\omega_{x} \omega_{y} \omega_{z} \right]^{{
\rm T} } $

\[\Omega \left(t\right)=\left[\begin{array}{cccc} {0} & {-\omega_{x} } & {-\omega 
_{y} } & {-\omega_{z} } \\ {\omega_{x} } & {0} & {\omega_{z} } & {-\omega_{y} 
} \\ {\omega_{y} } & {-\omega_{z} } & {0} & {\omega_{x} } \\ {\omega_{z} } & 
{\omega_{y} } & {-\omega_{x} } & {0} \end{array}\right];\lambda =\left[\begin{array}{c} 
{\lambda_{0} } \\ {\lambda_{1} } \\ {\lambda_{2} } \\ {\lambda_{3} } \end{array}
\right]\] 

Дане вираження  являє собою кватерніонне однорідне лінійне диференціальне рівняння 
першого порядку зі змінним коефіцієнтом у вигляді гіперкомплексного числа з дійсною 
частиною, рівною нулю. У скалярній формі це рівняння  має такий вигляд

\[\begin{array}{l} {\dot{\lambda }_{0} =0,5\, \, \left(\omega_{x} \lambda_{1} +
\omega_{y} \lambda_{2} +\omega_{z} \lambda_{3} \right);} \\ {\dot{\lambda }_{1} 
=0,5\, \, \left(\omega_{x} \lambda_{0} +\omega_{z} \lambda_{2} +\omega_{y} \lambda 
_{3} \right);} \\ {\dot{\lambda }_{2} =0,5\, \, \left(\omega_{y} \lambda_{0} +
\omega_{x} \lambda_{3} -\omega_{z} \lambda_{1} \right);} \\ {\dot{\lambda }_{3} 
=0,5\, \, \left(\omega_{z} \lambda_{0} +\omega_{y} \lambda_{1} -\omega_{x} \lambda 
_{2} \right);} \end{array}\] 

Легко визначити, що зворотні співвідношення мають вигляд

\[\begin{array}{l} {\omega_{x} =2\left(\lambda_{0} \dot{\lambda }_{1} -\lambda 
_{1} \dot{\lambda }_{0} +\dot{\lambda }_{2} \lambda_{3} -\dot{\lambda }_{3} \lambda 
_{2} \right);} \\ {\omega_{y} =2\left(\lambda_{0} \dot{\lambda }_{2} -\lambda_{2} 
\dot{\lambda }_{0} +\dot{\lambda }_{3} \lambda_{1} -\dot{\lambda }_{1} \lambda_{3} 
\right);} \\ {\omega_{z} =2\left(\lambda_{0} \dot{\lambda }_{3} -\lambda_{3} \dot{
\lambda }_{0} +\dot{\lambda }_{1} \lambda_{2} -\dot{\lambda }_{2} \lambda_{1} \right).} 
\end{array}\] 

Динаміка зміни параметрів кватерніона у випадку, коли кватерніон характеризує взаємне 
положення пов'язаних з ЛА осей $XYZ$ і навігаційних осей \textit{LR$\Phi $}, описується 
рівняннями

\[\left[\begin{array}{c} {\dot{\lambda }_{0} } \\ {\dot{\lambda }_{1} } \\ {\dot{
\lambda }_{2} } \\ {\dot{\lambda }_{3} } \end{array}\right]=\frac{1}{2} \left[\begin{array}{cccc} 
{0} & {-\omega_{x\Sigma } } & {-\omega_{y\Sigma } } & {-\omega_{z\Sigma } } \\ 
{\omega_{x\Sigma } } & {0} & {\omega_{z\Sigma } } & {-\omega_{y\Sigma } } \\ {
\omega_{y\Sigma } } & {-\omega_{z\Sigma } } & {0} & {\omega_{x\Sigma } } \\ {
\omega_{z\Sigma } } & {\omega_{y\Sigma } } & {-\omega_{x\Sigma } } & {0} \end{array}
\right]\cdot \left[\begin{array}{c} {\lambda_{0} } \\ {\lambda_{1} } \\ {\lambda 
_{2} } \\ {\lambda_{3} } \end{array}\right]\] 

\underbar{У свою чергу  }

\[\begin{array}{l} {\, \omega_{x_{\Sigma } } =\omega_{x_{{\rm }} } -\omega_{x_{
\$ LR} } ;\, \, \, } \\ {\omega_{y_{\Sigma } } =\omega_{y_{{\rm }} } -\omega_{y_{
\$ LR} } ;\, \, \, \, } \\ {\omega_{z_{\Sigma } } =\omega_{z_{{\rm }} } -\omega 
_{z_{\$ LR} } .} \end{array}\] 

де  $\omega_{y_{{\rm }} } ,{\rm \; }\omega_{x_{{\rm }} } ,{\rm \; }\omega_{z_{{
\rm }} } $$-$ проекції кутової швидкості ЛА відносно інерціального простору на осі 
зв'язаної системи координат, що вимірюються датчиками кутових швидкостей;

$\omega_{x_{\$ LR} } ,\omega_{y_{\$ LR} } ,\omega_{z_{\$ LR} } $$-$ проекції кутової 
швидкості навігаційної системи координат відносно інерціального простору на осі зв'язаної 
системи координат, які визначаються в результаті рішення матричного рівняння 

\[\left[\begin{array}{c} {\omega_{x_{\$ LR} } } \\ {\omega_{y_{\$ LR} } } \\ {
\omega_{z_{\$ LR} } } \end{array}\right]=B^{{\rm B}} \left[\begin{array}{c} {\omega 
_{\$_{V} } +\Omega_{\$ } } \\ {\omega_{R_{V} } +\Omega_{R} } \\ {\omega_{L_{V} 
} +\Omega_{L} } \end{array}\right].\] 

Дані складові розраховуються й у раніше розроблених алгоритмах.


\textbf{У скалярній формі динаміка зміни параметрів кватерніона описується  системою 
рівнянь, що має наступний вид}

\[\begin{array}{l} {\dot{\lambda }_{0} =-0,5(\omega_{x\Sigma } \lambda_{1} +\omega 
_{y\Sigma } \lambda_{2} +\omega_{z\Sigma } \lambda_{3} );} \\ {\dot{\lambda }_{1} 
=-0,5(\omega_{x\Sigma } \lambda_{0} +\omega_{z\Sigma } \lambda_{2} +\omega_{y
\Sigma } \lambda_{3} );} \\ {\dot{\lambda }_{2} =-0,5(\omega_{y\Sigma } \lambda 
_{0} +\omega_{z\Sigma } \lambda_{1} +\omega_{x\Sigma } \lambda_{3} );} \\ {\dot{
\lambda }_{3} =-0,5(\omega_{z\Sigma } \lambda_{0} +\omega_{y\Sigma } \lambda_{1} 
+\omega_{x\Sigma } \lambda_{2} ).} \end{array}\] 

По елементах матриці  \textit{B} можна визначити кути орієнтації ЛА $-$  крен g, 
тангаж  J, рискання (курс) $\psi $: 

\[\psi =-{\rm arctg}\left(\frac{b_{31} }{b_{11} } \right){\rm \; \; };\, \, \, \, 
\gamma ={\rm arctg}\left(\frac{b_{23} }{b_{22} } \right){\rm \; \; };\, \, \, \, 
\vartheta ={\rm arctg}\left(\frac{b_{21} }{\sqrt{b_{22}^{2} +b_{33}^{2} } } \right){
\rm \; \; }.\] 

Однак якщо крен і тангаж, внаслідок особливостей динаміки польоту літаків цивільної 
авіації не повинні перевищувати \textbar $\gamma$\textbar   $\leq$  30\dots 35$\circ$, 
\textbar $\vartheta$\textbar  $\leq$  6\dots 10$\circ$, то курс літака може змінюватися 
необмежено. Тому при розрахунках кутових координат літака доцільніше користуватися 
раніше отриманими співвідношеннями, принаймні, при розрахунках ортодромічного курсу 



\[\begin{array}{l} 
{\dot{\psi }=\left(\omega_{y_{\Sigma } } \cos \gamma -\omega_{z_{\Sigma } } \sin 
\gamma \right)\sec \vartheta ;} \\ {\dot{\gamma }=\omega_{x_{\Sigma } } +{\rm tg}
\vartheta {\rm \; }\left(\omega_{z_{\Sigma } } \sin \gamma -\omega_{y_{\Sigma } 
} \cos \gamma \right);} \\ {\dot{\vartheta }=\omega_{y_{\Sigma } } \sin \gamma +
\omega_{z_{\Sigma }} \cos \gamma .} \end{array}\] 

На основі вище викладеного було проведене уточнення розроблених алгоритмів числення 
навігаційних параметрів польоту, як у географічній так й в ортодромічній системі 
координат. При цьому основний блок $-$ блок проведення навігаційних розрахунків залишається 
незмінним.



$\begin{array}{l} {\omega_{y_{\Sigma } } =\omega_{y_{{\rm }} } -\omega_{y_{\$ 
LR} } ;} \\ {\omega_{x_{\Sigma } } =\omega_{x_{{\rm }} } -\omega_{x_{\$ LR} } 
;} \\ {\omega_{z_{\Sigma } } =\omega_{z_{{\rm }} } -\omega_{z_{\$ LR} } .} \end{array}$                                                      (I)



$\begin{array}{l} 
{\dot{\psi }=\left(\omega_{y_{\Sigma } } \cos \gamma -\omega_{z_{\Sigma } } \sin 
\gamma \right)\sec \vartheta ;} \\ {\dot{\gamma }=\omega_{x_{\Sigma } } +{\rm tg}
\vartheta {\rm \; }\left(\omega_{z_{\Sigma } } \sin \gamma -\omega_{y_{\Sigma } 
} \cos \gamma \right);} \\ {\dot{\vartheta }=\omega_{y_{\Sigma } } \sin \gamma +
\omega_{z_{\Sigma }} \cos \gamma ;} \\ {\psi_{{\rm >@B}} =-\psi .} \end{array}$                                              (II)



$\begin{array}{l} 
{\dot{\lambda }_{0} =-0,5(\omega_{x\Sigma } \lambda_{1} +\omega_{y\Sigma } \lambda 
_{2} +\omega_{z\Sigma } \lambda_{3} );} \\ {\dot{\lambda }_{1} =-0,5(\omega_{x
\Sigma } \lambda_{0} +\omega_{z\Sigma } \lambda_{2} +\omega_{y\Sigma } \lambda 
_{3} );} \\ {\dot{\lambda }_{2} =-0,5(\omega_{y\Sigma } \lambda_{0} +\omega_{z
\Sigma } \lambda_{1} +\omega_{x\Sigma } \lambda_{3} );} \\ {\dot{\lambda }_{3} 
=-0,5(\omega_{z\Sigma } \lambda_{0} +\omega_{y\Sigma } \lambda_{1} +\omega_{x
\Sigma } \lambda_{2} ).} \end{array}$                           (III)



$B=\left[\begin{array}{ccc} {\lambda_{0}^{2} +\lambda_{1}^{2} -\lambda_{2}^{2} 
-\lambda_{3}^{2} } & {2(\lambda_{1} \lambda_{2} -\lambda_{0} \lambda_{3} )} 
& {2(\lambda_{0} \lambda_{2} +\lambda_{1} \lambda_{3} )} \\ {2(\lambda_{0} \lambda 
_{3} +\lambda_{1} \lambda_{2} )} & {\lambda_{0}^{2} -\lambda_{1}^{2} +\lambda 
_{2}^{2} -\lambda_{3}^{2} } & {2(\lambda_{2} \lambda_{3} -\lambda_{0} \lambda 
_{1} )} \\ {2(\lambda_{1} \lambda_{3} -\lambda_{0} \lambda_{2} )} & {2(\lambda 
_{2} \lambda_{3} +\lambda_{0} \lambda_{1} )} & {\lambda_{0}^{2} -\lambda_{1}^{2} 
-\lambda_{2}^{2} +\lambda_{3}^{2} } \end{array}\right]$           (III)



\textbf{$\left[\begin{array}{c} {0_{L} } \\ {0_{R} } \\ {0_{\$ } } \end{array}\right]=
\left[\begin{array}{c} {0_{x_{{\rm }} } } \\ {0_{y_{{\rm }} } } \\ {0_{z_{{\rm }} 
} } \end{array}\right]$}                                                   (IV) 



$\begin{array}{l} 
{\dot{V}_{L} =a_{L} +V_{R} \left(\omega_{\$_{V} } +2\Omega_{\$ } \right)-V_{\Phi 
} \left(\omega_{R_{V} } +2\Omega_{R} \right);} \\ {\dot{V}_{R} =a_{R} +V_{\Phi 
} \omega_{L_{V} } -V_{L} \left(\omega_{\$_{V} } +2\Omega_{\$ } \right)+g_{R} 
;} \\ {\dot{V}_{\Phi } =a_{\Phi } +V_{L} \left(\omega_{R_{V} } +2\Omega_{R} \right)-V_{R} 
\omega_{L_{V} } .} \end{array}$                       (V)





$\begin{array}{l} {\dot{\lambda }_{{\rm >@B}} =-\omega_{\$_{V} } =\frac{V_{L} }{R_{{
\rm ?}} } ;} \\ {\dot{}=V_{R} ;} \\ {\omega_{R_{V}} =\omega_{\$_{V} } {\rm sin}
\varphi_{{\rm C}} ;} \\ {\dot{\varphi }_{{\rm >@B}} =\omega_{L_{V} } =\frac{V_{
\$ } }{R_{{\rm <}} } .} \end{array}$                                                      (VI) 



$\begin{array}{l} 
{\Omega_{L} =\Omega_{L_{0} } \cos \lambda_{{\rm >@B}} -\Omega_{R_{0} } \sin \lambda 
_{{\rm >@B}} ;} \\ {\Omega_{R} =(\Omega_{L_{0} } \sin \lambda_{{\rm >@B}} +\Omega 
_{R_{0} } \cos \lambda_{{\rm >@B}} )\cos \varphi_{{\rm >@B}} -\Omega_{\$_{0} 
} \sin \varphi_{{\rm >@B}} ;} \\ {\Omega_{\$ } =(\Omega_{L_{0} } \sin \lambda 
_{{\rm >@B}} +\Omega_{R_{0} } \cos \lambda_{{\rm >@B}} )\sin \varphi_{{\rm >@B}} 
+\Omega_{\$_{0} } \cos \varphi_{{\rm >@B}} ;} \end{array}$       (VII) 



$\left[\begin{array}{c} {\omega_{x_{\$ LR} } } \\ {\omega_{y_{\$ LR} } } \\ {\omega 
_{z_{\$ LR} } } \end{array}\right]=B^{{\rm B}} \left[\begin{array}{c} {\omega_{
\$_{V} } +\Omega_{\$ } } \\ {\omega_{R_{V} } +\Omega_{R} } \\ {\omega_{L_{V} 
} } \end{array}\right]$                                             (VIII)



$\begin{array}{c} {R_{{\rm ?}} =R_{{\rm <}} =R_{0} =a_{e} \left(1-\frac{\alpha_{e} 
}{2} \sin {\rm \varphi }_{} \right)} \\ {g_{R} =-g\left(1+5,2884\cdot 10^{-3} \sin 
^{2} \varphi \right){\rm \; }\left[1-\frac{2_{{\rm 7}} }{R_{0} } \left(1-e{\rm \; 
}\sin ^{2} \varphi \right)\right]{\rm \; }} \end{array}$                    (IX)



Структурна 
схема алгоритму (порядок проведення навігаційних розрахунків) наведена на рис. 2.2.

\eject \textbf{}

--$\omega 
_{x_{\$ LR} } \, \omega_{y_{\$ LR} } \, \omega_{z_{\$ LR} } $$\omega_{y_{\Sigma 
} } ;\, \, \omega_{x_{\Sigma } } ;\, \, \omega_{z_{\Sigma } } $--$\omega_{x_{
\$ LR} } \, \omega_{y_{\$ LR} } \, \omega_{z_{\$ LR} } $$\omega_{y_{\Sigma } } 
;\, \, \omega_{x_{\Sigma } } ;\, \, \omega_{z_{\Sigma } } $


